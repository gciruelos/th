\section*{}

El espacio de cómputos de un programa puede tener una estructura compleja.
Considere un lenguaje de programación sin efectos secundarios,
como lo es por ejemplo el cálculo $\lambda$ simplemente tipado.
Los cómputos que se pueden realizar a partir de una tupla $(A,B)$
son secuencias $(A, B) \to \hdots \to (A', B')$.
Estas secuencias siempre pueden descomponerse
en dos cómputos independientes,
$A \to \hdots \to A'$ y $B \to \hdots \to B'$.
La razón por la cual podemos hacer esto es que las sub-expresiones $A$ y $B$
no pueden interactuar la una con la otra.
Dicho de forma más general, el espacio de cómputos de $(A,B)$ puede
verse como el producto de los espacios de $A$ y $B$.

Por otro lado, el espacio de cómputos de la aplicación $f(A)$
no es tan fácil de caracterizar.
El problema es que $f$ puede usar o no el valor de A,
y en caso de usarlo puede posiblemente hacerlo más de una vez.


La motivación de esta tesis es intentar \textbf{entender el espacio de cómputos} de programas.
El objetivo de entender el espacio de cómputos del cálculo $\lambda$ es comprender
las propiedades de distintas \textit{estrategias de evaluación},
como son call-by-name o call-by-value desde un punto de vista cuantitativo.
Una mayor comprensión de estas estrategias desde un punto de vista cuantitativo puede
sugerir optimizaciones a programas y podría permitir justificar que
ciertas conversiones de programas son correctas (por ejemplo, que no
convierten un programa que termina en uno que no).

Otro objetivo para entender el espacio de cómputos del cálculo $\lambda$ es comprender mejor
distintas nociones de equivalencias de programas~\cite{Tesis:Morris,Bre16}.
Usualmente, se pueden definir diferentes equivalencias de
programas respecto a cómo un programa se evalúa en un contexto:
por ejemplo,
podríamos definir que dos $A$ y $B$ son (contextualmente) equivalentes
si, para todo contexto $\con$, $\conof{A} =_{\beta} \conof{B}$.
Es claro que poder descomponer el espacio de cómputos de $\conof{A}$ (resp. $\conof{B}$) en una
estructura que dependa del espacio de cómputos de $A$ (resp. $B$) sería conveniente
para ganar intuición sobre las distintas nociones de equivalencia observacional.

\vspace{1cm}

Es sabido que los espacios de derivación del cálculo $\lambda$ presentan
ciertas regularidades~\cite{Tesis:Levy:1978,DBLP:journals/tcs/Zilli84,laneve1994distributive,thesismellies,levy_redex_stability,DBLP:conf/lics/AspertiL13}.
En su tesis de doctorado, L\'evy~\cite{Tesis:Levy:1978}
probó que el espacio de derivaciones de un término es un semi-reticulado superior: todo par de derivaciones
$\redseq,\redseqtwo$ de un término $\tm$
tienen una cota inferior mínima $\redseq \sqcup \redseqtwo$, definida como $\redseq(\redseqtwo/\redseq)$,
que es única módulo \textit{permutation equivalence}.

Pero el espacio de derivación de un término no es fácil de entender en general: el Problema~2 de la
lista de problemas abiertos de la RTA~\cite{dershowitz1991open}
propone el asunto de investigar las propiedades de \textit{spectra}, es decir, espacios de derivación.

En cálculo $\lambda$, el ejemplo que dimos antes se traduce en
relacionar el espacio de derivación de la aplicación $\tm \tmtwo$
con los espacios de derivación de $\tm$ y $\tmtwo$, lo cual parece ser un problema difícil en general.
De nuevo, esto se debe a que $\tm$ bien puede borrar o duplicar a $\tmtwo$,
lo cual sugiere que el problema está relacionado con el de manejo de \textit{recursos}:
si supiéramos exactamente cómo $\tm$ usa a $\tmtwo$ el problema se simplificaría.
Por eso en esta tesis nos proponemos explorar la hipótesis de que tener una representación
explícita del manejo de recursos puede proveer una mejor intuición
sobre la estructura de los espacios de derivación.

Existen varios cálculos $\lambda$ que abarcan el problema del manejo de recursos
de manera explícita~\cite{boudol1993lambda,ehrhard2003differential,kesner2007resource,kesner2009prismoid},
la mayoría de los cuales están inspirados en la lógica lineal de Girard~\cite{girard1987linear}.

Recientemente, una familia de estos formalismos, cálculos dotados con
\textit{sistemas de tipos intersección no idempotentes},
ha recibido atención~\cite{ehrhard2012collapsing,bernadet2013non,bucciarelli2014inhabitation,bucciarelli2017non,kesner2016reasoning,thesisvial,KRV18}.
Estos sistemas de tipos permiten capturar, de manera estática, propiedades dinámicas no triviales de los términos
(en particular distintos grados de \textit{normalización}), mientras que al mismo tiempo se prestan a
demostraciones simples por inducción.

Los sistemas de tipos intersección fueron originalmente propuestos por
Coppo y Dezani-Ciancaglini~\cite{DBLP:journals/aml/CoppoD78}
para estudiar la terminación del cálculo $\lambda$.
Se caracterizan por la presencia de un constructor de tipos \textit{intersección} $\typ \cap \typtwo$.
Los sistemas de tipos intersección \textit{no idempotentes} se distinguen de los (clásicamente) idempotentes
por el hecho de que la intersección de tipos no es idempotente, es decir que $\typ$ y $\typ \cap \typ$
no son tipos equivalentes.
En cambio, la intersección se comporta como un conectivo multiplicativo en lógica lineal.
Los argumentos de las funciones típicamente tendrán varios tipos dependiendo de cuántas veces y cómo son usados,
lo cual da a estos sistemas una notable capacidad a la hora de rastrear cómo son utilizados los recursos
de un programa.

Los sistemas de tipos intersección no idempotentes fueron originalmente formulados por
Gardner~\cite{gardner1994discovering},
y mas tarde reintroducidos por de Carvalho~\cite{carvalho2007semantiques}.

\vspace{1cm}

En sistemas de tipos intersección no idempotentes típicos, la normalización de pruebas no es confluente.
En esta tesis presentamos un sistema confluente de tipos intersección no idempotentes para el cálculo $\lambda$.
Escribimos las derivaciones de tipos usando una sintaxis concisa de términos de prueba.
El sistema gozará de buenas propiedades: progreso, será fuertemente normalizante, y tendrá una teoría de
residuos muy regular.
Estableceremos una correspondencia con el cálculo lambda mediante teoremas de simulación.


Además, los espacios de derivación de los términos del sistema propuesto tendrán una estructura muy simple.
Esa simplicidad junto con los teoremas de simulación nos ayudará a estudiar un fragmento de interés
de los espacios de derivación de los términos del cálculo $\lambda$ con mucha facilidad.

La maquinaria de los tipos intersección no idempotentes nos permite seguir el rastro del uso de los recursos
necesarios para obtener una respuesta.
En particular, induce una noción de \textit{basura}: un cómputo es basura si no contribuye a hallar una respuesta.
Usando estas nociones, mostraremos que el espacio de derivaciones de un término $\lambda$ puede ser factorizado
usando una variante de la construcción de Grothendieck para semi-reticulados.
Esto significa, en particular, que cualquier derivación del cálculo $\lambda$ puede ser escrita
de una única manera como un prefijo libre de basura, seguido de basura.

\vspace{0.5cm}
\noindent
\textbf{Palabras clave:} Cálculo lambda, Tipos intersección, Espacio de derivación, Reticulado
