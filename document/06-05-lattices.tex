

We introduce the following definition mostly to fix nomenclature/notation:

\begin{definition}[Upper semilattices]
An \defn{upper semilattice} is a triple $(X,\leq,\lor)$
where $X$ is a set, $\leq$ is a partial order on $X$,
and there are binary joins $x \lor y$ for all $x, y \in X$.
An \defn{upper semillatice with bottom} is an upper semilattice
with a bottom element $\bot \in X$.
As customary, by abuse of notation, we write $X$ for both the structure
and the underlying set, when clear from the context.
We may write $\bot_X$ to emphasize that $\bot$ is the bottom element of $X$.

A \defn{morphism of upper semilattices} $f : X \to Y$
is a monotonic function $f : X \to Y$
(\ie $x \leq y$ implies $f(x) \leq f(y)$)
preserving joins,
that is $f(x \lor y) = f(x) \lor f(y)$.
A \defn{morphism of upper semilattices with bottom}
moreover preserves the bottom element, \ie $f(\bot) = \bot$.

Upper semmilattices provided with morphisms form a category $\USL$.
Upper semmilattices with bottom provided with morphisms form a category $\USLB$.
Any upper semilattice may be regarded as a category
whose objects are the elements of $X$, and such that there is a (unique) morphism
$x \to y$ if and only if $x \leq y$.
We write $\pt{y}{x}$ for such morphism.

As usual, if $X$ and $Y$ are posets, the set of functions $f : X \to Y$
is also a poset with $f \leq g$ defined as $f(x) \leq g(x)$ for all $x \in X$.
Thus $\USL$ forms a $2$-category
in which $0$-cells are upper semilattices,
$1$-cells are morphisms of upper semilattices $f : X \to Y$, 
and there is a $2$-cell $f \totwo g$ if and only if $f \leq g$.
\end{definition}

\begin{definition}[Lax 2-functor]
Let $\Cat$ be a category and $\CatTwo$ a $2$-category.
A lax 2-functor $F : \Cat \to \CatTwo$
is a pair $(F_1,F_2)$ where $F_1 : \Ob(\Cat) \to \Ob(\CatTwo)$
is a function, and for every morphism $f : X \to Y$ in $\Cat$,
\[
  F_2(f) : F_1(X) \to F_1(Y) \HS\text{ is a 1-cell in $\CatTwo$}
\]
such that functor laws hold up to $2$-cells.
We will be interested in the following notion of lax $2$-functor:
\begin{enumerate}
\item $F_2(\id_X) = \id_{F_1(X)}$ for all $X \in \Ob(\Cat)$.
\item $F_2(f \circ g) \leq F_2(f) \circ F_2(g)$ is a $2$-cell
      for any two morphisms $g : X \to Y$, $f : Y \to Z$ in $\Cat$.
\end{enumerate}
As usual with functors, we may write $F$ to stand for either $F_1$ or $F_2$,
when clear from the context.
\end{definition}

\begin{definition}[Grothendieck construction for $\USL$s]
Let $A$ be an upper semilattice, and let $B : A \to \USL$ be a lax 2-functor.
The Grothendieck construction $\grothy{A}{B}$
is defined as the upper semilattice given by the
set:
\[
  \set{(a,b) \ST a \in A,\ b \in B(a)}
\]
and the following structure:
\[
\begin{array}{rcl}
(a, b) \leq (a', b')    & \iffdef & a \leq a' \text{ and } B(\pt{a'}{a})(b) \leq b' \\
(a,b) \lor (a',b')       & \eqdef & (a \lor a', B(\pt{(a \lor a')}{a})(b) \lor B(\pt{(a \lor a')}{a'})(b')) \\
\end{array}
\]
Let us check that this forms indeed an upper semilattice:
\begin{enumerate}
\item {\bf Partial order.}
  Let us show that $\leq$ is a partial order.
  \begin{enumerate}
  \item {\bf Reflexivity.}
    Note that $a \leq a$. Moreover the morphism $\pt{a}{a} \in \Hom_A(a,a)$ is the identity because $A$ is a poset.
    So $b = B(\id_a)(b) = B(\pt{a}{a})(b) \leq b$.
  \item {\bf Antisymmetry.}
    Let $(a,b) \leq (a',b') \leq (a,b)$.
    First note that $a \leq a' \leq a$, so $a = a'$.
    Then since $A$ is a poset,
    we have that $\pt{a'}{a} \in \Hom_A(a,a')$ and $\pt{a}{a'} \in \Hom_A(a',a)$ are both the identity morphism.
    Moreover, we know that $b = B(\pt{a'}{a})(b) \leq b'$
    and symmetrically $b' = B(\pt{a}{a'})(b') \leq b$, so $b = b'$.
  \item {\bf Transitivity.}
    Let $(a,b) \leq (a',b') \leq (a'',b'')$.
    First note that $a \leq a' \leq a''$ so $a \leq a''$.
    Note that $\pt{a''}{a} \in \Hom_A(a,a'')$ and $(\pt{a''}{a'}) \circ (\pt{a'}{a}) \in \Hom_A(a,a'')$
    both witness the inequality $a \leq a''$.
    Since $A$ is a poset, necessarily $\pt{a''}{a} = (\pt{a''}{a'}) \circ (\pt{a'}{a})$.
    Moreover, note that we know that $B(\pt{a'}{a})(b) \leq b'$ and that $B(\pt{a''}{a'})(b') \leq b''$.
    Hence:
    \[
    \begin{array}{rcll}
      B(\pt{a''}{a})(b)
                  & =    & B((\pt{a''}{a'}) \circ (\pt{a'}{a}))(b) & \text{since $\pt{a''}{a} = (\pt{a''}{a'}) \circ (\pt{a'}{a})$} \\
                  & \leq & B(\pt{a''}{a'})(B(\pt{a'}{a})(b))       & \text{since $B((\pt{a''}{a'}) \circ (\pt{a'}{a})) \leq B(\pt{a''}{a'}) \circ B(\pt{a'}{a})$} \\
                                                     &&& \text{ as $B$ is a lax 2-functor} \\
                  & \leq & B(\pt{a''}{a'})(b')       & \text{since $B(\pt{a''}{a'})$ is a morphism of $\USL$s,} \\
                                                     &&& \text{ hence monotonic} \\
                  & \leq & b''
    \end{array}
    \]
  \end{enumerate}
\item {\bf Join.}
  Let us show that $(a, b) \lor (a',b') = (a \lor a', B(\pt{(a \lor a')}{a})(b) \lor B(\pt{(a \lor a')}{a'})(b'))$.
  is the supremum of $\set{(a, b), (a',b')}$.
  \begin{enumerate}
  \item {\bf Upper bound.}
    Let us show that $(a, b) \leq (a, b) \lor (a',b')$.
    Checking that $(a', b') \leq (a, b) \lor (a',b')$ is symmetrical.

    First, $a \leq a \lor a'$.
    Moreover, $B(\pt{(a \lor a')}{a})(b) \leq B(\pt{(a \lor a')}{a})(b) \lor B(\pt{(a \lor a')}{a'})(b')$.
  \item {\bf Least upper bound.}
    Let $(a'',b'')$ be another upper bound, \ie an element such that $(a, b) \leq (a'', b'')$ and $(a', b') \leq (a'', b'')$.
    Let us show that $(a, b) \lor (a',b') \leq (a'', b'')$.

    First, $a \leq a''$ and $a' \leq a''$, so $a \lor a' \leq a''$.
    Moreover, we know that
    $B(\pt{a''}{a})(b) \leq b''$ and $B(\pt{a''}{a'})(b') \leq b''$.
    \TODO{
    FALLA
    Pero vale en nuestro caso particular.\\
    %%%
    Let $(\cls{\redseq''},\cls{\redseqtwo''})$
    such that $(\cls{\redseq},\cls{\redseqtwo}) \leq (\cls{\redseq''},\cls{\redseqtwo''})$
    and $(\cls{\redseq'},\cls{\redseqtwo'}) \leq (\cls{\redseq''},\cls{\redseqtwo''})$,
    and let us show that
    $(\cls{\redseq},\cls{\redseqtwo}) \lor (\cls{\redseq'},\cls{\redseqtwo'}) \leq (\cls{\redseq''},\cls{\redseqtwo''})$.
    First note that $\cls{\redseq} \leqF \cls{\redseq''}$ and $\cls{\redseq'} \leqF \cls{\redseq''}$
    so $\cls{\redseq} \lorF \cls{\redseq'} \leqF \cls{\redseq''}$.\\
    Moreover, we know that:
    \[
      \cls{\redseq\redseqtwo/\redseq''} = \ulbG(\pt{\cls{\redseq''}}{\cls{\redseq}})(\cls{\redseqtwo}) \permle \cls{\redseqtwo''}
      \HS\text{ and }\HS
      \cls{\redseq'\redseqtwo'/\redseq''} = \ulbG(\pt{\cls{\redseq''}}{\cls{\redseq'}})(\cls{\redseqtwo'}) \permle \cls{\redseqtwo''}
    \]
    Let $\alpha := (\redseq \sqcup \redseq') \sieve \tm'$.
    First we claim that $\alpha \permle \redseq\redseqtwo \sqcup \redseq'\redseqtwo'$.
    Indeed, $\alpha = (\redseq \sqcup \redseq') \sieve \tm' \permle \redseq \sqcup \redseq'$
    by \rlem{sieve_is_prefix}, and it is easy to check that
    $\redseq \sqcup \redseq' \permle \redseq\redseqtwo \sqcup \redseq'\redseqtwo'$.
    \\
    What we have to check is the following inequality:
    \[
      \ulbG(\pt{\cls{\redseq''}}{\cls{\alpha}})(
        \ulbG(\pt{\cls{\alpha}}{\cls{\redseq}})(\cls{\redseqtwo})
        \sqcup
        \ulbG(\pt{\cls{\alpha}}{\cls{\redseq'}})(\cls{\redseqtwo'})
      )
      \permle
      \cls{\redseqtwo''}
    \]
    \
    Indeed:
    \[
      \begin{array}{rcll}
      &&
      \ulbG(\pt{\cls{\redseq''}}{\cls{\alpha}})(
        \ulbG(\pt{\cls{\alpha}}{\cls{\redseq}})(\cls{\redseqtwo})
        \sqcup
        \ulbG(\pt{\cls{\alpha}}{\cls{\redseq'}})(\cls{\redseqtwo'})
      ) \\
      & = &
      \cls{\alpha((\redseq\redseqtwo/\alpha) \sqcup (\redseq'\redseqtwo'/\alpha))/\redseq''}
      \\
      & = &
      \cls{\alpha((\redseq\redseqtwo \sqcup \redseq'\redseqtwo')/\alpha)/\redseq''}
      & \text{since $A/C \sqcup B/C \permeq (A \sqcup B)/C$}
      \\
      & = &
      \cls{(\redseq\redseqtwo \sqcup \redseq'\redseqtwo')(\alpha/(\redseq\redseqtwo \sqcup \redseq'\redseqtwo'))/\redseq''}
      & \text{since $A(B/A) \permeq B(A/B)$}
      \\
      & = &
      \cls{(\redseq\redseqtwo \sqcup \redseq'\redseqtwo')/\redseq''}
      & \text{since $\alpha \permle \redseq\redseqtwo \sqcup \redseq'\redseqtwo'$}
      \\
      & = &
      \cls{\redseq\redseqtwo/\redseq'' \sqcup \redseq'\redseqtwo'/\redseq''}
      & \text{since $A/C \sqcup B/C \permeq (A \sqcup B)/C$}
      \\
      & \permle &
      \cls{\redseqtwo''}
      & \text{since $\cls{\redseq\redseqtwo/\redseq''} \permle \cls{\redseqtwo''}$} \\
      &&& \text{ and $\cls{\redseq'\redseqtwo'/\redseq''} \permle \cls{\redseqtwo''}$}
      \end{array}
    \]
    %%%
    }
  \end{enumerate}
\end{enumerate}
\end{definition}

\begin{proposition}[Derivations form an $\USLB$]
\lprop{derivations_form_an_ulb}
Let $\cls{\redseq}$ denote the permutation equivalence class of a derivation $\redseq$.
Let $\tm$ be a $\lambda$-term.
Then derivations starting on $\tm$, modulo permutation equivalence,
form an upper semilattice with bottom $\ulbDeriv{\tm}$:
\[
  \ulbDeriv{\tm} \eqdef \set{\cls{\redseq} \ST \src(\redseq) = \tm}
\]
where $\permle$ is the usual prefix order on derivations ($\cls{\redseq} \permle \cls{\redseqtwo}$ if $\redseq \permle \redseqtwo$),
$\bot$ is the empty derivation starting on $\tm$,
and $\sqcup$ is the usual join of derivations ($\cls{\redseq} \sqcup \cls{\redseqtwo} = \cls{\redseq \sqcup \redseqtwo}$).
\end{proposition}
\begin{proof}
Straightforward, using well-known results, see for example~\cite[Chapter~12]{Barendregt:1984}.
\end{proof}

\begin{proposition}[Garbage derivations form an $\USLB$]
\lprop{garbage_derivations_form_an_ulb}
Let $\tm' \refines \tm$.
Then garbage derivations starting on $\tm$, modulo permutation equivalence,
form an upper semilattice with bottom $\ulbGarbage{\tm'}{\tm}$:
\[
  \ulbGarbage{\tm'}{\tm} \eqdef \set{\cls{\redseq} \ST \src(\redseq) = \tm \text{ and } \redseq \text{ is $\tm'$-garbage}}
\]
The lattice structure is given as for $\ulbDeriv{\tm}$ in \rprop{derivations_form_an_ulb}.
\end{proposition}
\begin{proof}
Note that the notion of being garbage is well-defined modulo permutation equivalence
by \rprop{compatibility_of_simulation_residuals_and_permutation_equivalence},
so the representative of $\cls{\redseq}$ may be chosen liberally.
Checking that $\permle$ is a partial order and that $\bot$ is the bottom element is
a consequence of \rprop{derivations_form_an_ulb}.
Checking that $\cls{\redseq} \lor \cls{\redseqtwo}$ is the supremum of $\set{\cls{\redseq},\cls{\redseqtwo}}$
is also a consequence of \rprop{derivations_form_an_ulb},
provided that the set $\ulbGarbage{\tm'}{\tm}$ is closed under joins ($\sqcup$).

So we are left to check that $\cls{\redseq} \sqcup \cls{\redseqtwo} \in \ulbGarbage{\tm'}{\tm}$
for all $\cls{\redseq},\cls{\redseqtwo} \in \ulbGarbage{\tm'}{\tm}$.
That is,
we must check that if $\redseq$ and $\redseqtwo$ are garbage
then $\redseq \sqcup \redseqtwo$ is garbage.
We have already seen this in~\rprop{properties_of_garbage}.
\end{proof}

We aim to show that garbage-free derivations form an $\USLB$.
This is not as easy as for garbage derivations.
In particular, the join of garbage-free derivations is not necessarily a
garbage-free derivation.
For instance, in the diagram of \rexample{naive_notion_of_non_garbage_fails},
the derivations $\redex$ and $\redextwo$ are both garbage-free,
but their join $\redex\redex_1\redextwo'_2$ is not.

\begin{proposition}[Garbage-free derivations form an $\USLB$]
\lprop{garbage_free_derivations_form_an_ulb}
Let $\tm' \refines \tm$.
Then garbage-free derivations starting on $\tm$, modulo permutation equivalence,
form an upper semilattice with bottom $\ulbFree{\tm'}{\tm}$:
\[
  \ulbFree{\tm'}{\tm} \eqdef \set{\cls{\redseq} \ST \src(\redseq) = \tm \text{ and } \redseq \text{ is $\tm'$-garbage-free}}
\]
The lattice structure is given by:
\[
  \begin{array}{rcl}
    \cls{\redseq} \leqF \cls{\redseqtwo} & \iffdef & \redseq/\redseqtwo \text{ is $(\tm'/\redseqtwo)$-garbage} \\
    \bot_{(\ulbFree{\tm'}{\tm})}         & \eqdef  & \cls{\emptyDerivation} \\
    \cls{\redseq} \lorF \cls{\redseqtwo}  & \eqdef  & \cls{(\redseq \sqcup \redseqtwo) \sieve \tm'} \\
  \end{array}
\]
\end{proposition}
\begin{proof}
Let us show that this forms an upper semilattice with bottom.
\begin{enumerate}
\item {\bf Partial order.}
  First let us show that $\leqF$ is a partial order.
  \begin{enumerate}
  \item {\bf Reflexivity.}
    It is immediate that $\cls{\redseq} \leqF \cls{\redseq}$ holds since $\redseq/\redseq = \emptyDerivation$ is garbage.
  \item {\bf Antisymmetry.}
    Let $\cls{\redseq} \leqF \cls{\redseqtwo} \leqF \cls{\redseq}$.
    This means that $\redseq/\redseqtwo$ and $\redseqtwo/\redseq$ are garbage.
    Then:
    \[
      \begin{array}{rcll}
      \redseq
      & \permeq & \redseq \sieve \tm' & \text{ since $\redseq$ is garbage-free, by \rprop{characterization_of_garbage_free_derivations}} \\
      & \permeq & \redseq(\redseqtwo/\redseq) \sieve \tm' & \text{ since $\redseqtwo/\redseq$ is garbage, by \rlem{sieving_trailing_garbage}} \\
      & \permeq & \redseqtwo(\redseq/\redseqtwo) \sieve \tm' & \text{ since $A(B/A) \permeq B(A/B)$ in general, using \rlem{sieving_is_compatible_with_permutation_equivalence}} \\
      & \permeq & \redseqtwo \sieve \tm' & \text{ since $\redseq/\redseqtwo$ is garbage, by \rlem{sieving_trailing_garbage}} \\
      & \permeq & \redseqtwo & \text{ since $\redseqtwo$ is garbage-free, by \rprop{characterization_of_garbage_free_derivations}} \\
      \end{array}
    \]
    Since $\redseq \permeq \redseqtwo$ we conclude that $\cls{\redseq} = \cls{\redseqtwo}$,
    as required.
  \item {\bf Transitivity.}
    Let $\cls{\redseq} \leqF \cls{\redseqtwo} \leqF \cls{\redseqthree}$
    and let us show that $\cls{\redseq} \leqF \cls{\redseqthree}$.
    Note that $\redseq/\redseqtwo$ and $\redseqtwo/\redseqthree$ are garbage.
    By the fact that the projection of garbage is garbage (\rprop{properties_of_garbage})
    the derivation $(\redseq/\redseqtwo)/(\redseqtwo/\redseqthree)$ is garbage.
      Moreover, by the fact that the composition of garbage is garbage (\rprop{properties_of_garbage}),
    we have that $(\redseqtwo/\redseqthree)((\redseq/\redseqtwo)/(\redseqtwo/\redseqthree))$ is also garbage.
    In general the following holds:
    \[
      \begin{array}{rcll}
      \redseq/\redseqthree
      & \permle & (\redseq/\redseqthree)((\redseqtwo/\redseqthree)/(\redseq/\redseqthree))    & \text{ since $A \permle AB$ in general} \\
      & \permeq & (\redseqtwo/\redseqthree)((\redseq/\redseqthree)/(\redseqtwo/\redseqthree)) & \text{ since $A(B/A) \permeq B(A/B)$ in general} \\
      & \permeq & (\redseqtwo/\redseqthree)((\redseq/\redseqtwo)/(\redseqthree/\redseqtwo))   & \text{ since $A(B/A) \permeq B(A/B)$ in general} \\
      \end{array}
    \]
    So since any prefix of a garbage derivation is garbage (\rprop{properties_of_garbage})
    we conclude that $\redseq/\redseqthree$ is garbage.
    This means that $[\redseq] \leqF [\redseqthree]$, as required.
  \end{enumerate}
\item {\bf Bottom element.}
  First, observe that $\bot_{(\ulbFree{\tm'}{\tm})}$ is well-defined,
  since $\bot_{(\ulbFree{\tm'}{\tm})} = [\emptyDerivation]$
  and $\emptyDerivation$ is $\tm'$-garbage-free.

  Moreover, let us show that $\bot_{(\ulbFree{\tm'}{\tm})}$ is the least element.
  Let $[\redseq]$ be an arbitrary element of $\ulbFree{\tm'}{\tm}$
  and let us check that $\bot_{(\ulbFree{\tm'}{\tm})} \leqF [\redseq]$.
  This is immediate since,
  by definition, $\bot_{(\ulbFree{\tm'}{\tm})} \leqF [\redseq]$
  if and only if $\emptyDerivation/\redseq$ is garbage.
  But $\emptyDerivation/\redseq = \emptyDerivation$ is trivially garbage.
\item {\bf Join.}
  Let $[\redseq],[\redseqtwo]$ be arbitrary elements of $\ulbFree{\tm'}{\tm}$,
  and let us check that $[\redseq] \lorF [\redseqtwo]$ is the join.
  First observe that $[\redseq] \lorF [\redseqtwo]$ is well-defined
  \ie that $(\redseq \sqcup \redseqtwo) \sieve \tm'$ is $\tm'$-garbage-free,
  which is an immediate consequence of \rprop{characterization_of_garbage_free_derivations}.
  Moreover, it is indeed the supremum of $\set{[\redseq],[\redseqtwo]}$:
  \begin{enumerate}
  \item {\bf Upper bound.}
    Let us show that $[\redseq] \leqF [\redseq] \lorF [\redseqtwo]$; the proof for $[\redseqtwo]$ is symmetrical.
    We must show that $\redseq/((\redseq \sqcup \redseqtwo) \sieve \tm')$ is garbage.
    Note that $\redseq \permle \redseq \sqcup \redseqtwo$,
    so in particular $\redseq/((\redseq \sqcup \redseqtwo) \sieve \tm') \permle (\redseq \sqcup \redseqtwo)/((\redseq \sqcup \redseqtwo) \sieve \tm')$.
    Given that any prefix of a garbage derivation is garbage (\rprop{properties_of_garbage}),
    it suffices to show that $(\redseq \sqcup \redseqtwo)/((\redseq \sqcup \redseqtwo) \sieve \tm')$ is garbage.
    This is a straightforward consequence of the fact that projecting after a sieve is garbage (\rlem{projection_after_sieving_is_garbage}).
  \item {\bf Least upper bound.}
    Let $[\redseq], [\redseqtwo] \leqF [\redseqthree]$,
    and let us show that $[\redseq] \lorF [\redseqtwo] \leqF [\redseqthree]$.
    We know that $\redseq/\redseqthree$ and $\redseqtwo/\redseqthree$ are garbage,
    and we are to show that $((\redseq \sqcup \redseqtwo) \sieve \tm')/\redseqthree$
    is garbage.
    Note that $(\redseq \sqcup \redseqtwo) \sieve \tm' \permle \redseq \sqcup \redseqtwo$
    as a consequence of the fact that the sieve is a prefix (\rlem{sieve_is_prefix}).
    So in particular $((\redseq \sqcup \redseqtwo) \sieve \tm')/\redseqthree \permle (\redseq \sqcup \redseqtwo)/\redseqthree$.
    Given that any prefix of a garbage derivation is garbage (\rprop{properties_of_garbage}),
    it suffices to show that $(\redseq \sqcup \redseqtwo)/\redseqthree$ is garbage.
    But $(\redseq \sqcup \redseqtwo)/\redseqthree \permeq \redseq/\redseqthree \sqcup \redseqtwo/\redseqthree$
    so we conclude by the fact that the join of garbage is garbage (\rprop{properties_of_garbage}).
  \end{enumerate}
\end{enumerate}
\end{proof}

\begin{lemma}
\llem{lorf_ignores_garbage}
Let $\redseq,\redseqtwo$ be coinitial derivations and $\tm' \refines \src(\redseq)$.
Then $\cls{(\redseq \sqcup \redseqtwo) \sieve \tm'} = \cls{\redseq \sieve \tm'} \lorF \cls{\redseqtwo \sieve \tm'}$.
\end{lemma}
\begin{proof}
Let:
\[
  \begin{array}{rcl}
    \alpha & := & \redseq/(\redseq \sieve \tm') \\
    \beta  & := & \redseqtwo/(\redseqtwo \sieve \tm') \\
    \gamma & := & (\alpha/((\redseqtwo \sieve \tm')/(\redseq \sieve \tm'))) \sqcup (\beta/((\redseq \sieve \tm')/(\redseqtwo \sieve \tm'))) \\
  \end{array}
\]
Note that $\alpha$ and $\beta$ are garbage by \rlem{projection_after_sieving_is_garbage}
and hence $\gamma$ is also garbage, as a consequence of
the facts that the join of garbage is garbage~(\rprop{properties_of_garbage})
and that the projection of garbage is garbage~(\rprop{properties_of_garbage}).
Remark that, in general, $AB \sqcup CD \permeq (A \sqcup C)(B/(C/A) \sqcup D/(A/C))$.
Then the statement of this lemma is a consequence of the following chain of equalities:
\[
  \begin{array}{rcll}
  \cls{(\redseq \sqcup \redseqtwo) \sieve \tm'}
  & = &
  \cls{((\redseq \sieve \tm')\alpha \sqcup (\redseqtwo \sieve \tm')\beta) \sieve \tm'}
  & \text{by \rthm{factorization_of_garbage}}
  \\
  & = &
  \cls{((\redseq \sieve \tm') \sqcup (\redseqtwo \sieve \tm'))\gamma \sieve \tm'}
  & \text{by the previous remark}
  \\
  & = &
  \cls{((\redseq \sieve \tm') \sqcup (\redseqtwo \sieve \tm')) \sieve \tm'}
  & \text{since sieving ignores trailing garbage~\rlem{sieving_trailing_garbage}}
  \\
  & = & \cls{\redseq \sieve \tm'} \lorF \cls{\redseqtwo \sieve \tm'}
  & \text{by definition of $\lorF$}
  \end{array}
\]
\end{proof}

\begin{theorem}[Factorization of the $\USLB$ of derivations]
\lthm{factorization_ulb_derivations}
Let $\tm' \refines \tm$. Then the following is an isomorphism of $\USLB$s:
\begin{equation}
  \leqn{factorization_ulb_derivations:isomorphism}
  \ulbDeriv{\tm} \simeq \grothy{ \ulbF }{ \ulbG }
\end{equation}
where $\ulbF = \ulbFree{\tm'}{\tm}$
is the upper semilattice of garbage-free derivations,
and $\ulbG$ is defined as the following lax 2-functor:
\[
 \begin{array}{rrrcll}
   \ulbG & : & \ulbF & \to & \USL \\
               &   & \cls{\redseq}                  & \mapsto & \ulbGarbage{\tm'/\redseq}{\tgt(\redseq)} & \text{ for $\cls{\redseq} \in \ulbF$,} \\
               &   & \ptF{\cls{\redseqtwo}}{\cls{\redseq}} & \mapsto &
                     \left(\begin{array}{rrrcl}
                       \ulbG(\ptF{\cls{\redseqtwo}}{\cls{\redseq}}) & : & \ulbG(\cls{\redseq}) & \to & \ulbG(\cls{\redseqtwo}) \\
                                                                   &   & [\alpha] & \mapsto & [\redseq\alpha/\redseqtwo] \\
                     \end{array}\right)
                     & \text{ for all $\cls{\redseq} \leqF \cls{\redseqtwo}$.} \\
 \end{array}
\]
Note that the isomorphism \reqn{factorization_ulb_derivations:isomorphism}
is of upper semilattices with bottom---in particular $\grothy{\ulbF}{\ulbG}$ is shown to have a bottom element.
\end{theorem}
\begin{proof}
Firstly, let us show that the expression on the right-hand side is well-defined.
This amounts to checking that $\ulbG$ is indeed a lax 2-functor:
\begin{enumerate}
\item {\bf Well-defined.}
  To show that $\ulbG$ is well-defined,
  note first that $\ulbG$ when applied on an object $\cls{\redseq} \in \ulbF$ yields the
  upper semilattice $\ulbGarbage{\tm'/\redseq}{\tgt(\redseq)}$.
  This is indeed an $\USL$ as has been shown in \rprop{garbage_derivations_form_an_ulb}.
  Moreover, the choice of representative does not matter,
  since if $\redseq$ and $\redseqtwo$ are permutation equivalent derivations,
  then $\tm'/\redseq = \tm'/\redseqtwo$ (by \rprop{compatibility_of_simulation_residuals_and_permutation_equivalence})
  and $\tgt(\redseq) = \tgt(\redseqtwo)$.

  On the other hand, let us check that given $\cls{\redseq},\cls{\redseqtwo} \in \ulbF$
  such that $\cls{\redseq} \leqF \cls{\redseqtwo}$
  then
  \[
    \ulbG(\ptF{\cls{\redseqtwo}}{\cls{\redseq}}) : \ulbG(\cls{\redseq}) \to \ulbG(\cls{\redseqtwo})
  \]
  is a morphism of $\USL$s.

  First, we can see that it is well-defined,
  since if $\cls{\alpha} \in \ulbG([\redseq])$
  then the image $\ulbG(\ptF{\cls{\redseqtwo}}{\cls{\redseq})(\cls{\alpha}}) = \cls{\redseq\alpha/\redseqtwo}$
  is an element of $\ulbG([\redseqtwo])$,
  since:
  \[
    \begin{array}{rcll}
      \redseq\alpha/\redseqtwo & = & (\redseq/\redseqtwo)(\alpha/(\redseqtwo/\redseq)) \\
    \end{array}
  \]
    is garbage, as it is the composition of garbage derivations~(\rprop{properties_of_garbage}):
  the derivation $\redseq/\redseqtwo$ is garbage since $\cls{\redseq} \leqF \cls{\redseqtwo}$ (by definition),
  and the derivation $\alpha/(\redseqtwo/\redseq)$ is garbage since $\alpha$ is garbage~(\rprop{properties_of_garbage}).
  Moreover, the choice of representative does not matter,
  since if $\redseq_1 \permeq \redseq_2$ and $\redseqtwo_1 \permeq \redseqtwo_2$ and $\alpha_1 \permeq \alpha_2$
  then $\redseq_1\alpha_1/\redseqtwo_1 \permeq \redseq_2\alpha_2/\redseqtwo_2$.

  Furthermore, let us verify that it respects the conditions for morphisms of $\USL$s:
  \begin{enumerate}
  \item {\bf Monotonic.}
    Let $\cls{\alpha}, \cls{\beta} \in \ulbG([\redseq])$
    such that $\cls{\alpha} \permle \cls{\beta}$, and let us show that
    $\ulbG(\ptF{\cls{\redseqtwo}}{\cls{\redseq}})(\cls{\alpha}) \permle \ulbG(\ptF{[\redseqtwo]}{[\redseq]})(\cls{\beta})$.
    Indeed, $\alpha \permle \beta$, so:
    \[
      \redseq\alpha/\redseqtwo
      = (\redseq/\redseqtwo)(\alpha/(\redseqtwo/\redseq))
      \permle (\redseq/\redseqtwo)(\beta/(\redseqtwo/\redseq))
      = \redseq\beta/\redseqtwo
    \]
  \item {\bf Preserves joins.}
    Let $\cls{\alpha}, \cls{\beta} \in \ulbG(\cls{\redseq})$,
    and let us show that
    $\ulbG(\ptF{\cls{\redseqtwo}}{\cls{\redseq}})(\cls{\alpha} \sqcup \cls{\beta}) =
     \ulbG(\ptF{\cls{\redseqtwo}}{\cls{\redseq}})(\cls{\alpha}) \sqcup
     \ulbG(\ptF{\cls{\redseqtwo}}{\cls{\redseq}})(\cls{\beta})$.
    Indeed:
    \[
      \redseq(\alpha \sqcup \beta)/\redseqtwo
      \permeq
      (\redseq\alpha \sqcup \redseq\beta)/\redseqtwo
      \permeq
      \redseq\alpha/\redseqtwo \sqcup \redseq\beta/\redseqtwo
    \]
  \end{enumerate}
\item {\bf Identity.}
  Let $\cls{\redseq} \in \ulbF$.
  Let us check that $\ulbG(\ptF{\cls{\redseq}}{\cls{\redseq}}) = \identity_{\ulbG(\cls{\redseq})}$
  is the identity morphism.
  Indeed, if $\cls{\alpha} \in \ulbG(\cls{\redseq})$,
  then $\ulbG(\ptF{\cls{\redseq}}{\cls{\redseq}})(\cls{\alpha}) = \cls{\redseq\alpha/\redseq} = \cls{\alpha}$.
\item {\bf Composition.}
  Let $\cls{\redseq}, \cls{\redseqtwo}, \cls{\redseqthree} \in \ulbF$
  such that $\cls{\redseq} \leqF \cls{\redseqtwo} \leqF \cls{\redseqthree}$.
  Let us check that
  $\ulbG((\ptF{\cls{\redseqthree}}{\cls{\redseqtwo}}) \circ (\ptF{\cls{\redseqtwo}}{\cls{\redseq}})) \permle
   \ulbG(\ptF{\cls{\redseqthree}}{\cls{\redseqtwo}}) \circ \ulbG(\ptF{\cls{\redseqtwo}}{\cls{\redseq}})$.
  Note that
  $(\ptF{\cls{\redseqthree}}{\cls{\redseqtwo}}) \circ (\ptF{\cls{\redseqtwo}}{\cls{\redseq}}) : \cls{\redseq} \leqF \cls{\redseqthree}$
  is a morphism in the upper semilattice $\ulbF$ seen as a category.
  Moreover, since it is a semilattice, there a unique morphism $\cls{\redseq} \leqF \cls{\redseqthree}$,
  namely $\ptF{\cls{\redseqthree}}{\cls{\redseq}}$, so we have that:
  \[
    (\ptF{\cls{\redseqthree}}{\cls{\redseqtwo}}) \circ (\ptF{\cls{\redseqtwo}}{\cls{\redseq}}) = \ptF{\cls{\redseqthree}}{\cls{\redseq}}
  \]
  Now if $\cls{\alpha} \in \ulbG{\cls{\redseq}}$, then:
  \[
    \begin{array}{rcll}
    \ulbG((\ptF{\cls{\redseqthree}}{\cls{\redseqtwo}}) \circ (\ptF{\cls{\redseqtwo}}{\cls{\redseq}}))(\cls{\alpha})
    & = & \ulbG(\ptF{\cls{\redseqthree}}{\cls{\redseq}})(\cls{\alpha}) \\
    & = & \redseq\alpha/\redseqthree \\
    & \permle & (\redseq\alpha/\redseqthree)((\redseqtwo/\redseq\alpha)/(\redseqthree/\redseq\alpha)) \\
    & = & \redseq\alpha(\redseqtwo/\redseq\alpha)/\redseqthree \\
    & \permeq & \redseqtwo(\redseq\alpha/\redseqtwo)/\redseqthree & \text{ since $A(B/A) \permeq B(A/B)$} \\
    & = & \ulbG(\ptF{\cls{\redseqthree}}{\cls{\redseqtwo}})(\redseq\alpha/\redseqtwo) \\
    & = & (\ulbG(\ptF{\cls{\redseqthree}}{\cls{\redseqtwo}}) \circ \ulbG(\ptF{\cls{\redseqtwo}}{\cls{\redseq}}))(\cls{\alpha})
    \end{array}
  \]
\end{enumerate}
so
$\ulbG((\ptF{\cls{\redseqthree}}{\cls{\redseqtwo}}) \circ (\ptF{\cls{\redseqtwo}}{\cls{\redseq}}))
\permle
\ulbG(\ptF{\cls{\redseqthree}}{\cls{\redseqtwo}}) \circ \ulbG(\ptF{\cls{\redseqtwo}}{\cls{\redseq}})$
as required.

Secondly, let us show that $\grothy{\ulbF}{\ulbG}$ has a bottom element.
Indeed, we argue that $(\bot_{\ulbF},\bot_{\ulbG(\bot_\ulbF)})$ is the bottom
element. Let $(\cls{\redseq},\cls{\redseqtwo}) \in \grothy{\ulbF}{\ulbG}$.
Then clearly $\bot_{\ulbF} \leqF \cls{\redseq}$. Moreover:
\[
 \ulbG(\ptF{[\redseq]}{[\bot_{\ulbF}]})(\bot_\ulbG) =
 [\emptyDerivation/\redseq] = [\emptyDerivation] \permle \cls{\redseqtwo}
\]

Finally, let us show that $\ulbDeriv{\tm} \simeq \grothy{\ulbF}{\ulbG}$,
for which we construct an explicit isomorphism.
Let:
\[
\begin{array}{rclcl}
  \varphi & : & \ulbDeriv{\tm}        & \to     & \grothy{\ulbF}{\ulbG} \\
             &&    \cls{\redseq} & \mapsto & (\cls{\redseq \sieve \tm'}, \cls{\redseq/(\redseq \sieve \tm')}) \\
\\
  \psi    & : & \grothy{\ulbF}{\ulbG} & \to & \ulbDeriv{\tm} \\
             &&  (\cls{\redseq},\cls{\redseqtwo}) & \mapsto & [\redseq\redseqtwo] \\
\end{array}
\]
Note that $\varphi$ and $\psi$ are well-defined mappings, since their value does not
depend on the choice of representative, due, in particular, to the fact that sieving
is compatible with permutation equivalence~(\rlem{sieving_is_compatible_with_permutation_equivalence}).
Let us check that $\varphi$ and $\psi$ are morphisms of $\USLB$s, and that they are mutual inverses:
\begin{enumerate}
\item {\bf $\varphi$ is a morphism of $\USLB$s.}
  Let us check the three conditions:
  \begin{enumerate}
  \item {\bf Monotonic.}
    Let $\cls{\redseq} \permle \cls{\redseqtwo}$ in $\ulbDeriv{\tm}$,
    and let us show that the following inequality holds:
    \[
      \varphi(\cls{\redseq}) =
      (\cls{\redseq \sieve \tm'}, \cls{\redseq/(\redseq \sieve \tm')})
      \leq
      (\cls{\redseqtwo \sieve \tm'}, \cls{\redseqtwo/(\redseqtwo \sieve \tm')})
      = \varphi(\cls{\redseqtwo})
    \]
    We check the two conditions (by definition of $\grothy{\ulbF}{\ulbG}$):
    \begin{enumerate}
    \item On the first hand,
          $\cls{\redseq \sieve \tm'} \leqF \cls{\redseqtwo \sieve \tm'}$
          since
          \[
            \begin{array}{rcll}
            (\redseq \sieve \tm')/(\redseqtwo \sieve \tm')
            & \permle &
            \redseq/(\redseqtwo \sieve \tm') & \text{since $\redseq \sieve \tm' \permle \redseq$ by \rlem{sieve_is_prefix}} \\
            & \permle & \redseqtwo/(\redseqtwo \sieve \tm') & \text{since $\redseq \permle \redseqtwo$ by hypothesis} \\
            \end{array}
          \]
          Note that this is garbage by \rlem{projection_after_sieving_is_garbage}.
          So by \rprop{properties_of_garbage}, $(\redseq \sieve \tm')/(\redseqtwo \sieve \tm')$ is also garbage,
          as required.
    \item On the other hand, let us show that
          $\ulbG(\ptF{\cls{\redseqtwo \sieve \tm'}}{\cls{\redseq \sieve \tm'}})(\cls{\redseq/(\redseq \sieve \tm')})
          \permle
          \redseqtwo/(\redseqtwo \sieve \tm')$.
          In fact:
          \[
            \begin{array}{rcll}
              \ulbG(\ptF{\cls{\redseqtwo \sieve \tm'}}{\cls{\redseq \sieve \tm'}})(\cls{\redseq/(\redseq \sieve \tm')})
            & = &
              \cls{ (\redseq \sieve \tm')(\redseq/(\redseq \sieve \tm')) / (\redseqtwo \sieve \tm') } & \text{ by definition} \\
            & = &
              \cls{ \redseq / (\redseqtwo \sieve \tm') } & \text{ by \rthm{factorization_of_garbage}} \\
            & \permle &
              \cls{ \redseqtwo / (\redseqtwo \sieve \tm') } & \text{ since $\redseq \permle \redseqtwo$} \\
            \end{array}
          \]
           
    \end{enumerate}
  \item {\bf Preserves bottom.}
    By definition:
    $\varphi(\bot_{\ulbDeriv{\tm}})
     = (\cls{\emptyDerivation \sieve \tm'}, \cls{\emptyDerivation/(\emptyDerivation \sieve \tm')})
     = (\cls{\emptyDerivation},\cls{\emptyDerivation})
     = (\bot_{\ulbF},\bot_{\ulbG(\bot_\ulbF)})$.
  \item {\bf Preserves joins.}
    Let $\cls{\redseq},\cls{\redseqtwo} \in \ulbDeriv{\tm}$, and let us show that
    $\varphi(\cls{\redseq} \sqcup \cls{\redseqtwo}) = \varphi(\cls{\redseqtwo}) \lor \varphi(\cls{\redseqtwo})$.
    Indeed, note that:
    \[
      \varphi(\cls{\redseq} \sqcup \cls{\redseqtwo}) = (\alpha,\beta)
    \]
    where
    \[
      \begin{array}{rcl}
      \alpha & = & \cls{(\redseq \sqcup \redseqtwo) \sieve \tm'} \\
      \beta  & = & \cls{(\redseq \sqcup \redseqtwo) / ((\redseq \sqcup \redseqtwo) \sieve \tm')} \\
      \end{array}
    \]
    and
    \[
      \varphi(\cls{\redseq}) \lor \varphi(\cls{\redseqtwo})
      = (\cls{\redseq \sieve \tm'}, \cls{\redseq / (\redseq \sieve \tm')}) \lor
        (\cls{\redseqtwo \sieve \tm'}, \cls{\redseqtwo / (\redseqtwo \sieve \tm')}) \\
      = (\alpha', \beta') \\
    \]
    where
    \[
      \begin{array}{rcl}
      \alpha' & = & \cls{\redseq \sieve \tm'} \lorF \cls{\redseqtwo \sieve \tm'} \\
      \beta'  & = &
             \ulbG(\ptF{\alpha}{\cls{\redseq \sieve \tm'}})(\cls{\redseq / (\redseq \sieve \tm')}) \sqcup \ulbG(\ptF{\alpha}{\cls{\redseqtwo \sieve \tm'}})(\cls{\redseqtwo / (\redseqtwo \sieve \tm')})
      \end{array}
    \]
    It suffices to show that $\alpha = \alpha'$ and $\beta = \beta'$.
    Let us show each separately:
    \begin{enumerate}
    \item {\bf Proof of $\alpha = \alpha'$.}
      The equality
      $
        \alpha
        = \cls{(\redseq \sqcup \redseqtwo) \sieve \tm'}
        = \cls{\redseq \sieve \tm'} \lorF \cls{\redseqtwo \sieve \tm'}
        = \alpha'
      $
      is an immediate consequence of \rlem{lorf_ignores_garbage}.
    \item {\bf Proof of $\beta = \beta'$.}
      Note that:
      \[
        \begin{array}{rcll}
        \beta'
              & = & \ulbG(\ptF{\alpha'}{\cls{\redseq \sieve \tm'}})(\cls{\redseq / (\redseq \sieve \tm')}) \sqcup
                    \ulbG(\ptF{\alpha'}{\cls{\redseqtwo \sieve \tm'}})(\cls{\redseqtwo / (\redseqtwo \sieve \tm')}) \\
              & = &
                   \cls{
                     (\redseq \sieve \tm')(\redseq / (\redseq \sieve \tm'))/\alpha'
                     \sqcup
                     (\redseqtwo \sieve \tm')(\redseqtwo / (\redseqtwo \sieve \tm'))/\alpha'
                   } \\
              & = &
                   \cls{\redseq/\alpha' \sqcup \redseqtwo/\alpha'}
                   &\hspace{-5cm}\text{by \rthm{factorization_of_garbage}} \\
              & = &
                   \cls{(\redseq \sqcup \redseqtwo)/\alpha'}
                   &\hspace{-5cm}\text{since $A/C \sqcup B/C \permeq (A \sqcup B)/C$} \\
              & = &
                   \cls{(\redseq \sqcup \redseqtwo)/((\redseq \sqcup \redseqtwo) \sieve \tm')}
                   &\hspace{-5cm}\text{since $\alpha' = \alpha = (\redseq \sqcup \redseqtwo) \sieve \tm'$} \\
              & = & \beta \\
        \end{array}
      \]
      as required. 
    \end{enumerate}
  \end{enumerate}
\item {\bf $\psi$ is a morphism of $\USLB$s.}
  Let us check the three conditions:
  \begin{enumerate}
  \item {\bf Monotonic.}
    Let $(\cls{\redseq_1},\cls{\redseqtwo_1}) \leq (\cls{\redseq_2},\cls{\redseqtwo_2})$ in $\grothy{\ulbF}{\ulbG}$
    and let us show that
    $\psi(\cls{\redseq_1},\cls{\redseqtwo_1}) \permle \psi(\cls{\redseq_2},\cls{\redseqtwo_2})$
    in $\ulbDeriv{\tm}$.
    Indeed, we know that $\ulbG(\ptF{\cls{\redseq_2}}{\cls{\redseq_1}})(\cls{\redseqtwo_1}) \permle \cls{\redseqtwo_2}$,
    that is to say $\redseq_1\redseqtwo_1/\redseq_2 \permle \redseqtwo_2$.
    Then:
    \[
      \redseq_1\redseqtwo_1/\redseq_2\redseqtwo_2
      = (\redseq_1\redseqtwo_1/\redseq_2)/\redseqtwo_2
      = \emptyDerivation
    \]
    which means that $\redseq_1\redseqtwo_1 \permle \redseq_2\redseqtwo_2$.
    This immediately implies that
    $\psi(\cls{\redseq_1},\cls{\redseqtwo_1}) \permle \psi(\cls{\redseq_2},\cls{\redseqtwo_2})$.
  \item {\bf Preserves bottom.}
    Recall that the bottom element $\bot_{(\grothy{\ulbF}{\ulbG})}$
    is defined as $(\bot_\ulbF, \bot_{\ulbG(\bot_\ulbF)})$, that is
    $(\cls{\emptyDerivation}, \cls{\emptyDerivation})$.
    Therefore $\psi(\bot_{(\grothy{\ulbF}{\ulbG})}) = \cls{\emptyDerivation} = \bot_{\ulbDeriv{\tm}}$.
  \item {\bf Preserves joins.}
    Let $(\cls{\redseq_1},\cls{\redseqtwo_1})$ and $(\cls{\redseq_2},\cls{\redseqtwo_2})$
    be elements of $\grothy{\ulbF}{\ulbG}$, and let us show that
    $\psi((\cls{\redseq_1},\cls{\redseqtwo_1}) \lor (\cls{\redseq_2},\cls{\redseqtwo_2})) =
     \psi(\cls{\redseq_1},\cls{\redseqtwo_1}) \sqcup \psi(\cls{\redseq_2},\cls{\redseqtwo_2}))$.

    Let:
    \[
    \begin{array}{rcl}
      \alpha & := & (\redseq_1 \sqcup \redseq_2) \sieve \tm'
    \end{array}
    \]

    First we claim that $\alpha \permle \redseq_1\redseqtwo_1 \sqcup \redseq_2\redseqtwo_2$.
    This is because by \rlem{sieve_is_prefix}
    we know that $\alpha = (\redseq_1 \sqcup \redseq_2) \sieve \tm' \permle \redseq_1 \sqcup \redseq_2$.
    Moreover, it is easy to check that
    $\redseq_1 \sqcup \redseq_2 \permle \redseq_1\redseqtwo_1 \sqcup \redseq_2\redseqtwo_2$.
    Using this fact we have:
    \[
    \begin{array}{rcll}
      \psi((\cls{\redseq_1},\cls{\redseqtwo_1}) \lor (\cls{\redseq_2},\cls{\redseqtwo_2}))
    & = &
      \psi(\cls{\alpha},\ulbG(\ptF{\cls{\alpha}}{\cls{\redseq_1}})(\cls{\redseqtwo_1}) \sqcup \ulbG(\ptF{\cls{\alpha}}{\cls{\redseq_2}})(\cls{\redseqtwo_2}))
    \\
    & = &
      \psi(\cls{\alpha}, \cls{(\redseq_1\redseqtwo_1/\alpha) \sqcup (\redseq_2\redseqtwo_2/\alpha)})
    \\
    & = &
      \psi(\cls{\alpha}, \cls{(\redseq_1\redseqtwo_1 \sqcup \redseq_2\redseqtwo_2)/\alpha})
    \\&&\text{ since $A/C \sqcup B/C \permle (A \sqcup B)/C$} \\
    & = &
      \cls{\alpha((\redseq_1\redseqtwo_1 \sqcup \redseq_2\redseqtwo_2)/\alpha)} \\
    & = &
      \cls{(\redseq_1\redseqtwo_1 \sqcup \redseq_2\redseqtwo_2)(\alpha/(\redseq_1\redseqtwo_1 \sqcup \redseq_2\redseqtwo_2))} \\
    & = &
      \cls{\redseq_1\redseqtwo_1 \sqcup \redseq_2\redseqtwo_2}
      \\&&\text{ since
                   $\alpha \permle \redseq_1\redseqtwo_1 \sqcup \redseq_2\redseqtwo_2$,
                 so $\alpha/(\redseq_1\redseqtwo_1 \sqcup \redseq_2\redseqtwo_2) = \emptyDerivation$} \\
    & = &
      \psi(\cls{\redseq_1},\cls{\redseqtwo_1}) \sqcup \psi(\cls{\redseq_2},\cls{\redseqtwo_2}))
    \end{array}
    \]
    as required.
  \end{enumerate}
\item {\bf Left inverse: $\psi \circ \varphi = \id$.}
  Let $\cls{\redseq} \in \ulbDeriv{\tm}$.
  Then by \rthm{factorization_of_garbage}:
  \[
    \psi(\varphi(\cls{\redseq}))
    = \psi(\cls{\redseq \sieve \tm'}, \cls{\redseq/(\redseq \sieve \tm')})
    = \cls{(\redseq \sieve \tm')(\redseq/(\redseq \sieve \tm'))}
    = \cls{\redseq}
  \]
\item {\bf Right inverse: $\varphi \circ \psi = \id$.}
  Let $(\cls{\redseq},\cls{\redseqtwo}) \in \grothy{\ulbF}{\ulbG}$.
  Note that
  $\redseq$ is $\tm'$-garbage-free
  and $\redseqtwo$ is $\tm'$-garbage,
  so by~\rlem{sieving_trailing_garbage}
  and \rprop{characterization_of_garbage_free_derivations}
  we know that $\redseq\redseqtwo \sieve \tm' = \redseq \sieve \tm' \permeq \redseq$.
  Hence:
  \[
    \varphi(\psi(\cls{\redseq},\cls{\redseqtwo}))
    = \varphi(\cls{\redseq\redseqtwo})
    = (\cls{\redseq\redseqtwo \sieve \tm'}, \cls{\redseq\redseqtwo/(\redseq\redseqtwo \sieve \tm')})
    = (\cls{\redseq}, \cls{\redseq\redseqtwo/\redseq})
    = (\cls{\redseq}, \cls{\redseqtwo})
  \]
\end{enumerate}
\end{proof}

