
As we previously said, we want to enunciate a general
factorization theorem for derivation spaces.
The factorization will consist in writing any derivation as
the concatenation of two derivations, such that the first one is garbage-free,
and the second one is garbage.

This factorization can be considered over derivation spaces because
it \emph{glues} well: the factorizations of two different derivations will
be compatible in a strong sense.

Let's recall why this factorization theorem is important.
\TODO{hnf ...}.


We introduce the following definition mostly to fix nomenclature and notation:

\begin{definition}[Upper semilattices]
An \defn{upper semilattice} is a triple $(X,\leq,\lor)$
where $X$ is a set, $\leq$ is a partial order on $X$,
and there are binary joins $x \lor y$ for all $x, y \in X$.
An \defn{upper semillatice with bottom} is an upper semilattice
with a bottom element $\bot \in X$.
As customary, by abuse of notation, we write $X$ for both the structure
and the underlying set, when clear from the context.
We may write $\bot_X$ to emphasize that $\bot$ is the bottom element of $X$.

A \defn{morphism of upper semilattices} $f : X \to Y$
is a monotonic function $f : X \to Y$
(\ie $x \leq y$ implies $f(x) \leq f(y)$)
preserving joins,
that is $f(x \lor y) = f(x) \lor f(y)$.
A \defn{morphism of upper semilattices with bottom}
moreover preserves the bottom element, \ie $f(\bot) = \bot$.

Upper semmilattices provided with morphisms form a category $\USL$.
Upper semmilattices with bottom provided with morphisms form a category $\USLB$.
Any upper semilattice may be regarded as a category
whose objects are the elements of $X$, and such that there is a (unique) morphism
$x \to y$ if and only if $x \leq y$.
We write $\pt{y}{x}$ for such morphism.

As usual, if $X$ and $Y$ are posets, the set of functions $f : X \to Y$
is also a poset with $f \leq g$ defined as $f(x) \leq g(x)$ for all $x \in X$.
Thus $\USL$ forms a $2$-category
in which $0$-cells are upper semilattices,
$1$-cells are morphisms of upper semilattices $f : X \to Y$, 
and there is a $2$-cell $f \totwo g$ if and only if $f \leq g$.
\end{definition}

\begin{definition}[Lax 2-functor]
Let $\Cat$ be a category and $\CatTwo$ a $2$-category.
A lax 2-functor $F : \Cat \to \CatTwo$
is a pair $(F_1,F_2)$ where $F_1 : \Ob(\Cat) \to \Ob(\CatTwo)$
is a function, and for every morphism $f : X \to Y$ in $\Cat$,
\[
  F_2(f) : F_1(X) \to F_1(Y) \HS\text{ is a 1-cell in $\CatTwo$}
\]
such that functor laws hold up to $2$-cells.
We will be interested in the following notion of lax $2$-functor:
\begin{enumerate}
\item $F_2(\id_X) = \id_{F_1(X)}$ for all $X \in \Ob(\Cat)$.
\item $F_2(f \circ g) \leq F_2(f) \circ F_2(g)$ is a $2$-cell
      for any two morphisms $g : X \to Y$, $f : Y \to Z$ in $\Cat$.
\end{enumerate}
As usual with functors, we may write $F$ to stand for either $F_1$ or $F_2$,
when clear from the context.
\end{definition}

Recall that we will have a factorization for each derivation,
and we want to somehow glue those factorizations such that
factorizations behave well under semilattice operations.
That can be done with an adaptation of Grothendieck's construction
for partially ordered sets.

\begin{definition}[Grothendieck construction for partially ordered sets]
Let $A$ be a poset, and let $B : A \to \Poset$ be a mapping associating
each object $a \in A$ to a poset $B(a)$.
Suppose moreover that $B$ is a {\em lax 2-functor}.
More precisely,
for each $a \leq b$ in $A$ let $B(\pt{b}{a}) : B(a) \to B(b)$
be a monotonic function such that:
\begin{enumerate}
\item $B(\pt{a}{a}) = \id_a$ for all $a \in A$.
\item $B((\pt{c}{b}) \circ (\pt{b}{a})) \totwocell B(\pt{c}{b}) \circ B(\pt{b}{a})$ is a 2-cell for all $a \leq b \leq c$ in $A$.
\end{enumerate}
The {\em Grothendieck construction} $\grothy{A}{B}$
is defined as the poset
given by
$
  \set{(a,b) \ST a \in A,\ b \in B(a)}
$
and such that $(a, b) \leq (a', b') \iffdef a \leq a' \text{ and } B(\pt{a'}{a})(b) \leq b'$.
\end{definition}
It is routine to check that $\grothy{A}{B}$ is indeed a poset.


\begin{proposition}[Semilattices of garbage-free and garbage derivations]
\lprop{semilattices_of_garbage_free_and_garbage_derivations}
Let $\tm' \refines \tm$. Then:
\begin{enumerate}
\item The set $\ulbFree{\tm'}{\tm} = \set{\cls{\redseq} \ST \src(\redseq) = \tm \text{ and } \redseq \text{ is $\tm'$-garbage-free}}$ is a finite lattice,
with the order $\cls{\redseq} \leqF \cls{\redseqtwo} \iffdef \redseq/\redseqtwo \text{ is $(\tm'/\redseqtwo)$-garbage}$,
the join $\cls{\redseq} \lorF \cls{\redseqtwo} = \cls{(\redseq \sqcup \redseqtwo) \sieve \tm'}$,
and the meet $\cls{\redseq} \landF \cls{\redseqtwo}$
given by the join of all the $\cls{\redseqthree}$ such that $\cls{\redseqthree} \leqF \cls{\redseq}$ and $\cls{\redseqthree} \leqF \cls{\redseqtwo}$.
\item The set $\ulbGarbage{\tm'}{\tm} = \set{\cls{\redseq} \ST \src(\redseq) = \tm \text{ and } \redseq \text{ is $\tm'$-garbage}}$ is an upper semilattice,
with the structure inherited from $\ulbDerivLam{\tm}$.
\end{enumerate}
\end{proposition}
\begin{proof}
\SeeAppendixRef{semilattices_of_garbage_free_and_garbage_derivations_proof}
The proof relies on the properties of garbage~(\rprop{properties_of_garbage})
and sieving~(\rprop{properties_of_sieving}).
\end{proof}

Suppose that $\tm' \refines \tm$,
and let $\ulbF \eqdef \ulbFree{\tm'}{\tm}$ denote the lattice
of $\tm'$-garbage-free derivations.
Let $\ulbG : \ulbF \to \Poset$ be the lax 2-functor
$
  \ulbG(\cls{\redseq}) \eqdef \ulbGarbage{\tm'/\redseq}{\tgt(\redseq)}
$
with the action on morphisms
$\ulbG(\ptF{\cls{\redseqtwo}}{\cls{\redseq}}) : \ulbG(\cls{\redseq}) \to \ulbG(\cls{\redseqtwo})$
given by $\cls{\alpha} \mapsto \cls{\redseq\alpha/\redseqtwo}$.
Then:

\begin{theorem}[Factorization]
\lthm{factorization_ulb_derivations}
The Grothendieck construction $\grothy{\ulbF}{\ulbG}$ is an upper semilattice.
The join is given by $(a,b) \lor (a',b') = (a \lorF a', \ulbG(\ptF{a \lorF a'}{a})(b) \sqcup \ulbG(\ptF{a \lorF a'}{a'})(b'))$.
Moreover, the following is an isomorphism of upper semilattices:
\[
  \begin{array}{rcl}
    \ulbDerivLam{\tm}       & \to    & \grothy{\ulbF}{\ulbG} \\
    \cls{\redseq}           & \mapsto   & (\cls{\redseq \sieve \tm'}, \cls{\redseq / (\redseq \sieve \tm')} \\
  \end{array}
  \HS
  \begin{array}{rcl}
    \grothy{\ulbF}{\ulbG}             & \to    & \ulbDerivLam{\tm} \\
    (\cls{\redseq}, \cls{\redseqtwo}) & \mapsto & \cls{\redseq\redseqtwo} \\
  \end{array}
\]
\end{theorem}
\begin{proof}
\SeeAppendixRef{factorization_ulb_derivations_proof}
The proof consists in verifying that $\ulbG$ is a lax 2-functor,
then that $\grothy{\ulbF}{\ulbG}$ is an upper semilattice,
and finally that the mappings above are an isomorphism.
\end{proof}
As an immediate consequence of this theorem,
any derivation $\redseq$ in the $\lambda$-calculus
may be decomposed as $\redseq \permeq \redseq_1\redseq_2$,
where $\redseq_1$ is $\tm'$-garbage-free
and $\redseq_2$ is $(\tm'/\redseq_1)$-garbage.
Moreover, $\redseq_1$ and $\redseq_2$ are unique, modulo permutation equivalence.
Note that this is exactly what we stated in the factorization theorem in last section,
\rthm{factorization_of_garbage_naive}.

\begin{example}
Let $\tm = (\lam{\var}{\vartwo\var\var})(I\varthree)$
and $\tm'$ be as in \rexample{algebraic_simulation_example}.
The upper semilattice $\ulbDerivLam{\tm}$ can be factorized as $\grothy{\ulbF}{\ulbG}$
as follows. Here posets are represented by their Hasse diagrams:
\[
\begin{array}{ccc}
  \xymatrix@R=.3cm@C=.3cm{
  &
    \cls{\emptyDerivation}
    \ar@/_.25cm/[dl]
    \ar@/^.25cm/[dr]
  &
  &
  \\
    \cls{\redex_1}
    \ar[d]
    \ar[r]
  &
    \cls{\redex_1\redextwo_{21}}
    \ar[d]
  &
    \cls{\redextwo}
    \ar@/^.25cm/[dl]
  \\
    \cls{\redex_1\redextwo_{11}}
    \ar[r]
  &
    \cls{\redex_1 \sqcup \redextwo}
  }
&
  \raisebox{-.85cm}{$\simeq$}
&
  \xymatrix@R=.3cm@C=.3cm{
  &
    (\cls{\emptyDerivation},\cls{\emptyDerivation})
    \ar@/_.25cm/[dl]
    \ar@/^.25cm/[dr]
  &
  &
  \\
    (\cls{\redex_1},\cls{\emptyDerivation})
    \ar[d]
    \ar[r]
  &
    (\cls{\redex_1},\cls{\redextwo_{21}})
    \ar[d]
  &
    (\cls{\redextwo},\cls{\emptyDerivation})
    \ar@/^.25cm/[dl]
  \\
    (\cls{\redex_1\redextwo_{11}},\cls{\emptyDerivation})
    \ar[r]
  &
    (\cls{\redex_1\redextwo_{11}},\cls{\redextwo_{22}})
  }
\end{array}
\]
Note for example
that $(\cls{\redextwo},\cls{\emptyDerivation}) \leq (\cls{\redex_1\redextwo_{11}},\cls{\redextwo_{22}})$
because $\cls{\redextwo} \leqF \cls{\redex_1\redextwo_{11}}$, that is,
$\redextwo/\redex_1\redextwo_{11} = \redextwo_{22}$ is garbage,
and
$\ulbG(\ptF{\cls{\redex_1\redextwo_{11}}}{\cls{\redextwo}})(\cls{\emptyDerivation})
= \cls{\redextwo/\redex_1\redextwo_{11}} = \cls{\redextwo_{22}} \permle \cls{\redextwo_{22}}$.
\end{example}

