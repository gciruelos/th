When we defined what a refinement is, we saw that some terms had refinements and other
did not, but the reason behind that was not clear.

In this section we will see that terms that can be refined are exactly the terms
that have a normal form.
This is coherent with the fact that typability in system~$\mathcal{W}$ characterizes
head normalization \cite[Corollary 5.5]{bucciarelli2017non}.

\begin{definition}[Head normal forms]
A term $\tm \in \terms$ of the lambda-calculus is a \defn{head normal form}
if it is of the form:
%there exist an integer $n \geq 0$, a sequence of $n$ variables $\var_1,\hdots,\var_n$,
%a variable $\vartwo$,
%an integer $m \geq 0$, and a sequence of $m$ terms $\tm_1,\hdots,\tm_m \in \terms$ such that:
\[
  \tm = \lam{\var_1}{\hdots\lam{\var_n}{\vartwo\,\tm_1\,\hdots\,\tm_m}}
\]
Note that $\vartwo$ might be or not be among the $\var_1,\hdots,\var_n$.
Similarly, a term $\tm \in \termsdist$ of the distributive lambda-calculus is a \defn{head normal form}
if it is of the form:
%there exist an integer $n \geq 0$,
%labels $\lab_1,\hdots,\lab_n$,
%variables $\var_1,\hdots,\var_n$,
%a variable $\vartwo$, a type $\typ$,
%an integer $m \geq 0$, and a sequence of $m$ lists of terms $\ls{\tm}_1,\hdots,\ls{\tm}_m \in \ls{\termsdist}$
%such that:
\[
  \tm = \lamp{\lab_1}{\var_1}{\hdots\lamp{\lab_n}{\var_n}{\vartwo^{\typ}\,\ls{\tm}_1\,\hdots\,\ls{\tm}_m}}
\]
\end{definition}

We say that a term \emph{has} a head normal form if it can be reduced to a head normal form.
And the claim that we are putting forward is that a term can be refined if and only if it has
a head normal form.

Note that for terms that \emph{are} head normal forms this is easy. The idea is that
if \[\lam{\var_1}{\hdots\lam{\var_n}{\vartwo\,\tm_1\,\hdots\,\tm_m}} \in \terms\]
then it is refined by
\[\lamp{\lab_1}{\var_1}{\hdots\lamp{\lab_n}{\var_n}{\vartwo^{[] \tolab{1} \hdots \tolab{n} \alpha^1}\,[]\,\hdots\,[]}} \in \termsdist\]
if $\var_i \neq \vartwo$ for every $i$ (otherwise the type of $\vartwo$ is slightly different,
but the term has the same shape).


\begin{lemma}
\llem{hnf_has_refinement}
Let $\tm \in \terms$ such that it is in head normal form.
 Then there exists $\tm' \in \termsdist$ such that $\tm \refinedby \tm'$.
\end{lemma}
\begin{proof}
Suppose $\tm \in \terms$ is in head normal form.
Then $\tm = \lam{\var_1}{\hdots\lam{\var_n}{\vartwo\,\tm_1\,\hdots\,\tm_m}}$.

We will prove that there exists a correct $\tctx \vdash \tm' : \typtwo$ such that $\tm' \refines \tm$.
To make the proof easier, we are going to prove something stronger:
  (a) that there exists such $\tm'$,
  (b) that for every application in $\tm'$ the argument list is empty, and
  (c) that if $\var_i = \vartwo$ for any $i$, $\tctx$ will be empty,
      otherwise it will be a singleton of the form $\set{\vartwo : [\typ]}$.

We proceed by induction on the pair $(n, m)$, that is, by induction on
  $\mathbb{N}^2$ with lexicographic order.
\SeeAppendixRef{hnf_has_refinement_proof}
\end{proof}

\begin{proposition}[Refinability characterizes head normalization]
\lprop{has_hnf_has_refinement}
The following are equivalent:
\begin{enumerate}
  \item There exists $\tm' \in \termsdist$ such that $\tm' \refines \tm$.
  \item There exists $\tm' \in \termsdist$ such that $\tm' \refines \tm$
    and $\tm' \tomdist
         \lamp{\lab_1}{\var_1}{\hdots \lamp{\lab_n}{\var_n}{\vartwo^\typ}{[] \hdots []}}$.
  \item There exists a head normal form $\tmtwo$ such that $\tm \tomulti_\beta \tmtwo$, \ie
    $\tm$ has a head normal form.
\end{enumerate}
\end{proposition}
\begin{proof}
\SeeAppendixRef{has_hnf_has_refinement_proof}
$(1 \implies 3)$ relies on Simulation (\rprop{simulation}).
$(2 \implies 1)$ is obvious.
$(3 \implies 2)$ uses that head normal forms are refinable, plus a subject expansion lemma.
Subject expansion requires formulating a strongeer property than correctness,
invariant by $\todist$-expansion (\ie by $\todist^{-1}$).
\end{proof}


Note that, in general, there is no need to refine a head normal form using empty lists for
all parameters.
A head normal form has a corresponding B\"om tree, from which one
can obtain an \defn{approximant},
$\lam{\var_1}{\hdots\lam{\var_n}{\Omega\,\Omega\,\hdots\,\Omega}}$
\cite{Barendregt:1984,dezani-ciancaglini-bohm-approximant}.
A finite approximant is essentially any finite prefix of the B\"om tree, given by the grammar
$A ::= \Omega |\lam{\var_1}{\hdots\lam{\var_n}{A_1\,A_2\,\hdots\,A_m}}$.
$(2 \implies 3)$ from \rprop{has_hnf_has_refinement} may be generalized to arbitrary
finite approximations. We did not delve into this theory.



