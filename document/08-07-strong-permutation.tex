
\subsection{Proof of \rprop{strong_permutation} -- Permutation}
\label{appendix_strong_permutation}

Let us extend the operation of substitution to operate on contexts
by declaring that $\varlabel{\var}{\conbase} = [\,]$
and $\subs{\conbase}{\var}{[\,]} = \conbase$.
We need two auxiliary lemmas:

\begin{lemma}[Substitution lemma for contexts]
\llem{substitution_lemma_for_contexts}
\llem{substitution_lemma_alt}
If both sides of the equation are defined
and $(\ls{\tmtwo}_1,\ls{\tmtwo}_2)$ is a partition of $\ls{\tmtwo}$
then
$\subs{\conof{\tm}}{\var}{\ls{\tmtwo}} = \subs{\con}{\var}{\ls{\tmtwo_1}}\of{\subs{\tm}{\var}{\ls{\tmtwo}_2}}$.
\end{lemma}
\begin{proof}
The proof is similar to the Substitution Lemma, by induction on $\con$.
\end{proof}

\begin{lemma}[Reduction inside a substitution]
\llem{reduce_in_substitution}
Let $1 \leq i \leq n$ and $\tmtwo_i \tolab{\lab} \tmtwo'_i$.
Then:
\[
 \subs{\tm}{\var}{[\tmtwo_1,\hdots,\tmtwo_{i-1},\tmtwo_i,\tmtwo_{i+1},\hdots,\tmtwo_n]} \tolab{\lab}
 \subs{\tm}{\var}{[\tmtwo_1,\hdots,\tmtwo_{i-1},\tmtwo'_i,\tmtwo_{i+1},\hdots,\tmtwo_n]}
\].
\end{lemma}
\begin{proof}
Straightforward by induction on $\tmtwo$.
\end{proof}

The proof of \rprop{strong_permutation} proceeds as follows.
Let $\redex : \tm_0 \todistl{\lab} \tm_1$ and $\redextwo : \tm_0 \todistl{\lab'} \tm_2$
be coinitial steps, and let us show that the diagram may be closed.
The step $\redex$ is of the form
$
  \tm_0 = \conof{(\lamp{\lab}{\var}{\tm})\ls{\tmtwo}}
           \todistl{\lab} \conof{\subs{\tm}{\var}{\ls{\tmtwo}}} = \tm_1
$.
Recall that $\redex \neq \redextwo$ by hypothesis.
We proceed by induction on $\con$.
\begin{enumerate}
\item {\bf Empty context, $\con = \conbase$.}
  Then
  $
    \redex : \tm_0 = (\lamp{\lab}{\var}{\tm})\ls{\tmtwo}
             \todistl{\lab} \subs{\tm}{\var}{\ls{\tmtwo}} = \tm_1
  $.
  There are two subcases, depending on whether the pattern of $\redextwo$ is inside $\tm$
  or inside $\ls{\tmtwo}$:
  \begin{enumerate}
  \item {\bf The pattern of $\redextwo$ is in $\tm$.}
    \label{strong_permutation__case_redextwo_inside_tm}
    Using \rlem{substitution_lemma_for_contexts}, the situation is:
    \[
    \xymatrix@C=.25cm@R=.5cm{
     (\lamp{\lab}{\var}{\conof{(\lamp{\lab'}{\vartwo}{\tmfour}) \ls{\tmthree}}}) \ls{\tmtwo}
                        \ar[d]_{\lab}
                        \ar[r]^{\lab'} &
     (\lamp{\lab}{\var}{\conof{\subs{\tmfour}{\vartwo}{\ls{\tmthree}}}}) \ls{\tmtwo}
                        \ar@{.>}[d]_{\lab} \\
     \subs{\conof{(\lamp{\lab'}{\vartwo}{\tmfour}) \ls{\tmthree}}}{\var}{\ls{\tmtwo}}
                        \ar@{=}[d] &
     \subs{\conof{\subs{\tmfour}{\vartwo}{\ls{\tmthree}}}}{\var}{\ls{\tmtwo}}
                        \ar@{=}[d] \\
     \subs{\con}{\var}{\ls{\tmtwo}_1}
       \of{(\lamp{\lab'}{\vartwo}{\subs{\tmfour}{\var}{\ls{\tmtwo}_2}})
           \subs{\ls{\tmthree}}{\var}{\ls{\tmtwo}_3}}
                        \ar@{.>}[d]_-{\lab'} &
     \subs{\con}{\var}{\ls{\tmtwo}_1}
       \of{\subs{\subs{\tmfour}{\vartwo}{\ls{\tmthree}}}{\var}{\ls{\tmtwo}_4}} \\
     \subs{\con}{\var}{\ls{\tmtwo}_1}
       \of{\subs{\subs{\tmfour}{\var}{\ls{\tmtwo}_2}}{\vartwo}
                 {\subs{\ls{\tmthree}}{\var}{\ls{\tmtwo}_3}}}
                         &  \\
    }
    \]
    where $(\ls{\tmtwo}_1,\ls{\tmtwo}_2,\ls{\tmtwo}_3)$ and $(\ls{\tmtwo}_1,\ls{\tmtwo}_4)$ are partitions of $\ls{\tmtwo}$.
    Note that $(\ls{\tmtwo}_2, \ls{\tmtwo}_3)$ is a partition of $\ls{\tmtwo}_4$,
    so it suffices to show that
    $\subs{\subs{\tmfour}{\vartwo}{\ls{\tmthree}}}{\var}{\ls{\tmtwo}_4}
      =
      \subs{\subs{\tmfour}{\var}{\ls{\tmtwo}_2}}{\vartwo}
        {\subs{\ls{\tmthree}}{\var}{\ls{\tmtwo}_3}}
    $, which is an immediate consequence of the Substitution Lemma (\rlem{substitution_lemma}).
  \item {\bf The pattern of $\redextwo$ is inside $\ls{\tmtwo}$.}
    \label{strong_permutation__case_redextwo_inside_tmtwo}
    In this case, $\ls{\tmtwo} = [\ls{\tmtwo}_1, \conof{(\lamp{\lab'}{\vartwo}{\tmfour}) \ls{\tmthree}}, \ls{\tmtwo}_2]$.
    \[
    \xymatrix@C=.5cm@R=.5cm{
     (\lamp{\lab}{\var}{\tm}) [\ls{\tmtwo}_1,
                               \conof{(\lamp{\lab'}{\vartwo}{\tmfour}) \ls{\tmthree}},
                               \ls{\tmtwo}_2]
                        \ar[d]_-{\lab}
                        \ar[r]^{\lab'} &
     (\lamp{\lab}{\var}{\tm}) [\ls{\tmtwo}_1,
                               \subs{\tmfour}{\vartwo}{\ls{\tmthree}},
                               \ls{\tmtwo}_2]
                        \ar@{.>}[d]_-{\lab} \\
     \subs{\tm}{\var}{[\ls{\tmtwo}_1,
                       \conof{(\lamp{\lab'}{\vartwo}{\tmfour}) \ls{\tmthree}},
                       \ls{\tmtwo}_2]}
                       \ar@{.>}[r]^-{\lab'}
                       &
      \subs{\tm}{\var}{[\ls{\tmtwo}_1,
                        \subs{\tmfour}{\vartwo}{\ls{\tmthree}},\ls{\tmtwo}_2]}        \\
    }
    \]
    The arrow of the bottom of the diagram exists as a consequence of \rlem{reduce_in_substitution}.
  \end{enumerate}
\item {\bf Under an abstraction, $\con = \lamp{\lab''}{\vartwo}{\con'}$.}
  Straightforward by \ih.
\item {\bf Left of an application, $\con = \con'\ls{\tmthree}$.}
    There are three subcases, depending on whether the redex $\redextwo$ is
    at the root, to the left of the application, or to the right of the application.
    \begin{enumerate}
    \item {\bf The pattern of $\redextwo$ is at the root}.
      Then $\contwoof{(\lamp{\lab'}{\var}{\tm})\ls{\tmtwo}}$ must have a lambda at the root,
      so it is of the form $\lamp{\lab'}{\vartwo}{\con'''\of{(\lamp{\lab}{\var}{\tm}) \ls{\tmtwo}}}$.
      Hence, the starting term is
      $(\lamp{\lab'}{\vartwo}{\con'''\of{(\lamp{\lab}{\var}{\tm}) \ls{\tmtwo}}}) \ls{\tmthree}$.
      The symmetric case has already been studied in item~\refcase{strong_permutation__case_redextwo_inside_tm}.
    \item {\bf The pattern of $\redextwo$ is inside $\con'$.} Straightforward by \ih.
    \item {\bf The pattern of $\redextwo$ is inside $\ls{\tmthree}$.}
      It is immediate to close the diagram since the steps are at disjoint positions:
        \[
        \xymatrix@C=.5cm@R=.5cm{
         \contwoof{(\lamp{\lab}{\var}{\tm}) \ls{\tmtwo}}
           [\ls{\tmthree}_1, \con''\of{(\lamp{\lab'}{\vartwo}{\tmfour}) \ls{\tmfive}}, \ls{\tmthree}_2]
                            \ar[d]_-{\lab}
                            \ar[r]^{\lab'} &
         \contwoof{(\lamp{\lab}{\var}{\tm})\ls{\tmtwo}}
             [\ls{\tmthree}_1, \con''\of{\subs{\tmfour}{\vartwo}{\ls{\tmfive}}}, \ls{\tmthree}_2]
                           \ar@{.>}[d]_-{\lab} \\
         \contwoof{\subs{\tm}{\var}{\ls{\tmtwo}}}
           [\ls{\tmthree}_1, \con''\of{(\lamp{\lab'}{\vartwo}{\tmfour}) \ls{\tmfive}}, \ls{\tmthree}_2]
                            \ar@{.>}[r]^{\lab'}
                           &
         \contwoof{\subs{\tm}{\var}{\ls{\tmtwo}}}
             [\ls{\tmthree}_1, \con''\of{\subs{\tmfour}{\vartwo}{\ls{\tmfive}}}, \ls{\tmthree}_2]
        }
        \]
    \end{enumerate}
\item {\bf Right of an application,
  $\con = \tmfour[\tmthree_1, \hdots, \tmthree_{i-1}, \con',\tmthree_{i+1}, \hdots, \tmthree_n]$}.
  There are four subcases, depending on whether the redex $\redextwo$ is
  at the root, to the left of the application (that is, inside $\tmfour$),
  or to the right of the application (that is, either inside $\contwo$ or $\tmthree_j$ for some $j$).
  \begin{enumerate}
  \item {\bf The pattern of $\redextwo$ is at the root.}
    Then $\tmfour$ has a lambda at the root, \ie it is of the form $\lamp{\lab'}{\vartwo}{\tmthree}$.
    Hence the starting term is
    $
      (\lamp{\lab'}{\vartwo}{\tmthree}) [\tmthree_{1:i-1}, \contwoof{(\lamp{\lab}{\var}{\tm})\tmtwo},\tmthree_{i+1:n}]
    $.
    The symmetric case has already been studied in item~\refcase{strong_permutation__case_redextwo_inside_tmtwo}.
  \item {\bf The pattern of $\redextwo$ is inside $\tmfour$.}
    The steps are disjoint, so it is immediate.
  \item {\bf The pattern of $\redextwo$ is inside $\tmthree_j$ for some $j \neq i$.}
    The steps are disjoint, so it is immediate.
  \item {\bf The pattern of $\redextwo$ is inside $\contwo$.}
    Straightforward by \ih.
  \end{enumerate}
\end{enumerate}






%Recall that we had two steps $\redex : \tm_0 \todistl{\lab} \tm_1$ and $\redextwo : \tm_0 \todistl{\lab'} \tm_2$, going out from $\tm_0$, and we wanted to show that $\tm_1$ and $\tm_2$ have one reduct in common one step away.
%
%We use some auxiliary lemmas that are stated and proven at the end of this section.
%
%\begin{proof}
%We suppose the step $\redex$ is of the form:
%\[
%  \redex : \tm_0 = \conof{(\lamp{\lab}{\var}{\tm})\ls{\tmtwo}}
%           \todistl{\lab} \conof{\subs{\tm}{\var}{\ls{\tmtwo}}} = \tm_1
%\]
%Recall that $\redex \neq \redextwo$ by hypothesis, so we know it must reduce some other lambda.
%We proceed by induction on $\con$.
%\begin{enumerate}
%\item {\bf Empty context, $\con = \conbase$.}
%  Then
%  \[
%    \redex : \tm_0 = (\lamp{\lab}{\var}{\tm})\ls{\tmtwo}
%             \todistl{\lab} \subs{\tm}{\var}{\ls{\tmtwo}} = \tm_1
%  \]
%  There are two subcases, depending on whether the redex $\redextwo$ is inside $\tm$
%  or inside $\ls{\tmtwo}$:
%  \begin{enumerate}
%  \item {\bf $\redextwo$ is inside $\tm$.}
%    \[
%    \xymatrix@C=.25cm{
%     (\lamp{\lab}{\var}{\conof{(\lamp{\lab'}{\vartwo}{\tmfour}) \ls{\tmthree}}}) \ls{\tmtwo}
%                        \ar[d]^{\lab}
%                        \ar[r]^{\lab'} &
%     (\lamp{\lab}{\var}{\conof{\subs{\tmfour}{\vartwo}{\ls{\tmthree}}}}) \ls{\tmtwo}
%                        \ar@{-->}[d]^{\lab} \\
%     \subs{\conof{(\lamp{\lab'}{\vartwo}{\tmfour}) \ls{\tmthree}}}{\var}{\ls{\tmtwo}}
%                        \ar@{=}[d] &
%     \subs{\conof{\subs{\tmfour}{\vartwo}{\ls{\tmthree}}}}{\var}{\ls{\tmtwo}}
%                        \ar@{=}[d] \\
%     \subs{\con}{\var}{\ls{\tmtwo}_1}
%       \of{(\lamp{\lab'}{\vartwo}{\subs{\tmfour}{\var}{\ls{\tmtwo}_2}})
%           \subs{\ls{\tmthree}}{\var}{\ls{\tmtwo}_3}}
%                        \ar@{-->}[d]^{\lab'} &
%     \subs{\con}{\var}{\ls{\tmtwo}_1}
%       \of{\subs{\subs{\tmfour}{\vartwo}{\ls{\tmthree}}}{\var}{\ls{\tmtwo}_0}} \\
%     \subs{\con}{\var}{\ls{\tmtwo}_1}
%       \of{\subs{\subs{\tmfour}{\var}{\ls{\tmtwo}_2}}{\vartwo}
%                 {\subs{\ls{\tmthree}}{\var}{\ls{\tmtwo}_3}}}
%                         &  \\
%    }
%    \]
%    Note that $\ls{\tmtwo}_2 + \ls{\tmtwo}_3$ is a permutation of $\ls{\tmtwo}_0$.
%    It is enough to see that
%    \[\subs{\subs{\tmfour}{\vartwo}{\ls{\tmthree}}}{\var}{\ls{\tmtwo}_0}
%      =
%      \subs{\subs{\tmfour}{\var}{\ls{\tmtwo}_2}}{\vartwo}
%        {\subs{\ls{\tmthree}}{\var}{\ls{\tmtwo}_3}}
%    \]
%    But that is true because of the Substitution lemma (\rlem{substitution_lemma}).
%  \item {\bf $\redextwo$ is inside $\ls{\tmtwo}$.}
%    In this case,
%
%    \[ \ls{\tmtwo} = [\tmtwo_1, \hdots, \tmtwo_{j-1},
%                      \conof{(\lamp{\lab'}{\vartwo}{\tmfour}) \ls{\tmthree}},
%                      \tmtwo_{j+1}, \hdots, \tmtwo_n]
%                   = [\ls{\tmtwo}_{1:j-1},
%                      \conof{(\lamp{\lab'}{\vartwo}{\tmfour}) \ls{\tmthree}},
%                      \ls{\tmtwo}_{j+1:n}]
%    \]
%    \[
%    \xymatrix@C=1cm{
%     (\lamp{\lab}{\var}{\tm}) [\ls{\tmtwo}_{1:j-1},
%                               \conof{(\lamp{\lab'}{\vartwo}{\tmfour}) \ls{\tmthree}},
%                               \ls{\tmtwo}_{j+1:n}]
%                        \ar[d]^{\lab}
%                        \ar[r]^{\lab'} &
%     (\lamp{\lab}{\var}{\tm}) [\ls{\tmtwo}_{1:j-1},
%                               \subs{\tmfour}{\vartwo}{\ls{\tmthree}},
%                               \ls{\tmtwo}_{j+1:n}]
%                        \ar@{-->}[d]^{\lab} \\
%     \subs{\tm}{\var}{[\ls{\tmtwo}_{1:j-1},
%                       \conof{(\lamp{\lab'}{\vartwo}{\tmfour}) \ls{\tmthree}},
%                       \ls{\tmtwo}_{j+1:n}]}
%                       &
%      \subs{\tm}{\var}{[\ls{\tmtwo}_{1:j-1},
%                        \subs{\tmfour}{\vartwo}{\ls{\tmthree}},\ls{\tmtwo}_{j+1:n}]}        \\
%    }
%    \]
%    We can close the diagram reducing from left to right using \rlem{reduce_in_substitution}.
%  \end{enumerate}
%\item {\bf Under an abstraction, $\con = \lamp{\lab''}{\vartwo}{\con'}$.}
%    Then for some $\tm_1$ and $\tm_2$,
%    \[
%    \xymatrix@C=1.5cm{
%        \lamp{\lab''}{\vartwo}{\contwoof{(\lamp{\lab}{\var}{\tm}) \ls{\tmtwo}}}
%                        \ar[d]^{\lab}
%                        \ar[r]^{\lab'} &
%        \lamp{\lab''}{\vartwo}{\tm_2} \\
%        \lamp{\lab''}{\vartwo}{\tm_1} & \\
%    }
%    \]
%    But by inductive hypothesis,
%    \[
%    \xymatrix@C=1.5cm{
%        \contwoof{(\lamp{\lab}{\var}{\tm}) \ls{\tmtwo}}
%                        \ar[d]^{\lab}
%                        \ar[r]^{\lab'} &
%        {\tm_2} \ar@{-->}[d]^{\lab} \\ 
%        {\tm_1} \ar@{-->}[r]^{\lab'} &
%        {\tm_3} \\
%    }
%    \]
%
%    So if we consider those steps in the full term we get our desired diagram.
%    \[
%    \xymatrix@C=1.5cm{
%        \lamp{\lab''}{\vartwo}{\contwoof{(\lamp{\lab}{\var}{\tm}) \ls{\tmtwo}}}
%                        \ar[d]^{\lab}
%                        \ar[r]^{\lab'} &
%        \lamp{\lab''}{\vartwo}{\tm_2} \ar@{-->}[d]^{\lab} \\ 
%        \lamp{\lab''}{\vartwo}{\tm_1} \ar@{-->}[r]^{\lab'} &
%        \lamp{\lab''}{\vartwo}{\tm_3} \\
%    }
%    \]
%
%\item {\bf Left of an application, $\con = \con'\ls{\tmthree}$.}
%    There are three subcases, depending on whether the redex $\redextwo$ is
%    at the root, to the left of the application, or to the right of the application.
%    %inside $\con'$ or $\ls{\tmthree}$, \pablo{or at the root}.
%    \begin{enumerate}
%    \item {\bf $\redextwo$ is inside $\con'$.} Then by inductive hypothesis,
%        we have the following diagram for some $\tm_1$ and $\tm_2$:
%        \[
%        \xymatrix@C=1.5cm{
%            \contwoof{(\lamp{\lab}{\var}{\tm}) \ls{\tmtwo}}
%                            \ar[d]^{\lab}
%                            \ar[r]^{\lab'} &
%            {\tm_2} \ar@{-->}[d]^{\lab} \\ 
%            {\tm_1} \ar@{-->}[r]^{\lab'} &
%            {\tm_3} \\
%        }
%        \]
%        
%        So if we consider those steps in the full term we get our desired diagram.
%        \[
%        \xymatrix@C=1.5cm{
%            {\contwoof{(\lamp{\lab}{\var}{\tm}) \ls{\tmtwo}}} \ls{\tmthree}
%                            \ar[d]^{\lab}
%                            \ar[r]^{\lab'} &
%            \tm_2 \ls{\tmthree} \ar@{-->}[d]^{\lab} \\ 
%            \tm_1 \ls{\tmthree} \ar@{-->}[r]^{\lab'} &
%            \tm_3 \ls{\tmthree} \\
%        }
%        \]
%    \item {\bf $\redextwo$ is inside $\ls{\tmthree}$.}
%      What we have can be expressed in the following diagram.
%        \[
%        \xymatrix@C=1cm{
%         \contwoof{(\lamp{\lab}{\var}{\tm}) \ls{\tmtwo}} [\ls{\tmtwo}_{1:j-1},
%                                   \con''\of{(\lamp{\lab'}{\vartwo}{\tmfour}) \ls{\tmthree}},
%                                   \ls{\tmtwo}_{j+1:n}]
%                            \ar[d]^{\lab}
%                            \ar[r]^{\lab'} &
%         (\lamp{\lab}{\var}{\tm}) [\ls{\tmtwo}_{1:j-1},
%                                   \subs{\tmfour}{\vartwo}{\ls{\tmthree}},
%                                   \ls{\tmtwo}_{j+1:n}]
%                           \ar[d]^{\lab} \\
%         \subs{\tm}{\var}{[\ls{\tmtwo}_{1:j-1},
%                           \con''\of{(\lamp{\lab'}{\vartwo}{\tmfour}) \ls{\tmthree}},
%                           \ls{\tmtwo}_{j+1:n}]}
%                           &
%         \subs{\tm}{\var}{[\ls{\tmtwo}_{1:j-1},
%                           \subs{\tmfour}{\vartwo}{\ls{\tmthree}},
%                           \ls{\tmtwo}_{j+1:n}]} \\
%        }
%        \]
%    We can close the diagram reducing from left to right using \rlem{reduce_in_substitution}.
%    \item {\bf $\redextwo$ is at the root}.
%    What this means is that $\contwoof{(\lamp{\lab'}{\var}{\tm})\ls{\tmtwo}}$ has a lambda at its root,
%    so it is equal to $\lamp{\lab'}{\vartwo}{\con'''\of{(\lamp{\lab}{\var}{\tm}) \ls{\tmtwo}}}$
%    for some, $\con'''$ and $\vartwo$.
%
%    Hence, the term we have is
%    $(\lamp{\lab'}{\vartwo}{\con'''\of{(\lamp{\lab}{\var}{\tm}) \ls{\tmtwo}}}) \ls{\tmthree}$,
%    which has already been dealt with in case 1.1.
%    \end{enumerate}
%\item {\bf Right of an application,
%  $\con = \tmfour[\tmthree_1, \hdots, \tmthree_{i-1}, \con',\tmthree_{i+1}, \hdots, \tmthree_n] =
%          \tmfour[\tmthree_{1:i-1}, \con',\tmthree_{i+1:n}]$}.
%  There are four subcases, depending on whether the redex $\redextwo$ is
%  at the root, to the left of the application (that is, inside $\tmfour$),
%  or to the right of the application (that is, either inside $\contwo$ or $\tmthree_j$ for some $j$).
%  \begin{enumerate}
%  \item {\bf $\redextwo$ is inside $\tmfour$.}
%    Notice that the steps are disjoint, so we may close the diagram in a straightforward manner.
%  \item {\bf $\redextwo$ is inside $\tmthree_j$ for some $j \neq i$.}
%    Notice that, again, the steps are disjoint,
%    so we may close the diagram in a straightforward manner.
%  \item {\bf $\redextwo$ is inside $\contwo$.} Then by inductive hypothesis,
%        we have the following diagram for some $\tm_1$ and $\tm_2$:
%        \[
%        \xymatrix@C=1.5cm{
%            \contwoof{(\lamp{\lab}{\var}{\tm}) \ls{\tmtwo}}
%                            \ar[d]^{\lab}
%                            \ar[r]^{\lab'} &
%            {\tm_2} \ar@{-->}[d]^{\lab} \\ 
%            {\tm_1} \ar@{-->}[r]^{\lab'} &
%            {\tm_3} \\
%        }
%        \]
%        
%        So if we consider those steps in the full term we get our desired diagram.
%        \[
%        \xymatrix@C=1.5cm{
%          \tmfour \lset{\tmthree_{1:i-1}, \contwoof{(\lamp{\lab}{\var}{\tm}) \ls{\tmtwo}}, \tmthree_{i+1:n}}
%                          \ar[d]^{\lab}
%                          \ar[r]^{\lab'} &
%          \tmfour \lset{\tmthree_{1:i-1}, \tm_2, \tmthree_{i+1:n}} \ar@{-->}[d]^{\lab} \\ 
%          \tmfour \lset{\tmthree_{1:i-1}, \tm_1, \tmthree_{i+1:n}} \ar@{-->}[r]^{\lab'} &
%          \tmfour \lset{\tmthree_{1:i-1}, \tm_3, \tmthree_{i+1:n}} \\
%        }
%        \]
%  \item {\bf $\redextwo$ is at the root.}
%    Hence $\tm$ has a lambda at the root, so
%    $\tmfour[\tmthree_{1:i-1}, \contwoof{(\lamp{\lab}{\var}{\tm})\tmtwo},\tmthree_{i+1:n}] =
%     (\lamp{\lab'}{\vartwo}{\tmthree}) [\tmthree_{1:i-1}, \contwoof{(\lamp{\lab}{\var}{\tm})\tmtwo},\tmthree_{i+1:n}]$,
%     which is the same case we already proved in 1.2.
%  \end{enumerate}
%\end{enumerate}
%\end{proof}
%
%\subsection*{Auxiliary lemma}
%
%\begin{lemma}[Reduction]
%\llem{reduce_in_substitution}
%Suppose that $\tmtwo_k \todistl{\lab} \tmtwo'_k$.
%Then if
%$\subs{\tm}{\var}{\lset{\tmtwo_1, \hdots, \tmtwo_{k-1}, \tmtwo_k, \tmtwo_{k+1}, \hdots, \tmtwo_n}}$
%is defined, so is
%$\subs{\tm}{\var}{\lset{\tmtwo_1, \hdots, \tmtwo_{k-1}, \tmtwo'_k, \tmtwo_{k+1}, \hdots, \tmtwo_n}}$
%and
%\[
%\subs{\tm}{\var}{\lset{\tmtwo_1, \hdots, \tmtwo_{k-1}, \tmtwo_k, \tmtwo_{k+1}, \hdots, \tmtwo_n}}
%\todistl{\lab}
%\subs{\tm}{\var}{\lset{\tmtwo_1, \hdots, \tmtwo_{k-1}, \tmtwo'_k, \tmtwo_{k+1}, \hdots, \tmtwo_n}}
%\]
%\end{lemma}
%\begin{proof}
%By induction on $\tm$. 
%We will write $\lset{\tmtwo_1, \hdots, \tmtwo_{k-1}, \tmtwo_k, \tmtwo_{k+1}, \hdots, \tmtwo_n}$ as
%$\lset{\tmtwo_{1:k-1}, \tmtwo_k, \tmtwo_{k+1:n}}$
%\begin{enumerate}
%\item \Case{Variable (same) $\tm = \var$}
%  \begin{equation*}\begin{split}
%    \subs{\var}{\var}{[\tmtwo_k]}
%       & = \tmtwo_k \\
%       & \todistl{\lab} \tmtwo'_k \\
%       & = \subs{\var}{\var}{[\tmtwo'_k]} \\
%  \end{split}\end{equation*}
%\item \Case{Variable (different) $\tm = \vartwo$}
%  This case cannot happen because $\lset{\tmtwo_1, \hdots, \tmtwo_n}$ must
%  be empty but at the same time must contain $\tmtwo_k$.
%\item \Case{Abstraction, $\tm = \lamp{\lab}{\vartwo}{\tmfour}$}
%  \begin{equation*}\begin{split}
%    \subs{(\lamp{\lab}{\vartwo}{\tmfour})}{\var}{\lset{\tmtwo_{1:k-1}, \tmtwo_k, \tmtwo_{k+1:n}}}
%       & = \subs{\lamp{\lab}{\vartwo}{\tmfour}}{\var}{\lset{\tmtwo_{1:k-1}, \tmtwo_k, \tmtwo_{k+1:n}}} \\
%       & \todistih \subs{\lamp{\lab}{\vartwo}{\tmfour}}{\var}{\lset{\tmtwo_{1:k-1}, \tmtwo'_k, \tmtwo_{k+1:n}}} \\
%       & = \subs{(\lamp{\lab}{\vartwo}{\tmfour})}{\var}{\lset{\tmtwo_{1:k-1}, \tmtwo'_k, \tmtwo_{k+1:n}}} \\
%  \end{split}\end{equation*}
%\item \Case{Application, $\tm = \tmfour [\tmfour_1,\hdots,\tmfour_m]$}
%  \begin{equation*}\begin{split}
%    \subs{(\tmfour [\tmfour_1, \hdots, \tmfour_m])}{\var}{\lset{\tmtwo_{1:k-1}, \tmtwo_k, \tmtwo_{k+1:n}}}
%       & = \subs{\tmfour}{\var}{\ls{\tmtwo}_0} [\subs{\tmfour_i}{\var}{\ls{\tmtwo}_i}]_{i=0}^m \\
%  \end{split}\end{equation*}
%  Where $\tmtwo_k$ is one of the terms in the list $\ls{\tmtwo}_i$ for some $i \in \{0, \hdots, m\}$.
%  Suppose that it is in $\ls{\tmtwo}_0$ (if it is in $\ls{\tmtwo}_i$ for some $i > 0$, it is analogous).
%  This means that
%  $\ls{\tmtwo}_0 = [\tmtwo_{0,1}, \hdots, \tmtwo_k, \hdots, \tmtwo_{0, l}]$,
%  where $l = |\ls{\tmtwo}_0|$.
%
%  By inductive hypothesis,
%  \begin{equation*}\begin{split}
%    \subs{\tmfour}{\var}{[\tmtwo_{0,1}, \hdots, \tmtwo_k, \hdots, \tmtwo_{0, l}]} [\subs{\tmfour_i}{\var}{\ls{\tmtwo}_i}]_{i=0}^m
%      & \todistl{\lab} \subs{\tmfour}{\var}{[\tmtwo_{0,1}, \hdots, \tmtwo'_k, \hdots, \tmtwo_{0, l}]}
%      [\subs{\tmfour_i}{\var}{\ls{\tmtwo}_i}]_{i=0}^m \\
%  \end{split}\end{equation*}
%  \begin{equation*}\begin{split}
%      & = \subs{(\tmfour [\tmfour_1, \hdots, \tmfour_m])}{\var}{[\tmtwo_{0,1}, \hdots, \tmtwo'_k, \hdots, \tmtwo_{0, l}]
%      + \ls{\tmtwo}_1 + \hdots + \ls{\tmtwo}_m} \\
%      & = \subs{(\tmfour [\tmfour_1, \hdots, \tmfour_m])}{\var}{[\tmtwo_{1:k-1}, \hdots, \tmtwo'_k, \hdots,
%      \tmtwo_{k+1:n}]}\\
%  \end{split}\end{equation*}
%\end{enumerate}
%\end{proof}
%
%
%
