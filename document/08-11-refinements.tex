
\begin{lemma}[Refinement of a substitution]
\llem{refinement_substitution}
If $\tm' \refines \tm$ and $\tmtwo'_i \refines \tmtwo$ for all $1 \leq i \leq n$
  then $\subs{\tm'}{\var}{[\tmtwo_1', \hdots, \tmtwo_n']} \refines \subs{\tm}{\var}{\tmtwo}$.
\end{lemma}
\begin{proof}
  By induction on $\tm$.
  \begin{enumerate}
  \item {\bf Variable (same), $\tm = \var$.} In this case, $\tm' = \var$.
    Then, $\lsetcomp{\tmtwo'}{i}{1}{n} = [\tmtwo']$
      and $\subs{\var}{\var}{[\tmtwo']} = \tmtwo'$,
      which by hypothesis refines $\tmtwo = \subs{\var}{\var}{\tmtwo}$.
  \item {\bf Variable (different), $\tm = \vartwo$.} In this  case $\tm' = \vartwo$.
    Then, $\lsetcomp{\tmtwo'}{i}{1}{n} = \emptylset$
      and $\subs{\vartwo}{\var}{\emptylset} = \vartwo$,
      which refines $\vartwo = \subs{\vartwo}{\var}{\tmtwo}$.
  \item {\bf Abstraction, $\tm = \lam{\vartwo}{\tmfour}$.}
    In this case $\tm' = \lamp{\lab}{\vartwo}{\tmfour'}$,
      where $\tmfour \refinedby \tmfour'$.

    Also $\subs{(\lam{\vartwo}{\tmfour})}{\var}{\tmtwo} = \lam{\vartwo}{\subs{\tmfour}{\var}{\tmtwo}}$,
    and $\subs{(\lamp{\lab}{\vartwo}{\tmfour'})}{\var}{[\tmtwo_1', \hdots, \tmtwo_n']} =
         \lamp{\lab}{\vartwo}{\subs{\tmfour'}{\var}{[\tmtwo_1', \hdots, \tmtwo_n']}}$.
    By inductive hypothesis $\subs{\tmfour'}{\var}{[\tmtwo_1', \hdots, \tmtwo_n']} \refines \subs{\tmfour}{\var}{\tmtwo}$,
    so we are done.
  \item {\bf Application, $\tm = \tmfour \tmthree$.}
    In this case $\tm' = \tmfour' [\tmthree'_1, \hdots, \tmthree'_n]$,
      where $\tmfour \refinedby \tmfour'$
      and $\tmthree \refinedby \tmthree'_i$.

    Also, $\subs{(\tmfour \tmthree)}{\var}{\tmtwo} =
             \subs{\tmfour}{\var}{\tmtwo} \subs{\tmthree}{\var}{\tmtwo}$
    and $\subs{(\tmfour' [\tmthree'_1, \hdots, \tmthree'_n])}{\var}{\ls{\tmtwo}}
           = \subs{\tmfour'}{\var}{\ls{\tmtwo}_0} \subs{\tmthree_i}{\var}{\ls{\tmtwo}_i}$.

    But by inductive hypothesis
      $\subs{\tmfour}{\var}{\tmtwo} \refinedby \subs{\tmfour'}{\var}{\ls{\tmtwo}_0}$
      and $\subs{\tmthree}{\var}{\tmtwo} \refinedby \subs{\tmthree_i}{\var}{\ls{\tmtwo}_i}$
        for all $i \in \set{1, \hdots, n}$,
    so we are done.
  \end{enumerate}
\end{proof}

\begin{lemma}[Refinement of a context]
\llem{refinement_context}
The following are equivalent, which relates refinement and contexts:
  \begin{enumerate}
  \item $\tm' \refines \conof{\tmtwo}$,
  \item $\tm'$ is of the form $\con'\of{\tmtwo'_{1},\hdots,\tmtwo'_n}$,
        where $\con'$ is an $n$-hole context such that $\con' \refines \con$ and
        $\tmtwo'_i \refines \tmtwo$ for all $1 \leq i \leq n$.
        Note that $n$ might be $0$, in which case $\con'$ is a term.
  \end{enumerate}
Moreover, in the implication $(1 \implies 2)$,
the decomposition is unique,
\ie the context $\con'$, number of holes $n \geq 0$, and terms $\tmtwo'_1,\hdots,\tmtwo'_n$
are the unique possible such objects.
\end{lemma}
\begin{proof}
Remark that, in general, if $\con' \refines \tm$, where $\con'$ is a many-hole context and $\tm$ is a term,
then $\con'$ must be actually a $0$-hole context. This can be easily checked by induction on $\con'$.
We prove each direction separately:
\begin{itemize}
\item $(1 \implies 2)$
  By induction on $\con$.
  \begin{enumerate}
  \item {\bf Empty, $\con = \conbase$.}
    Then $\tm' \refines \tmtwo$.
    Taking $\con' = \conbase$ and $\tmtwo'_1 := \tm'$,
    we have that $\tm' = \con'\of{\tmtwo'_1}$.
    Moreover, the decomposition is unique.
  \item {\bf Abstraction, $\con = \lam{\var}{\con_1}$.}
    Then $\tm'$ must be of the form $\lamp{\lab}{\var}{\tmthree'}$
    where $\tmthree' \refines \conof{\tmtwo}$.
    By \ih, write $\tmthree'$ as $\tmthree' = \con'_1\of{\tmtwo'_1,\hdots,\tmtwo'_n}$
    where $\con'_1 \refines \con_1$ and $\tmtwo'_i \refines \tmtwo$ for all $1 \leq i \leq n$.
    Taking $\con' := \lamp{\lab}{\var}{\con'_1}$ we conclude.
    %Moreover, note that $\con'$ must start with $\lamp{\lab}{\var}{\conbase}$.
    By \ih the decomposition is unique for $\tmtwo'$, so the decomposition is also unique for $\tm'$.
  \item {\bf Left of an application, $\con = \con_1\,\tmthree$.}
    Then $\tm'$ must be of the form $\tmfour'[\tmthree'_1,\hdots,\tmthree'_m]$
    with $\tmfour' \refines \con_1\of{\tmtwo}$
    and $\tmthree'_i \refines \tmthree$ for all $1 \leq i \leq m$.
    By \ih, write $\tmfour'$ as $\tmfour' = \con'_1\of{\tmtwo'_1,\hdots,\tmtwo'_n}$
    with $\con'_1 \refines \con_1$ and $\tmtwo'_i \refines \tmtwo$ for all $1 \leq i \leq n$.
    Taking $\con' := \con'_1\,[\tmthree'_1,\hdots,\tmthree'_m]$ we conclude.
    %Moreover, note that $\con'$ must start with $\conbase[\tmthree'_1,\hdots,\tmthree'_m]$
    %where, by the previous remark, all the $\tmthree'_i$ must be $0$-hole contexts.
    By \ih the decomposition is unique for $\tmfour'$, so the decomposition is also unique for $\tm'$.
  \item {\bf Right of an application, $\con = \tmthree\,\con_1$.}
    Then $\tm'$ must be of the form $\tmthree'\,[\tmfour'_1,\hdots,\tmfour'_m]$
    where $\tmthree' \refines \tmthree$ and $\tmfour'_i \refines \con_1\of{\tmtwo}$ for all $1 \leq i \leq m$.
    By \ih, we can write each of the $\tmfour'_i$ as $\tmfour'_i = \con'_{(1,i)}\of{\tmtwo'_{(i,1)},\hdots,\tmtwo'_{(i,k_i)}}$
    where $\con'_{(1,i)}$ is a context of $k_i$ holes
    such that $\con'_{(1,i)} \refines \con_1$ and $\tmtwo'_{(i,j)} \refines \tmtwo$ for all $1 \leq i \leq m$ and all $1 \leq j \leq k_i$.
    Taking $\con' = \tmthree'[\con'_{(1,1)},\hdots,\con'_{(1,m)}]$ as a context with $m = \sum_{i=1}^{n} k_i$ holes
    we conclude.
    %Moreover, note that $\con'$ must start with $\tmthree'[\conbase,\hdots,\conbase]$
    %where the application has exactly $m$ arguments,
    %and, by the previous remark, $\tmthree'$ must be a $0$-hole context.
    By \ih the decomposition fo reach $\tmthree'_i$ is unique,
    so the decomposition is also unique for $\tm'$.
  \end{enumerate}
\item $(2 \implies 1)$
  By induction on $\con$.
  \begin{enumerate}
  \item {\bf Empty, $\con = \conbase$.}
    Then $\con' = \conbase$ and $\tmtwo'_1 = \tm'$, so we are done.
  \item {\bf Abstraction, $\con = \lam{\var}{\con_1}$.}
    Then $\con' = \lamp{\lab}{\var}{\con'_1}$ where $\con'_1 \refines \con_1$.
    By \ih $\con'_1\of{\tmtwo'_1,\hdots,\tmtwo'_n} \refines \con_1\of{\tmtwo}$,
    so $\tm' = \lamp{\lab}{\var}{\con'_1\of{\tmtwo'_1,\hdots,\tmtwo'_n}} \refines \lam{\var}{\con_1\of{\tmtwo}}$.
  \item {\bf Left of an application, $\con = \con_1\,\tmthree$.}
    Then $\con' = \con'_1\,[\tmthree'_1,\hdots,\tmthree'_m]$ where $\con'_1 \refines \con_1\of{\tmtwo}$ and
    $\tmthree'_i \refines \tmthree$ for all $1 \leq i \leq m$.
    Note that the $\tmthree'_i$ are terms (\ie they are $0$-hole contexts), by the previous remark.
    Then by \ih, $\con'_1\of{\tmtwo'_1,\hdots,\tmtwo'_n} \refines \con_1\of{\tmtwo}$,
    so $\tm' = \con'_1\of{\tmtwo'_1,\hdots,\tmtwo'_n}[\tmthree'_1,\hdots,\tmthree'_m] \refines \con_1\of{\tmtwo}\,\tmthree$.
  \item {\bf Right of an application, $\con = \tmthree\,\con_1$.}
    Then $\con' = \tmthree'[\con'_{(1,1)},\hdots,\con'_{(1,m)}]$
    where $\tmthree' \refines \tmthree$ and $\con'_{(1,j)} \refines \con_1$ for all $1 \leq j \leq m$.
    Note that $\tmthree'$ is a term (\ie it is a $0$-hole context), by the previous remark. 
    Moreover, for each $1 \leq j \leq m$, the context $\con'_{(1,j)}$ is a context with $k_j \geq 0$ holes,
    in such a way that $\sum_{j=1}^{m} k_j = n$.
    Let us split the list $[\tmtwo'_1,\hdots,\tmtwo'_n]$ in $m$ lists, such that the $j$-th list has $k_j$ elements,
    \ie $[\tmtwo'_1,\hdots,\tmtwo'_n] = [\tmtwo'_{(1,1)},\hdots,\tmtwo'_{(1,k_1)},\hdots,\tmtwo'_{(m,1)},\hdots,\tmtwo'_{(m,k_m)}]$.
    By \ih, for each $1 \leq j \leq m$ we have that $\con'_{(1,j)}\of{ \tmtwo'_{(j,1)},\hdots,\tmtwo'_{(j,k_j)} } \refines \con_1\of{\tmtwo}$.
    So:
    \[
      \tm' = \tmthree'[\con'_{(1,1)}\of{ \tmtwo'_{(1,1)},\hdots,\tmtwo'_{(1,k_1)} },\hdots,\con'_{(1,m)}\of{ \tmtwo'_{(m,1)},\hdots,\tmtwo'_{(j,k_m)} }]
      \refines \tmthree\,\con_1\of{\tmtwo}
    \]
    which concludes the proof.
  \end{enumerate}
\end{itemize}
\end{proof}

\begin{lemma}[Contexts refined by terms may be filled]
\llem{contexts_refined_by_terms_may_be_filled}
Suppose that $\tm' \refines \con$ where $\tm'$ is a term (with no holes).
Then $\tm' \refines \con\of{\Anon}$ for any term or context $\Anon$.
\end{lemma}
\begin{proof}
By induction on $\con$.
\begin{enumerate}
\item {\bf Empty, $\con = \conbase$.}
  This case is impossible, since it implies $\tm' = \conbase$.
\item {\bf Abstraction, $\con = \lam{\var}{\con'}$.}
  Then $\tm' = \lamp{\lab}{\var}{\tm''}$ where $\tm'' \refines \con'$.
  By \ih, $\tm'' \refines \con'\of{\Anon}$ so $\tm' = \lamp{\lab}{\var}{\tm''} \refines \lam{\var}{\con'\of{\Anon}} = \con\of{\Anon}$.
\item {\bf Left of an application, $\con = \con'\,\tmtwo$.}
  Then $\tm' = \tm''\,[\tmtwo'_1,\hdots,\tmtwo'_n]$ with $\tm'' \refines \con'$ and $\tmtwo'_i \refines \tmtwo$ for all $1 \leq i \leq n$.
  By \ih, $\tm'' \refines \con'\of{\Anon}$ so $\tm' = \tm''\,[\tmtwo'_1,\hdots,\tmtwo'_n] \refines \con'\of{\Anon}\,\tmtwo = \con\of{\Anon}$.
\item {\bf Right of an application, $\con = \con'\,\tmtwo$.}
  Then $\tm' = \tmtwo'\,[\tm''_1,\hdots,\tm''_n]$ with $\tmtwo' \refines \tmtwo$ and $\tm''_i \refines \con'$ for all $1 \leq i \leq n$.
  By \ih $\tm''_i \refines \con'\of{\Anon}$ for all $1 \leq i \leq n$,
  so $\tm' = \tmtwo'\,[\tm''_1,\hdots,\tm''_n] \refines \tmtwo\,\con'\of{\Anon} = \con\of{\Anon}$.
\end{enumerate}
\end{proof}
