Generally, intersection type systems are used on the pure $\lambda$-calculus.
In such environment, a term may be able to be typed:
for example, in system~$\mathcal{W}$,
$\lam{x}{x x}$ can have the type $((\alpha \to \beta) \cap \alpha) \to \beta$.
On the other hand, the term $\Omega = (\lam{x}{x x}) (\lam{x}{x x})$ cannot
be typed.

In contrast, what we will do in this work is define a slightly different
calculus, such that all terms of that calculus can be typed using
the non idempotent intersection type system defined above.

The informal idea behind the definition of the terms of the calculus is that
in an application, for each type in the domain of the function there will be
a different argument. So if we have a term $\tm$ with type
$(\typ_1 \cap \typ_2) \to \typ_3$, we can apply it to a \textit{list}
of arguments, one with type $\typ_1$ and the other with type $\typ_2$.

\begin{definition}[Distributive type system]
The set of \defn{distributive terms},
ranged over by ($\tm,\tmtwo,\tmthree,\hdots$) is given by the following abstract syntax:
\[
  \tm ::= \var^{\typ} \mid \lamp{\lab}{\var}{\tm} \mid \tm\,\ls{\tm}
\]
Typing rules are defined inductively as follows.
\[
  \indrule{var}{
  }{
    \var : \lset{\typ} \vdash \var^\typ : \typ
  }
  \indrule{\toI}{
    \tctx \oplus \var : \mtyp \vdash \tm : \typtwo
  }{
    \tctx \vdash \lamp{\lab}{\var}{\tm} : \mtyp \tolab{\lab} \typtwo
  }
\]
\[
  \indrule{\toE}{
    \tctx \vdash \tm : \lset{\typtwo_1,\hdots,\typtwo_n} \tolab{\lab} \typ
    \HS
    \left( \tctxtwo_i \vdash \tmtwo_i : \typtwo_i \right)_{i=1}^{n}
  }{
    \tctx +_{i=1}^{n} \tctxtwo_i \vdash \tm[\tmtwo_1,\hdots,\tmtwo_n] : \typ
  }
\]
Moreover, we introduce a judgment of the form
$[\tctx_1,\hdots,\tctx_n] \vdash [\tm_1,\hdots,\tm_n] : [\typ_1,\hdots,\typ_n]$
with the following rule:
\[
  \indrule{t-multi}{
    \tctx_i \vdash \tm_i : \typ_i \text{ for all $i=1..n$}
  }{
    [\tctx_1,\hdots,\tctx_n] \vdash [\tm_1,\hdots,\tm_n] : [\typ_1,\hdots,\typ_n]
  }
\]
\end{definition}

\begin{example}
Uusing integer labels,
\[\vdash \lamp{1}{\var}{\var^{[\alpha^2,\alpha^3] \tolab{4} \beta^5}[\var^{\alpha^3},\var^{\alpha^2}]}
: [[\alpha^2,\alpha^3] \tolab{4} \beta^5, \alpha^2, \alpha^3] \tolab{1} \beta^5\]
is a derivable judgment.
For another example,
\[\var : [] \tolab{1} \alpha^2 \vdash \var^{[] \tolab{1} \alpha^2} [] : \alpha^2\]
is a derivable judgment.
\end{example}

Note that writing all labels is rather cumbersome, so we will omit or simplify them when possible.

\subsection{Correctness}
Observe that the definition we gave has a fatal flaw:
we cannot uniquely associate arguments with variables in the body of the lambdas.
For example, consider the term
$(\lamp{1}{x}{y^{[\alpha^2, \alpha^2] \tolab{3} \alpha^4} [x^{\alpha^2}, x^{\alpha^2}]})
[a^{\alpha^2}, b^{\alpha^2}]$.
We don't know which parameter to associate which eaich $x$---which parameter goes in the first $x$, $a$ or $b$?

To solve that problem we introduce an invariant that will ensure
that problem does not manifest.
We will call that invariant \defn{correctness},
and we will consider $\lambdadist$ to be the system of all correct terms.

Note that the problem is that the function in the application expects two arguments
with exactly the same type. A related problem is that, in the body of the function,
the variable $x$ has the same type twice (remember that in a non idempotent context repetition matters).
In fact, it is enough to ask that those two anomalies don't show up for the system to work.

We will also ask that the labels of the lambdas do not repeat which
will come in handy later.

\begin{definition}[Correct term]
\ldef{sequentiality_and_correctness}
A multiset of types $[\typ_1,\hdots,\typ_n]$ is \defn{sequential}
if the external labels of $\typ_i$ and $\typ_j$ are different for all $i \neq j$.
A typing context $\tctx$ is sequential if $\tctx(\var)$ is sequential for every $\var \in \dom\tctx$.
The set of \defn{correct} terms is denoted by $\termsdist$.
A term $\tm$ is correct if it is typable and it verifies the following three conditions:
\begin{enumerate}
\item {\em Uniquely labeled lambdas.}
  If $\lamp{\lab}{\var}{\tmtwo}$ and $\lamp{\labtwo}{\vartwo}{\tmthree}$ 
  are subterms of $\tm$ at different positions, then $\lab$ and $\labtwo$
  must be different labels.
\item {\em Sequential contexts.}
  If $\tmtwo$ is a subterm of $\tm$ and $\tctx \vdash \tmtwo : \typ$
  is derivable, then $\tctx$ must be sequential.
\item {\em Sequential types.}
  If $\tmtwo$ is a subterm of $\tm$,
  the judgment $\tctx \vdash \tmtwo : \typ$ is derivable,
  and there exists a type such that
  $(\mtyp \tolab{\lab} \typtwo \occursin \tctx)$ or $(\mtyp \tolab{\lab} \typtwo \occursin \typ)$,
  then $\mtyp$ must be sequential.
\end{enumerate}
\end{definition}

\begin{example}
$\var^{[\alpha^1] \tolab{2} \beta^3}[\var^{\alpha^1}]$ is a correct term.
The example we gave above,
\[(\lamp{1}{x}{y^{[\alpha^2, \alpha^2] \tolab{3} \alpha^4} [x^{\alpha^2}, x^{\alpha^2}]})
[a^{\alpha^2}, b^{\alpha^2}]\]
is not.
One problem is that the type of $y$ is not sequential.
For a last example,
$\lamp{1}{\var}{\lamp{1}{\vartwo}{\vartwo^{\alpha^2}}}$
is not a correct term since labels for lambdas are not unique.
\end{example}


Having defined the types and the terms of our system
sheds some light on the resource management capabilities that we claimed it will enjoy:
note that we can track very precisely how (\ie with which type) a term will be used or
a bounded variable will be evaluated.
This will prove very useful to analyze the $\lambda$-calculus.

The next lemma shows that a term is uniquely typable: this means that for a given term
there is only one type and typing context that satisfy the typing judgement.
Moreover, all proof trees are the same.

Note that for all proof trees to be the same it is crucial that
we only consider correct terms, because otherwise when we apply the rule
$\toE$ we may have the possiblity to choose between different orders for the
parameters.\footnote{This implies that a weaker uniqueness result holds for
incorrect terms: proof trees are equal modulo permutations when the rule $\toE$
is applied, but as we don't care about incorrect terms we don't need to consider this.}

\begin{lemma}[Unique typing]
\llem{unique_typing}
Let $\tm$ be typable, \ie suppose that there exist a context $\tctx$ and a type $\typ$ such that $\tctx \vdash \tm : \typ$. Furthermore, suppose that $\tm$ is correct.
Then there is a unique derivation that types $\tm$.
In particular, if $\tctx' \vdash \tm : \typ'$, then $\tctx = \tctx'$ and $\typ = \typ'$.
\end{lemma}
\begin{proof}
\SeeAppendixRef{unique_typing_proof}
By induction on $\tm$.
\end{proof}

