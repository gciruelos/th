As we stated in the introduction, the goal of labeling terms and types
is that we obtain a confluent calculus. Indeed, the resulting calculus
is confluent, and it is the purpose of this section to prove so.


First, we need to prove an adaptation of the Substitution Lemma for our calculus,
which is a key tool to prove properties about coinitial steps.
The substitution lemma for the pure lambda calculus \cite[Lemma 2.1.16]{Barendregt:1984}  states that, provided that
$\var \neq \vartwo$ and $\var \not\in \fv{\tmthree}$, then
  $\subs{\subs{\tm}{\var}{\tmtwo}}{\vartwo}{\tmthree} =
    \subs{\subs{\tm}{\vartwo}{\tmthree}}{\var}{\subs{\tmtwo}{\vartwo}{\tmthree}}$

In our case variables get substituted by a \emph{list} of terms, so we need to adapt it.
Particularly, the list of terms that will take the place of $\tmthree$ needs to be
divided up in several lists: one for the corresponding $\vartwo$s in $\tm$, and the rest
for the $\vartwo$s in each of the elements of $\tmtwo$, which recall that now will be a list.

\begin{notation}
We extend the substitution operator to work on lists,
defining
\[\subs{[\tm_i]_{i=1}^{n}}{\var}{\ls{\tmtwo}} \eqdef [\subs{\tm_i}{\var}{\ls{\tmtwo}_i}]_{i=1}^{n}\]
where $(\ls{\tmtwo}_1,\hdots,\ls{\tmtwo}_n)$ is a partition of $\ls{\tmtwo}$
such that $\varlabel{\var}{\tm_i} = \tmlabel{\ls{\tmtwo}_i}$ for all $i \in \set{1,\hdots,n}$.
\end{notation}

\begin{lemma}[Substitution Lemma]
\llem{substitution_lemma}
Let $\var \neq \vartwo$ and $\var \not\in \fv{\ls{\tmthree}}$.
If $(\ls{\tmthree}_1,\ls{\tmthree}_2)$ is a partition of $\ls{\tmthree}$
then
\[
  \subs{\subs{\tm}{\var}{\ls{\tmtwo}}}{\vartwo}{\ls{\tmthree}}
  =
  \subs{\subs{\tm}{\vartwo}{\ls{\tmthree}_1}}{\var}{\subs{\ls{\tmtwo}}{\vartwo}{\ls{\tmthree}_2}}
\]
provided that both sides of the equation are defined.
{\em Note:} there exists a list $\ls{\tmthree}$ that makes the left-hand side defined
if and only if there exist lists $\ls{\tmthree}_1,\ls{\tmthree}_2$ that make the
right-hand side defined.
\end{lemma}
\begin{proof}
\SeeAppendixRef{substitution_lemma_proof}
By induction on $\tm$.
\end{proof}

\begin{example} For example, consider the term
\[
  \tm =
    (\lamp{\lab}{\var}{\varthree^{[\alpha] \to [\beta] \to \delta} [\var^\alpha] [\var^\beta]})
      [z^\beta,y^\alpha]
\]
We can perform the following substitution (where $w,a,b$ are all variables).
\[
  \subs{
    \subs{\tm}{\varthree}{[w^{[\alpha] \to [\beta] \to \delta}, \vartwo^\beta]}
  }{
    \vartwo
  }{
    [a^\beta,b^\alpha]
  }
  =
   (\lamp{\lab}{\var}{w^{[\alpha] \to [\beta] \to \delta} [\var^\alpha] [\var^\beta]})
    [a^\beta,b^\alpha]
\]
Note that, as we replaced one $\varthree$ by a $\vartwo$, if we wanted to invert the order
of the substitutions we would need to separate the list $[a^\beta, b^\alpha]$ in two.
\[
  \subs{
    \subs{\tm}{\vartwo}{[b^\alpha]}
  }{
    \varthree
  }{
    \subs{[w^{[\alpha] \to [\beta] \to \delta}, \vartwo^\beta]}{\vartwo}{a^\beta}
  }
  =
   (\lamp{\lab}{\var}{w^{[\alpha] \to [\beta] \to \delta} [\var^\alpha] [\var^\beta]})
    [a^\beta,b^\alpha]
\]
\end{example}



\TODO{Charlar como escribir bien esto (permutation / confluence) (pregunta anotada en el cuaderno)}

