
\begin{lemma}[Different redexes have disjoint simulation residuals]
\llem{different_redexes_have_disjoint_images}
If $\redex \neq \redextwo$, then $\names(\redex/\tm')$ and $\names(\redextwo/\tm')$ are disjoint.
\end{lemma}
\begin{proof}
\begin{gonzaenv}
Let $\redex : \conof{(\lam{\var}{\tmfour})\tmtwo} \to \conof{\subs{\tmfour}{\var}{\tmtwo}}$.
We proceed by induction on $\con$.
\begin{enumerate}
  \item {\bf $\con = \conbase$.}
    We have that $\redex : (\lam{\var}{\tmfour})\tmtwo \to \subs{\tmfour}{\var}{\tmtwo}$.
    Furthermore,
    $\tm' = (\lamp{\lab}{\var}{\tmfour'}) \ls{\tmtwo}$
    and $\tm' / \redex = \subs{\tmfour'}{\var}{\ls{\tmtwo}}$, hence $\names(\redex / \tm') = \set{\lab}$.
    Now there are two cases depending on where $\redextwo$ is located.
    \begin{enumerate}
      \item {\bf $\redextwo$ is in $\tmfour$.}
        So $\tmfour = \con_{\tmfour}\of{(\lam{\vartwo}{\tmthree}) \tmthreevariant}$.
        Now, $\redextwo : (\lam{\var}{\con_{\tmfour}\of{(\lam{\vartwo}{\tmthree}) \tmthreevariant}}) \tmtwo
             \to (\lam{\var}{\con_{\tmfour}\of{\subs{\tmthree}{\vartwo}{\tmthreevariant}}}) \tmtwo$.
        That implies that
          $\tm' = (\lamp{\lab}{\var}{\con'_{\tmfour}\of{
            (\lamp{\lab_1}{\vartwo}{\tmthree_1}) \tmthreevariant_1, \hdots,
            (\lamp{\lab_n}{\vartwo}{\tmthree_n}) \tmthreevariant_n}}) \ls{\tmtwo}$,
        and, in turn, the simulated step yields
          $\tm' / \redextwo = (\lamp{\lab}{\var}{\con'_{\tmfour}\of{
            \subs{\tmthree_1}{\vartwo}{\tmthreevariant_1}, \hdots,
            \subs{\tmthree_n}{\vartwo}{\tmthreevariant_n}}}) \ls{\tmtwo}$,
        so $\names(\redextwo / \tm') = \set{\lab_1, \hdots, \lab_n}$, which does not contain $\lab$ because
        labels are pairwise distinct.
      \item {\bf $\redextwo$ is in $\tmtwo$.}
        So $\tmtwo = \con_{\tmtwo}\of{(\lam{\vartwo}{\tmthree}) \tmthreevariant)}$.
        Like before,
          $\redextwo : (\lam{\var}{\tmfour}) \con_{\tmtwo}\of{(\lam{\vartwo}{\tmthree}) \tmthreevariant}
             \to (\lam{\var}{\tmfour}) \con_{\tmtwo}\of{\subs{\tmthree}{\vartwo}{\tmthreevariant}}$.
        By \rlem{refinement_context} we have that $\tm' = (\lamp{\lab}{\var}{\tmfour'}) [\con'_{\tmtwo}\of{
          (\lamp{\lab_{i,1}}{\vartwo}{\tmthree_{i,1}}) \tmthreevariant_{i,1}, \hdots,
          (\lamp{\lab_{i,m_i}}{\vartwo}{\tmthree_{i,m_i}}) \tmthreevariant_{i,m_i}}]_{i=1}^n$, then
        $\names(\redextwo / \tm') = \bigcup_{i=1}^n \set{\lab_{i,1}, \hdots, \lab_{i, m_i}}$, which
        does not contain $\lab$ because labels are pairwise distinct.
    \end{enumerate}
  \item {\bf $\con = \lam{\varthree}{\con_1}$.} This case is straightforward by inductive hypothesis.
  \item {\bf $\con = \tmfive \ \con_1$.}
    Now we have that
    $\redex : \tmfive \ \con_1\of{(\lam{\vartwo}{\tmthree}) \tmthreevariant}
      \to \tmfive \ \con_1\of{\subs{\tmthree}{\vartwo}{\tmthreevariant}}$.
    And we can again separate in cases depending on where $\redextwo$ is.
    \begin{enumerate}
      \item {\bf $\redextwo$ is in $\tmfive$.} What that means is that
        $\redextwo : \tmfive \ \con_1\of{(\lam{\vartwo}{\tmthree}) \tmthreevariant}
          \to \tmsix \ \con_1\of{(\lam{\vartwo}{\tmthree}) \tmthreevariant}$.
        Using the same reasoning as before, we know that
        $\tm' = \tmfive' [\con'_{1,i}\of{
          (\lamp{\lab_{i,1}}{\vartwo}{\tmthree_{i,1}}) \tmthreevariant_{i,1}, \hdots,
          (\lamp{\lab_{i,m_i}}{\vartwo}{\tmthree_{i,m_i}}) \tmthreevariant_{i,m_i}}
          ]_{i=1}^n$.
        As $\redextwo / \tm'$ only reduces lambdas in $\tmfive'$, and $\redextwo / \tm'$ only reduces lambdas
        on the argument list of the application, the set of names are disjoint.
      \item {\bf $\redextwo$ is in $\con_1\of{(\lam{\vartwo}{\tmthree}) \tmthreevariant}$.}
        This case is straightforward by inductive hypothesis.
    \end{enumerate}
  \item {\bf $\con = \con_1 \ \tmfive$.}
    In this case
    $\redex : \con_1\of{\lam{\vartwo}{\tmthree}) \tmthreevariant} \ \tmfive \to
              \con_1\of{\subs{\tmthree}{\vartwo}{\tmthreevariant}} \ \tmfive$,
    and we will, once again, divide in two cases, depending on where the redex $\redextwo$ is located.
    \begin{enumerate}
      \item {\bf $\redextwo$ is in $\con_1\of{(\lam{\vartwo}{\tmthree}) \tmthreevariant}$.}
        This case is straightforward by inductive hypothesis.
      \item {\bf $\redextwo$ is in $\tmfive$.}
        More precisely, what we have is that
        $\redextwo : \con_1\of{(\lam{\vartwo}{\tmthree}) \tmthreevariant} \ \tmfive
          \to \con_1\of{(\lam{\vartwo}{\tmthree}) \tmthreevariant} \ \tmsix$.
        Moreover,
        $\tm' = \con_1\of{
          (\lamp{\lab_1}{\vartwo}{\tmthree_1}) \tmthreevariant_1, \hdots,
          (\lamp{\lab_m}{\vartwo}{\tmthree_m}) \tmthreevariant_m
        } [\tmfive_1, \hdots, \tmfive_n]$.
        Hence, $\names(\redex / \tm') = \set{\lab_1, \hdots, \lab_m}$. Also, $\redextwo / \tm'$
        only reduces lambdas in the terms $\tmfive_1$, ..., $\tmfive_n$, which must be different
        than the ones reduced by $\redex / \tm'$ because lambda labels are pairwise distinct.
    \end{enumerate}
\end{enumerate}
\end{gonzaenv}
\end{proof}

\subsection*{A set of steps and a disjoint derivation have disjoint simulation residuals}
\begin{lemma}[A set of steps and a disjoint derivation have disjoint simulation residuals]
\llem{disjoint_redex_derivation_have_disjoint_images}
Let $\redexset$ be a set of coinitial steps such that no residuals of any step in $\redexset$
are contracted along a derivation $\redseqtwo$.
Then $\names(\redexset/\tm')$ and $\names(\redseqtwo/\tm')$ are disjoint.
\end{lemma}
\begin{proof}
We proceed by induction on $\redseqtwo$. If $\redseqtwo$ is empty, it is immediate,
so suppose that $\redseqtwo = \redexthree\redseqthree$.
Note that $\redexthree \not\in \redexset$,
since residuals of redexes in $\redexset$ are never contracted in the derivation $\redexthree\redseqthree$.
This implies that $\names(\redexset/\tm') \cap \names(\redexthree/\tm') = \emptyset$
by \rlem{disjoint_redex_derivation_have_disjoint_images}.
Moreover, this means that:
\[
  \begin{array}{rcll}
  \names(\redexset/\tm')
  & = & \names(\redexset/\tm') \setminus \names(\redexthree/\tm') & \text{since they are disjoint sets} \\
  & = & \names((\redexset/\tm')/(\redexthree/\tm')) & \text{by \rlem{names_after_projection_along_a_step}} \\
  & = & \names((\redexset/\redexthree)/(\tm'/\redexthree)) & \text{by \rcoro{permutation_equivalence_in_terms_of_names} and \rlem{generalized_cube_lemma}} \\
  \end{array}
\]
Then:
\[
  \begin{array}{rcll}
  \names(\redexset/\tm') \cap \names(\redseqtwo/\tm')
  & = & \names(\redexset/\tm') \cap \names(\redexthree\redseqthree/\tm') \\
  & = & \names(\redexset/\tm') \cap \names((\redexthree/\tm')(\redseqthree/(\tm'/\redexthree))) \\
  & = & (\names(\redexset/\tm') \cap \names(\redexthree/\tm')) \cup \\
    && (\names(\redexset/\tm') \cap \names(\redseqthree/(\tm'/\redexthree))) \\
  & = & \names(\redexset/\tm') \cap \names(\redseqthree/(\tm'/\redexthree)) \\
  && \text{ since $\names(\redexset/\tm') \cap \names(\redexthree/\tm')$ is empty} \\
  & = & \names((\redexset/\redexthree)/(\tm'/\redexthree)) \cap \names(\redseqthree/(\tm'/\redexthree)) \\
  & = & \emptyset \HS\text{ by \ih} \\
  \end{array}
\]
To justify the last step, observe that residuals of redexes in the set $\redexset/\redexthree$
may not be contracted along the derivation $\redseqthree$,
since this would imply that a residual of a redex in the set $\redexset$
is contracted along the derivation $\redexthree\redseqthree$,
contradicting the hypothesis.
\end{proof}

\subsection*{Proof of \rprop{characterization_of_garbage} --- Characterization of garbage}
\label{characterization_of_garbage_proof}
Let $\redseq : \tm \rtobeta \tmtwo$ and $\tm' \refines \tm$.
The following are equivalent:
\begin{enumerate}
\item $\redseq \sieve \tm' = \emptyDerivation$.
\item There are no coarse steps for $(\redseq,\tm')$.
\item The derivation $\redseq$ is $\tm'$-garbage.
\end{enumerate}
\begin{proof}
It is immediate to check that items 1 and 2 are equivalent, by definition of sieving,
so let us prove $2 \implies 3$ and $3 \implies 2$:
\begin{itemize}
\item $(2 \implies 3)$
  We prove the contrapositive, namely that if $\redseq$ is not garbage,
  there is a coarse step for $(\redseq,\tm')$.
  Suppose that $\redseq$ is not garbage, \ie $\redseq/\tm' \neq \emptyDerivation$.
  Then by \rprop{properties_of_garbage}
  we have that $\redseq$ can be written as $\redseq = \redseq_1 \redex \redseq_2$ where
  all the steps in $\redseq_1$ are garbage and $\redex$ is not garbage.
  By the fact that garbage only creates garbage~(\rlem{garbage_only_creates_garbage})
  the step $\redex$ has an ancestor $\redex_0$, \ie $\redex \in \redex_0/\redseq_1$.
  Moreover, since garbage only duplicates garbage~(\rlem{garbage_only_duplicates_garbage})
  we have that $\redex_0/\redseq_1 = \redex$.
  Given that $\redex$ is not garbage, we have that:
  \[
    \begin{array}{rcll}
          (\redex_0/\tm')/(\redseq_1/\tm')
    & = & (\redex_0/\redseq_1)/(\tm'/\redseq_1) & \text{ by \rlem{generalized_cube_lemma}} \\
    & = & \redex/(\tm'/\redseq_1) \\
    & \neq & \emptyset \\
    \end{array}
  \]
  Since $(\redex_0/\tm')/(\redseq_1/\tm') \neq \emptyset$, in particular,
  $\redex_0/\tm' \neq \emptyset$, which means that $\redex_0$ is not garbage.
  Moreover, $\redex_0 \permle \redseq_1\redex\redseq_2 = \redseq$.
  So $\redex_0$ is coarse for $(\redseq,\tm')$.
\item $(3 \implies 2)$
  Let $\redseq$ be garbage, suppose that there is a coarse step $\redex$ for $(\redseq,\tm')$,
  and let us derive a contradiction.
  Since $\redex$ is coarse for $(\redseq,\tm')$,
  we have that $\redex \permle \redseq$,
  so $\redex/\tm' \permle \redseq/\tm'$ by \rcoro{simulation_residuals_and_prefixes}.
  But $\redseq/\tm'$ is empty because $\redseq$ is $\tm'$-garbage,
  that is, $\redex/\tm' \permle \redseqtwo/\tm' = \emptyDerivation$,
  which means that $\redex$ is also $\tm'$-garbage.
  This contradicts the fact that $\redex$ is coarse for $(\redseq,\tm')$.
\end{itemize}
\end{proof}


\subsection*{Proof of \rprop{characterization_of_garbage_free_derivations} --- Characterization of garbage-free derivations}
\label{characterization_of_garbage_free_derivations_proof}
Let $\redseq : \tm \rtobeta \tmtwo$ and $\tm' \refines \tm$.
The following are equivalent:
\begin{enumerate}
\item $\redseq$ is $\tm'$-garbage-free.
\item $\redseq \permeq \redseq \sieve \tm'$.
\item $\redseq \permeq \redseqtwo \sieve \tm'$ for some derivation $\redseqtwo$.
\end{enumerate}
\begin{proof}
Let us prove $1 \implies 2 \implies 3 \implies 1$:
\begin{itemize}
\item $(1 \implies 2)$
  Suppose that $\redseq$ is $\tm'$-garbage-free,
  and let us show that $\redseq \permeq \redseq \sieve \tm'$
  by induction on the length of $\redseq \sieve \tm'$.

  If there are no coarse steps for $(\redseq,\tm')$,
  By \rprop{characterization_of_garbage}, any derivation with no coarse steps is garbage.
  So $\redseq$ is $\tm'$-garbage.
  Since $\redseq$ is garbage-free, this means that $\redseq = \emptyDerivation$.
  Hence $\redseq = \emptyDerivation = \redseq \sieve \tm'$, as required.

  If there exists a coarse step for $(\redseq,\tm')$, let $\redex_0$ be the leftmost such step.
  Note that $\redseq \permeq \redex_0(\redseq/\redex_0)$ since $\redex_0 \permle \redseq$.
  Moreover, we claim that $\redseq/\redex_0$ is $(\tm'/\redex_0)$-garbage-free.
  Let $\redseqtwo \permle \redseq/\redex_0$ such that $(\redseq/\redex_0)/\redseqtwo$ is garbage
  with respect to the term $(\tm'/\redex_0)/\redseqtwo = \tm'/\redex_0\redseqtwo$,
  and let us show that $(\redseq/\redex_0)/\redseqtwo$ is empty.
  Note that:
  \[
    \begin{array}{rcll}
    \redex_0\redseqtwo
    & \permle & \redex_0(\redseq/\redex_0) & \text{since $\redseqtwo \permle \redseq/\redex_0$} \\
    & \permeq & \redseq                    & \text{as already noted} \\
    \end{array}
  \]
  Moreover, we know that the derivation $\redseq/\redex_0\redseqtwo = (\redseq/\redex_0)/\redseqtwo$ is
  $(\tm'/\redex_0\redseqtwo)$-garbage.
  So, given that $\redseq$ is $\tm'$-garbage-free, we conclude that
  $\redseq/\redex_0\redseqtwo = \emptyDerivation$, that is $(\redseq/\redex_0)/\redseqtwo = \emptyDerivation$,
  which completes the proof of the claim that $\redseq/\redex_0$ is $(\tm'/\redex_0)$-garbage-free.
  We conclude as follows:
  \[
    \begin{array}{rcll}
      \redseq & \permeq & \redex_0(\redseq/\redex_0)                          & \text{as already noted} \\
              & \permeq & \redex_0((\redseq/\redex_0) \sieve (\tm'/\redex_0)) & \text{by \ih since $\redseq/\redex_0$ is $(\tm'/\redex_0)$-garbage-free} \\
              & \permeq & \redseq \sieve \tm'                                 & \text{by definition of sieving} \\
    \end{array}
  \]
\item $(2 \implies 3)$ Obvious, taking $\redseqtwo := \redseq$.
\item $(3 \implies 1)$
  Let $\redseq \permeq \redseqtwo \sieve \tm'$.
  Let us show that $\redseq$ is garbage-free by induction on the length of $\redseqtwo \sieve \tm'$.

  If there are no coarse steps for $(\redseqtwo,\tm')$,
  then $\redseq \permeq \redseqtwo \sieve \tm' = \emptyDerivation$, which means that $\redseq = \emptyDerivation$.
  Observe that the empty derivation is trivially garbage-free.

  If there exists a coarse step for $(\redseqtwo,\tm')$,
  let $\redex_0$ be the leftmost such step.
  Then $\redseq \permeq \redseqtwo \sieve \tm' = \redex_0((\redseqtwo/\redex_0) \sieve (\tm'/\redex_0))$.
  To see that $\redseq$ is $\tm'$-garbage-free,
  let $\redseqthree \permle \redseq$ such that $\redseq/\redseqthree$ is garbage,
  and let us show that $\redseq/\redseqthree$ is empty. 
  We know that $\redseq/\redseqthree$ is of the following form (modulo permutation equivalence):
  \[
     \redseq/\redseqthree \permeq
     \frac{\redex_0((\redseqtwo/\redex_0) \sieve (\tm'/\redex_0))}{\redseqthree}
     =
     \left(\frac{\redex_0}{\redseqthree}\right) \left(\frac{(\redseqtwo/\redex_0) \sieve (\tm'/\redex_0)}{\redseqthree/\redex_0}\right)
  \]
  That is, we know that the following derivation is $(\tm'/\redseqthree)$-garbage,
  and it suffices to show that it is empty:
  \[
    \left(\frac{\redex_0}{\redseqthree}\right) \left(\frac{(\redseqtwo/\redex_0) \sieve (\tm'/\redex_0)}{\redseqthree/\redex_0}\right)
  \]
  Recall that, in general, $AB$ is garbage if and only if $A$ and $B$ are garbage (\rprop{properties_of_garbage}).
  Similarly, $AB$ is empty if and only if $A$ and $B$ are empty.
  So it suffices to prove the two following implications:
  \[
    \begin{array}{clll}
    {\bf (A)} &
    \text{ If } & \redex_0/\redseqthree \text{ is garbage,} & \text{ then it is empty.} \\
    {\bf (B)} &
    \text{ If } & ((\redseqtwo/\redex_0) \sieve (\tm'/\redex_0))/(\redseqthree/\redex_0) \text{ is garbage,} & \text{ then it is empty.}
    \end{array}
  \]
  Let us check that each implication holds:
  \begin{itemize}
  \item {\bf (A)}
    Suppose that $\redex_0/\redseqthree$ is $(\tm'/\redseqthree)$-garbage,
    and let us show that $\redex_0/\redseqthree$ is empty.
    Knowing that the derivation $\redex_0/\redseqthree$ is garbage means that
    $(\redex_0/\redseqthree)/(\tm'/\redseqthree) = \emptyset$.
    Since $\redex_0$ is the leftmost step coarse for $(\redseqtwo,\tm')$,
    by \rlem{leftmost_coarse_step_has_at_most_one_residual}
    we have that $\#(\redex_0/\redseqthree) \leq 1$.
    If $\redex_0/\redseqthree$ is empty we are done, since this is what we wanted to prove.

    The remaining possibility is that $\redex_0/\redseqthree$ be a singleton.
    We argue that this case is impossible.
    Note that for every prefix $\redseqthree_1 \permle \redseqthree$,
    the set $\redex_0/\redseqthree_1$ is also a singleton,
    since otherwise it would be empty, as a consequence of \rlem{leftmost_coarse_step_has_at_most_one_residual}.
    So we may apply \rlem{stable_non_garbage} and conclude that,
    since $\redex_0$ is not $\tm'$-garbage
    then $\redex_0/\redseqthree$ is not $(\tm'/\redseqthree)$-garbage.
    This contradicts the hypothesis.
  \item {\bf (B)}
    Suppose that $((\redseqtwo/\redex_0) \sieve (\tm'/\redex_0))/(\redseqthree/\redex_0)$ is garbage
    with respect to the term $(\tm'/\redex_0)/(\redseqthree/\redex_0)$, and let us show that it is empty.
    Since $(\redseqtwo/\redex_0) \sieve (\tm'/\redex_0)$
    is a shorter derivation than $\redseqtwo \sieve \tm'$,
    we may apply the \ih we obtain that
    $(\redseqtwo/\redex_0) \sieve (\tm'/\redex_0)$ is $(\tm'/\redex_0)$-garbage-free.
    Moreover, the following holds:
    \[\redseqthree/\redex_0 \permle \redseq/\redex_0 \permeq (\redseqtwo/\redex_0) \sieve (\tm'/\redex_0)\]

    So, by definition of $(\redseqtwo/\redex_0) \sieve (\tm'/\redex_0)$
    being garbage-free,
    the fact that
    the derivation
    $((\redseqtwo/\redex_0) \sieve (\tm'/\redex_0))/(\redseqthree/\redex_0)$ is garbage
    implies that it is empty, as required.
  \end{itemize}
\end{itemize}
\end{proof}
 


\subsection*{Proof of \rprop{properties_of_sieving} --- Properties of sieving}
\label{properties_of_sieving_proof}
To prove \rprop{properties_of_sieving} we first prove various auxiliary results.

\begin{lemma}[The sieve is a prefix]
\llem{sieve_is_prefix}
Let $\redseq : \tm \rtobeta \tmtwo$ and $\tm' \refines \tm$.
Then $\redseq \sieve \tm' \permle \redseq$.
\end{lemma}
\begin{proof}
By induction on the length of $\redseq \sieve \tm'$.
There are two cases, depending on whether there exists a coarse step for $(\redseq,\tm')$.
\begin{enumerate}
\item {\bf If there are no coarse steps for $(\redseq,\tm')$.}
  Then trivially $\redseq \sieve \tm' = \emptyDerivation \permle \redseq$.
\item {\bf If there exists a coarse step for $(\redseq,\tm')$.}
  Let $\redseq_0$ be the leftmost coarse step for $(\redseq,\tm')$.
  Then:
  \[
    \begin{array}{rcll}
      \redseq \sieve \tm'
      & = & \redex_0((\redseq/\redex_0) \sieve (\tm'/\redex_0)) \\
      & \permle & \redex_0(\redseq/\redex_0) & \text{by \ih} \\
      & \equiv  & \redseq(\redex_0/\redseq)  & \text{since $A(B/A) \permeq B(A/B)$ in general} \\
      & =       & \redseq                    & \text{since $\redex_0 \permle \redseq$ as $\redex_0$ is coarse for $(\redseq,\tm')$} \\
    \end{array}
  \]
\end{enumerate}
\end{proof}

\begin{lemma}[Garbage only interacts with garbage]
\label{garbage_only_creates_garbage_proof}
\label{garbage_only_duplicates_garbage_proof}
The following hold:
\begin{enumerate}
\item {\bf Garbage only creates garbage.}
  Let $\redex$ and $\redextwo$ be composable steps in the $\lambda$-calculus,
  and let $\tm' \refines \src(\redex)$.
  If $\redex$ creates $\redextwo$ and $\redex$ is $\tm'$-garbage,
  then $\redextwo$ is $(\tm'/\redex)$-garbage.
\item {\bf Garbage only duplicates garbage.}
  Let $\redex$ and $\redextwo$ be coinitial steps in the $\lambda$-calculus
  and let $\tm' \refines \src(\redex)$.
  If $\redex$ duplicates $\redextwo$, \ie $\#(\redextwo/\redex) > 1$,
  and $\redex$ is $\tm'$-garbage,
  then $\redextwo$ is $(\tm'/\redex)$-garbage.
\end{enumerate}
\end{lemma}
\begin{proof}
We prove each item separately:
\begin{enumerate}
\item
  According to L\'evy~\cite{Tesis:Levy:1978},
  there are three creation cases in the $\lambda$-calculus.
  We consider the three possibilities for $\redex$ creating $\redextwo$:
  \begin{enumerate}
  \item
    {\bf Case I,
      $
         \conof{(\lam{\var}{\var})(\lam{\vartwo}{\tmtwo})\tmthree}
       \toabeta{\redex}
         \conof{(\lam{\vartwo}{\tmtwo})\tmthree}
       \toabeta{\redextwo}
         \conof{\subs{\tmtwo}{\vartwo}{\tmthree}}
      $.}
    Then by \rlem{refinement_context},
    the term $\tm'$ is of the form
    $\con'\of{\Delta_1,\hdots,\Delta_n}$
    where $\con'$ is an $n$-hole context such that $\con' \refines \conof{\conbase\,\tmthree}$
    and $\Delta_i \refines (\lam{\var}{\var})(\lam{\vartwo}{\tmtwo})$ for all $1 \leq i \leq n$.
    Since $\redex$ is garbage, we know that actually $n = 0$.
    So $\tm' \refines \conof{\conbase\,\tmthree}$ and $\redex/\tm' : \tm' \rtodist \tm' = \tm'/\redex$ in zero steps.
    Hence $\tm' \refines \conof{(\lam{\vartwo}{\tmtwo})\tmthree}$,
    so by \rlem{refinement_context},
    the term $\tm'$ can be written in a unique way as
    $\con''\of{\Sigma_1,\hdots,\Sigma_m}$, where $\con''$ is an $m$-hole context
    such that $\con'' \refines \conof{\conbase\,\tmthree}$ and $\Sigma_i \refines \lam{\vartwo}{\tmtwo}$ for all $1 \leq i \leq m$.
    Since the decomposition is unique and $\tm' \refines \conof{\conbase\,\tmthree}$,
    we conclude that $m = 0$.
    Hence $\redextwo$ is $(\tm'/\redex)$-garbage.
  \item
    {\bf Case II,
      $
         \conof{(\lam{\var}{\lam{\vartwo}{\tmtwo}})\,\tmthree\,\tmfour}
       \toabeta{\redex}
         \conof{(\lam{\vartwo}{\subs{\tmtwo}{\var}{\tmthree}})\,\tmfour}
       \toabeta{\redextwo}
         \conof{\subs{\subs{\tmtwo}{\var}{\tmthree}}{\vartwo}{\tmfour}}
      $.}
    Then by \rlem{refinement_context},
    the term $\tm'$ is of the form
    $\con'\of{\Delta_1,\hdots,\Delta_n}$
    where $\con'$ is an $n$-hole context such that $\con' \refines \conof{\conbase\,\tmfour}$
    and $\Delta_i \refines (\lam{\var}{\lam{\vartwo}{\tmtwo}})\,\tmthree$ for all $1 \leq i \leq n$.
    Since $\redex$ is garbage, we know that actually $n = 0$.
    So $\tm' \refines \conof{\conbase\,\tmfour}$ and $\redex/\tm' : \tm' \rtodist \tm' = \tm'/\redex$
    in zero steps. Hence $\tm' \refines \conof{(\lam{\vartwo}{\subs{\tmtwo}{\var}{\tmthree}})\,\tmfour}$,
    so by \rlem{refinement_context}, the term $\tm'$ can be written in a unique way as
    $\con''\of{\Sigma_1,\hdots,\Sigma_n}$, where $\con''$ is an $m$-hole context such that
    $\con'' \refines \conof{\conbase\,\tmfour}$ and $\Sigma_i \refines \lam{\vartwo}{\subs{\tmtwo}{\var}{\tmthree}}$
    for all $1 \leq i \leq m$.
    Since the decomposition is unique and $\tm' \refines \conof{\conbase\,\tmfour}$,
    we conclude that $m = 0$.
    Hence $\redextwo$ is $(\tm'/\redex)$-garbage.
  \item {\bf Case III, $
         \con_1\of{(\lam{\var}{\con_2\of{\var\,\tmtwo}})\,(\lam{\vartwo}{\tmthree})}
       \toabeta{\redex}
         \con_1\of{\hat{\con}_2\of{(\lam{\vartwo}{\tmthree})\,\hat{\tmtwo}}}
       \toabeta{\redextwo}
         \con_1\of{\hat{\con}_2\of{\subs{\tmthree}{\vartwo}{\hat{\tmtwo}}}}
      $,
      where
        $\hat{\con}_2 = \subs{\con_2}{\var}{\lam{\vartwo}{\tmthree}}$
      and
        $\hat{\tm} = \subs{\tm}{\var}{\lam{\vartwo}{\tmthree}}$.}
    Then by \rlem{refinement_context},
    the term $\tm'$ is of the form $\con'\of{\Delta_1,\hdots,\Delta_n}$
    where $\con'$ is an $n$-hole context such that $\con' \refines \con_1$
    and $\Delta_i \refines (\lam{\var}{\con_2\of{\var\,\tmtwo}})\,(\lam{\vartwo}{\tmthree})$ for all $1 \leq i \leq n$.
    Since $\redex$ is garbage, we know that actually $n = 0$.
    So $\tm' \refines \con_1$ and $\redex/\tm' : \tm' \rtodist \tm' = \tm'/\redex$ in zero steps.
    Hence $\tm' \refines \con_1\of{\hat{\con}_2\of{(\lam{\vartwo}{\tmthree})\,\hat{\tmtwo}}}$,
    so by \rlem{refinement_context}, the term $\tm'$ can be written in a unique way as
    $\con''\of{\Sigma_1,\hdots,\Sigma_m}$, where $\con'' \refines \con_1$ and
    $\Sigma_i \refines \hat{\con}_2\of{(\lam{\vartwo}{\tmthree})\,\hat{\tmtwo}}$ for all $1 \leq i \leq m$.
    Since the decomposition is unique and $\tm' \refines \con_1$, we conclude that $m = 0$.
    Hence $\redextwo$ is $(\tm'/\redex)$-garbage.
  \end{enumerate}
\item
  Since $\redex$ duplicates $\redextwo$,
  the redex contracted by $\redextwo$ 
  lies inside the argument of $\redex$, that is,
  the source term is of the form $\con_1\of{(\lam{\var}{\tm})\con_2\of{(\lam{\vartwo}{\tmtwo})\tmthree}}$
  where the pattern of $\redex$ is $(\lam{\var}{\tm})\con_2\of{(\lam{\vartwo}{\tmtwo})\tmthree}$,
  and the pattern of $\redextwo$ is $(\lam{\vartwo}{\tmtwo})\tmthree$.
  By \rlem{refinement_context},
  the term $\tm'$ is of the form $\con'\of{\Delta_1,\hdots,\Delta_n}$
  where $\con'$ is an $n$-hole context
  such that $\con' \refines \con$ and $\Delta_i \refines (\lam{\var}{\tm})\con_2\of{(\lam{\vartwo}{\tmtwo})\tmthree}$ for all $1 \leq i \leq n$.
  Since $\redex$ is garbage, we know that $n = 0$.
  By \rlem{refinement_context},
  the term $\tm'$ can be written as $\con''\of{\Sigma_1,\hdots,\Sigma_m}$
  where $\con'' \refines \con_1\of{(\lam{\var}{\tm})\con_2}$
  and $\Sigma_i \refines (\lam{\vartwo}{\tmtwo})\tmthree$ for all $1 \leq i \leq m$.
  Note that $\tm' \refines \con_1$ so $\tm' \refines \con_1\of{(\lam{\var}{\tm})\con_2}$,
  as can be checked by induction on $\con_1$.
  Since the decomposition is unique, this means that $m = 0$,
  and thus $\redextwo$ is garbage.
\end{enumerate}
\end{proof}


\begin{lemma}[The leftmost coarse step has at most one residual]
\llem{leftmost_coarse_step_has_at_most_one_residual}
Let $\redex_0$ be the leftmost coarse step for $(\redseq,\tm')$,
and let $\redseqtwo \permle \redseq$.
Then $\#(\redex_0/\redseqtwo) \leq 1$.
%:%Moreover, if $\redex_0 \permle \redseqtwo$ then $\redex_0 \in \redseqtwo$.
%:%Recall that $\redex \in \redseq$ means that $\redseq$ can be written as $\redseq_1\redextwo\redseqtwo_2$
%:%such that $\redextwo$ is a residual of $\redex$, \ie $\redextwo \in \redex/\redseq_1$.
\end{lemma}
\begin{proof}
By induction on the length of $\redseqtwo$. The base case is immediate.
For the inductive step, let $\redseqtwo = \redextwo \redseqthree \permle \redseq$.
Then in particular $\redextwo \permle \redseq$.
We consider two cases, depending on whether $\redex_0 = \redextwo$.
\begin{enumerate}
\item {\bf Equal, $\redex_0 = \redextwo$.}
  Then $\redex_0/\redseqtwo = \redex_0/\redex_0\redseqthree = \emptyset$ and we are done.
  %:%Moreover, $\redex_0 \in \redex_0\redseqthree$.
\item {\bf Non-equal, $\redex_0 \neq \redextwo$.}
  First we argue that $\redex_0/\redextwo$ has exactly one residual $\redex_1 \in \redex_0/\redextwo$.
  To see this, we consider two further cases, depending on whether
  $\redextwo$ is $\tm'$-garbage or not:
  \begin{enumerate}
  \item {\bf If $\redextwo$ is not $\tm'$-garbage.}
    Then $\redextwo \permle \redseq$ and $\redextwo/\tm' = \emptyset$, so $\redextwo$ is coarse for $(\redseq,\tm')$.
    Since $\redex_0$ is the leftmost coarse step, this means that $\redex$ is to the left of $\redextwo$.
    So $\redex_0$ has exactly one residual $\redex_1 \in \redex_0/\redextwo$.
  \item {\bf If $\redextwo$ is $\tm'$-garbage.}
    Let us write the term $\tm$ as $\tm = \conof{(\lam{\var}{\tmtwo})\tmthree}$, where
    $(\lam{\var}{\tmtwo})\tmthree$ is the pattern of the redex $\redextwo$.
    By \rlem{refinement_context}
    the term $\tm'$ is of the form $\tm' = \con'[\Delta_1,\hdots,\Delta_n]$,
    where $\con'$ is a many-hole context such that $\con' \refines \con$
    and $\Delta_i \refines (\lam{\var}{\tmtwo})\tmthree$ for all $1 \leq i \leq n$.
    We know that $\redextwo$ is garbage, so $n = 0$ and $\tm' = \con'$ is actually
    a $0$-hole context (\ie a term).
    On the other hand, $\redex_0$ is coarse for $(\redseq,\tm')$, so in particular
    it is not $\tm'$-garbage.
    This means that the pattern of the redex $\redex_0$ cannot occur inside the argument $\tmthree$
    of the redex $\redextwo$.
    So $\redextwo$ does not erase or duplicate $\redex$,
    \ie $\redex_0$ has exactly one residual $\redex_1 \in \redex_0/\redextwo$.
  \end{enumerate}
  Now we have that $\redex_0/\redextwo\redseqthree = \redex_1/\redseqthree$.
  We are left to show that $\#(\redex_0/\redextwo\redseqthree) \leq 1$.
  Let us show that we may apply the \ih on $\redex_1$.
  More precisely, observe that $\redseqthree \permle \redseq/\redextwo$ since $\redextwo\redseqthree \permle \redseq$.
  To apply the \ih it suffices to show that $\redex_1$ is coarse for $(\redseq/\redextwo,\tm'/\redextwo)$.
  Indeed, we may check the two conditions
  for the definition that $\redex_1$ is coarse for $(\redseq/\redextwo,\tm'/\redextwo)$.
  \begin{enumerate}
  \item Firstly, $\redex_1 = \redex_0/\redextwo \permle \redseq/\redextwo$ holds, as a consequence of the fact that $\redex_0 \permle \redseq$.
  \item Secondly, we may check that $\redex_1$ is not $(\tm'/\redextwo)$-garbage.
        To see this, \ie that $\redex_1/(\tm'/\redextwo)$ is non-empty,
        we check that $\names(\redex_1/(\tm'/\redextwo))$ is non-empty.
        \[
          \begin{array}{rcll}
          \names(\redex_1/(\tm'/\redextwo))
          & = & \names((\redex_0/\redextwo)/(\tm'/\redextwo)) & \text{ by definition of $\redex_1$} \\
          & = & \names((\redex_0/\tm')/(\redextwo/\tm')) & \text{ by \rlem{generalized_cube_lemma}} \\
          & = & \names(\redex_0/\tm') \setminus \names(\redextwo/\tm') & \text{ by \rlem{names_after_projection_along_a_step}} \\
          & = & \names(\redex_0/\tm')
          \end{array}
        \]
        For the last step, note that $\names(\redex_0/\tm')$ and $\names(\redextwo/\tm')$ are disjoint,
        since $\redex_0 \neq \redextwo$.
  \end{enumerate}
  Applying the \ih, we obtain that $\#(\redex_0/\redextwo\redseqthree) = \#(\redex_1/\redseqthree) \leq 1$.  
\end{enumerate}
\end{proof}

\begin{lemma}
\llem{stable_non_garbage}
Let $\redex$ be a step, let $\redseq$ a coinitial derivation,
and let $\tm' \refines \src(\redex)$.
Suppose that $\redex$ is not $\tm'$-garbage,
and that $\redex / \redseq_1$ is a singleton for every prefix $\redseq_1 \permle \redseq$.
Then $\redex/\redseq$ is not $(\tm'/\redseq)$-garbage.
\end{lemma}
\begin{proof}
By induction on $\redseq$.
The base case, when $\redseq = \emptyDerivation$, is immediate
since we know that $\redex$ is not garbage.
For the inductive step, suppose that $\redseq = \redextwo\redseqtwo$.
We know that $\redex/\redextwo$ is a singleton,
so let $\redex_1 = \redex/\redextwo$.
Note that
$
  \redex_1/(\tm'/\redextwo)
  = (\redex/\redextwo)/(\tm'/\redextwo)
  = (\redex/\tm')/(\redextwo/\tm')
$ by \rlem{generalized_cube_lemma}.
We know that $\redex/\tm'$ is non-empty, because $\redex$ is not garbage.
Moreover, $\names(\redex/\tm')$ and $\names(\redextwo/\tm')$ are disjoint
since $\redex \neq \redextwo$.
So $\#\names((\redex/\tm')/(\redextwo/\tm')) = \names(\redex/\tm')$ by \rlem{names_after_projection_along_a_step}.
This means that the set $\redex_1/(\tm'/\redextwo)$ is non-empty,
so $\redex_1$ is not $(\tm'/\redextwo)$-garbage .
By \ih we obtain that $\redex_1/\redseqtwo$ is not $((\tm'/\redextwo)/\redseqtwo)$-garbage,
which means that
$(\redex/\redextwo\redseqtwo)/(\tm'/\redextwo\redseqtwo) \neq \emptyset$,
\ie that $\redex/\redextwo\redseqtwo$ is not garbage, as required.
To be able to apply the \ih, observe that if $\redseqtwo_1$ is a prefix of $\redseqtwo$,
then $\redextwo\redseqtwo_1$ is a prefix of $\redseq$,
so the fact that $\redex$ has a single residual after $\redextwo\redseqtwo_1$
implies that the step $\redex_1 = \redex/\redextwo$ has a single residual after $\redseqtwo_1$.
\end{proof}

\begin{lemma}[The projection after a sieve is garbage]
\llem{projection_after_sieving_is_garbage}
Let $\redseq : \tm \rtobeta \tmtwo$ and $\tm' \refines \tm$.
Then $\redseq/(\redseq \sieve \tm')$ is $(\tm'/(\redseq \sieve \tm'))$-garbage.
\end{lemma}
\begin{proof}
By induction on the length of $\redseq \sieve \tm'$.

If there are no coarse steps for $(\redseq,\tm')$,
then $\redseq \sieve \tm' = \emptyDerivation$.
Moreover, by \rprop{characterization_of_garbage}, a derivation with no
coarse steps is garbage. So $\redseq/(\redseq \sieve \tm') = \redseq$
is garbage, as required.

If there exists a coarse step for $(\redseq,\tm')$,
let $\redex_0$ be the leftmost such step,
and let $\redseqtwo = \redseq/\redex_0$ and $\tmtwo' = \tm'/\redex_0$.
Then:
\[
  \begin{array}{rcll}
  \redseq/(\redseq \sieve \tm')
  & = & \redseq/(\redex_0((\redseq/\redex_0) \sieve (\tm'/\redex_0)) & \text{ by definition} \\
  & = & (\redseq/\redex_0)/((\redseq/\redex_0) \sieve (\tm'/\redex_0)) \\
  & = & \redseqtwo/(\redseqtwo \sieve \tmtwo') \\
  \end{array}
\]
By \ih, $\redseqtwo/(\redseqtwo \sieve \tmtwo')$ is $(\tmtwo'/(\redseqtwo \sieve \tmtwo'))$-garbage.
That is:
\[
  \frac{
    \redseqtwo/(\redseqtwo \sieve \tmtwo')
  }{
    \tmtwo'/(\redseqtwo \sieve \tmtwo')
  } = \emptyDerivation
\]
Unfolding the definitions of $\redseqtwo$ and $\tmtwo'$ we have
that:
\[
  \frac{
    (\redseq/\redex_0)/((\redseq/\redex_0) \sieve (\tm'/\redex_0))
  }{
    (\tm'/\redex_0)/((\redseq/\redex_0) \sieve (\tm'/\redex_0))
  } = \emptyDerivation
\]
Equivalently:
\[
  \frac{
    \redseq/\redex_0((\redseq/\redex_0) \sieve (\tm'/\redex_0))
  }{
    \tm'/\redex_0((\redseq/\redex_0) \sieve (\tm'/\redex_0))
  } = \emptyDerivation
\]
Finally, by definition of sieve,
\[
  \frac{
    \redseq/(\redseq \sieve \tm')
  }{
    \tm'/(\redseq \sieve \tm')
  } = \emptyDerivation
\]
which means that $\redseq/(\redseq \sieve \tm')$ is $(\tm'/(\redseq \sieve \tm'))$-garbage,
as required.
\end{proof}

\begin{lemma}[Sieving trailing garbage]
\llem{sieving_trailing_garbage}
Let $\redseq$ and $\redseqtwo$ be composable derivations, and let $\tm' \refines \src(\redseq)$.
If $\redseqtwo$ is $(\tm'/\redseq)$-garbage, then $\redseq\redseqtwo \sieve \tm' = \redseq \sieve \tm'$.
\end{lemma}
\begin{proof}
By induction on the length of $\redseq \sieve \tm'$.
There are two cases, depending on whether there exists a coarse step for $(\redseq,\tm')$.
\begin{enumerate}
\item {\bf If there are no coarse steps for $(\redseq,\tm')$.}
  Then by \rprop{characterization_of_garbage},
  the derivation $\redseq$ is $\tm'$-garbage.
    Recall that the composition of garbage is again garbage (\rprop{properties_of_garbage}),
  so $\redseq\redseqtwo$ is $\tm'$-garbage.
  Resorting to \rprop{characterization_of_garbage}
  we obtain that $\redseq\redseqtwo \sieve \tm' = \emptyDerivation = \redseq \sieve \tm'$,
  as required.
\item {\bf If there exists a coarse step for $(\redseq,\tm')$.}
  Let $\redex_0$ be the leftmost coarse step for $(\redseq,\tm')$.
  Then since $\redex_0 \permle \redseq$ also $\redex_0 \permle \redseq\redseqtwo$,
  so $\redex_0$ is a coarse step for $(\redseq\redseqtwo,\tm')$.
  In particular, since there exists at least one coarse step for $(\redseq\redseqtwo,\tm')$,
  let $\redextwo_0$ be the leftmost such step.
  We argue that $\redex_0 = \redextwo_0$.
  We consider two cases, depending on whether $\redextwo_0$ is coarse for $(\redseq,\tm')$:
  \begin{enumerate}
  \item {\bf If $\redextwo_0$ is coarse for $(\redseq,\tm')$.}
    Then $\redex_0$ and $\redextwo_0$ are both simultaneously
    coarse for $(\redseq,\tm')$ and for $(\redseq\redseqtwo,\tm')$.
    Note that $\redex_0$ cannot be to the left of $\redextwo_0$,
    since then $\redextwo_0$ would not be the leftmost coarse step for $(\redseq\redseqtwo,\tm')$.
    Symmetrically, $\redextwo_0$ cannot be to the left of $\redex_0$,
    since then $\redex_0$ would not be the leftmost coarse step for $(\redseq,\tm')$.
    Hence $\redex_0 = \redextwo_0$ as claimed.
  \item {\bf If $\redextwo_0$ is not coarse for $(\redseq,\tm')$.}
    We argue that this case is impossible.
    Note that $\redextwo_0 \permle \redseq\redseqtwo$ but it is not the case that $\redextwo_0 \permle \redseq$,
    so $\redextwo_0/\redseq$ is not empty.
    Note also that $\redextwo_0/\redseq \permle \redseqtwo$,
    so by \rcoro{simulation_residuals_and_prefixes}
    we have that $(\redextwo_0/\redseq)/(\tm'/\redseq) \permle \redseqtwo/(\tm'/\redseq)$.
    Moreover, $\redseqtwo/(\tm'/\redseq) = \emptyDerivation$ is empty because $\redseqtwo$ is $(\tm'/\redseq)$-garbage.
    This means that $(\redextwo_0/\redseq)/(\tm'/\redseq) = \emptyDerivation$.
    Then we have the following chain of equalities:
    \[
      \begin{array}{rcll}
        \emptyset & = & \names((\redextwo_0/\redseq)/(\tm'/\redseq)) \\
                  & = & \names((\redextwo_0/\tm')/(\redseq/\tm'))               & \text{by \rcoro{permutation_equivalence_in_terms_of_names} and \rlem{generalized_cube_lemma}} \\
                  & = & \names(\redextwo_0/\tm') \setminus \names(\redseq/\tm') & \text{by \rlem{names_after_projection_along_a_step}} \\
                  & = & \names(\redextwo_0/\tm')
      \end{array}
    \]
    To justify the last step, start by noting that no residual of $\redextwo_0$ is contracted along the derivation $\redseq$.
    Indeed, if $\redextwo_0$ had a residual, then $\redseq = \redseq_1\redextwo_1\redseq_2$ where $\redextwo_1 \in \redextwo_0/\redseq$.
    But recall that $\redextwo_0$ is the leftmost coarse step for $(\redseq\redseqtwo,\tm')$ and
    $\redseq_1 \permle \redseq\redseqtwo$, so it has at most one residual (\rlem{leftmost_coarse_step_has_at_most_one_residual}).
    This means that $\redextwo_0/\redseq_1 = \redextwo_1$, so $\redextwo_0/\redseq = \emptyset$, which is a contradiction.
    Given that no residuals of $\redextwo_0$ are contracted along the derivation $\redseq$,
    we have that the sets $\names(\redextwo_0/\tm')$ and $\names(\redseq/\tm')$ are disjoint
    which justifies the last step by resorting to \rlem{disjoint_redex_derivation_have_disjoint_images}.
    
    According to the chain of equalities above, we have that $\names(\redextwo_0/\tm') = \emptyset$.
    This means that $\redextwo_0$ is $\tm'$-garbage.
    This in turn contradicts the fact that $\redextwo_0$ is coarse for $(\redseq\redseqtwo,\tm')$,
    confirming that this case is impossible.
  \end{enumerate}
  We have just claimed that $\redex_0 = \redextwo_0$.
  Then:
  \[
    \begin{array}{rcll}
    \redseq\redseqtwo \sieve \tm'
    & = & \redex_0((\redseq\redseqtwo/\redex_0) \sieve (\tm'/\redex_0)) & \text{by definition of sieving} \\
    & = & \redex_0(((\redseq/\redex_0)(\redseqtwo/(\redex_0/\redseq))) \sieve (\tm'/\redex_0)) \\
    & = & \redex_0((\redseq/\redex_0) \sieve (\tm'/\redex_0)) & \text{by \ih since $\redseqtwo/(\redex_0/\redseq)$ is garbage by \rprop{properties_of_garbage}} \\
    & = & \redseq \sieve \tm' & \text{by definition of sieving}
    \end{array}
  \]
  which concludes the proof.
\end{enumerate}
\end{proof}


