What we want to do know is give the operational semantics for this calculus.
The idea is straightforward: in order apply a lambda abstraction to a list of arguments we just
replace each occurrence of the variable bounded by the lambda with the corresponding argument.
As the term types and is correct we know that we can do it successfully.

As substitution will be \emph{type-directed} we need to define notation for the type
of ocurrences of a free variable. If $\tm$ is typable,
$\varlabel{\var}{\tm}$
stands for the multiset
of types of the free occurrences of $\var$ in $\tm$.
If $\tm_1,\hdots,\tm_n$ are typable,
$\tmlabel{[\tm_1,\hdots,\tm_n]}$ stands for the multiset
of types of $\tm_1,\hdots,\tm_n$.
For example,
$\varlabel{\var}{\var^{[\alpha^1] \tolab{2} \beta^3}[\var^{\alpha^1}]} =
\tmlabel{[\vartwo^{\alpha^1},\varthree^{[\alpha^1] \tolab{2} \beta^3}]} = [[\alpha^1] \tolab{2} \beta^3,\alpha^1]$.
To perform a substitution $\subs{\tm}{\var}{[\tmtwo_1,\hdots,\tmtwo_n]}$
we will require that $\varlabel{\var}{\tm} = \tmlabel{[\tmtwo_1,\hdots,\tmtwo_n]}$.


Now we will define exactly what we mean by \emph{substituting}.
\begin{definition}[Substitution]
\ldef{substitution}
Let $\tm$ and $\tmtwo_1,\hdots,\tmtwo_n$ be correct terms such that $\varlabel{\var}{\tm} = \tmlabel{[\tmtwo_1,\hdots,\tmtwo_n]}$.
The capture-avoiding substitution of $\var$ in $\tm$ by $\ls{\tmtwo} = [\tmtwo_1,\hdots,\tmtwo_n]$
is denoted by $\subs{\tm}{\var}{\ls{\tmtwo}}$ and defined as follows:
\[
  \begin{array}{rcll}
    \subs{\var^\typ}{\var}{[\tmtwo]} & \eqdef & \tmtwo
  \\
    \subs{\vartwo^\typ}{\var}{[]} & \eqdef & \vartwo^\typ
    & \text{ if $\var \neq \vartwo$}
  \\
    \subs{(\lamp{\lab}{\vartwo}{\tmthree})}{\var}{\ls{\tmtwo}} & \eqdef &  \lamp{\lab}{\vartwo}{ \subs{\tmthree}{\var}{\ls{\tmtwo}} }
    & \text{ if $\var \neq \vartwo$ and $\vartwo \not\in \fv{\ls{\tmtwo}}$}
  \\
    \subs{\tmthree_0[\tmthree_j]_{j=1}^{m}}{\var}{\ls{\tmtwo}} & \eqdef &
    \subs{\tmthree_0}{\var}{\ls{\tmtwo}_0}[\subs{\tmthree_j}{\var}{\ls{\tmtwo}_j}]_{j=1}^{m}
  \end{array}
\]
In the last case, $(\ls{\tmtwo}_0, \hdots, \ls{\tmtwo}_m)$
is a partition of $\ls{\tmtwo}$
such that $\varlabel{\var}{\tmthree_j} = \tmlabel{\ls{\tmtwo}_j}$ for all $j \in {0,\hdots,m}$.
\end{definition}

For example,
\[\subs{(\var^{[\alpha^1] \tolab{2} \beta^3}[\var^{\alpha^1}])}{\var}{[\vartwo^{[\alpha^1] \tolab{2} \beta^3},\varthree^{\alpha^1}]}
= \vartwo^{[\alpha^1] \tolab{2} \beta^3}\varthree^{\alpha^1}\]
and
\[\subs{(\var^{[\alpha^1] \tolab{2} \beta^3}[\var^{\alpha^1}])}{\var}{[\vartwo^{\alpha^1},\varthree^{[\alpha^1] \tolab{2} \beta^3}]}
= \varthree^{[\alpha^1] \tolab{2} \beta^3}\vartwo^{\alpha^1}.\]


\begin{remark}
Substitution is {\em type-directed}: arguments $[\tmtwo_1,\hdots,\tmtwo_n]$
are propagated throughout the term
so that $\tmtwo_i$ reaches the free occurrence of $\var$
that has the same type as $\tmtwo_i$.
There exists one such occurrence for each $i \in \set{1, \hdots, n}$ because
$\varlabel{\var}{\tm} = \tmlabel{[\tmtwo_1,\hdots,\tmtwo_n]}$.
Moreover, the fact that $\tm$ is correct ensures that such occurrence is unique,
since $\varlabel{\var}{\tm}$ is sequential.
Hence there is essentially a unique way to split $\ls{\tmtwo}$
into $(\ls{\tmtwo}_0, \ls{\tmtwo}_1, \hdots, \ls{\tmtwo}_n)$.
More precisely, if $(\ls{\tmtwo}_0, \ls{\tmtwo}_1, \hdots, \ls{\tmtwo}_n)$
and $(\ls{\tmthree}_0, \ls{\tmthree}_1, \hdots, \ls{\tmthree}_n)$
are two partitions of $\ls{\tmtwo}$ with the stated property,
then $\ls{\tmtwo}_i$ is a permutation of $\ls{\tmthree}_i$ for all $i \in \set{0,\hdots,n}$.
It is easy to check by induction on $\tm$
that the value of $\subs{\tm}{\var}{\ls{\tmtwo}}$ does
not depend on this choice.
The following lemma formalizes what we just said.
\end{remark}


\begin{lemma}[Substitution is well-defined]
\llem{distrSubstitutionWellDefined}
If $\tctx \oplus \var : \ls{\typtwo} \vdash \tm : \typ$
and $\ls{\tctxtwo} \vdash \ls{\tmtwo} : \ls{\typtwo}$
are derivable,
then $\tctx + \ls{\tctxtwo} \vdash \subs{\tm}{\var}{\ls{\tmtwo}} : \typ$
is derivable.
\end{lemma}
\begin{proof}
By induction on $\tm$, straightforward using the ideas in the last remark.
\end{proof}

\begin{lemma}[Parameter/argument correspondence]
\llem{parameter_argument_correspondence}
If $(\lamp{\lab}{\var}{\tm})[\tmtwo_1,\hdots,\tmtwo_n]$ is a correct term then
there are exactly $n$ free occurrences of $\var$ in $\tm$.
More precisely, $\tm$ can be written as $\conhat\of{\var^{\typ_1},\hdots,\var^{\typ_n}}$
where $\conhat$ is an $n$-hole context
and type $\typ_i$ is the type of $\tmtwo_i$ for all $1 \leq i \leq n$.
\end{lemma}
\begin{proof}
Consider the (unique) type derivation for $(\lamp{\lab}{\var}{\tm})[\tmtwo_1,\hdots,\tmtwo_n]$.
The last rule must be:
\[
  \indrule{\toE}{
    \indrule{\toI}{
      \tctx,\var:[\typtwo_1,\hdots,\typtwo_n] \vdash \tm : \typ
    }{
      \tctx \vdash \lamp{\lab}{\var}{\tm} : [\typtwo_1,\hdots,\typtwo_n] \tolab{\lab} \typ
    }
    \HS
    \tctxtwo_i \vdash \tmtwo_i : \typtwo_i \HS\text{ for all $i = 1..n$ }
  }{
    \tctx +_{i=1}^{n} \tctxtwo_i \vdash (\lamp{\lab}{\var}{\tm})[\tmtwo_1,\hdots,\tmtwo_n] : \typ
  }
\]
It is then easy to conclude by resorting to Linearity~\rlem{linearity}.
\end{proof}

Substitution as we presented it can be hard to work with in some environments,
as we have to track how substitution terms get distributed over the term they are
being substituted in.

The following alternative definition of deals with this by
taking advantage of the fact that substitution is type-directed: it
takes all terms with itself and doesn't distribute it in the application case, but rather
``picks'' the correct term to substitute in the base case.

\begin{definition}[Alternative definition of substitution]
Let $\tctx,\var : [\typ_1,\hdots,\typ_n] \vdash \tm : \typtwo$
and let $\tctxtwo_i \vdash \tmtwo_i : \typthree_i$ for each $i \in \set{1,\hdots,m}$
in such a way that $\varlabel{\var}{\tm} \subseteq \tmlabel{[\tmtwo_1,\hdots,\tmtwo_m]}$
and $[\typtwo_1,\hdots,\typtwo_n]$ is sequential.
Let us write $\ls{\tmtwo}$ for $[\tmtwo_1,\hdots,\tmtwo_m]$.
Then an alternative definition for substitution $\tm\sub{\var}{\ls{\tmtwo}}$
may be defined as follows:
\[
  \begin{array}{rcll}
    \var^{\typ}\sub{\var}{\ls{\tmtwo}}                 & \eqdef & \tmtwo_i & \text{ where $i$ is the unique index such that} \\
                                                                         &&& \text{ $\tmlabel{\tmtwo_i}$ is the external label of $\typ$}\\
    \vartwo^{\typ}\sub{\var}{\ls{\tmtwo}}              & \eqdef & \vartwo^\typ & \text{ if $\var \neq \vartwo$} \\
    (\lamp{\lab}{\vartwo}{\tm})\sub{\var}{\ls{\tmtwo}} & \eqdef & \lamp{\lab}{\vartwo}{\tm\sub{\var}{\ls{\tmtwo}}} & \text{ if $\var \neq \vartwo$ and there is no capture}\\
    (\tm[\tmthree_i]_{i=1}^{k})\sub{\var}{\ls{\tmtwo}} & \eqdef & \tm\sub{\var}{\ls{\tmtwo}}[\tmthree_i\sub{\var}{\ls{\tmtwo}}]_{i=1}^{k}.
  \end{array}
\]
Moreover,
$[\tm_1,\hdots,\tm_n]\sub{\var}{\ls{\tmtwo}}$
stands for $[\tm_1\sub{\var}{\ls{\tmtwo}},\hdots,\tm_n\sub{\var}{\ls{\tmtwo}}]$,
whenever each substitution $\tm_i\sub{\var}{\ls{\tmtwo}}$ is well-defined.
\end{definition}

\begin{lemma}
$\subs{\tm}{\var}{\ls{\tmtwo}} = \tm\sub{\var}{\ls{\tmtwo}'}$
whenever $\varlabel{\var}{\tm} = \tmlabel{\ls{\tmtwo}}$ and $\ls{\tmtwo}$ is a sublist of $\ls{\tmtwo}'$.
\end{lemma}
\begin{proof}
By induction on $\tm$.
\begin{enumerate}
\item {\bf Variable (same), $\tm = \var^\typ$.}
  First $\subs{\var^\typ}{\var}{[\tmtwo]} = \tmtwo$.
  Also,$\var^\typ\sub{\var}{\ls{\tmtwo}'} = \tmtwo$,
  because $\tmtwo$ is in $\ls{\tmtwo}'$ and it must be the only one for which the external label is the external label of $\typ$.
\item {\bf Variable (different), $\tm = \vartwo$.}
  On the left hand side, $\subs{\vartwo^\typtwo}{\var}{\emptylset} = \vartwo^\typtwo$.
  On the right hand side, $\vartwo^\typtwo\sub{\var}{\ls{\tmtwo}'} = \vartwo^\typtwo$.
\item {\bf Abstraction, $\tm = \lamp{\lab}{\vartwo}{\tmthree}$.}
  On the left, $\subs{(\lamp{\lab}{\vartwo}{\tmthree})}{\var}{\ls{\tmtwo}} = \lamp{\lab}{\vartwo}{\subs{\tmthree}{\var}{\ls{\tmtwo}}}$.
  On the right $(\lamp{\lab}{\vartwo}{\tmthree})\sub{\var}{\ls{\tmtwo}'} = \lamp{\lab}{\vartwo}{\tmthree\sub{\var}{\ls{\tmtwo}}'}$.
  The right-hand side of both equantions are the same by inductive hypothesis.
\item {\bf Application, $\tm = \tmfour [\tmthree_1, \hdots, \tmthree_n]$.}
  On the left side we have that
    $\subs{(\tmfour [\tmthree_i]_{i=1}^n)}{\var}{\ls{\tmtwo}} =
      \subs{\tmfour}{\var}{\ls{\tmtwo}_0} [\subs{\tmthree_i}{\var}{\ls{\tmtwo}_i}]_{i=1}^n$, where
      $\ls{\tmtwo}_0 +_{i=1}^{n} \ls{\tmtwo}_i$ is a permutation of $\ls{\tmtwo}$,
      $\varlabel{\var}{\tm} = \tmlabel{\ls{\tmtwo}_0}$ and
      $\varlabel{\var}{\tmthree_i} = \tmlabel{\ls{\tmtwo}_i}$ for all $i = \set{1,...,n}$.
  On the other hand, $((\tmfour [\tmthree_i]_{i=1}^n))\sub{\var}{\ls{\tmtwo}'} =
      \tmfour\sub{\var}{\ls{\tmtwo}'} [\tmthree_i\sub{\var}{\ls{\tmtwo}'}]_{i=1}^n$.

  Note that for every $i \in \set{0, ..., n}$, $\ls{\tmtwo}_i$ is a sublist of $\ls{\tmtwo}$.
  Then, by inductive hypothesis, $\subs{\tmfour}{\var}{\ls{\tmtwo}_0} = \tmfour\sub{\var}{\ls{\tmtwo}'}$
  and $\subs{\tmthree_i}{\var}{\ls{\tmtwo}_i} = \tmthree_i\sub{\var}{\ls{\tmtwo}'}$, which is what we wanted.
\end{enumerate}
\end{proof}

\begin{lemma}[Substitution lemma for the alternative notion of substitution]
\llem{substitution_lemma_alt}
The following variant of the substitution lemma holds
when $\var \not\in \fv{\ls{\tmthree}}$:
\[
 \tm\sub{\var}{\ls{\tmtwo}}\sub{\vartwo}{\ls{\tmthree}} =
 \tm\sub{\vartwo}{\ls{\tmthree}}\sub{\var}{\ls{\tmtwo}\sub{\vartwo}{\ls{\tmthree}}}
\]
This equation is intended to mean, in particular, that one side is well-defined
if and only if the other side is well-defined.
\end{lemma}
\begin{proof}
Straightforward by induction on $\tm$.
\end{proof}

\begin{remark}
$\conof{\tm}\sub{\var}{\ls{\tmtwo}} = \con\sub{\var}{\ls{\tmtwo}}\of{\tm\sub{\var}{\ls{\tmtwo}}}$.
\end{remark}

\bigskip


\begin{definition}[The $\lambdadist$-calculus]
The \defn{$\lambdadist$-calculus} (distributive $\lambda$-calculus)
is given by the set of correct typable terms $\termsdist$.
For each label $\lab \in \labelset$, we define a reduction relation $\todistl{\lab}\ \subseteq \termsdist \times \termsdist$
as follows:
\[
  \conof{(\lamp{\lab}{\var}{\tm})\vec{\tmtwo}}
  \ \todistl{\lab}\ %
  \conof{\subs{\tm}{\var}{\vec{\tmtwo}}}
\]
where $\con$ stands for a \defn{context}.
The binary relation $\todist$ is the union of all the $\todistl{\lab}$:
\[
  \todist \ \eqdef\ \bigcup_{\lab \in \labelset}\todistl{\lab}
\]
We sometimes drop the subscript for $\todist$, writing just $\to$, when clear from the context.
The set of contexts is given by the grammar:
\[
  \con ::= \conbase \mid \lamp{\lab}{\var}{\con} \mid \con\,\vec{\tm} \mid \tm[\tmtwo_1,\hdots,\tmtwo_{i-1},\con,\tmtwo_{i+1},\hdots,\tmtwo_n]
\]
Contexts can be thought as terms with a single free occurrence of a distinguished variable $\conbase$.
The notation $\conof{\tm}$ stands for the capturing substitution of the occurrence of $\conbase$ in $\con$ by $\tm$.

In general an $n$-hole context is a term $\con$ with exactly $n \geq 0$ free occurrences of the distinguished variable $\conbase$.
If $\con$ is an $n$-hole context, $\conof{\tm_1,\hdots,\tm_n}$ stands for the term
that results from performing the substitution of the $i$-th occurrence of $\conbase_i$ (from left to right)
by $\tm_i$.
If $\con$ is an $n$-hole contexts for some $n$, we say that it is a many-hole context.
\end{definition}
