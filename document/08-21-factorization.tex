
We need a few auxiliary lemmas:

\begin{lemma}[Garbage-free/garbage decomposition]
\llem{garbage_free_garbage_decomposition}
Let $\redseq : \tm \rtobeta \tmtwo$ and $\tm' \refines \tm$.
Then there exist $\redseq_1,\redseq_2$ such that:
\begin{enumerate}
\item $\redseq \permeq \redseq_1\redseq_2$
\item $\redseq_1$ is $\tm'$-garbage-free,
\item $\redseq_2$ is $\tm'$-garbage.
\end{enumerate}
Moreover, $\redseq_1$ and $\redseq_2$ are unique modulo permutation equivalence,
and we have that
$\redseq_1 \permeq \redseq \sieve \tm'$ and $\redseq_2 \permeq \redseq/(\redseq \sieve \tm')$.
\end{lemma}
\begin{proof}
Let us prove that said decomposition exists and that it is unique:
\begin{itemize}
\item {\bf Existence.}
  Take $\redseq_1 := \redseq \sieve \tm'$ and $\redseq_2 := \redseq/(\redseq \sieve \tm')$.
  Then we may check the three conditions in the statement:
  \begin{enumerate}
  \item
    Recall that $\redseq \sieve \tm' \permle \redseq$ holds by \rlem{sieve_is_prefix}, so:
    \[
      \begin{array}{rcll}
      \redseq_1\redseq_2
      & = & (\redseq \sieve \tm')(\redseq/(\redseq \sieve \tm')) \\
      & \permeq & \redseq((\redseq \sieve \tm')/\redseq) & \text{ since $A(B/A) \permeq B(A/B)$ holds in general} \\
      & \permeq & \redseq                                & \text{ since $\redseq \sieve \tm' \permle \redseq$}
      \end{array}
    \]
    as required.
  \item
    The derivation $\redseq_1 = \redseq \sieve \tm'$
    is $\tm'$-garbage-free.
    This is as an immediate consequence of \rprop{characterization_of_garbage_free_derivations},
    namely the result of sieving is always garbage-free.
  \item
    The derivation $\redseq_2 = \redseq/(\redseq \sieve \tm')$ is garbage.
    This is an immediate consequence of \rlem{projection_after_sieving_is_garbage},
    namely projection after a sieve is always garbage.
  \end{enumerate}
\item {\bf Uniqueness, modulo permutation equivalence.}
  Let $\redseq \permeq \redseqtwo_1,\redseqtwo_2$
  where $\redseqtwo_1$ is $\tm'$-garbage-free,
  and $\redseqtwo_2$ is $\tm'$-garbage.
  Then we argue that $\redseqtwo_1 \permeq \redseq_1$ and $\redseqtwo_2 \permeq \redseq_2$.

  Since $\redseq_1\redseq_2 \permeq \redseq \permeq \redseqtwo_1\redseqtwo_2$
  and sieving is compatible with permutation equivalence (\rlem{sieving_is_compatible_with_permutation_equivalence})
  we have that $\redseq_1\redseq_2 \sieve \tm' \permeq \redseqtwo_1\redseqtwo_2 \sieve \tm'$.
  By \rlem{sieving_trailing_garbage}, we know that trailing garbage does not affect sieving,
  hence $\redseq_1 \sieve \tm' \permeq \redseqtwo_1 \sieve \tm'$.
  Moreover, $\redseq_1$ and $\redseqtwo_1$ are garbage-free, which
  by \rprop{characterization_of_garbage_free_derivations} means
  that $\redseq_1 \permeq \redseqtwo_1$.
  This means that $\redseq_1$ is unique, modulo permutation equivalence.
  Finally, since $\redseq_1\redseq_2 \permeq \redseqtwo_1\redseqtwo_2$ and $\redseq_1 \permeq \redseqtwo_1$,
  we have that $\redseq_1\redseq_2/\redseq_1 \permeq \redseqtwo_1\redseqtwo_2/\redseqtwo_1$,
  that is $\redseq_2 \permeq \redseqtwo_2$.
  This means that $\redseq_2$ is unique, modulo permutation equivalence.
\end{itemize}
\end{proof}

\begin{lemma}
\llem{lorf_ignores_garbage}
Let $\tm' \refines \src(\redseq) = \src(\redseqtwo)$.
Then $\cls{(\redseq \sqcup \redseqtwo) \sieve \tm'} = \cls{\redseq \sieve \tm'} \lorF \cls{\redseqtwo \sieve \tm'}$.
\end{lemma}
\begin{proof}
Let:
\[
  \begin{array}{rcl}
    \alpha & := & \redseq/(\redseq \sieve \tm') \\
    \beta  & := & \redseqtwo/(\redseqtwo \sieve \tm') \\
    \gamma & := & (\alpha/((\redseqtwo \sieve \tm')/(\redseq \sieve \tm'))) \sqcup (\beta/((\redseq \sieve \tm')/(\redseqtwo \sieve \tm'))) \\
  \end{array}
\]
Note that $\alpha$ and $\beta$ are garbage by \rlem{projection_after_sieving_is_garbage}
and hence $\gamma$ is also garbage, as a consequence of
the facts that the join of garbage is garbage
and that the projection of garbage is garbage~(\rprop{properties_of_garbage}).
Remark that, in general, $AB \sqcup CD \permeq (A \sqcup C)(B/(C/A) \sqcup D/(A/C))$.
Then the statement of this lemma is a consequence of the following chain of equalities:
\[
  \begin{array}{rcll}
  \cls{(\redseq \sqcup \redseqtwo) \sieve \tm'}
  & = &
  \cls{((\redseq \sieve \tm')\alpha \sqcup (\redseqtwo \sieve \tm')\beta) \sieve \tm'}
  & \text{by \rlem{garbage_free_garbage_decomposition}}
  \\
  & = &
  \cls{((\redseq \sieve \tm') \sqcup (\redseqtwo \sieve \tm'))\gamma \sieve \tm'}
  & \text{by the previous remark}
  \\
  & = &
  \cls{((\redseq \sieve \tm') \sqcup (\redseqtwo \sieve \tm')) \sieve \tm'}
  & \text{by~\rlem{sieving_trailing_garbage}}
  \\
  & = & \cls{\redseq \sieve \tm'} \lorF \cls{\redseqtwo \sieve \tm'}
  & \text{by definition of $\lorF$}
  \end{array}
\]
\end{proof}

To prove \rthm{factorization_ulb_derivations},
let us check that $\grothy{\ulbF}{\ulbG}$ is well-defined,
that it is an upper semilattice,
and finally that $\ulbDerivLam{\tm} \simeq \grothy{\ulbF}{\ulbG}$
are isomorphic as upper semilattices.

\begin{proof}
\begin{enumerate}
\item {\bf The Grothendieck construction $\grothy{\ulbF}{\ulbG}$ is well-defined.}
  This amounts to checking that $\ulbG$ is indeed a lax 2-functor:
  \begin{enumerate}
  \item {\bf The mapping $\ulbG$ is well-defined on objects.}
    Note that $\ulbG(\cls{\redseq}) = \ulbGarbage{\tm'/\redseq}{\tgt(\redseq)}$, which is a poset.
    Moreover, the choice of the representative $\redseq$ of the equivalence class $\cls{\redseq}$
    does not matter,
    since if $\redseq$ and $\redseqtwo$ are permutation equivalent derivations,
    then $\tm'/\redseq = \tm'/\redseqtwo$ (by \rprop{compatibility_of_simulation_residuals_and_permutation_equivalence})
    and $\tgt(\redseq) = \tgt(\redseqtwo)$.
  \item {\bf The mapping $\ulbG$ is well-defined on morphisms.}
    Let us check that given $\cls{\redseq},\cls{\redseqtwo} \in \ulbF$
    such that $\cls{\redseq} \leqF \cls{\redseqtwo}$
    then
    $
      \ulbG(\ptF{\cls{\redseqtwo}}{\cls{\redseq}}) : \ulbG(\cls{\redseq}) \to \ulbG(\cls{\redseqtwo})
    $ is a morphism of posets, \ie a monotonic function.

    First, we can see that it is well-defined,
    since if $\cls{\alpha} \in \ulbG([\redseq])$
    then the image $\ulbG(\ptF{\cls{\redseqtwo}}{\cls{\redseq})(\cls{\alpha}}) = \cls{\redseq\alpha/\redseqtwo}$
    is an element of $\ulbG([\redseqtwo])$,
    since
    $
      \redseq\alpha/\redseqtwo = (\redseq/\redseqtwo)(\alpha/(\redseqtwo/\redseq))
    $
    is garbage, as it is the composition of garbage derivations~(\rprop{properties_of_garbage}):
    the derivation $\redseq/\redseqtwo$ is garbage since $\cls{\redseq} \leqF \cls{\redseqtwo}$ (by definition),
    and the derivation $\alpha/(\redseqtwo/\redseq)$ is garbage since $\alpha$ is garbage~(\rprop{properties_of_garbage}).
    Moreover, the choice of representative does not matter,
    since if $\redseq_1 \permeq \redseq_2$ and $\redseqtwo_1 \permeq \redseqtwo_2$ and $\alpha_1 \permeq \alpha_2$
    then $\redseq_1\alpha_1/\redseqtwo_1 \permeq \redseq_2\alpha_2/\redseqtwo_2$.

    We are left to verify that $\ulbG(\ptF{\cls{\redseqtwo}}{\cls{\redseq}})$ is monotonic.
    Let $\cls{\alpha}, \cls{\beta} \in \ulbG([\redseq])$
    such that $\cls{\alpha} \permle \cls{\beta}$, and let us show that
    $\ulbG(\ptF{\cls{\redseqtwo}}{\cls{\redseq}})(\cls{\alpha}) \permle \ulbG(\ptF{[\redseqtwo]}{[\redseq]})(\cls{\beta})$.
    Indeed, $\alpha \permle \beta$, so
    $
      \redseq\alpha/\redseqtwo
      = (\redseq/\redseqtwo)(\alpha/(\redseqtwo/\redseq))
      \permle (\redseq/\redseqtwo)(\beta/(\redseqtwo/\redseq))
      = \redseq\beta/\redseqtwo
    $.
  \item {\bf Identity.}
    Let $\cls{\redseq} \in \ulbF$.
    Let us check that $\ulbG(\ptF{\cls{\redseq}}{\cls{\redseq}}) = \identity_{\ulbG(\cls{\redseq})}$
    is the identity morphism.
    Indeed, if $\cls{\alpha} \in \ulbG(\cls{\redseq})$,
    then $\ulbG(\ptF{\cls{\redseq}}{\cls{\redseq}})(\cls{\alpha}) = \cls{\redseq\alpha/\redseq} = \cls{\alpha}$.
  \item {\bf Composition.}
    Let $\cls{\redseq}, \cls{\redseqtwo}, \cls{\redseqthree} \in \ulbF$
    such that $\cls{\redseq} \leqF \cls{\redseqtwo} \leqF \cls{\redseqthree}$.
    Let us check that there is a 2-cell
    $\ulbG((\ptF{\cls{\redseqthree}}{\cls{\redseqtwo}}) \circ (\ptF{\cls{\redseqtwo}}{\cls{\redseq}})) \permle
     \ulbG(\ptF{\cls{\redseqthree}}{\cls{\redseqtwo}}) \circ \ulbG(\ptF{\cls{\redseqtwo}}{\cls{\redseq}})$.
    Note that
    $(\ptF{\cls{\redseqthree}}{\cls{\redseqtwo}}) \circ (\ptF{\cls{\redseqtwo}}{\cls{\redseq}}) : \cls{\redseq} \leqF \cls{\redseqthree}$
    is a morphism in the upper semilattice $\ulbF$ seen as a category.
    Moreover, since it is a semilattice, there a unique morphism $\cls{\redseq} \leqF \cls{\redseqthree}$,
    namely $\ptF{\cls{\redseqthree}}{\cls{\redseq}}$, so we have that
    $
      (\ptF{\cls{\redseqthree}}{\cls{\redseqtwo}}) \circ (\ptF{\cls{\redseqtwo}}{\cls{\redseq}}) = \ptF{\cls{\redseqthree}}{\cls{\redseq}}
    $.
    Now if $\cls{\alpha} \in \ulbG{\cls{\redseq}}$, then:
    \[
      \begin{array}{rcll}
      \ulbG((\ptF{\cls{\redseqthree}}{\cls{\redseqtwo}}) \circ (\ptF{\cls{\redseqtwo}}{\cls{\redseq}}))(\cls{\alpha})
      & = & \ulbG(\ptF{\cls{\redseqthree}}{\cls{\redseq}})(\cls{\alpha}) \\
      & = & \redseq\alpha/\redseqthree \\
      & \permle & (\redseq\alpha/\redseqthree)((\redseqtwo/\redseq\alpha)/(\redseqthree/\redseq\alpha)) \\
      & = & \redseq\alpha(\redseqtwo/\redseq\alpha)/\redseqthree \\
      & \permeq & \redseqtwo(\redseq\alpha/\redseqtwo)/\redseqthree \HS\text{ since $A(B/A) \permeq B(A/B)$} \\
      & = & \ulbG(\ptF{\cls{\redseqthree}}{\cls{\redseqtwo}})(\redseq\alpha/\redseqtwo) \\
      & = & (\ulbG(\ptF{\cls{\redseqthree}}{\cls{\redseqtwo}}) \circ \ulbG(\ptF{\cls{\redseqtwo}}{\cls{\redseq}}))(\cls{\alpha})
      \end{array}
    \]
  \end{enumerate}
  so
  $\ulbG((\ptF{\cls{\redseqthree}}{\cls{\redseqtwo}}) \circ (\ptF{\cls{\redseqtwo}}{\cls{\redseq}}))
  \permle
  \ulbG(\ptF{\cls{\redseqthree}}{\cls{\redseqtwo}}) \circ \ulbG(\ptF{\cls{\redseqtwo}}{\cls{\redseq}})$
  as required.

\item {\bf The Grothendieck construction $\grothy{\ulbF}{\ulbG}$ is an upper semilattice.}
  \begin{enumerate}
  \item {\bf Partial order.}
    Recall that $\grothy{\ulbF}{\ulbG}$ is always poset
    with the order given by $(a,b) \leq (a',b')$ if and only if
    $a \leqF a'$ and $\ulbG(\ptF{a'}{a})(b) \permle b'$.
  \item {\bf Bottom element.}
    We argue that $(\bot_{\ulbF},\bot_{\ulbG(\bot_\ulbF)})$ is the bottom
    element. Let $(\cls{\redseq},\cls{\redseqtwo}) \in \grothy{\ulbF}{\ulbG}$.
    Then clearly $\bot_{\ulbF} \leqF \cls{\redseq}$. Moreover,
    $
     \ulbG(\ptF{[\redseq]}{[\bot_{\ulbF}]})(\bot_\ulbG) =
     [\emptyDerivation/\redseq] = [\emptyDerivation] \permle \cls{\redseqtwo}
    $.
  \item {\bf Join.}
    Let us show that $(a, b) \lor (a',b') = (a \lorF a', \ulbG(\ptF{(a \lorF a')}{a})(b) \sqcup \ulbG(\ptF{(a \lorF a')}{a'})(b'))$.
    is the least upper bound of $\set{(a, b), (a',b')}$.
    \begin{enumerate}
    \item {\bf Upper bound.}
      Let us show that $(a, b) \leq (a, b) \lor (a',b')$.
      Recall that $(a, b) \lor (a', b') = (a \lorF a, \ulbG(\ptF{(a \lor a')}{a})(b) \sqcup \ulbG(\ptF{(a \lorF a')}{a'})(b'))$.
      First, we have $a \leqF a \lorF a'$.
      Moreover, $\ulbG(\ptF{(a \lorF a')}{a})(b) \permle \ulbG(\ptF{(a \lor a')}{a})(b) \sqcup \ulbG(\ptF{(a \lorF a')}{a'})(b')$, as required.
    \item {\bf Least upper bound.}
      Let $(a,b) = (\cls{\redseq},\cls{\redseqtwo})$
      and $(a',b') = (\cls{\redseq'},\cls{\redseqtwo'})$.
      Moreover, let $(\cls{\redseq''},\cls{\redseqtwo''})$ be an upper bound,
      \ie an element
      such that $(\cls{\redseq},\cls{\redseqtwo}) \leq (\cls{\redseq''},\cls{\redseqtwo''})$
      and $(\cls{\redseq'},\cls{\redseqtwo'}) \leq (\cls{\redseq''},\cls{\redseqtwo''})$.
      Let us show that
      $(\cls{\redseq},\cls{\redseqtwo}) \lor (\cls{\redseq'},\cls{\redseqtwo'}) \leq (\cls{\redseq''},\cls{\redseqtwo''})$.
      First note that $\cls{\redseq} \leqF \cls{\redseq''}$ and $\cls{\redseq'} \leqF \cls{\redseq''}$
      so $\cls{\redseq} \lorF \cls{\redseq'} \leqF \cls{\redseq''}$.\\
      Moreover, we know that:
      \[
        \cls{\redseq\redseqtwo/\redseq''} = \ulbG(\pt{\cls{\redseq''}}{\cls{\redseq}})(\cls{\redseqtwo}) \permle \cls{\redseqtwo''}
        \ \text{ and }\ %
        \cls{\redseq'\redseqtwo'/\redseq''} = \ulbG(\pt{\cls{\redseq''}}{\cls{\redseq'}})(\cls{\redseqtwo'}) \permle \cls{\redseqtwo''}
      \]
      Let $\alpha := (\redseq \sqcup \redseq') \sieve \tm'$.
      First we claim that $\alpha \permle \redseq\redseqtwo \sqcup \redseq'\redseqtwo'$.
      Indeed, $\alpha = (\redseq \sqcup \redseq') \sieve \tm' \permle \redseq \sqcup \redseq'$
      by \rlem{sieve_is_prefix}, and it is easy to check that
      $\redseq \sqcup \redseq' \permle \redseq\redseqtwo \sqcup \redseq'\redseqtwo'$.
      What we have to check is the following inequality:
      \[
        \ulbG(\pt{\cls{\redseq''}}{\cls{\alpha}})(
          \ulbG(\pt{\cls{\alpha}}{\cls{\redseq}})(\cls{\redseqtwo})
          \sqcup
          \ulbG(\pt{\cls{\alpha}}{\cls{\redseq'}})(\cls{\redseqtwo'})
        )
        \permle
        \cls{\redseqtwo''}
      \]
      Indeed:
      \[
        \begin{array}{rcll}
        &&
        \ulbG(\pt{\cls{\redseq''}}{\cls{\alpha}})(
          \ulbG(\pt{\cls{\alpha}}{\cls{\redseq}})(\cls{\redseqtwo})
          \sqcup
          \ulbG(\pt{\cls{\alpha}}{\cls{\redseq'}})(\cls{\redseqtwo'})
        ) \\
        & = &
        \cls{\alpha((\redseq\redseqtwo/\alpha) \sqcup (\redseq'\redseqtwo'/\alpha))/\redseq''}
        \\
        & = &
        \cls{\alpha((\redseq\redseqtwo \sqcup \redseq'\redseqtwo')/\alpha)/\redseq''}
        &\hspace{-3cm} \text{since $A/C \sqcup B/C \permeq (A \sqcup B)/C$}
        \\
        & = &
        \cls{(\redseq\redseqtwo \sqcup \redseq'\redseqtwo')(\alpha/(\redseq\redseqtwo \sqcup \redseq'\redseqtwo'))/\redseq''}
        &\hspace{-3cm} \text{since $A(B/A) \permeq B(A/B)$}
        \\
        & = &
        \cls{(\redseq\redseqtwo \sqcup \redseq'\redseqtwo')/\redseq''}
        &\hspace{-3cm} \text{since $\alpha \permle \redseq\redseqtwo \sqcup \redseq'\redseqtwo'$}
        \\
        & = &
        \cls{\redseq\redseqtwo/\redseq'' \sqcup \redseq'\redseqtwo'/\redseq''}
        &\hspace{-3cm} \text{since $A/C \sqcup B/C \permeq (A \sqcup B)/C$}
        \\
        & \permle &
        \cls{\redseqtwo''}
        &\hspace{-3cm} \text{since $\cls{\redseq\redseqtwo/\redseq''} \permle \cls{\redseqtwo''}$, $\cls{\redseq'\redseqtwo'/\redseq''} \permle \cls{\redseqtwo''}$}
        \end{array}
      \]
    \end{enumerate}
  \end{enumerate}

\item {\bf There is an isomorphism $\ulbDerivLam{\tm} \simeq \grothy{\ulbF}{\ulbG}$ of upper semilattices.}
  As stated, the isomorphism is given by:
  \[
  \begin{array}{rclcl}
    \varphi & : & \ulbDerivLam{\tm}        & \to     & \grothy{\ulbF}{\ulbG} \\
               &&    \cls{\redseq} & \mapsto & (\cls{\redseq \sieve \tm'}, \cls{\redseq/(\redseq \sieve \tm')}) \\
  \\
    \psi    & : & \grothy{\ulbF}{\ulbG} & \to & \ulbDerivLam{\tm} \\
               &&  (\cls{\redseq},\cls{\redseqtwo}) & \mapsto & [\redseq\redseqtwo] \\
  \end{array}
  \]
  Note that $\varphi$ and $\psi$ are well-defined mappings, since their value does not
  depend on the choice of representative, due, in particular, to the fact that sieving
  is compatible with permutation equivalence~(\rlem{sieving_is_compatible_with_permutation_equivalence}).
  Let us check that $\varphi$ and $\psi$ are morphisms of upper semilatices,
  and that they are mutual inverses:
  \begin{enumerate}
  \item {\bf $\varphi$ is a morphism of upper semilattices.}
    Let us check the three conditions:
    \begin{enumerate}
    \item {\bf Monotonic.}
      Let $\cls{\redseq} \permle \cls{\redseqtwo}$ in $\ulbDerivLam{\tm}$,
      and let us show that the following inequality holds:
      \[
        \varphi(\cls{\redseq}) =
        (\cls{\redseq \sieve \tm'}, \cls{\redseq/(\redseq \sieve \tm')})
        \leq
        (\cls{\redseqtwo \sieve \tm'}, \cls{\redseqtwo/(\redseqtwo \sieve \tm')})
        = \varphi(\cls{\redseqtwo})
      \]
      We check the two conditions (by definition of $\grothy{\ulbF}{\ulbG}$):
      \begin{enumerate}
      \item On the first hand,
            $\cls{\redseq \sieve \tm'} \leqF \cls{\redseqtwo \sieve \tm'}$
            since
            \[
              \begin{array}{rcll}
              (\redseq \sieve \tm')/(\redseqtwo \sieve \tm')
              & \permle &
              \redseq/(\redseqtwo \sieve \tm') & \text{since $\redseq \sieve \tm' \permle \redseq$ by \rlem{sieve_is_prefix}} \\
              & \permle & \redseqtwo/(\redseqtwo \sieve \tm') & \text{since $\redseq \permle \redseqtwo$ by hypothesis} \\
              \end{array}
            \]
            Note that this is garbage by \rlem{projection_after_sieving_is_garbage}.
            So by \rprop{properties_of_garbage}, $(\redseq \sieve \tm')/(\redseqtwo \sieve \tm')$ is also garbage,
            as required.
      \item On the other hand, let us show that
            $\ulbG(\ptF{\cls{\redseqtwo \sieve \tm'}}{\cls{\redseq \sieve \tm'}})(\cls{\redseq/(\redseq \sieve \tm')})
            \permle
            \redseqtwo/(\redseqtwo \sieve \tm')$.
            In fact:
            \[
              \begin{array}{rcll}
                \ulbG(\ptF{\cls{\redseqtwo \sieve \tm'}}{\cls{\redseq \sieve \tm'}})(\cls{\redseq/(\redseq \sieve \tm')})
              & = &
                \cls{ (\redseq \sieve \tm')(\redseq/(\redseq \sieve \tm')) / (\redseqtwo \sieve \tm') } & \text{ by definition} \\
              & = &
                \cls{ \redseq / (\redseqtwo \sieve \tm') } & \text{ by \rlem{garbage_free_garbage_decomposition}} \\
              & \permle &
                \cls{ \redseqtwo / (\redseqtwo \sieve \tm') } & \text{ since $\redseq \permle \redseqtwo$} \\
              \end{array}
            \]
             
      \end{enumerate}
    \item {\bf Preserves bottom.}
      By definition:
      $\varphi(\bot_{\ulbDerivLam{\tm}})
       = (\cls{\emptyDerivation \sieve \tm'}, \cls{\emptyDerivation/(\emptyDerivation \sieve \tm')})
       = (\cls{\emptyDerivation},\cls{\emptyDerivation})
       = (\bot_{\ulbF},\bot_{\ulbG(\bot_\ulbF)})$.
    \item {\bf Preserves joins.}
      Let $\cls{\redseq},\cls{\redseqtwo} \in \ulbDerivLam{\tm}$, and let us show that
      $\varphi(\cls{\redseq} \sqcup \cls{\redseqtwo}) = \varphi(\cls{\redseqtwo}) \lor \varphi(\cls{\redseqtwo})$.
      Indeed, note that:
      \[
        \varphi(\cls{\redseq} \sqcup \cls{\redseqtwo}) = (\alpha,\beta)
      \]
      where
      \[
        \begin{array}{rcl}
        \alpha & = & \cls{(\redseq \sqcup \redseqtwo) \sieve \tm'} \\
        \beta  & = & \cls{(\redseq \sqcup \redseqtwo) / ((\redseq \sqcup \redseqtwo) \sieve \tm')} \\
        \end{array}
      \]
      and
      \[
        \varphi(\cls{\redseq}) \lor \varphi(\cls{\redseqtwo})
        = (\cls{\redseq \sieve \tm'}, \cls{\redseq / (\redseq \sieve \tm')}) \lor
          (\cls{\redseqtwo \sieve \tm'}, \cls{\redseqtwo / (\redseqtwo \sieve \tm')}) \\
        = (\alpha', \beta') \\
      \]
      where
      \[
        \begin{array}{rcl}
        \alpha' & = & \cls{\redseq \sieve \tm'} \lorF \cls{\redseqtwo \sieve \tm'} \\
        \beta'  & = &
               \ulbG(\ptF{\alpha}{\cls{\redseq \sieve \tm'}})(\cls{\redseq / (\redseq \sieve \tm')}) \sqcup \ulbG(\ptF{\alpha}{\cls{\redseqtwo \sieve \tm'}})(\cls{\redseqtwo / (\redseqtwo \sieve \tm')})
        \end{array}
      \]
      It suffices to show that $\alpha = \alpha'$ and $\beta = \beta'$.
      Let us show each separately:
      \begin{enumerate}
      \item {\bf Proof of $\alpha = \alpha'$.}
        The equality
        $
          \alpha
          = \cls{(\redseq \sqcup \redseqtwo) \sieve \tm'}
          = \cls{\redseq \sieve \tm'} \lorF \cls{\redseqtwo \sieve \tm'}
          = \alpha'
        $
        is an immediate consequence of \rlem{lorf_ignores_garbage}.
      \item {\bf Proof of $\beta = \beta'$.}
        Note that:
        \[
          \begin{array}{rcll}
          \beta'
                & = & \ulbG(\ptF{\alpha'}{\cls{\redseq \sieve \tm'}})(\cls{\redseq / (\redseq \sieve \tm')}) \sqcup
                      \ulbG(\ptF{\alpha'}{\cls{\redseqtwo \sieve \tm'}})(\cls{\redseqtwo / (\redseqtwo \sieve \tm')}) \\
                & = &
                     \cls{
                       (\redseq \sieve \tm')(\redseq / (\redseq \sieve \tm'))/\alpha'
                       \sqcup
                       (\redseqtwo \sieve \tm')(\redseqtwo / (\redseqtwo \sieve \tm'))/\alpha'
                     } \\
                & = &
                     \cls{\redseq/\alpha' \sqcup \redseqtwo/\alpha'}
                     &\hspace{-5cm}\text{by \rlem{garbage_free_garbage_decomposition}} \\
                & = &
                     \cls{(\redseq \sqcup \redseqtwo)/\alpha'}
                     &\hspace{-5cm}\text{since $A/C \sqcup B/C \permeq (A \sqcup B)/C$} \\
                & = &
                     \cls{(\redseq \sqcup \redseqtwo)/((\redseq \sqcup \redseqtwo) \sieve \tm')}
                     &\hspace{-5cm}\text{since $\alpha' = \alpha = (\redseq \sqcup \redseqtwo) \sieve \tm'$} \\
                & = & \beta \\
          \end{array}
        \]
        as required. 
      \end{enumerate}
    \end{enumerate}
  \item {\bf $\psi$ is a morphism of upper semilattices.}
    Let us check the three conditions:
    \begin{enumerate}
    \item {\bf Monotonic.}
      Let $(\cls{\redseq_1},\cls{\redseqtwo_1}) \leq (\cls{\redseq_2},\cls{\redseqtwo_2})$ in $\grothy{\ulbF}{\ulbG}$
      and let us show that
      $\psi(\cls{\redseq_1},\cls{\redseqtwo_1}) \permle \psi(\cls{\redseq_2},\cls{\redseqtwo_2})$
      in $\ulbDerivLam{\tm}$.
      Indeed, we know that $\ulbG(\ptF{\cls{\redseq_2}}{\cls{\redseq_1}})(\cls{\redseqtwo_1}) \permle \cls{\redseqtwo_2}$,
      that is to say $\redseq_1\redseqtwo_1/\redseq_2 \permle \redseqtwo_2$.
      Then:
      \[
        \redseq_1\redseqtwo_1/\redseq_2\redseqtwo_2
        = (\redseq_1\redseqtwo_1/\redseq_2)/\redseqtwo_2
        = \emptyDerivation
      \]
      which means that $\redseq_1\redseqtwo_1 \permle \redseq_2\redseqtwo_2$.
      This immediately implies that
      $\psi(\cls{\redseq_1},\cls{\redseqtwo_1}) \permle \psi(\cls{\redseq_2},\cls{\redseqtwo_2})$.
    \item {\bf Preserves bottom.}
      Recall that the bottom element $\bot_{(\grothy{\ulbF}{\ulbG})}$
      is defined as $(\bot_\ulbF, \bot_{\ulbG(\bot_\ulbF)})$, that is
      $(\cls{\emptyDerivation}, \cls{\emptyDerivation})$.
      Therefore $\psi(\bot_{(\grothy{\ulbF}{\ulbG})}) = \cls{\emptyDerivation} = \bot_{\ulbDerivLam{\tm}}$.
    \item {\bf Preserves joins.}
      Let $(\cls{\redseq_1},\cls{\redseqtwo_1})$ and $(\cls{\redseq_2},\cls{\redseqtwo_2})$
      be elements of $\grothy{\ulbF}{\ulbG}$, and let us show that
      $\psi((\cls{\redseq_1},\cls{\redseqtwo_1}) \lor (\cls{\redseq_2},\cls{\redseqtwo_2})) =
       \psi(\cls{\redseq_1},\cls{\redseqtwo_1}) \sqcup \psi(\cls{\redseq_2},\cls{\redseqtwo_2}))$.

      Let:
      \[
      \begin{array}{rcl}
        \alpha & := & (\redseq_1 \sqcup \redseq_2) \sieve \tm'
      \end{array}
      \]

      First we claim that $\alpha \permle \redseq_1\redseqtwo_1 \sqcup \redseq_2\redseqtwo_2$.
      This is because by \rlem{sieve_is_prefix}
      we know that $\alpha = (\redseq_1 \sqcup \redseq_2) \sieve \tm' \permle \redseq_1 \sqcup \redseq_2$.
      Moreover, it is easy to check that
      $\redseq_1 \sqcup \redseq_2 \permle \redseq_1\redseqtwo_1 \sqcup \redseq_2\redseqtwo_2$.
      Using this fact we have:
      \[
      \begin{array}{rcll}
        \psi((\cls{\redseq_1},\cls{\redseqtwo_1}) \lor (\cls{\redseq_2},\cls{\redseqtwo_2}))
      & = &
        \psi(\cls{\alpha},\ulbG(\ptF{\cls{\alpha}}{\cls{\redseq_1}})(\cls{\redseqtwo_1}) \sqcup \ulbG(\ptF{\cls{\alpha}}{\cls{\redseq_2}})(\cls{\redseqtwo_2}))
      \\
      & = &
        \psi(\cls{\alpha}, \cls{(\redseq_1\redseqtwo_1/\alpha) \sqcup (\redseq_2\redseqtwo_2/\alpha)})
      \\
      & = &
        \psi(\cls{\alpha}, \cls{(\redseq_1\redseqtwo_1 \sqcup \redseq_2\redseqtwo_2)/\alpha})
      \\&&\text{ since $A/C \sqcup B/C \permle (A \sqcup B)/C$} \\
      & = &
        \cls{\alpha((\redseq_1\redseqtwo_1 \sqcup \redseq_2\redseqtwo_2)/\alpha)} \\
      & = &
        \cls{(\redseq_1\redseqtwo_1 \sqcup \redseq_2\redseqtwo_2)(\alpha/(\redseq_1\redseqtwo_1 \sqcup \redseq_2\redseqtwo_2))} \\
      & = &
        \cls{\redseq_1\redseqtwo_1 \sqcup \redseq_2\redseqtwo_2}
        \\&&\text{ since
                     $\alpha \permle \redseq_1\redseqtwo_1 \sqcup \redseq_2\redseqtwo_2$,
                   so $\alpha/(\redseq_1\redseqtwo_1 \sqcup \redseq_2\redseqtwo_2) = \emptyDerivation$} \\
      & = &
        \psi(\cls{\redseq_1},\cls{\redseqtwo_1}) \sqcup \psi(\cls{\redseq_2},\cls{\redseqtwo_2}))
      \end{array}
      \]
      as required.
    \end{enumerate}
  \item {\bf Left inverse: $\psi \circ \varphi = \id$.}
    Let $\cls{\redseq} \in \ulbDerivLam{\tm}$.
    Then by \rlem{garbage_free_garbage_decomposition}:
    \[
      \psi(\varphi(\cls{\redseq}))
      = \psi(\cls{\redseq \sieve \tm'}, \cls{\redseq/(\redseq \sieve \tm')})
      = \cls{(\redseq \sieve \tm')(\redseq/(\redseq \sieve \tm'))}
      = \cls{\redseq}
    \]
  \item {\bf Right inverse: $\varphi \circ \psi = \id$.}
    Let $(\cls{\redseq},\cls{\redseqtwo}) \in \grothy{\ulbF}{\ulbG}$.
    Note that
    $\redseq$ is $\tm'$-garbage-free
    and $\redseqtwo$ is $\tm'$-garbage,
    so by~\rlem{sieving_trailing_garbage}
    and \rprop{characterization_of_garbage_free_derivations}
    we know that $\redseq\redseqtwo \sieve \tm' = \redseq \sieve \tm' \permeq \redseq$.
    Hence:
    \[
      \varphi(\psi(\cls{\redseq},\cls{\redseqtwo}))
      = \varphi(\cls{\redseq\redseqtwo})
      = (\cls{\redseq\redseqtwo \sieve \tm'}, \cls{\redseq\redseqtwo/(\redseq\redseqtwo \sieve \tm')})
      = (\cls{\redseq}, \cls{\redseq\redseqtwo/\redseq})
      = (\cls{\redseq}, \cls{\redseqtwo})
    \]
  \end{enumerate}
\end{enumerate}
\end{proof}
