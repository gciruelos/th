We will now present the type system we will work with, which as we said, is a variant of the system presented
in \cite{bucciarelli2017non}.

To make the notation easier, we will write $\typ_1 \cap ... \cap \typ_n$ as the
multiset $\lset{\typ_1, \hdots, \typ_n}$.
Recall that we work in a non idempotent environment so we may have repetitions.

\begin{definition}[Types]
Let $\labelset = \set{\lab,\lab',\lab'',\hdots}$ be a denumerable set of labels.
The sets of \defn{types}, ranged over by $\typ,\typtwo,\typthree,\hdots$,
and \defn{finite sets of types}, ranged over by $\mtyp,\mtyptwo,\mtypthree,\hdots$,
are given mutually inductively by the following abstract syntax:
\[
  \typ ::= \basetyp^{\lab} \mid \mtyp \tolab{\lab} \typ
\]
\[
  \mtyp ::= \lset{\typ_i}_{i=1}^{n} \HS \text{for some $n \geq 0$}
\]
In a type like $\basetyp^{\lab}$ and $\mtyp \tolab{\lab} \typ$,
the label $\lab$ is called the \defn{external label}.
\end{definition}

One remark that deserves mention is that the difference with the system~$\mathcal{W}$
at type-level is that we include labels.
Like in system~$\mathcal{W}$, note that it will not be possible for a term to have multiple types,
like the name \textit{intersection type system} would suggest.
Rather, what happens is that function terms will receive a parameter that can be interpreted
as having several types.

Later we will need to look closely at these types,
for which purpose the following definition will be useful.

\begin{definition}
A type $\typ$ is said to \defn{occur} in another type $\typtwo$ if
$\typ \occursin \typtwo$ holds,
where $(\occursin)$ is the reflexive and transitive closure of
the axioms
$\typ_i \occursin [\typ_1,\hdots,\typ_n] \to \typtwo$ for all $i \in \set{1,\hdots,n}$,
and $\typtwo \occursin [\typ_1,\hdots,\typ_n] \to \typtwo$.
This is extended to say that a type $\typ$ occurs in a multiset $[\typtwo_1,\hdots,\typtwo_n]$,
defined by $\typ \occursin [\typtwo_1,\hdots,\typtwo_n]$ if $\typ \occursin \typtwo_i$
for some $i\in\set{1,\hdots,n}$,
and that a type $\typ$ occurs in a typing context $\tctx$,
defined by $\typ \occursin \tctx$ if $\typ \occursin \tctx(\var)$ for some $\var \in \dom\tctx$.
\end{definition}
