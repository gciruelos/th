As we stated in the introduction, the goal of labeling terms and types
is that we obtain a confluent calculus.
Confluence means that every two reduction sequences from a term
can be extended to a common reduct.
Indeed, the $\lambdadist$-calculus
is confluent, and it is the purpose of this section to prove so.


First, we need to prove an adaptation of the Substitution Lemma for our calculus,
which is a key tool to prove properties about coinitial steps.
The substitution lemma for the pure lambda calculus \cite[Lemma 2.1.16]{Barendregt:1984}  states that, provided that
$\var \neq \vartwo$ and $\var \not\in \fv{\tmthree}$, then
  $\subs{\subs{\tm}{\var}{\tmtwo}}{\vartwo}{\tmthree} =
    \subs{\subs{\tm}{\vartwo}{\tmthree}}{\var}{\subs{\tmtwo}{\vartwo}{\tmthree}}$

In our case variables get substituted by a \emph{list} of terms, so we need to adapt it.
Particularly, the list of terms that will take the place of $\tmthree$ needs to be
divided up in several lists: one for the corresponding $\vartwo$s in $\tm$, and the rest
for the $\vartwo$s in each of the elements of $\tmtwo$, which recall that now will be a list.

\begin{notation}
We extend the substitution operator to work on lists,
defining
\[\subs{[\tm_i]_{i=1}^{n}}{\var}{\ls{\tmtwo}} \eqdef [\subs{\tm_i}{\var}{\ls{\tmtwo}_i}]_{i=1}^{n}\]
where $(\ls{\tmtwo}_1,\hdots,\ls{\tmtwo}_n)$ is a partition of $\ls{\tmtwo}$
such that $\varlabel{\var}{\tm_i} = \tmlabel{\ls{\tmtwo}_i}$ for all $i \in \set{1,\hdots,n}$.
\end{notation}

\begin{lemma}[Substitution Lemma]
\llem{substitution_lemma}
Let $\var \neq \vartwo$ and $\var \not\in \fv{\ls{\tmthree}}$.
If $(\ls{\tmthree}_1,\ls{\tmthree}_2)$ is a partition of $\ls{\tmthree}$
then
\[
  \subs{\subs{\tm}{\var}{\ls{\tmtwo}}}{\vartwo}{\ls{\tmthree}}
  =
  \subs{\subs{\tm}{\vartwo}{\ls{\tmthree}_1}}{\var}{\subs{\ls{\tmtwo}}{\vartwo}{\ls{\tmthree}_2}}
\]
provided that both sides of the equation are defined.
{\em Note:} there exists a list $\ls{\tmthree}$ that makes the left-hand side defined
if and only if there exist lists $\ls{\tmthree}_1,\ls{\tmthree}_2$ that make the
right-hand side defined.
\end{lemma}
\begin{proof}
\SeeAppendixRef{substitution_lemma_proof}
By induction on $\tm$.
\end{proof}

\begin{example} For example, consider the term
\[
  \tm =
    (\lamp{\lab}{\var}{\varthree^{[\alpha] \to [\beta] \to \delta} [\var^\alpha] [\var^\beta]})
      [z^\beta,y^\alpha].
\]
We can perform the following substitution (where $w,a,b$ are all variables):
\[
  \subs{
    \subs{\tm}{\varthree}{[w^{[\alpha] \to [\beta] \to \delta}, \vartwo^\beta]}
  }{
    \vartwo
  }{
    [a^\beta,b^\alpha]
  }
  =
   (\lamp{\lab}{\var}{w^{[\alpha] \to [\beta] \to \delta} [\var^\alpha] [\var^\beta]})
    [a^\beta,b^\alpha].
\]
Note that, as we replaced one $\varthree$ by a $\vartwo$, if we wanted to invert the order
of the substitutions we would need to separate the list $[a^\beta, b^\alpha]$ in two:
\[
  \subs{
    \subs{\tm}{\vartwo}{[b^\alpha]}
  }{
    \varthree
  }{
    \subs{[w^{[\alpha] \to [\beta] \to \delta}, \vartwo^\beta]}{\vartwo}{a^\beta}
  }
  =
   (\lamp{\lab}{\var}{w^{[\alpha] \to [\beta] \to \delta} [\var^\alpha] [\var^\beta]})
    [a^\beta,b^\alpha].
\]
\end{example}

\bigskip

Altough are going to prove that the $\lambdadist$-calculus is confluent,
actually a strictly stronger property holds: strong permutation.
This property says that if we "open" a diagram with two (different) steps,
we can "close" it with two steps, too; and it also gives us information
about which are those closing steps, via their labels.
The fact that this stronger property holds gives us the chance to
develop a functorial residual theory, as we will learn in the next chapter.



\begin{proposition}[Strong Permutation]
\lprop{strong_permutation}
If $\tm_0 \todistl{\lab_1} \tm_1$
and $\tm_0 \todistl{\lab_2} \tm_2$
are different steps, then there exists a term $\tm_3 \in \termsdist$ such that
$\tm_1 \todistl{\lab_2} \tm_3$ and $\tm_2 \todistl{\lab_1} \tm_3$.
Diagramatically,
\[
    \xymatrix{
      \tm_0 \ar@{->}[d]^{\lab_2}
            \ar@{->}[r]^{\lab_1} &
      \tm_1 \ar@{-->}[d]^{\lab_2} \\
            \ar@{-->}[r]^{\lab_1}
      \tm_2 &
      \tm_3. \\
    }
\]
\end{proposition}
\begin{proof}
Let $\redex : \tm_0 \todistl{\lab} \tm_1$ and $\redextwo : \tm_0 \todistl{\lab'} \tm_2$
be steps going out from $\tm_0$, and let us show that the peak may be closed.
The step $\redex$ is of the form:
\[
  \redex : \tm_0 = \conof{(\lamp{\lab}{\var}{\tm})\ls{\tmtwo}}
           \todistl{\lab} \conof{\subs{\tm}{\var}{\ls{\tmtwo}}} = \tm_1
\]
We proceed by induction on $\con$, and within each case we separate in different
cases depending on where $\redextwo$ is located (which recall that is different
than $\redex$ by hypothesis)
\SeeAppendixRef{strong_permutation_proof}
\end{proof}

\bigskip


The Strong Permutation property may also be called \emph{sub-commutativity}
or \emph{WCR$^{\leq 1}$} in some contexts.
\footnote{In reality, sub-commutativity and WCR$^{\leq 1}$ are slightly different,
but equivalent:
as they don't ask for the steps to be different,
but allow the closing steps to be at most one (instead of exactly one).}

As a consequence of strong permutation, reduction is subcommutative,
\ie $(\lefttodist \circ \todist)\ \subseteq (\todist^= \circ \lefttodist^=)$
where
$\lefttodist$ denotes $(\todist)^{-1}$
and $R^=$ denotes the reflexive closure of $R$.

Moreover, it is well-known that subcommutativity implies confluence,
\ie $(\lefttodist^* \circ \todist^*)\ \subseteq (\todist^* \circ \lefttodist^*)$
(\cf \cite[Proposition 1.1.10]{Terese}). Note that the
inverse does not hold, hence confluence is weaker than strong permutation, as we previously stated.

\begin{corollary}[Confluence]
$\lambdadist$ is confluent, \ie,
if $\tm_0 \tomdist \tm_1$ and $\tm_0 \tomdist \tm_2$, then there exists a term $\tm_3$
such that $\tm_1 \tomdist \tm_3$ and $\tm_2 \tomdist \tm_3$.
Diagramatically,
\[
    \xymatrix{
      \tm_0 \ar@{->>}[d]
            \ar@{->>}[r] &
      \tm_1 \ar@{-->>}[d] \\
            \ar@{-->>}[r]
      \tm_2 &
      \tm_3. \\
    }
\]
\end{corollary}


