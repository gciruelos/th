In this section we will develop the theory of residuals of the $\lambdadist$-calculus.
The basic concepts of general rewriting theory where outlined in \rsec{rewriting_theory_preliminaries},
and we will now concentrate in the specifics of our calculus.

Informally, the residual of a step after another is what is left of a step after executing another one; it is a set of steps.

\begin{example} If we consider the term of the pure lambda calculus:
\[(\lam{\var}{\vartwo})((\lam{\varthree}{\varthree}) w)\]
Then there are two steps we can perform (we shall call them $\redex$ and $\redextwo$).
\begin{align*}
  \redex &:
    (\lam{\var}{\vartwo})((\lam{\varthree}{\varthree}) w) \to y \\
  \redextwo &:
    (\lam{\var}{\vartwo})((\lam{\varthree}{\varthree}) w) \to (\lam{\var}{\vartwo}) w
\end{align*}
Then the residual of $\redex$ after $\redextwo$ is the step $(\lam{\var}{\vartwo}) w \to w$.
On the other hand, the residual of $\redextwo$ after $\redex$ is the empty set:
$\redextwo$ was erased by $\redex$.
\end{example}

It is interesting to develop a theory of residuals for $\lambdadist$ because
residuals allow us to trace a redex through the reduction of a term.
Given that the purpose of $\lambdadist$ is to be able to track
how resources are used in $\lambda$-terms, it is crucial to
be able to learn how these resources interact during a reduction.

As it turns out, the theory of residuals of $\lambdadist$ will prove to be powerful
enough to represent meaningful information, but simple enough to have a
comprehensible structure.


\begin{definition}
If $\redex : \tm \todistl{\lab} \tm'$ is a step in the distributive lambda
calculus, then:
\[
  \begin{array}{rcll}
    \src(\redex) & \eqdef & \tm & \text{ is the \defn{source} of $\redex$} \\
    \tgt(\redex) & \eqdef & \tm' & \text{ is the \defn{target} of $\redex$} \\
    \name(\redex) & \eqdef & \lab & \text{ is the \defn{name} of $\redex$}
  \end{array}
\]
Two steps $\redex, \redextwo$
are \defn{coinitial} if $\src(\redex) = \src(\redextwo)$
and \defn{cofinal} if $\tgt(\redex) = \tgt(\redextwo)$.
\end{definition}

\begin{definition}[Residuals in the distributive lambda-calculus]
\ldef{residuals}
Given coinitial steps $\redex, \redextwo$,
the set $\redex/\redextwo$ of \defn{residuals of $\redex$ after $\redextwo$} is defined as follows:
\[
  \redex/\redextwo = \set{\redex' \ST \src(\redex') = \tgt(\redextwo) \text{ and } \name(\redex) = \name(\redex')}
\]
\end{definition}

\begin{remark}
Recall that the name of a step is the label that decorates the lambda reduced by the step,
and that in correct terms all lambdas have pairwise distinct labels.
Given that our calculus has no duplication or erasure (as per the following lemma),
names of steps will be useful to name reductions.
\end{remark}


\begin{lemma}[Cardinality of the set of residuals]
\llem{cardinality_of_the_set_of_residuals}
\[
  \#(\redex/\redextwo) = \begin{cases}
                         0 & \text{ if $\redex = \redextwo$} \\
                         1 & \text{ otherwise}
                         \end{cases}
\]
\end{lemma}
\begin{proof}
Recall that, by definition, lambdas in a correct term have pairwise distinct labels.
Consider first the case when $\redex = \redextwo$.
Then
$
  \redex = \redextwo : \conof{(\lamp{\lab}{\var}{\tm})\ls{\tmtwo}} \todistl{\lab} \conof{\subs{\tm}{\var}{\ls{\tmtwo}}}
$.
There is only one lambda decorated with $\lab$ in the source,
so there are no lambdas decorated with $\lab$ in the target.
Hence $\redex/\redextwo = \emptyset$.

On the other hand if, $\redex \neq \redextwo$,
by the Strong Permutation property (\rprop{strong_permutation})
there exists a step $\redex' \in \redex/\redextwo$
with the same name as $\redex$.
There are no other lambdas decorated with $\lab$ in the target.
Hence $\redex/\redextwo = \set{\redex'}$.
\end{proof}

\subsubsection{Orthogonality of $\lambdadist$}

\TODO{definir development? definir diagrama de permutacion?}

\begin{lemma}[Finite developments]
\llem{finite_developments}
Let $\redexset$ be a set of coinitial steps.
Then the length of every complete development of $\redexset$
is precisely the cardinality of $\redexset$.
In particular, developments are finite.
\end{lemma}
\begin{proof}
By induction on the cardinality of $\redexset$.
If $\redexset = \emptyset$, the only complete development of $\redexset$ is $\emptyDerivation$ and we are done.
Otherwise, if $\redseq$ is a complete development of $\redexset$, it is a non-empty derivation, \ie
$\redseq = \redex\redseqtwo$
where $\redex \in \redexset$
and such that $\redseqtwo$ is a complete development of $\redexset/\redex$.
Since residuals of distinct redexes have distinct names (and hence they are distinct)
we have that
$
  \redexset/\redex = \uplus_{\redextwo \in \redexset} (\redextwo/\redex)
$,
where $\uplus$ denotes the disjoint union of sets.
Moreover, $\#(\redextwo/\redex) = 1$ if and only if $\redex \neq \redextwo$
by \rlem{cardinality_of_the_set_of_residuals}, so:
\[
  \#(\redexset/\redex) =
  \Sigma_{\redextwo \in \redexset} \#(\redextwo/\redex) =
  \#(\redexset \setminus \set{\redex}) =
  \#(\redexset) - 1
\]
Hence by \ih the length of $\redseqtwo$ is $\#(\redexset) - 1$
and we conclude.
\end{proof}

\begin{proposition}
The distributive lambda-calculus is an Orthogonal Axiomatic Rewrite System
in the sense of Melli\`es.
\end{proposition}
\begin{proof}
There are four axioms to check:
\begin{enumerate}
\item {\bf Autoerasure.}
  Immediate from the cardinality of residuals lemma (\rlem{cardinality_of_the_set_of_residuals}).
\item {\bf Finite Residuals.}
  Immediate from the cardinality of residuals lemma (\rlem{cardinality_of_the_set_of_residuals}).
\item {\bf Finite Developments.}
  Proved in the Finite Developments lemma (\rlem{finite_developments}).
\item {\bf Semantic Orthogonality.}
  A consequence of the Strong Permutation property (\rprop{strong_permutation}).
\end{enumerate}
\end{proof}

\begin{definition}
A step $\redex$ \defn{belongs to} a coinitial derivation $\redseq$,
written $\redex \in \redseq$,
if and only if some residual of $\redex$ is contracted along $\redseq$. 
More precisely, $\redex \in \redseq$ if there exist $\redseq_1,\redex',\redseq_2$
such that $\redseq = \redseq_1\redex'\redseq_2$ and $\redex' \in \redex/\redseq_1$.
We write $\redex \not\in \redseq$ if it is not the case that $\redex \in \redseq$.
\end{definition}


