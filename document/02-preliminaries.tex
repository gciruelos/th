
\chapter{Preliminaries}
\lcha{preliminaries}

\section{Order theory}
We are interested in understanding the derivation spaces
of $\lambda$-terms.
These derivation spaces, as we briefly mentioned earlier, have an structure that can be seen as an order.
More specifically, the poset we will consider will be the one of derivations, where the order is giving by
the amount of \textit{work} each derivation does.

But the structures we will work with are more than just posets, they have a richer structure,
some of which we will define now.

An \defn{upper semilattice} is a poset (\ie, a set $A$ with an order $\leq$)
with a least element or \defn{bottom} $\bot \in A$,
such that for every two elements $a, b \in A$
there is a least upper bound or \defn{join} $(a \lor b) \in A$.


A \defn{lattice} is an upper semilattice
with a greatest element or \defn{top} $\top \in A$,
and such that for every two elements $a, b \in A$
there is a greatest lower bound or \defn{meet} $(a \land b) \in A$.
A lattice is \defn{distributive} if $\land$ distributes over $\lor$ and vice versa.

In the introduction we claimed that derivation spaces of $\lambda$-terms where
upper semilattices, but in general were not lattices.

A \defn{morphism} of upper semilattices
is given by a monotonic function $f : A \to B$,
\ie $a \leq b$ implies $f(a) \leq f(b)$,
preserving the bottom element, \ie $f(\bot) = \bot$,
and joins, \ie $f(a \lor b) = f(a) \lor f(b)$ for all $a, b \in A$.
Similarly for morphisms of lattices (and distributive lattices).

Any poset $(A,\leq)$ may be regarded as category whose objects are the elements of $A$
and morphisms are of the form $\pt{b}{a}$ for all $a \leq b$. 
The category of posets with monotonic functions is denoted by $\Poset$.
In fact, we view it as a 2-category:
given morphisms $f,g : A \to B$ of posets,
there is a \textit{2-cell} $f \totwocell g$ if $f(a) \leq g(a)$ for all $a \in A$.
(A 2-category is just a category with morphisms between morphisms).
This notion is more technical and will only be used in the last chapter.


\section{Rewriting theory}
\lsec{rewriting_theory_preliminaries}
The $\lambda$-calculus is a particular case of a more general mathematical concept, called \defn{rewriting system}.
Informally, a rewriting system is a set of objects and a set of rules that let you transform some objects into others.

More formally, an \defn{axiomatic rewrite system} (\cf~\cite[Def.~2.1]{thesismellies})
is given by a set of objects $\Obj$, a set of steps $\Stp$,
two functions $\src,\tgt : \Stp \to \Obj$ indicating the source and target of each step,
and a \defn{residual function} $(/)$ such that given any two steps $\redex,\redextwo \in \Stp$
with the same source, yields a set of steps $\redex/\redextwo$ such that $\src(\redex') = \tgt(\redextwo)$
for all $\redex' \in \redex/\redextwo$.
Steps are ranged over by $\redex,\redextwo,\redexthree,\hdots$.
A step $\redex' \in \redex/\redextwo$ is called a \defn{residual} of $\redex$ after $\redextwo$,
and $\redex$ is called an \defn{ancestor} of $\redex'$.
Steps are \defn{coinitial} (resp. \defn{cofinal}) if they have the same source (resp. target).
A \defn{derivation} is a possibly empty sequence of composable steps $\redex_1\hdots\redex_n$.
Derivations are ranged over by $\redseq,\redseqtwo,\redseqthree,\hdots$.
The functions $\src$ and $\tgt$ are extended to derivations (noting that there is a different empty derivation for each element in $\Obj$).
We use $\emptyDerivation_\tm$ to denote the empty derivation with source (and target) $\tm$, and we will often drop the subscript when it is clear from the context.
the subscript.

Composition of derivations is defined when $\tgt(\redseq) = \src(\redseqtwo)$ and written $\redseq\redseqtwo$.
Residuals are extended for projecting after a derivation,
namely $\redex_n \in \redex_0/\redextwo_1\hdots\redextwo_n$
if and only if there exist $\redex_1,\hdots,\redex_{n-1}$
such that $\redex_{i+1} \in \redex_i/\redextwo_{i+1}$ for all $0 \leq i \leq n - 1$.
Let $\redexset$ be a set of coinitial steps.

A \defn{development} of $\redexset$ is a (possibly infinite) derivation $\redex_1\hdots\redex_n\hdots$
such that for every index $i$ there exists a step $\redextwo \in \redexset$
such that $\redex_i \in \redextwo/\redex_1\hdots\redex_{i-1}$.
Informally, a development of $\redexset$ is a derivation in which we only
do the work that the steps contained in $\redexset$ do.
A development is \defn{complete} if it is maximal.

The definition of an abstract rewriting system is very general, and because of that it does not give us
general properties for systems with such structure.
In his PhD thesis, Melliès gave a set of sufficient properties (which he called axioms)
that a rewriting system should have to behave properly.

An \defn{orthogonal} axiomatic rewrite system (\cf~\cite[Sec.~2.3]{thesismellies})
has four additional axioms\footnote{In \cite{thesismellies},
Autoerasure is called \textit{Petit~axiome~A},
Finite Residuals is called \textit{Petit~axiome~B},
and Semantic Orthogonality is called \textit{Axiome~PERM}.
We follow the nomenclature of~\cite{DBLP:conf/popl/AccattoliBKL14}}:
\begin{enumerate}
\item {\em Autoerasure}.
  $\redex/\redex = \emptyset$ for all $\redex \in \Stp$.
\item {\em Finite Residuals}.
  The set $\redex/\redextwo$ is finite for all coinitial $\redex,\redextwo \in \Stp$.
\item {\em Finite Developments}.
  If $\redexset$ is a set of coinitial steps,
  all developments of $\redexset$ are finite.
\item {\em Semantic Orthogonality}.
  Let $\redex,\redextwo \in \Stp$ be coinitial steps.
  Then there exist a complete development $\redseq$ of $\redex/\redextwo$
  and a complete development $\redseqtwo$ of $\redextwo/\redex$
  such that $\redseq$ and $\redseqtwo$ are cofinal.
  Moreover,
  for every step $\redexthree \in \Stp$ such that $\redexthree$ is coinitial to $\redex$,
  the following equality between sets holds:
  $\redexthree/\redex\redseqtwo = \redexthree/\redextwo\redseq$.
\end{enumerate}
In~\cite{thesismellies}, Melli\`es develops
the theory of orthogonal axiomatic rewrite systems.
A notion of \defn{projection}
$\redseq/\redseqtwo$ may be defined between coinitial derivations,
essentially by setting $\emptyDerivation/\redseqtwo \eqdef \emptyDerivation$
and $\redex\redseq'/\redseqtwo \eqdef (\redex/\redseqtwo)(\redseq'/(\redseqtwo/\redex))$
where, by abuse of notation, $\redex/\redseqtwo$ stands for a (canonical) complete development
of the set $\redex/\redseqtwo$.
\footnote{The projection $\redex / \redseqtwo$ essentially stands for the work from
$\redex$ that is left after doing $\redseqtwo$.}
Using this notion, one may define
a transitive relation of \defn{prefix} ($\redseq \permle \redseqtwo$),
a \defn{permutation equivalence} relation ($\redseq \permeq \redseqtwo$),
and the \defn{join} of derivations ($\redseq \sqcup \redseqtwo$).
Some of their properties are summed up in the figure
below:
\begin{center}
{\bf Summary of properties of orthogonal axiomatic rewrite systems}
\end{center}
{\footnotesize
\[
\begin{array}{|c|c|c|}
\hline
\begin{array}{r@{\hspace{0.2cm}}c@{\hspace{0.2cm}}l}
  \emptyDerivation\,\redseq & = & \redseq
\\
  \redseq\,\emptyDerivation & = & \redseq
\\
  \emptyDerivation/\redseq & = & \emptyDerivation
\\
  \redseq/\emptyDerivation & = & \redseq
\\
  \redseq/\redseqtwo\redseqthree & = & (\redseq/\redseqtwo)/\redseqthree
\\
  \redseq\redseqtwo/\redseqthree & = & (\redseq/\redseqthree)(\redseqtwo/(\redseqthree/\redseq))
\\
  \redseq/\redseq & = & \emptyDerivation
\end{array}
&
\begin{array}{r@{\hspace{0.2cm}}c@{\hspace{0.2cm}}l}
  \redseq \permle \redseqtwo & \iffdef & \redseq/\redseqtwo = \emptyDerivation
\\
  \redseq \permeq \redseqtwo & \iffdef & \redseq \permle \redseqtwo \land \redseqtwo \permle \redseq
\\
  \redseq \sqcup \redseqtwo  & \eqdef & \redseq(\redseqtwo/\redseq) 
\\
  \redseq \permeq \redseqtwo & \implies & \redseqthree/\redseq = \redseqthree/\redseqtwo
\\
  \redseq \permle \redseqtwo & \iff & \exists \redseqthree.\ \redseq\redseqthree \permeq \redseqtwo
\\
  \redseq \permle \redseqtwo & \iff & \redseq \sqcup \redseqtwo \permeq \redseqtwo
\end{array}
&
\begin{array}{r@{\hspace{0.2cm}}c@{\hspace{0.2cm}}l}
  \redseq \permle \redseqtwo & \implies & \redseq/\redseqthree \permle \redseqtwo/\redseqthree
\\
  \redseq \permle \redseqtwo & \iff & \redseqthree\redseq \permle \redseqthree\redseqtwo
\\
  \redseq \sqcup \redseqtwo  & \permeq & \redseqtwo \sqcup \redseq
\\
  (\redseq \sqcup \redseqtwo) \sqcup \redseqthree & = & \redseq \sqcup (\redseqtwo \sqcup \redseqthree)
\\
  \redseq & \permle & \redseq \sqcup \redseqtwo
\\
  (\redseq \sqcup \redseqtwo)/\redseqthree & = & (\redseq/\redseqthree) \sqcup (\redseqtwo/\redseqthree) \\
\end{array}
\\
\hline
\end{array}
\]
}
Let $\cls{\redseq} = \set{\redseqtwo \ST \redseq \permeq \redseqtwo}$
denote the permutation equivalence class of $\redseq$.
In an orthogonal axiomatic rewrite system,
the set $\ulbDeriv{x} = \set{\cls{\redseq} \ST \src(\redseq) = x}$
forms an upper semilattice.
The order $\cls{\redseq} \permle \cls{\redseqtwo}$ is given by $\redseq \permle \redseqtwo$,
the join is $\cls{\redseq} \sqcup \cls{\redseqtwo} = \cls{\redseq \sqcup \redseqtwo}$,
and the bottom is $\bot = \cls{\emptyDerivation}$.

The $\lambda$-calculus is an example of an orthogonal axiomatic rewrite system \cite{thesismellies}.
Our structures of interest are the semilattices of derivations of the $\lambda$-calculus,
written $\ulbDerivLam{\tm}$ for any given $\lambda$-term $\tm$.
As usual, $\beta$-reduction in the $\lambda$-calculus
is written $\tm \tobeta \tmtwo$
and defined by the contextual closure of the axiom $(\lam{\var}{\tm})\tmtwo \tobeta \subs{\tm}{\var}{\tmtwo}$.


\section{Lists and sets}

Throughout this work we will use lists and sets, so we establish here basic definitions and notations.

\begin{definition}[Lists and sets]
If $A$ is a sort, we write $\ls{A}$ for the sort of (finite) \defn{lists} over $A$,
defined inductively as:
\[
  \ls{A} ::= \emptyList \mid A \cons \ls{A}
  \]
We usually write $[a_1, \hdots, a_n]$,
abbreviated $[a_i]_{i=1}^{n}$,
to stand for $a_1 \cons (a_2 \cons \hdots (a_n \cons \emptyList))$.
If $\ls{a}$ and $\ls{b}$ are lists, $\ls{a} + \ls{b}$ stands for its concatenation,
and $\lengthof{\ls{a}}$ is the length of the list $\ls{a}$.
If $a,b,c,\hdots$ are the names of the metavariables ranging over a sort $A$,
then $\ls{a},\ls{b},\ls{c},\hdots$
are the names of the metavariables ranging over lists of $A$.
When there is no possibility of confusion, we may also write $[\ls{a},b,\ls{c}]$
for the list $\ls{a} + [b] + \ls{c}$.
We write $\ls{a} \approx \ls{b}$ if $\ls{a}$ is a permutation of $\ls{b}$.
Observe that $\approx$ is an equivalence relation.

In some cases we will work with (finite) \defn{sets},
which are defined to be lists,
considered modulo arbitrary permutations of their elements
and without regard of the number of repetitions of each element.
The notation for operations on lists will be lifted to operations
on sets.
In particular, $\ls{a} + \ls{b}$ denotes the union of sets,
and $\lengthof{\ls{a}}$ stands for the cardinal of $\ls{a}$.
This notation is chosen to resemble the multiset notation of existing
intersection type systems. Whether we are referring to sets or lists will be
clear from the context.
\end{definition}





\section{Typing}

In the framework of typed calculi, a typing judgement is a statement of the meta-theory that says that
with a given a typing context, we can prove that some term has some type.
For example, the judgement $\tctx \vdash \tm : \typ$ says that with the typing context
$\tctx$ we can prove that $\tm$ has type $\typ$.

\defn{Typing contexts}, or contexts for short,
ranged over by $\tctx,\tctxtwo,\tctxthree,\hdots$ are (total) functions from variables to finite sets of types.
We write $\dom\tctx$ for the set of variables $\var$ such that $\tctx(\var) \neq \lset{}$.
We write $\emptyContext$ for the context such that $\emptyContext(\var) = \lset{}$ for every variable $\var$.
The notation $\tctx + \tctxtwo$ stands for the \defn{sum of contexts}, defined as follows:
\[
    (\tctx + \tctxtwo)(\var) \eqdef \tctx(\var) + \tctxtwo(\var)
\]
The notation $\tctx \oplus \tctxtwo$ stands for the \defn{disjoint sum of contexts},
\ie it stands for $\tctx + \tctxtwo$ provided $\dom\tctx \cap \dom\tctxtwo = \emptyset$.
We also write $\tctx +_{i=1}^{n} \tctxtwo_i$ for $\tctx + \sum_{i=1}^{n} \tctxtwo_i$.
Moreover, $\var : \mtyp$ denotes the context such that $(\var : \mtyp)(\var) = \mtyp$
and $\dom(\var : \mtyp) = \set{\var}$.

