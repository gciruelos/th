\chapter{Conclusions}
\label{ch:conclusions}

In this thesis we defined a calculus based on non-idempotent intersection types,
which we called $\lambdadist$.
Admittedly, its syntax is complex because of the labeling of variables
and the correctness invariant, which where \textit{ad hoc}
additions so that the calculus was confluent.
However the derivation spaces of this calculus are very simple structures,
they are distributive lattices (\rthm{lattice_is_distributive}),
and are representable as rings of sets (\rthm{lattice_ring_of_sets}),

Then we proved that derivation spaces in the $\lambda$-calculus
can be mapped onto these simpler derivation spaces, via a strong
simulation result (\rcoro{algebraic_simulation}).
Using this, we proved how the derivation space of any $\lambda$-term that has a
head normal form can be factorized as a ``twisted product'' of garbage-free
and garbage derivations (\rthm{factorization_ulb_derivations}).

We think this validates the hypothesis that explicitly representing resource
management can shed some light on the structure of derivation spaces.
We would like to know what would happen if we changed $\lambdadist$ for
another resource calculus, whether or not similar results can be found.

The Factorization theorem (\rthm{factorization_ulb_derivations})
is reminiscent of Melliès’ \cite{DBLP:conf/ctcs/Mellies97} external--internal
factorization. It should be possible to establish a formal correspondence between these
notions. As noted by Melliès, any evaluation strategy that always picks external steps is
hypernormalizing. It should be easy to show that this holds for evaluation strategies picking
non-garbage steps, using the terminology of this work.

Open is the question that we posed after \rprop{has_hnf_has_refinement},
to study the relationship between refinements of a term of the $\lambda$-calculus
and its approximants.
It should not be hard to prove that there is a correspondence,
\ie that for a given term $\tm$ and an approximant $A$,
we can find one refinement of $\tm$ whose normal form refines a head normal form of $\tm$
that corresponds to $A$.

A related question is the one of whether it is possible to characterize garbage in the same
way we did in this work without using the $\lambdadist$-calculus.
We believe a plausible way to do this would be to use approximants:
instead of defining garbage with respect to a refinement we define garbage
with respect to an approximant.
Such a result would not have any $\lambdadist$ traces in its statement,
but $\lambdadist$ may help to prove it easily.

