
\subsection*{Proof of \rlem{sieving_is_well_defined} --- Sieving is well-defined}
\label{sieving_is_well_defined_proof}
The operation $\redseq \sieve \tm'$ is well-defined.
\begin{proof}
Proving this amounts to showing that the recursion scheme is well-founded.
It suffices to show that there is a measure $M$ ranging over the non-negative
integers such that
$M(\redseq,\tm') > M(\redseq/\redex_0,\tm'/\redex_0)$
whenever $\redex_0$ is the leftmost coarse step for $(\redseq,\tm')$.
Indeed, if we define:
\[
  M(\redseq,\tm') \eqdef \lengthof{\redseq/\tm'}
\]
where $\lengthof{\redseq/\tm'}$ stands for the length of the derivation $\redseq/\tm'$
in the distributive lambda-calculus, then we may conclude by checking that the following inequality holds:
\[
  \lengthof{\redseq/\tm'} > \lengthof{(\redseq/\redex_0)/(\tm'/\redex_0)}
\]
First observe that
$\redex_0 \permle \redseq$
so, by \rcoro{simulation_residuals_and_prefixes},
$\redex_0/\tm' \permle \redseq/\tm'$
and, by \rprop{prefixes_as_subsets},
$\names(\redex_0/\tm') \permle \names(\redseq/\tm')$.
Then we have:
\[
  \begin{array}{rcll}
  \lengthof{\redseq/\tm'}
   & = & \#\names(\redseq/\tm') & \text{by \rcoro{length_of_derivation_is_number_of_distinct_names}} \\
   & > & \#(\names(\redseq/\tm') \setminus \names(\redex_0/\tm')) & \text{since $\redex_0/\tm' \neq \emptyset$ and $\names(\redex_0/\tm') \subseteq \names(\redseq/\tm')$} \\
   & = & \#\names((\redseq/\tm') / (\redex_0/\tm')) & \text{by \rlem{names_after_projection_along_a_step}} \\
   & = & \#\names((\redseq/\redex_0) / (\tm'/\redex_0)) & \text{by \rcoro{permutation_equivalence_in_terms_of_names} and \rlem{generalized_cube_lemma}} \\
   & = & \lengthof{(\redseq/\redex_0) / (\tm'/\redex_0)} & \text{by \rcoro{length_of_derivation_is_number_of_distinct_names}}
  \end{array}
\]
\end{proof}

\subsection*{Proof of \rlem{sieving_is_compatible_with_permutation_equivalence} --- Sieving is compatible with permutation equivalence}
\label{sieving_is_compatible_with_permutation_equivalence_proof}
Let $\redseq \permeq \redseqtwo$. Then $\redseq \sieve \tm' \permeq \redseqtwo \sieve \tm'$.
\begin{proof}
First observe that, given two permutation equivalent derivations $\redseq$ and $\redseqtwo$,
a step $\redex$ is coarse for $(\redseq,\tm')$ if and only if $\redex$ is coarse for $(\redseqtwo,\tm')$,
since:
\[
  (\redex \permle \redseq) \iff (\redex/\redseq = \emptyset) \iff (\redex/\redseqtwo = \emptyset) \iff (\redex \permle \redseqtwo)
\]
We proceed by induction on the length of $\redseq \sieve \tm'$.
There are two cases, depending on whether there is a coarse step for $(\redseq,\tm')$.
\begin{enumerate}
\item {\bf If there are no coarse steps for $(\redseq,\tm')$.}
  Then there are no coarse steps for $(\redseqtwo,\tm')$,
  so $\redseq \sieve \tm' = \emptyDerivation = \redseqtwo \sieve \tm'$.
\item {\bf If there exists a coarse step for $(\redseq,\tm')$.}
  Let $\redex_0$ be the leftmost coarse step for $(\redseq,\tm')$.
  By the preceding observation, $\redex_0$ is also the leftmost coarse step for $(\redseqtwo,\tm')$.

  Note also that $\redseq/\redex_0 \permeq \redseqtwo/\redex_0$,
  as a consequence of the well-known properties of permutation equivalence.
  Moreover, the length of $(\redseq/\redex_0) \sieve (\tm'/\redex_0)$ is
  shorter than the length of $\redseq \sieve \tm'$,
  so by \ih, $(\redseq/\redex_0) \sieve (\tm'/\redex_0) = (\redseqtwo/\redex_0) \sieve (\tm'/\redex_0)$.
  To conclude the proof, note that:
  \[
    \redseq \sieve \tm' = \redex_0 ((\redseq/\redex_0) \sieve (\tm'/\redex_0))
    \eqih \redex_0 ((\redseqtwo/\redex_0) \sieve (\tm'/\redex_0)) = \redseqtwo \sieve \tm'
  \]
\end{enumerate}
\end{proof}

