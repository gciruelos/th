Below the reader can find some auxiliary lemmas used in the proofs of the main
results in \rsec{lattices_and_derivation_spaces}.

\begin{lemma}[Projections are decreasing]
\llem{projections_are_decreasing}
Let $\redex \in \redseq$. Then $\lengthof{\redseq} = 1 + \lengthof{\redseq/\redex}$.
\end{lemma}
\begin{proof}
Observe that $\redex \permle \redseq$ by \rlem{characterization_of_belonging}.
So $\redseq \permeq \redex(\redseq/\redex)$, which gives us that:
\[
  \begin{array}{rcll}
  \lengthof{\redseq} & = & \#\names(\redseq) & \text{ by \rcoro{length_of_derivation_is_number_of_distinct_names}} \\
                     & = & \#\names(\redex(\redseq/\redex)) & \text{ by \rcoro{permutation_equivalence_in_terms_of_names}, since $\redseq \permeq \redex(\redseq/\redex)$} \\
                     & = & \#(\names(\redex) \uplus \names(\redseq/\redex) & \text{ by \rlem{names_concatenation_disjoint_union}} \\
                     & = & 1 + \#\names(\redseq/\redex) \\
                     & = & 1 + \lengthof{\redseq/\redex} & \text{ by \rcoro{length_of_derivation_is_number_of_distinct_names}} \\
  \end{array}
\]
\end{proof}

\begin{lemma}[Properties of disjoint derivations]
\llem{properties_of_disjoint_derivations}
Let $\redseq,\redseqtwo$ be coinitial derivations.
Then the following are equivalent:
\begin{enumerate}
\item $\names(\redseq) \cap \names(\redseqtwo) = \emptyset$.
\item $\redseq \sqcap \redseqtwo = \emptyDerivation$.
\item There is no step common to $\redseq$ and $\redseqtwo$.
\end{enumerate}
In this case we say that $\redseq$ and $\redseqtwo$ are \defn{disjoint}.
\end{lemma}
\begin{proof}
The implication $(1 \implies 2)$ is immediate since if we suppose that $\redseq \sqcap \redseqtwo$ is non-empty
then the first step of $\redseq \sqcap \redseqtwo$ is a step $\redexthree$
such that $\redexthree \in \redseq$ and $\redexthree \in \redseqtwo$.
By Characterization of belonging~(\rlem{characterization_of_belonging}),
this means that $\name(\redexthree) \in \names(\redseq) \cap \names(\redseqtwo)$,
contradicting the fact that $\name(\redexthree)$ and $\names(\redseq)$ are disjoint.

The implication
$(2 \implies 3)$ is immediate by definition of $\redseq \sqcap \redseqtwo$.

Let us check that the implication $(3 \implies 1)$ holds.
By the contrapositive, suppose that $\names(\redseq)$ and $\names(\redseqtwo)$
are not disjoint, and let us show that there is a step common to $\redseq$ and $\redseqtwo$.
Since $\names(\redseq) \cap \names(\redseqtwo) \neq \emptyset$,
we know that the derivation $\redseq$ can be written as $\redseq = \redseq_1 \redex \redseq_2$
where $\name(\redex) \in \names(\redseqtwo)$.
Without loss of generality we may suppose
that $\redex$ is the first step in $\redseq$ with that property,
\ie that $\names(\redseq_1) \cap \names(\redseqtwo) = \emptyset$.
Moreover, let us write $\redseqtwo$ as $\redseqtwo = \redseqtwo_1 \redextwo \redseqtwo_2$ 
where $\name(\redex) = \name(\redextwo)$.

Observe that the name of $\redex$ does not appear anywhere along the sequence of steps $\redseq_1$,
\ie that $\name(\redex) \not\in \names(\redseq_1)$, as a consequence of the
fact that no names are ever repeated in any sequence of steps (\rlem{names_concatenation_disjoint_union}).
This implies that $\name(\redextwo) \not\in \names(\redseq_1/\redseqtwo_1)$.
Indeed:
\[
  \name(\redextwo)
  = \name(\redex)
  \not\in \names(\redseq_1)
  \supseteq \names(\redseq_1) \setminus \names(\redseqtwo_1)
  =^{(\text{\rlem{names_after_projection_along_a_step}})} \names(\redseq_1/\redseqtwo_1)
\]
This means that $\redextwo$ is not erased by the derivation $\redseq_1/\redseqtwo_1$.
More precisely, $\redextwo/(\redseq_1/\redseqtwo_1)$ is a singleton.

Symmetrically, $\redex/(\redseqtwo_1/\redseq_1)$ is a singleton.
Moreover, $\name(\redextwo/(\redseq_1/\redseqtwo_1)) = \name(\redextwo) = \name(\redex) = \name(\redex/(\redseqtwo_1/\redseq_1))$ 
so we have that $\redextwo/(\redseq_1/\redseqtwo_1) = \redex/(\redseqtwo_1/\redseq_1)$.
The situation is the following, where $\names(\redseq_1) \cap \names(\redseqtwo_1) = \emptyset$:
\[
  \xymatrix{
    &
    &
    & \ar@{->>}[ld]_{\redseq_1} \ar@{->>}[rd]^{\redseqtwo_1} &
  \\
    &
    \ar@{->>}[l]_{\redseq_2}
    &
    \ar[l]_{\redex}
    \ar@{->>}[rd]_{\redseqtwo_1/\redseq_1} & & \ar@{->>}[ld]^{\redseq_1/\redseqtwo_1} \ar[r]^{\redextwo}
    &
    \ar@{->>}[r]^{\redseqtwo_2}
    &
  \\
    &
    &
    &
    \ar[d]_{\redex/(\redseqtwo_1/\redseq_1) = \redextwo/(\redseq_1/\redseqtwo_1)}
    &
  \\
    &
    &&&
  }
\]
By Full stability~(\rlem{full_stability}) this means that there exists a step $\redexthree$
such that $\redexthree/\redseq_1 = \redex$ and $\redexthree/\redseqtwo_1 = \redextwo$.
Then $\redexthree \in \redseq_1\redex\redseq_2 = \redseq$
and also $\redexthree \in \redseqtwo_1\redextwo\redseqtwo_2 = \redseqtwo$
so $\redexthree$ is common to $\redseq$ and $\redseqtwo$,
by which we conclude.
\end{proof}


\begin{lemma}
\llem{names_of_meet_included_in_names}
Let $\redseq$ and $\redseqtwo$ be coinitial derivations.
Then $\names(\redseq \sqcap \redseqtwo) \subseteq \names(\redseq)$.
\end{lemma}
\begin{proof}
By induction on the length of $\redseq \sqcap \redseqtwo$:
\begin{enumerate}
\item {\bf Empty, $\redseq \sqcap \redseqtwo = \emptyDerivation$.}
      Then $\names(\redseq \sqcap \redseqtwo) = \emptyset \subseteq \names(\redseq)$ is immediate.
\item {\bf Non-empty, $\redseq \sqcap \redseqtwo = \redexthree(\redseq/\redexthree \sqcap \redseqtwo/\redexthree)$,
           where $\redexthree$ is a step common to $\redseq$ and $\redseqtwo$.}
      Then since $\redexthree$ is common to $\redseq$ and $\redseqtwo$,
      we have that $\name(\redexthree) \in \names(\redseq)$.
      Moreover, by \ih $\names(\redseq/\redexthree \sqcap \redseqtwo/\redexthree) \subseteq \names(\redseq/\redexthree)$.
      So:
      \[
        \begin{array}{rcll}
        \names(\redseq \sqcap \redseqtwo) & = & \set{\name(\redexthree)} \cup \names(\redseq/\redexthree \sqcap \redseqtwo/\redexthree) \\
                                        & \subseteq & \names(\redseq) \cup \names(\redseq/\redexthree) \\
                                        & = & \names(\redseq) \cup (\names(\redseq) \setminus \set{\name(\redexthree)} & \text{by \rlem{names_after_projection_along_a_step}} \\
                                        & = & \names(\redseq)
        \end{array}
      \]
      as required.
\end{enumerate}
\end{proof}

