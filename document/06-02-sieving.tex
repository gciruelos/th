
In the next subsection, we characterize garbage-free derivations by giving
a procedure (``sieving'') that, in some sense, erases all the garbage from a derivation.

\begin{definition}[Sieving]
Let $\tm' \refines \tm$ where $\tm'$ is a correct term,
and let $\redseq : \tm \rtobeta \tmtwo$ be an arbitrary derivation.
A step $\redex$ is \defn{coarse for $(\redseq,\tm')$} if
$\redex \permle \redseq$ and $\redex/\tm' \neq \emptyset$.

We define the \defn{sieve of $\redseq$ with respect to $\tm'$},
written $\redseq \sieve \tm'$, as a derivation in the $\lambda$-calculus,
going out from $\tm$, as follows, considering two cases, depending on whether there exists a
coarse step for $(\redseq,\tm')$.
\begin{itemize}
\item {\bf If there are no coarse steps for $(\redseq,\tm')$.}
  Then
  $(\redseq \sieve \tm') \eqdef \emptyDerivation$.
\item {\bf If there exists a coarse step for $(\redseq,\tm')$.}
  Let $\redex_0$ the leftmost coarse step.
  Then:
  \[
    (\redseq \sieve \tm') \eqdef \redex_0 ((\redseq/\redex_0) \sieve (\tm'/\redex_0))
  \]
  Note that, in the recursive invocation, the expression is well-formed because
  $(\tm'/\redex_0) \refines \src(\redseq/\redex_0)$,
  which is immediate from~\rdef{simulation_residuals_def}.
\end{itemize}
This definition is shown to be well-defined (terminating) in the following lemma.
\end{definition}

\begin{lemma}[Sieving is well-defined]
The operation $\redseq \sieve \tm'$ is well-defined.
\end{lemma}
\begin{proof}
Proving this amounts to showing that the recursion scheme is well-founded.
It suffices to show that there is a measure $M$ ranging over the non-negative
integers such that
$M(\redseq,\tm') > M(\redseq/\redex_0,\tm'/\redex_0)$
whenever $\redex_0$ is the leftmost coarse step for $(\redseq,\tm')$.
Indeed, if we define:
\[
  M(\redseq,\tm') \eqdef \lengthof{\redseq/\tm'}
\]
where $\lengthof{\redseq/\tm'}$ stands for the length of the derivation $\redseq/\tm'$
in the distributive lambda-calculus, then we may conclude by checking that the following inequality holds:
\[
  \lengthof{\redseq/\tm'} > \lengthof{(\redseq/\redex_0)/(\tm'/\redex_0)}
\]
First observe that
$\redex_0 \permle \redseq$
so, by \rcoro{simulation_residuals_and_prefixes},
$\redex_0/\tm' \permle \redseq/\tm'$
and, by \rprop{prefixes_as_subsets},
$\names(\redex_0/\tm') \permle \names(\redseq/\tm')$.
Then we have:
\[
  \begin{array}{rcll}
  \lengthof{\redseq/\tm'}
   & = & \#\names(\redseq/\tm') & \text{by \rcoro{length_of_derivation_is_number_of_distinct_names}} \\
   & > & \#(\names(\redseq/\tm') \setminus \names(\redex_0/\tm')) & \text{since $\redex_0/\tm' \neq \emptyset$ and $\names(\redex_0/\tm') \subseteq \names(\redseq/\tm')$} \\
   & = & \#\names((\redseq/\tm') / (\redex_0/\tm')) & \text{by \rlem{names_after_projection_along_a_step}} \\
   & = & \#\names((\redseq/\redex_0) / (\tm'/\redex_0)) & \text{by \rcoro{permutation_equivalence_in_terms_of_names} and \rlem{generalized_cube_lemma}} \\
   & = & \lengthof{(\redseq/\redex_0) / (\tm'/\redex_0)} & \text{by \rcoro{length_of_derivation_is_number_of_distinct_names}}
  \end{array}
\]
\end{proof}

\begin{lemma}[Sieving is compatible with permutation equivalence]
\llem{sieving_is_compatible_with_permutation_equivalence}
Let $\redseq \permeq \redseqtwo$. Then $\redseq \sieve \tm' \permeq \redseqtwo \sieve \tm'$.
\end{lemma}
\begin{proof}
First observe that, given two permutation equivalent derivations $\redseq$ and $\redseqtwo$,
a step $\redex$ is coarse for $(\redseq,\tm')$ if and only if $\redex$ is coarse for $(\redseqtwo,\tm')$,
since:
\[
  (\redex \permle \redseq) \iff (\redex/\redseq = \emptyset) \iff (\redex/\redseqtwo = \emptyset) \iff (\redex \permle \redseqtwo)
\]
We proceed by induction on the length of $\redseq \sieve \tm'$.
There are two cases, depending on whether there is a coarse step for $(\redseq,\tm')$.
\begin{enumerate}
\item {\bf If there are no coarse steps for $(\redseq,\tm')$.}
  Then there are no coarse steps for $(\redseqtwo,\tm')$,
  so $\redseq \sieve \tm' = \emptyDerivation = \redseqtwo \sieve \tm'$.
\item {\bf If there exists a coarse step for $(\redseq,\tm')$.}
  Let $\redex_0$ be the leftmost coarse step for $(\redseq,\tm')$.
  By the preceding observation, $\redex_0$ is also the leftmost coarse step for $(\redseqtwo,\tm')$.

  Note also that $\redseq/\redex_0 \permeq \redseqtwo/\redex_0$,
  as a consequence of the well-known properties of permutation equivalence.
  Moreover, the length of $(\redseq/\redex_0) \sieve (\tm'/\redex_0)$ is
  shorter than the length of $\redseq \sieve \tm'$,
  so by \ih, $(\redseq/\redex_0) \sieve (\tm'/\redex_0) = (\redseqtwo/\redex_0) \sieve (\tm'/\redex_0)$.
  To conclude the proof, note that:
  \[
    \redseq \sieve \tm' = \redex_0 ((\redseq/\redex_0) \sieve (\tm'/\redex_0))
    \eqih \redex_0 ((\redseqtwo/\redex_0) \sieve (\tm'/\redex_0)) = \redseqtwo \sieve \tm'
  \]
\end{enumerate}
\end{proof}

\begin{lemma}[The sieve is a prefix]
\llem{sieve_is_prefix}
Let $\redseq : \tm \rtobeta \tmtwo$ and $\tm' \refines \tm$.
Then $\redseq \sieve \tm' \permle \redseq$.
\end{lemma}
\begin{proof}
By induction on the length of $\redseq \sieve \tm'$.
There are two cases, depending on whether there exists a coarse step for $(\redseq,\tm')$.
\begin{enumerate}
\item {\bf If there are no coarse steps for $(\redseq,\tm')$.}
  Then trivially $\redseq \sieve \tm' = \emptyDerivation \permle \redseq$.
\item {\bf If there exists a coarse step for $(\redseq,\tm')$.}
  Let $\redseq_0$ be the leftmost coarse step for $(\redseq,\tm')$.
  Then:
  \[
    \begin{array}{rcll}
      \redseq \sieve \tm'
      & = & \redex_0((\redseq/\redex_0) \sieve (\tm'/\redex_0)) \\
      & \permle & \redex_0(\redseq/\redex_0) & \text{by \ih} \\
      & \equiv  & \redseq(\redex_0/\redseq)  & \text{since $A(B/A) \permeq B(A/B)$ in general} \\
      & =       & \redseq                    & \text{since $\redex_0 \permle \redseq$ as $\redex_0$ is coarse for $(\redseq,\tm')$} \\
    \end{array}
  \]
\end{enumerate}
\end{proof}
