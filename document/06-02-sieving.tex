
In this subsection, we will characterize garbage-free derivations by giving
a procedure--\emph{sieving}--that, in some sense, erases all the garbage from a derivation.

\begin{definition}[Sieving]
Let $\tm' \refines \tm$ where $\tm'$ is a correct term,
and let $\redseq : \tm \rtobeta \tmtwo$ be an arbitrary derivation.
A step $\redex$ is \defn{coarse for $(\redseq,\tm')$} if
$\redex \permle \redseq$ and $\redex/\tm' \neq \emptyset$.

We define the \defn{sieve of $\redseq$ with respect to $\tm'$},
written $\redseq \sieve \tm'$, as a derivation in the $\lambda$-calculus,
going out from $\tm$, as follows, considering two cases, depending on whether there exists a
coarse step for $(\redseq,\tm')$.
\begin{itemize}
\item {\bf If there are no coarse steps for $(\redseq,\tm')$.}
  Then
  $(\redseq \sieve \tm') \eqdef \emptyDerivation$.
\item {\bf If there exists a coarse step for $(\redseq,\tm')$.}
  Let $\redex_0$ the leftmost coarse step.
  Then:
  \[
    (\redseq \sieve \tm') \eqdef \redex_0 ((\redseq/\redex_0) \sieve (\tm'/\redex_0))
  \]
  Note that, in the recursive invocation, the expression is well-formed because
  $(\tm'/\redex_0) \refines \src(\redseq/\redex_0)$,
  which is immediate from~\rdef{simulation_residuals_def}.
\end{itemize}
This definition is shown to be well-defined (terminating and well behaved) in the following lemmas.
\end{definition}

\begin{lemma}[Sieving is well-defined]
\llem{sieving_is_well_defined}
The operation $\redseq \sieve \tm'$ is well-defined.
\end{lemma}
\begin{proof}
\SeeAppendixRef{sieving_is_well_defined_proof}
We show that the recursion is well-founded using the measure
  $M(\redseq,\tm') = \lengthof{\redseq/\tm'}$.
\end{proof}

\begin{lemma}[Sieving is compatible with permutation equivalence]
\llem{sieving_is_compatible_with_permutation_equivalence}
Let $\redseq \permeq \redseqtwo$. Then $\redseq \sieve \tm' \permeq \redseqtwo \sieve \tm'$.
\end{lemma}
\begin{proof}
\SeeAppendixRef{sieving_is_compatible_with_permutation_equivalence_proof}
By induction on the length of $\redseq \sieve \tm'$, observing that
$(\redex \permle \redseq) \iff (\redex/\redseq = \emptyset) \iff (\redex/\redseqtwo = \emptyset) \iff (\redex \permle \redseqtwo)$.
\end{proof}

\begin{example}
In \rexample{having_garbage_naive},
$\redextwo \sieve \tm' = \redextwo$
and
$\redextwo\redex_2 \sieve \tm' = \redex_1\redextwo_{11}$.
\end{example}

In the next section we will prove some further results about sieving.
