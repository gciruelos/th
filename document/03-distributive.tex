\chapter{A distributive $\lambda$-calculus}
\lcha{lambdadist}

In this chapter we present a
{\em distributive $\lambda$-calculus} ($\lambdadist$),
and we prove some basic properties it enjoys.
Following that, we give develop a theory of residuals for our calculus and talk
about the structure of the derivation spaces of its terms.
To finish, we establish a relation between our calculus and the pure lambda-calculus
that will allow us to prove interesting results in the next chapter.

Terms of the $\lambdadist$-calculus are typing derivations of a non-idempotent intersection type
system, written using proof term syntax.
The underlying type system is a variant of
system~$\mathcal{W}$ of \cite{bucciarelli2014inhabitation,bucciarelli2017non},
the main difference being that $\lambdadist$
uses {\em labels} and a suitable invariant on terms,
to ensure that the formal parameters of all functions
are in 1--1 correspondence with the actual arguments that they receive.

\section{Types}
We will now present the type system we will work with, which as we said, is a variant of the system presented
in \cite{bucciarelli2017non}.

To make the notation easier, we will write $\typ_1 \cap ... \cap \typ_n$ as the
multiset $\lset{\typ_1, \hdots, \typ_n}$.
Recall that we work in a non idempotent environment so we may have repetitions.

\begin{definition}[Types]
Let $\labelset = \set{\lab,\lab',\lab'',\hdots}$ be a denumerable set of labels.
The sets of \defn{types}, ranged over by $\typ,\typtwo,\typthree,\hdots$,
and \defn{finite sets of types}, ranged over by $\mtyp,\mtyptwo,\mtypthree,\hdots$,
are given mutually inductively by the following abstract syntax:
\[
  \typ ::= \basetyp^{\lab} \mid \mtyp \tolab{\lab} \typ
\]
\[
  \mtyp ::= \lset{\typ_i}_{i=1}^{n} \HS \text{for some $n \geq 0$}
\]
In a type like $\basetyp^{\lab}$ and $\mtyp \tolab{\lab} \typ$,
the label $\lab$ is called the \defn{external label}.
\end{definition}

One remark that deserves mention is that the difference with the system~$\mathcal{W}$
at type-level is that we include labels.
Like in system~$\mathcal{W}$, note that it will not be possible for a term to have multiple types,
like the name \textit{intersection type system} would suggest.
Rather, what happens is that function terms will receive a parameter that can be interpreted
as having several types.

Later we will need to look closely at these types,
for which purpose the following definition will be useful.

\begin{definition}
A type $\typ$ is said to \defn{occur} in another type $\typtwo$ if
$\typ \occursin \typtwo$ holds,
where $(\occursin)$ is the reflexive and transitive closure of
the axioms
$\typ_i \occursin [\typ_1,\hdots,\typ_n] \to \typtwo$ for all $i \in \set{1,\hdots,n}$,
and $\typtwo \occursin [\typ_1,\hdots,\typ_n] \to \typtwo$.
This is extended to say that a type $\typ$ occurs in a multiset $[\typtwo_1,\hdots,\typtwo_n]$,
defined by $\typ \occursin [\typtwo_1,\hdots,\typtwo_n]$ if $\typ \occursin \typtwo_i$
for some $i\in\set{1,\hdots,n}$,
and that a type $\typ$ occurs in a typing context $\tctx$,
defined by $\typ \occursin \tctx$ if $\typ \occursin \tctx(\var)$ for some $\var \in \dom\tctx$.
\end{definition}


\section{Syntax}
Generally, intersection type systems are used on the pure $\lambda$-calculus.
In such environment, a term may be typed:
for example, in system~$\mathcal{W}$,
$\lam{x}{x x}$ can have the type $((\alpha \to \beta) \cap \alpha) \to \beta$,
but the term $\Omega = (\lam{x}{x x}) (\lam{x}{x x})$ cannot
be typed.

In contrast, what we will do in this work is define a slightly different
calculus, such that all well-formed terms of that calculus can be typed using
the non idempotent intersection type system defined above.

The informal idea behind the definition of the terms of the calculus is that
in an application, for each type in the domain of the function there will be
a different argument. So if we have a term $\tm$ with type
$(\typ_1 \cap \typ_2) \to \typ_3$, we will apply it to a \textit{list}
of arguments, one with type $\typ_1$ and the other with type $\typ_2$.

\begin{definition}[Distributive type system]
The set of \defn{distributive terms},
ranged over by ($\tm,\tmtwo,\tmthree,\hdots$) is given by the following abstract syntax:
\[
  \tm ::= \var^{\typ} \mid \lamp{\lab}{\var}{\tm} \mid \tm\,\ls{\tm}
\]
Typing rules are defined inductively as follows.
\[
  \indrule{var}{
  }{
    \var : \lset{\typ} \vdash \var^\typ : \typ
  }
  \indrule{\toI}{
    \tctx \oplus \var : \mtyp \vdash \tm : \typtwo
  }{
    \tctx \vdash \lamp{\lab}{\var}{\tm} : \mtyp \tolab{\lab} \typtwo
  }
\]
\[
  \indrule{\toE}{
    \tctx \vdash \tm : \lset{\typtwo_1,\hdots,\typtwo_n} \tolab{\lab} \typ
    \HS
    \left( \tctxtwo_i \vdash \tmtwo_i : \typtwo_i \right)_{i=1}^{n}
  }{
    \tctx +_{i=1}^{n} \tctxtwo_i \vdash \tm[\tmtwo_1,\hdots,\tmtwo_n] : \typ
  }
\]
Moreover, we introduce a judgment of the form
$[\tctx_1,\hdots,\tctx_n] \vdash [\tm_1,\hdots,\tm_n] : [\typ_1,\hdots,\typ_n]$
with the following rule:
\[
  \indrule{t-multi}{
    \tctx_i \vdash \tm_i : \typ_i \text{ for all $i=1..n$}
  }{
    [\tctx_1,\hdots,\tctx_n] \vdash [\tm_1,\hdots,\tm_n] : [\typ_1,\hdots,\typ_n]
  }
\]
\end{definition}

The most noticeable feature of these terms is that in applications
we do not have an argument that can be typed in many ways (as we did in system~$\mathcal{W}$).
Rather, we have a different term for each type that the function expects its parameter to be.

\begin{example}
Using integer labels,
\[\vdash \lamp{1}{\var}{\var^{[\alpha^2,\alpha^3] \tolab{4} \beta^5}[\var^{\alpha^3},\var^{\alpha^2}]}
: [[\alpha^2,\alpha^3] \tolab{4} \beta^5, \alpha^2, \alpha^3] \tolab{1} \beta^5\]
is a derivable judgment.
\end{example}

Note that writing all the labels can be tiresome
(because writing all the labels is essentially writing a proof that the term is typed),
so we will omit or simplify them when possible.

\subsection*{Correctness}
Observe that the definition we gave has a fatal flaw:
we cannot uniquely associate arguments with variables in the body of the lambdas.
For example, consider the following term.
\[(\lamp{1}{x}{y^{[\alpha^2, \alpha^2] \tolab{3} \alpha^4} [x^{\alpha^2}, x^{\alpha^2}]})
[a^{\alpha^2}, b^{\alpha^2}]\]
We do not know which parameter to associate which each $x$---which parameter goes in the first $x$, $a$ or $b$?

To solve that problem we introduce an invariant that will ensure
that problem does not manifest.
We will call that invariant \defn{correctness}.

Note that the problem is that the function in the application expects two arguments
with exactly the same type. A related problem is that, in the body of the function,
the variable $x$ has the same type twice (remember that in a non idempotent context repetition matters).
In fact, it is enough to ask that those two anomalies do not show up for the system to work.

We will also ask that the labels of the lambdas do not repeat which
will come in handy later.

\begin{definition}[Correct term]
\ldef{sequentiality_and_correctness}
A multiset of types $[\typ_1,\hdots,\typ_n]$ is \defn{sequential}
if the external labels of $\typ_i$ and $\typ_j$ are different for all $i \neq j$.
A typing context $\tctx$ is sequential if $\tctx(\var)$ is sequential for every $\var \in \dom\tctx$.
A term $\tm$ is \defn{correct} if it is typable and it verifies the following three conditions:
\begin{enumerate}
\item {\em Uniquely labeled lambdas.}
  If $\lamp{\lab}{\var}{\tmtwo}$ and $\lamp{\labtwo}{\vartwo}{\tmthree}$
  are sub-terms of $\tm$ at different positions, then $\lab$ and $\labtwo$
  must be different labels.
  \footnote{This is not strictly necessary for our current purpose, but will be useful later.}
\item {\em Sequential contexts.}
  If $\tmtwo$ is a sub-term of $\tm$ and $\tctx \vdash \tmtwo : \typ$
  is derivable, then $\tctx$ must be sequential.
\item {\em Sequential types.}
  If $\tmtwo$ is a sub-term of $\tm$,
  the judgment $\tctx \vdash \tmtwo : \typ$ is derivable,
  and there exists a type such that
  $(\mtyp \tolab{\lab} \typtwo \occursin \tctx)$ or $(\mtyp \tolab{\lab} \typtwo \occursin \typ)$,
  then $\mtyp$ must be sequential.
\end{enumerate}
\end{definition}

Essentially, correctness says that for
any function that appears on the term, its parameters should be uniquely identifiable.

\begin{example}
$\var^{[\alpha^1] \tolab{2} \beta^3}[\var^{\alpha^1}]$ is a correct term.
The example we gave above,
\[(\lamp{1}{x}{y^{[\alpha^2, \alpha^2] \tolab{3} \alpha^4} [x^{\alpha^2}, x^{\alpha^2}]})
[a^{\alpha^2}, b^{\alpha^2}]\]
is not.
One problem is that the type of $y$ is not sequential.
For a last example,
$\lamp{1}{\var}{\lamp{1}{\vartwo}{\vartwo^{\alpha^2}}}$
is not a correct term since labels for lambdas are not unique.
\end{example}

\begin{remark}
We will consider $\termsdist$ to be the set of all correct terms.
In other words, we will only consider correct terms during the rest of the work.
\end{remark}


Having defined the types and the terms of our system
sheds some light on the resource management capabilities that we claimed it will enjoy:
note that we can track very precisely how (\ie with which type) a term will be used or
a bounded variable will be evaluated.
This will prove very useful to analyze the $\lambda$-calculus.

The next lemma shows that a term is uniquely typable: this means that for a given term
there is only one type and typing context that satisfy the typing judgment.
Moreover, all proof trees are the same.

Note that for all proof trees to be the same it is crucial that
we only consider correct terms, because otherwise when we apply the rule
$\toE$ we may have the possibility to choose between different orders for the
parameters.\footnote{This implies that a weaker uniqueness result holds for
incorrect terms: proof trees are equal modulo permutations when the rule $\toE$
is applied, but as we do not care about incorrect terms we do not need to consider this.}

\begin{lemma}[Unique typing]
\llem{unique_typing}
Let $\tm$ be typable, \ie suppose that there exist a context $\tctx$ and a type $\typ$ such that $\tctx \vdash \tm : \typ$. Furthermore, suppose that $\tm$ is correct.
Then there is a unique derivation that types $\tm$.
In particular, if $\tctx' \vdash \tm : \typ'$, then $\tctx = \tctx'$ and $\typ = \typ'$.
\end{lemma}
\begin{proof}
\SeeAppendixRef{unique_typing_proof}
By induction on $\tm$.
\end{proof}



\section{The calculus}
What we want to do know is give the operational semantics for this calculus.
The idea is straightforward: in order apply a lambda abstraction to a list of arguments we just
replace each occurrence of the variable bounded by the lambda with the corresponding argument.

To do that we need to define notation for the type
of ocurrences of a free variable. If $\tm$ is typable,
$\varlabel{\var}{\tm}$
stands for the multiset
of types of the free occurrences of $\var$ in $\tm$.
If $\tm_1,\hdots,\tm_n$ are typable,
$\tmlabel{[\tm_1,\hdots,\tm_n]}$ stands for the multiset
of types of $\tm_1,\hdots,\tm_n$.
For example,
$\varlabel{\var}{\var^{[\alpha^1] \tolab{2} \beta^3}[\var^{\alpha^1}]} =
\tmlabel{[\vartwo^{\alpha^1},\varthree^{[\alpha^1] \tolab{2} \beta^3}]} = [[\alpha^1] \tolab{2} \beta^3,\alpha^1]$.
To perform a substitution $\subs{\tm}{\var}{[\tmtwo_1,\hdots,\tmtwo_n]}$
we will require that $\varlabel{\var}{\tm} = \tmlabel{[\tmtwo_1,\hdots,\tmtwo_n]}$.


Let us define exactly what we mean by \emph{substitution}.
\begin{definition}[Substitution]
\ldef{substitution}
Let $\tm$ and $\tmtwo_1,\hdots,\tmtwo_n$ be correct terms such that $\varlabel{\var}{\tm} = \tmlabel{[\tmtwo_1,\hdots,\tmtwo_n]}$.
The capture-avoiding substitution of $\var$ in $\tm$ by $\ls{\tmtwo} = [\tmtwo_1,\hdots,\tmtwo_n]$
is denoted by $\subs{\tm}{\var}{\ls{\tmtwo}}$ and defined as follows:
\[
  \begin{array}{rcll}
    \subs{\var^\typ}{\var}{[\tmtwo]} & \eqdef & \tmtwo
  \\
    \subs{\vartwo^\typ}{\var}{[]} & \eqdef & \vartwo^\typ
    & \text{ if $\var \neq \vartwo$}
  \\
    \subs{(\lamp{\lab}{\vartwo}{\tmthree})}{\var}{\ls{\tmtwo}} & \eqdef &  \lamp{\lab}{\vartwo}{ \subs{\tmthree}{\var}{\ls{\tmtwo}} }
    & \text{ if $\var \neq \vartwo$ and $\vartwo \not\in \fv{\ls{\tmtwo}}$}
  \\
    \subs{\tmthree_0[\tmthree_j]_{j=1}^{m}}{\var}{\ls{\tmtwo}} & \eqdef &
    \subs{\tmthree_0}{\var}{\ls{\tmtwo}_0}[\subs{\tmthree_j}{\var}{\ls{\tmtwo}_j}]_{j=1}^{m}
  \end{array}
\]
In the last case, $(\ls{\tmtwo}_0, \hdots, \ls{\tmtwo}_m)$
is a partition of $\ls{\tmtwo}$
such that $\varlabel{\var}{\tmthree_j} = \tmlabel{\ls{\tmtwo}_j}$ for all $j \in {0,\hdots,m}$.
\end{definition}

For example,
\[\subs{(\var^{[\alpha^1] \tolab{2} \beta^3}[\var^{\alpha^1}])}{\var}{[\vartwo^{[\alpha^1] \tolab{2} \beta^3},\varthree^{\alpha^1}]}
= \vartwo^{[\alpha^1] \tolab{2} \beta^3}\varthree^{\alpha^1}\]
and
\[\subs{(\var^{[\alpha^1] \tolab{2} \beta^3}[\var^{\alpha^1}])}{\var}{[\vartwo^{\alpha^1},\varthree^{[\alpha^1] \tolab{2} \beta^3}]}
= \varthree^{[\alpha^1] \tolab{2} \beta^3}\vartwo^{\alpha^1}.\]


\begin{remark}
Substitution is {\em type-directed}: arguments $[\tmtwo_1,\hdots,\tmtwo_n]$
are propagated throughout the term
so that $\tmtwo_i$ reaches the free occurrence of $\var$
that has the same type as $\tmtwo_i$.
There exists one such occurrence for each $i \in \set{1, \hdots, n}$ because
$\varlabel{\var}{\tm} = \tmlabel{[\tmtwo_1,\hdots,\tmtwo_n]}$.
Moreover, the fact that $\tm$ is correct ensures that such occurrence is unique,
since $\varlabel{\var}{\tm}$ is sequential.
%Hence there is essentially a unique way to split $\ls{\tmtwo}$
%into $(\ls{\tmtwo}_0, \ls{\tmtwo}_1, \hdots, \ls{\tmtwo}_n)$.
%More precisely, if $(\ls{\tmtwo}_0, \ls{\tmtwo}_1, \hdots, \ls{\tmtwo}_n)$
%and $(\ls{\tmthree}_0, \ls{\tmthree}_1, \hdots, \ls{\tmthree}_n)$
%are two partitions of $\ls{\tmtwo}$ with the stated property,
%then $\ls{\tmtwo}_i$ is a permutation of $\ls{\tmthree}_i$ for all $i \in \set{0,\hdots,n}$.
%It is easy to check by induction on $\tm$
%that the value of $\subs{\tm}{\var}{\ls{\tmtwo}}$ does
%not depend on this choice.
The following lemma formalizes what we just said.
\end{remark}


\begin{lemma}[Substitution is well-defined]
\llem{distrSubstitutionWellDefined}
If $\tctx \oplus \var : \ls{\typtwo} \vdash \tm : \typ$
and $\ls{\tctxtwo} \vdash \ls{\tmtwo} : \ls{\typtwo}$
are derivable,
then $\tctx + \ls{\tctxtwo} \vdash \subs{\tm}{\var}{\ls{\tmtwo}} : \typ$
is derivable.
\end{lemma}
\begin{proof}
By induction on $\tm$, straightforward using the ideas in the last remark.
\end{proof}

Substitution as we presented it can be hard to work with in some environments,
as we have to track how substitution terms get distributed over the term they are
being substituted in.

The following alternative definition of deals with this by
taking advantage of the fact that substitution is type-directed: it
takes all terms all the way down and does not split them in the application case, but rather
``picks'' the correct term to substitute in the base case.

\begin{definition}[Alternative definition of substitution]
Let $\tctx,\var : [\typ_1,\hdots,\typ_n] \vdash \tm : \typtwo$
and let $\tctxtwo_i \vdash \tmtwo_i : \typthree_i$ for each $i \in \set{1,\hdots,m}$
in such a way that $\varlabel{\var}{\tm} \subseteq \tmlabel{[\tmtwo_1,\hdots,\tmtwo_m]}$
and $[\typtwo_1,\hdots,\typtwo_n]$ is sequential.
Let us write $\ls{\tmtwo}$ for $[\tmtwo_1,\hdots,\tmtwo_m]$.
Then an alternative definition for substitution $\tm\sub{\var}{\ls{\tmtwo}}$
may be defined as follows:
\[
  \begin{array}{rcll}
    \var^{\typ}\sub{\var}{\ls{\tmtwo}}                 & \eqdef & \tmtwo_i & \text{ where $i$ is the unique index such that} \\
                                                                         &&& \text{ $\tmlabel{\tmtwo_i} = \typ$}\\
    \vartwo^{\typ}\sub{\var}{\ls{\tmtwo}}              & \eqdef & \vartwo^\typ & \text{ if $\var \neq \vartwo$} \\
    (\lamp{\lab}{\vartwo}{\tm})\sub{\var}{\ls{\tmtwo}} & \eqdef & \lamp{\lab}{\vartwo}{\tm\sub{\var}{\ls{\tmtwo}}} & \text{ if $\var \neq \vartwo$ and there is no capture}\\
    (\tm[\tmthree_i]_{i=1}^{k})\sub{\var}{\ls{\tmtwo}} & \eqdef & \tm\sub{\var}{\ls{\tmtwo}}[\tmthree_i\sub{\var}{\ls{\tmtwo}}]_{i=1}^{k}.
  \end{array}
\]
Moreover,
$[\tm_1,\hdots,\tm_n]\sub{\var}{\ls{\tmtwo}}$
stands for $[\tm_1\sub{\var}{\ls{\tmtwo}},\hdots,\tm_n\sub{\var}{\ls{\tmtwo}}]$,
whenever each substitution $\tm_i\sub{\var}{\ls{\tmtwo}}$ is well-defined.
\end{definition}

\begin{lemma}
$\subs{\tm}{\var}{\ls{\tmtwo}} = \tm\sub{\var}{\ls{\tmtwo}'}$
whenever $\varlabel{\var}{\tm} = \tmlabel{\ls{\tmtwo}}$ and $\ls{\tmtwo}$ is a sublist of $\ls{\tmtwo}'$.
\end{lemma}
\begin{proof}
By induction on $\tm$.
\begin{enumerate}
\item {\bf Variable (same), $\tm = \var^\typ$.}
  First $\subs{\var^\typ}{\var}{[\tmtwo]} = \tmtwo$.
  Also,$\var^\typ\sub{\var}{\ls{\tmtwo}'} = \tmtwo$,
  because $\tmtwo$ is in $\ls{\tmtwo}'$ and it must be the only one for which the external label is the external label of $\typ$.
\item {\bf Variable (different), $\tm = \vartwo$.}
  On the left hand side, $\subs{\vartwo^\typtwo}{\var}{\emptylset} = \vartwo^\typtwo$.
  On the right hand side, $\vartwo^\typtwo\sub{\var}{\ls{\tmtwo}'} = \vartwo^\typtwo$.
\item {\bf Abstraction, $\tm = \lamp{\lab}{\vartwo}{\tmthree}$.}
  On the left, $\subs{(\lamp{\lab}{\vartwo}{\tmthree})}{\var}{\ls{\tmtwo}} = \lamp{\lab}{\vartwo}{\subs{\tmthree}{\var}{\ls{\tmtwo}}}$.
  On the right $(\lamp{\lab}{\vartwo}{\tmthree})\sub{\var}{\ls{\tmtwo}'} = \lamp{\lab}{\vartwo}{\tmthree\sub{\var}{\ls{\tmtwo}}'}$.
  The right-hand side of both equantions are the same by inductive hypothesis.
\item {\bf Application, $\tm = \tmfour [\tmthree_1, \hdots, \tmthree_n]$.}
  On the left side we have that
    $\subs{(\tmfour [\tmthree_i]_{i=1}^n)}{\var}{\ls{\tmtwo}} =
      \subs{\tmfour}{\var}{\ls{\tmtwo}_0} [\subs{\tmthree_i}{\var}{\ls{\tmtwo}_i}]_{i=1}^n$, where
      $\ls{\tmtwo}_0 +_{i=1}^{n} \ls{\tmtwo}_i$ is a permutation of $\ls{\tmtwo}$,
      $\varlabel{\var}{\tm} = \tmlabel{\ls{\tmtwo}_0}$ and
      $\varlabel{\var}{\tmthree_i} = \tmlabel{\ls{\tmtwo}_i}$ for all $i = \set{1,...,n}$.
  On the other hand, $((\tmfour [\tmthree_i]_{i=1}^n))\sub{\var}{\ls{\tmtwo}'} =
      \tmfour\sub{\var}{\ls{\tmtwo}'} [\tmthree_i\sub{\var}{\ls{\tmtwo}'}]_{i=1}^n$.

  Note that for every $i \in \set{0, ..., n}$, $\ls{\tmtwo}_i$ is a sublist of $\ls{\tmtwo}$.
  Then, by inductive hypothesis, $\subs{\tmfour}{\var}{\ls{\tmtwo}_0} = \tmfour\sub{\var}{\ls{\tmtwo}'}$
  and $\subs{\tmthree_i}{\var}{\ls{\tmtwo}_i} = \tmthree_i\sub{\var}{\ls{\tmtwo}'}$, which is what we wanted.
\end{enumerate}
\end{proof}

\begin{lemma}[Substitution lemma for the alternative notion of substitution]
\llem{substitution_lemma_alt}
The following variant of the substitution lemma holds
when $\var \not\in \fv{\ls{\tmthree}}$:
\[
 \tm\sub{\var}{\ls{\tmtwo}}\sub{\vartwo}{\ls{\tmthree}} =
 \tm\sub{\vartwo}{\ls{\tmthree}}\sub{\var}{\ls{\tmtwo}\sub{\vartwo}{\ls{\tmthree}}}
\]
This equation is intended to mean, in particular, that one side is well-defined
if and only if the other side is well-defined.
\end{lemma}
\begin{proof}
Straightforward by induction on $\tm$.
\end{proof}

\begin{remark}
$\conof{\tm}\sub{\var}{\ls{\tmtwo}} = \con\sub{\var}{\ls{\tmtwo}}\of{\tm\sub{\var}{\ls{\tmtwo}}}$.
\end{remark}

\bigskip


\begin{definition}[The $\lambdadist$-calculus]
The \defn{$\lambdadist$-calculus} (distributive $\lambda$-calculus)
is given by the set of correct typable terms $\termsdist$.
For each label $\lab \in \labelset$, we define a reduction relation $\todistl{\lab}\ \subseteq \termsdist \times \termsdist$
as follows:
\[
  \conof{(\lamp{\lab}{\var}{\tm})\vec{\tmtwo}}
  \ \todistl{\lab}\ %
  \conof{\subs{\tm}{\var}{\vec{\tmtwo}}}
\]
where $\con$ stands for a \defn{context}.
The binary relation $\todist$ is the union of all the $\todistl{\lab}$:
\[
  \todist \ \eqdef\ \bigcup_{\lab \in \labelset}\todistl{\lab}
\]
We sometimes drop the subscript for $\todist$, writing just $\to$, when clear from the context.
The set of contexts is given by the grammar:
\[
  \con ::= \conbase \mid \lamp{\lab}{\var}{\con} \mid \con\,\vec{\tm} \mid \tm[\tmtwo_1,\hdots,\tmtwo_{i-1},\con,\tmtwo_{i+1},\hdots,\tmtwo_n]
\]
Contexts can be thought as terms with a single free occurrence of a distinguished variable $\conbase$.
The notation $\conof{\tm}$ stands for the capturing substitution of the occurrence of $\conbase$ in $\con$ by $\tm$.

In general an $n$-hole context is a term $\con$ with exactly $n \geq 0$ free occurrences of the distinguished variable $\conbase$.
If $\con$ is an $n$-hole context, $\conof{\tm_1,\hdots,\tm_n}$ stands for the term
that results from performing the substitution of the $i$-th occurrence of $\conbase_i$ (from left to right)
by $\tm_i$.
If $\con$ is an $n$-hole contexts for some $n$, we say that it is a many-hole context.
\end{definition}


\section{Basic properties}

The first lemma talks about the shape of the typing context of a given typing judgment.
Specifically, that the typing context will contain $\var : \typ$ for each free variable
$\var^\typ$ that occurs in the term, and nothing else.

This lemma will be very useful to prove a lot of upcoming results.

\begin{lemma}[Linearity]
\llem{linearity}
Let $\tm \in \termsdist$ be a correct term,
and let $\tctx \vdash \tm : \typ$ be the (unique) type derivation for $\tm$.
Let $\var$ be any variable,
and consider the $n \geq 0$ free occurrences of the variable $\var$ in the term $\tm$,
more precisely, write $\tm$ as $\tm = \conhat\of{\var^{\typ_1},\hdots,\var^{\typ_n}}$,
where $\conhat$ is a context with $n$-holes such that $\var \not\in \fv{\conhat}$.
\end{lemma}
\begin{proof}
\SeeAppendixRef{linearity_proof}
By induction on $\tm$.
\end{proof}


\begin{remark}
  Given that variables are labeled with their types, it is more or less easy to obtain
  the type of a given (correct and typable) term.
  Moreover, Linearity (\rlem{linearity}) shows that it is easy to obtain the context of the typing judgment.

  In summary, given a term $\tm$ that is correct and typable
  it is straightforward to obtain its typing judgment
  $\tctx \vdash \tm : \typ$.
  It is also straightforward to find out whether the term is typable or not---the
  typability will manifest itself while we try to find $\typ$.
\end{remark}



The following properties show that the rewrite rule $\todist$ is well-defined.

\begin{lemma}[$\termsdist$ is closed under $\todist$]
\llem{closed_under_arrow}
Let $\tm \in \termsdist$ such that $\tm \todist \tm'$. Then $\tm' \in \termsdist$.
\end{lemma}
\begin{proof}
  \SeeAppendixRef{closed_under_arrow_proof}
Let $\tm = \conof{(\lamp{\lab}{\var}{\tmtwo}) \ls{\tmthree}}$. By induction on $\con$.
\end{proof}


\begin{lemma}[Subject reduction]
\llem{subject_reduction}
If $\tctx \vdash \conof{(\lamp{\lab}{\var}{\tm}) \vec{\tmtwo}} : \typ$
then $\tctx \vdash \conof{\subs{\tm}{\var}{\vec{\tmtwo}}} : \typ$.
Moreover, correctness is preserved.
\end{lemma}
\begin{proof}
\SeeAppendixRef{subject_reduction_proof} By induction on $\con$.
\end{proof}

These two lemmas are very important because they show that $\todist$ behaves
well, in particular that it preserves correctness of terms and their types.
That is, if $\tm \to \tm'$ and $\tm$ is correct and well-typed,
we know that $\tm'$ is correct, well-typed and has the same type as $\tm$.


\subsection{Termination}
The $\lambdadist$-calculus happens to be strongly normalizing.
This is because substitution is \emph{linear}, \ie the term
$\subs{\tm}{\var}{[\tmtwo_1, \hdots, \tmtwo_n]}$ uses
$\tmtwo_i$ exactly once for all $i \in \set{1,\hdots,n}$,
hence $\todist$ reduces the number of lambdas of a term in exactly one.


\begin{proposition}[Termination]
\lprop{termination}
There is no infinite reduction $\tm_0 \todist \tm_1 \todist \tm_2 \todist \hdots$.
\end{proposition}
\begin{proof}
It suffices to show that there is a function $d : \termsdist \to \Nat_0$ compatible with $\todist$,
\ie such that $\tm \todist \tmtwo$ implies $d(\tm) > d(\tmtwo)$.
In particular, we will show that taking $d(\tm)$ to be the number of $\lambda$s in $\tm$ works. More precisely:
\begin{equation*}\begin{split}
d(\var^\typ) & \eqdef 0 \\
d(\lamp{\lab}{\var}{\tm}) & \eqdef 1 + d(\tm) \\
d(\tm \lsetenum{\tmtwo}{1}{n}) & \eqdef d(\tm) + \sum_{i=1}^n d(\tmtwo_i) \\
\end{split}\end{equation*}
This definition may be extended to contexts, by taking $d(\conbase) \eqdef 0$.
It is straightforward to show, by induction on $\con$, that $d(\conof{\tm}) = d(\con) + d(\tm)$.

Now, to prove the proposition, suppose that
  \[
    \conof{(\lamp{\lab}{\var}{\tm}) \ls{\tmtwo}} \todist \conof{\subs{\tm}{\var}{\ls{\tmtwo}}}
  \]
We would like to see that
  $d(\conof{(\lamp{\lab}{\var}{\tm}) \ls{\tmtwo}}) > d(\conof{\subs{\tm}{\var}{\ls{\tmtwo}}})$.
But
  \[
    \begin{array}{rcl}
    d(\conof{(\lamp{\lab}{\var}{\tm})}) & = & d(\con) + d((\lamp{\lab}{\var}{\tm}) \ls{\tmtwo}) \\
                                        & = & d(\con) + d(\lamp{\lab}{\var}{\tm}) + \sum_{i=1}^n d(\tmtwo_i) \\
                                        & = & d(\con) + 1 + d(\tm) + \sum_{i=1}^n d(\tmtwo_i) \\
    \end{array}
  \]
Moreover
  \[
    d(\conof{\subs{\tm}{\var}{\ls{\tmtwo}}}) = d(\con) + d(\subs{\tm}{\var}{\ls{\tmtwo}})
  \]
So it suffices to show that $1 + d(\tm) + \sum_{i=1}^n d(\tmtwo_i) > d(\subs{\tm}{\var}{\ls{\tmtwo}})$.
As a matter of fact, a stronger proposition holds:
  $d(\tm) + \sum_{i=1}^n d(\tmtwo_i) = d(\subs{\tm}{\var}{\ls{\tmtwo}})$.
\SeeAppendixRef{termination_proof} We can prove this by induction on $\tm$.
\end{proof}



\begin{corollary}[Bound for the length of derivations]
\lcoro{bounded_length_of_derivations}
Let $\tm' \in \termsdist$ be a correct term.
Then there is a bound for the length of derivations starting at $\tm'$.
\end{corollary}
\begin{proof}
This is an immediate consequence of the fact that the distributive lambda-calculus
is strongly normalizing (\rprop{termination}) and finitely branching, by K\"onig's lemma.
\end{proof}

\begin{remark}
For example, a possible bound---albeit not very good---is the number of lambdas of the term.
This fact stems from the proof of \rprop{termination}.
\end{remark}




\subsection{Confluence}
As we stated in the introduction, the goal of labeling terms and types
is that we obtain a confluent calculus.
Confluence means that every two reduction sequences from a term
can be extended to a common reduct.
Indeed, the $\lambdadist$-calculus
is confluent, and it is the purpose of this section to prove so.


First, we need to prove an adaptation of the Substitution Lemma for our calculus,
which is a key tool to prove properties about coinitial steps.
The substitution lemma for the pure lambda calculus \cite[Lemma 2.1.16]{Barendregt:1984}  states that, provided that
$\var \neq \vartwo$ and $\var \not\in \fv{\tmthree}$, then
  $\subs{\subs{\tm}{\var}{\tmtwo}}{\vartwo}{\tmthree} =
    \subs{\subs{\tm}{\vartwo}{\tmthree}}{\var}{\subs{\tmtwo}{\vartwo}{\tmthree}}$

In our case variables get substituted by a \emph{list} of terms, so we need to adapt it.
Particularly, the list of terms that will take the place of $\tmthree$ needs to be
divided up in several lists: one for the corresponding $\vartwo$s in $\tm$, and the rest
for the $\vartwo$s in each of the elements of $\tmtwo$, which recall that now will be a list.

\begin{notation}
We extend the substitution operator to work on lists,
defining
\[\subs{[\tm_i]_{i=1}^{n}}{\var}{\ls{\tmtwo}} \eqdef [\subs{\tm_i}{\var}{\ls{\tmtwo}_i}]_{i=1}^{n}\]
where $(\ls{\tmtwo}_1,\hdots,\ls{\tmtwo}_n)$ is a partition of $\ls{\tmtwo}$
such that $\varlabel{\var}{\tm_i} = \tmlabel{\ls{\tmtwo}_i}$ for all $i \in \set{1,\hdots,n}$.
\end{notation}

\begin{lemma}[Substitution Lemma]
\llem{substitution_lemma}
Let $\var \neq \vartwo$ and $\var \not\in \fv{\ls{\tmthree}}$.
If $(\ls{\tmthree}_1,\ls{\tmthree}_2)$ is a partition of $\ls{\tmthree}$
then
\[
  \subs{\subs{\tm}{\var}{\ls{\tmtwo}}}{\vartwo}{\ls{\tmthree}}
  =
  \subs{\subs{\tm}{\vartwo}{\ls{\tmthree}_1}}{\var}{\subs{\ls{\tmtwo}}{\vartwo}{\ls{\tmthree}_2}}
\]
provided that both sides of the equation are defined.
{\em Note:} there exists a list $\ls{\tmthree}$ that makes the left-hand side defined
if and only if there exist lists $\ls{\tmthree}_1,\ls{\tmthree}_2$ that make the
right-hand side defined.
\end{lemma}
\begin{proof}
\SeeAppendixRef{substitution_lemma_proof}
By induction on $\tm$.
\end{proof}

\begin{example} For example, consider the term
\[
  \tm =
    (\lamp{\lab}{\var}{\varthree^{[\alpha] \to [\beta] \to \delta} [\var^\alpha] [\var^\beta]})
      [z^\beta,y^\alpha].
\]
We can perform the following substitution (where $w,a,b$ are all variables):
\[
  \subs{
    \subs{\tm}{\varthree}{[w^{[\alpha] \to [\beta] \to \delta}, \vartwo^\beta]}
  }{
    \vartwo
  }{
    [a^\beta,b^\alpha]
  }
  =
   (\lamp{\lab}{\var}{w^{[\alpha] \to [\beta] \to \delta} [\var^\alpha] [\var^\beta]})
    [a^\beta,b^\alpha].
\]
Note that, as we replaced one $\varthree$ by a $\vartwo$, if we wanted to invert the order
of the substitutions we would need to separate the list $[a^\beta, b^\alpha]$ in two:
\[
  \subs{
    \subs{\tm}{\vartwo}{[b^\alpha]}
  }{
    \varthree
  }{
    \subs{[w^{[\alpha] \to [\beta] \to \delta}, \vartwo^\beta]}{\vartwo}{a^\beta}
  }
  =
   (\lamp{\lab}{\var}{w^{[\alpha] \to [\beta] \to \delta} [\var^\alpha] [\var^\beta]})
    [a^\beta,b^\alpha].
\]
\end{example}

\bigskip

Although are going to prove that the $\lambdadist$-calculus is confluent,
actually a strictly stronger property holds: strong permutation.
This property says that if we "open" a diagram with two (different) steps,
we can "close" it with two steps too; and it also gives us information
about which are those closing steps, via their labels.
The fact that this stronger property holds gives us the chance to
develop a functorial residual theory (\cf Full Stability, \rlem{full_stability}),
as we will learn in the next chapter.



\begin{proposition}[Strong Permutation]
\lprop{strong_permutation}
If $\tm_0 \todistl{\lab_1} \tm_1$
and $\tm_0 \todistl{\lab_2} \tm_2$
are different steps, then there exists a term $\tm_3 \in \termsdist$ such that
$\tm_1 \todistl{\lab_2} \tm_3$ and $\tm_2 \todistl{\lab_1} \tm_3$.
Diagrammatically,
\[
    \xymatrix{
      \tm_0 \ar@{->}[d]^{\lab_2}
            \ar@{->}[r]^{\lab_1} &
      \tm_1 \ar@{-->}[d]^{\lab_2} \\
            \ar@{-->}[r]^{\lab_1}
      \tm_2 &
      \tm_3. \\
    }
\]
\end{proposition}
\begin{proof}
Let $\redex : \tm_0 \todistl{\lab} \tm_1$ and $\redextwo : \tm_0 \todistl{\lab'} \tm_2$
be steps going out from $\tm_0$, and let us show that the peak may be closed.
The step $\redex$ is of the form:
\[
  \redex : \tm_0 = \conof{(\lamp{\lab}{\var}{\tm})\ls{\tmtwo}}
           \todistl{\lab} \conof{\subs{\tm}{\var}{\ls{\tmtwo}}} = \tm_1
\]
We proceed by induction on $\con$, and within each case we separate in different
cases depending on where $\redextwo$ is located (which recall that is different
than $\redex$ by hypothesis).
\SeeAppendixRef{strong_permutation_proof}
\end{proof}

\bigskip


The Strong Permutation property may also be called \emph{sub-commutativity}
or \emph{WCR$^{\leq 1}$} in some contexts.
\footnote{In reality, sub-commutativity and WCR$^{\leq 1}$ are slightly different
than strong permutation,
but equivalent:
as they don't ask for the steps to be different,
but allow the closing steps to be at most one (instead of exactly one).}
As a consequence of strong permutation, reduction is sub-commutative,
\ie $(\lefttodist \circ \todist)\ \subseteq (\todist^= \circ \lefttodist^=)$
where
$\lefttodist$ denotes $(\todist)^{-1}$
and $R^=$ denotes the reflexive closure of $R$.

Moreover, it is well-known that sub-commutativity implies confluence,
\ie $(\lefttodist^* \circ \todist^*)\ \subseteq (\todist^* \circ \lefttodist^*)$
(\cf \cite[Proposition 1.1.10]{Terese}). Note that the
inverse does not hold, hence confluence is weaker than strong permutation, as we previously stated.

\begin{corollary}[Confluence]
$\lambdadist$ is confluent, \ie,
if $\tm_0 \tomdist \tm_1$ and $\tm_0 \tomdist \tm_2$, then there exists a term $\tm_3$
such that $\tm_1 \tomdist \tm_3$ and $\tm_2 \tomdist \tm_3$.
Diagrammatically,
\[
    \xymatrix{
      \tm_0 \ar@{->>}[d]
            \ar@{->>}[r] &
      \tm_1 \ar@{-->>}[d] \\
            \ar@{-->>}[r]
      \tm_2 &
      \tm_3. \\
    }
\]
\end{corollary}




