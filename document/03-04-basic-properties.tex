
The first lemma talks about the shape of the typing context of a given typing judgment.
Specifically, that the typing context will contain $\var : \typ$ for each free variable
$\var^\typ$ that occurs in the term, and nothing else.

This lemma will be very useful to prove a lot of upcoming results.

\begin{lemma}[Linearity]
\llem{linearity}
Let $\tm \in \termsdist$ be a correct term,
and let $\tctx \vdash \tm : \typ$ be the (unique) type derivation for $\tm$.
Let $\var$ be any variable,
and consider the $n \geq 0$ free occurrences of the variable $\var$ in the term $\tm$,
more precisely, write $\tm$ as $\tm = \conhat\of{\var^{\typ_1},\hdots,\var^{\typ_n}}$,
where $\conhat$ is a context with $n$-holes such that $\var \not\in \fv{\conhat}$.
\end{lemma}
\begin{proof}
\SeeAppendixRef{linearity_proof}
By induction on $\tm$.
\end{proof}


\begin{remark}
  Given that variables are labeled with their types, it is more or less easy to obtain
  the type of a given (correct and typable) term.
  Moreover, Linearity (\rlem{linearity}) shows that it is easy to obtain the context of the typing judgment.

  In summary, given a term $\tm$ that is correct and typable
  it is straightforward to obtain its typing judgment
  $\tctx \vdash \tm : \typ$.
  It is also straightforward to find out whether the term is typable or not---the
  typability will manifest itself while we try to find $\typ$.
\end{remark}



The following properties show that the rewrite rule $\todist$ is well-defined.

\begin{lemma}[$\termsdist$ is closed under $\todist$]
\llem{closed_under_arrow}
Let $\tm \in \termsdist$ such that $\tm \todist \tm'$. Then $\tm' \in \termsdist$.
\end{lemma}
\begin{proof}
  \SeeAppendixRef{closed_under_arrow_proof}
Let $\tm = \conof{(\lamp{\lab}{\var}{\tmtwo}) \ls{\tmthree}}$. By induction on $\con$.
\end{proof}


\begin{lemma}[Subject reduction]
\llem{subject_reduction}
If $\tctx \vdash \conof{(\lamp{\lab}{\var}{\tm}) \vec{\tmtwo}} : \typ$
then $\tctx \vdash \conof{\subs{\tm}{\var}{\vec{\tmtwo}}} : \typ$.
Moreover, correctness is preserved.
\end{lemma}
\begin{proof}
\SeeAppendixRef{subject_reduction_proof} By induction on $\con$.
\end{proof}

These two lemmas are very important because they show that $\todist$ behaves
well, in particular that it preserves correctness of terms and their types.
That is, if $\tm \to \tm'$ and $\tm$ is correct and well-typed,
we know that $\tm'$ is correct, well-typed and has the same type as $\tm$.
