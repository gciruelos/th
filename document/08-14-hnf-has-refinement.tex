
We had a term $\tm \in \terms$ that is in head normal form,
\ie $\tm = \lam{\var_1}{\hdots\lam{\var_n}{\vartwo\,\tm_1\,\hdots\,\tm_m}}$.

We wanted to prove that:
  \begin{itemize}
    \item there exists $\tm' \in \termsdist$ such that $\tm' \refines \tm$,
    \item for every application in $\tm'$ the argument list is empty, and
    \item if $\var_i = \vartwo$ for any $i$, $\tctx$ will be empty,
      otherwise it will be a singleton of the form $\set{\vartwo : [\typ]}$.
  \end{itemize}

\begin{proof}
We proceed by induction on the pair $(n, m)$, that is, by induction on
  $\mathbb{N}^2$ with lexicographic order.
\begin{enumerate}
  \item {\bf Base $(0, 0)$.} In this case $\tm = \vartwo$,
    so we can simply choose $\tm'$ to be $\vartwo^\typ$.
  \item {\bf Inductive step.} We want to see that the property holds for
    $(n, m) > (0,0)$ given that it holds for all $(n', m') < (n, m)$.
    Here there are two cases, either $n = 0$ or $n > 0$.
    \begin{enumerate}
      \item {\bf $n = 0$.} We have that $\tm = \vartwo\,\tm_1\,\hdots\,\tm_m$, $m \geq 1$.
        Let $\tmtwo = \vartwo\,\tm_1\,\hdots\,\tm_{m-1}$.
        By inductive hypothesis, there exists a correct term $\tmtwo'$ that
          refines $\tmtwo$.

        We can suppose, without loss of generality, that
          $\tmtwo' = \vartwo^\typ\,\ls{\tm}_1\,\hdots\,\ls{\tm}_{m-1}$,
        where each element of $\ls{\tm}_i$ refines $\tm_i$.
        If $\tmtwo'$ wasn't like that, it simply would not refine $\tmtwo$.
        If $m = 1$, then $\tmtwo' = \vartwo$ and the type of $\vartwo$ is $\typ$.

        If $m > 1$, the inductive hypothesis tells us that all $\ls{\tm}_i$ are empty, so
          $\tmtwo' = \vartwo \emptylset_1 \hdots \emptylset_{m-1}$
          (we number them to make it more clear which is which).
        Because of Unique typing, the type of $\vartwo$ has to be
          $\emptylset_1 \to \hdots \to \emptylset_{m-1} \to \typ_1$,
        where $\typ_1$ is the type of $\tmtwo'$.
        Diagramatically,
        \[
          \indrule{\toE}{
            \set{\vartwo : \emptylset_1 \to \hdots \to \emptylset_{m-1} \to \typ_1}
              \vdash
              \vartwo^{\emptylset_1 \to \hdots \to \emptylset_{m-1} \to \typ_1}
                \,\emptylset_1\,\hdots\,\emptylset_{m-2}
              : \emptylset_{m-1} \to \typ_1
          }{
            \set{\vartwo : \emptylset_1 \to \hdots \to \emptylset_{m-1} \to \typ_1}
              \vdash \vartwo^{\emptylset_1 \to \hdots \to \emptylset_{m-1} \to \typ_1}
                \,\emptylset_1\,\hdots\,\emptylset_{m-1}
              : \typ_1
          }
        \]
        Now, we can give a typing derivation for the following term,
          if we consider a derivation very similar to the last one but
          where $\typ_1$ has been replaced by $\emptylset \to \typ_2$.
        \[
          \indrule{\toE}{
            \set{\vartwo : \emptylset_1 \to \hdots \to \emptylset_{m-1} \to \emptylset \to \typ_2}
              \vdash \vartwo^{\emptylset_1 \to \hdots \to \emptylset_{m-1} \to \emptylset \to \typ_2}
                \,\emptylset_1\,\hdots\,\emptylset_{m-1}
              : \emptylset \to \typ_2
          }{
            \set{\vartwo : \emptylset_1 \to \hdots \to \emptylset_{m-1} \to \emptylset \to \typ_2} \vdash \vartwo^{\emptylset_1 \to \hdots \to \emptylset_{m-1} \to \emptylset \to \typ_2}\,\emptylset_1\,\hdots\,\emptylset_{m-1}\,\emptylset : \typ_2
          }
        \]
        Let $\tm' = \vartwo^{\emptylset_1 \to \hdots \to \emptylset_{m-1} \to \emptylset \to \typ_2}\,\emptylset_1\,\hdots\,\emptylset_{m-1}\,\emptylset$.

        Note that the derivation fo $\tm'$ can be completed very easily in the
          same way that $\tmtwo'$ was done, and it's easy to check that $\tm'$
          is correct (contexts are sequential, there are no lambdas, and types
          are sequential, because the derivation is essentially the same that
          the one of $\tmtwo'$).
        Moreover, $\tm'$ refines $\tm$, so we are done.
      \item {\bf $n > 0$.} We have that $\tm = \lam{\var_1}{\hdots\lam{\var_n}{\vartwo\,\tm_1\,\hdots\,\tm_m}}$.
        Let $\tmtwo = \lam{\var_2}{\hdots\lam{\var_n}{\vartwo\,\tm_1\,\hdots\,\tm_m}}$.
        By inductive hypothesis, there is a term $\tctx \vdash \tmtwo' : \typ$ that refines $\tmtwo$.
        The same way as before, we know that
          $\tmtwo' = \lamp{\lab_2}{\var_2}{\hdots\lamp{\lab_n}{\var_n}{\vartwo\,\emptylset\,\hdots\,\emptylset}}$.

        Let $\tm' = \lamp{\lab}{\var_1}{\tmtwo'}$, where $\lab$ is fresh.
        If $\var_1 = \vartwo$ and $\var_1 \neq \var_i$ for all $1 < i \leq n$,
          then $\lab$ is the external label of $\vartwo$, and in any other case it is a fresh label.
        Here to make the proof tidier we will divide in two new cases,
          depending on whether $\var_1 = \vartwo$ and $\var_1 \neq \var_i$ for all $1 < i \leq n$, or not.
        \begin{enumerate}
          \item {\bf $\var_1 = \vartwo$ and $\var_1 \neq \var_i$ for all $1 < i \leq n$.}
            In this case, by inductive hypothesis $\tctx$ must be a singleton $\set{y : \typ_1}$.
            \[
              \indrule{\toI}{
                \set{\vartwo : \typ_1} \vdash \tmtwo' : \typ
              }{
                \emptyset \vdash \lamp{\lab}{\var_1}{\tmtwo'} : [\typ_1] \tolab{\lab} \typ
              }
            \]
            We have that $\tm' \refines \tm$, and that the derivation of $\tm'$ has sequential types and contexts,
              because by inductive hypothesis $\tmtwo'$ does.
            Finally, $\tm'$ has pairwise distinct labels in all lambdas because $\tmtwo'$ does and $\lab$
              was chosen to be a fresh label.
            \item {\bf $\var_1 \neq \vartwo$ or $x_1 = x_i$ for some $1 < i \leq n$.}
            \[
              \indrule{\toI}{
                \set{\vartwo : \typ_1} \vdash \tmtwo' : \typ
              }{
                \set{\vartwo : \typ_1} \vdash \lamp{\lab}{\var_1}{\tmtwo'} : \emptylset \tolab{\lab} \typ
              }
            \]
            Like in the previous case, we have that $\tm' \refines \tm$, and that the derivation of
              $\tm'$ has sequential types and contexts, because by inductive hypothesis
              $\tmtwo'$ does, and that $\tm'$ has pairwise distinct labels in all lambdas
              because $\tmtwo'$ does and $\lab$ was chosen to be a fresh label.
        \end{enumerate}
    \end{enumerate}
\end{enumerate}
\end{proof}
