
Recall from the preliminaries that some abstract rewriting systems have the property of being
\emph{orthogonal}.
Being orthtogonal entails a myriad of properties and results that make working
with orthogonal rewrite systems very easy.
Informally, in an orthogonal rewrite system residuals behave and have the properties
that one would expect, some of them are summarized in the table that follows.

{\footnotesize
\[
\begin{array}{|c|c|c|}
\hline
\begin{array}{r@{\hspace{0.2cm}}c@{\hspace{0.2cm}}l}
  \emptyDerivation\,\redseq & = & \redseq
\\
  \redseq\,\emptyDerivation & = & \redseq
\\
  \emptyDerivation/\redseq & = & \emptyDerivation
\\
  \redseq/\emptyDerivation & = & \redseq
\\
  \redseq/\redseqtwo\redseqthree & = & (\redseq/\redseqtwo)/\redseqthree
\\
  \redseq\redseqtwo/\redseqthree & = & (\redseq/\redseqthree)(\redseqtwo/(\redseqthree/\redseq))
\\
  \redseq/\redseq & = & \emptyDerivation
\end{array}
&
\begin{array}{r@{\hspace{0.2cm}}c@{\hspace{0.2cm}}l}
  \redseq \permle \redseqtwo & \iffdef & \redseq/\redseqtwo = \emptyDerivation
\\
  \redseq \permeq \redseqtwo & \iffdef & \redseq \permle \redseqtwo \land \redseqtwo \permle \redseq
\\
  \redseq \sqcup \redseqtwo  & \eqdef & \redseq(\redseqtwo/\redseq) 
\\
  \redseq \permeq \redseqtwo & \implies & \redseqthree/\redseq = \redseqthree/\redseqtwo
\\
  \redseq \permle \redseqtwo & \iff & \exists \redseqthree.\ \redseq\redseqthree \permeq \redseqtwo
\\
  \redseq \permle \redseqtwo & \iff & \redseq \sqcup \redseqtwo \permeq \redseqtwo
\end{array}
&
\begin{array}{r@{\hspace{0.2cm}}c@{\hspace{0.2cm}}l}
  \redseq \permle \redseqtwo & \implies & \redseq/\redseqthree \permle \redseqtwo/\redseqthree
\\
  \redseq \permle \redseqtwo & \iff & \redseqthree\redseq \permle \redseqthree\redseqtwo
\\
  \redseq \sqcup \redseqtwo  & \permeq & \redseqtwo \sqcup \redseq
\\
  (\redseq \sqcup \redseqtwo) \sqcup \redseqthree & = & \redseq \sqcup (\redseqtwo \sqcup \redseqthree)
\\
  \redseq & \permle & \redseq \sqcup \redseqtwo
\\
  (\redseq \sqcup \redseqtwo)/\redseqthree & = & (\redseq/\redseqthree) \sqcup (\redseqtwo/\redseqthree) \\
\end{array}
\\
\hline
\end{array}
\]
}

As we stated in the preliminaries, one must check four axioms in order to prove that
a given system (in this case $\lambdadist$) is orthogonal.

Let's see why the axiom called Finite Developments is interesting.
As stated in \cite{thesismellies}, the \emph{Axiome FD}, or finite developments axiom,
asks that for every set of coinitial steps $\redexset$, then all developments of $\redexset$
are finite.

This axiom is important because suppose we have two coinitial steps $\redex$ and $\redextwo$,
such that $\redex : \tm \to \tmtwo$. Then $\redextwo / \redex$ is a set of steps with
the same source ($\tmtwo$). The idea is that if we had $\tm$ and wanted to ``execute'' both
$\redex$ and $\redextwo$, if we didn't have finite developments, then in particular
we don't have finite complete developments, so a reduction that tries to
execute what's left of $\redextwo$ after $\redex$ may not finish.

The $\lambdadist$-calculus not only enjoys of finite developments, but we can
give the exact length a complete development of a set of steps will have.

\begin{lemma}[Finite developments]
\llem{finite_developments}
Let $\redexset$ be a set of coinitial steps.
Then the length of every complete development of $\redexset$
is precisely the cardinality of $\redexset$.
In particular, developments are finite.
\end{lemma}
\begin{proof}
By induction on the cardinality of $\redexset$.
If $\redexset = \emptyset$, the only complete development of $\redexset$ is $\emptyDerivation$ and we are done.
Otherwise, if $\redseq$ is a complete development of $\redexset$, it is a non-empty derivation, \ie
$\redseq = \redex\redseqtwo$
where $\redex \in \redexset$
and such that $\redseqtwo$ is a complete development of $\redexset/\redex$.
Since residuals of distinct redexes have distinct names (and hence they are distinct)
we have that
$
  \redexset/\redex = \uplus_{\redextwo \in \redexset} (\redextwo/\redex)
$,
where $\uplus$ denotes the disjoint union of sets.
Moreover, $\#(\redextwo/\redex) = 1$ if and only if $\redex \neq \redextwo$
by \rlem{cardinality_of_the_set_of_residuals}, so:
\[
  \#(\redexset/\redex) =
  \Sigma_{\redextwo \in \redexset} \#(\redextwo/\redex) =
  \#(\redexset \setminus \set{\redex}) =
  \#(\redexset) - 1
\]
Hence by \ih the length of $\redseqtwo$ is $\#(\redexset) - 1$
and we conclude.
\end{proof}

\begin{proposition}
The distributive lambda-calculus is an Orthogonal Axiomatic Rewrite System
in the sense of Melli\`es.
\end{proposition}
\begin{proof}
There are four axioms to check:
\begin{enumerate}
\item {\bf Autoerasure.}
  Immediate from the cardinality of residuals lemma (\rlem{cardinality_of_the_set_of_residuals}).
\item {\bf Finite Residuals.}
  Immediate from the cardinality of residuals lemma (\rlem{cardinality_of_the_set_of_residuals}).
\item {\bf Finite Developments.}
  Proved in the Finite Developments lemma (\rlem{finite_developments}).
\item {\bf Semantic Orthogonality.}
  A consequence of the Strong Permutation property (\rprop{strong_permutation}).
\end{enumerate}
\end{proof}



\bigskip
\begin{example}
\lexample{derivation-ids-ex}
Next we present an example that will be useful to illustrate
some of the upcoming propositions. Consider the following term:

\[
  (\lamp{\lab_1}{\var}{\var^{[\alpha] \to \alpha} [\var^\alpha])}
    [\lamp{\lab_2}{\vartwo}{\vartwo^\alpha},
    (\lamp{\lab_3}{\varthree}{\varthree^\alpha}) [w^\alpha]].
\]

What follows is its derivation space, \ie,
all possible derivations that have the previous term as a source.

{\footnotesize
\xymatrix{
  &
  (\lamp{\lab_1}{\var}{\var^{[\alpha] \to \alpha} [\var^\alpha])}
    [\lamp{\lab_2}{\vartwo}{\vartwo^\alpha},
    (\lamp{\lab_3}{\varthree}{\varthree^\alpha}) [w^\alpha]]
    \ar@[->][rd]^{\lab_3}_{\redextwo}
    \ar@[->][ld]_{\lab_1}^{\redex}
  & \\
  (\lamp{\lab_2}{\vartwo}{\vartwo^\alpha})
   ((\lamp{\lab_3}{\varthree}{\varthree^\alpha}) [w^\alpha])
   \ar@[->][rd]_{\lab_3}^{\redextwo'}
   \ar@[->][dd]_{\lab_2}^{\redexthree}
   &
   &
  (\lamp{\lab_1}{\var}{\var^{[\alpha] \to \alpha} [\var^\alpha])}
    [\lamp{\lab_2}{\vartwo}{\vartwo^\alpha},
    w^\alpha]
   \ar@[->][ld]^{\lab_1}_{\redex'} \\
  &
  (\lamp{\lab_2}{\vartwo}{\vartwo^\alpha})  [w^\alpha]
  \ar@[->][d]_{\lab_2}^{\redexthree'}
  & \\
  (\lamp{\lab_3}{\varthree}{\varthree^\alpha})  [w^\alpha]
  \ar@[->][r]_{\lab_3}^{\redextwo''}
  &
  w^\alpha
  & \\}
}
\end{example}

Some facts about these derivations:

\begin{itemize}
  \item $\redex$ creates $\redexthree$: we cannot perform $\redexthree$ until we performed $\redex$. We will expand on the topic of creation later.
  \item $\redex \redextwo' \permeq \redextwo \redex'$. Remember that two derivations $\redseq$,
    $\redseqtwo$ are permutation equivalent if $\redseq \permle \redseqtwo \permle \redseq$,
    \ie if they perform the same amount of work.
  \item $\redextwo / \redex = \set{\redex''}$ and
    $\redextwo / \redex \redexthree = \set{\redextwo''}$. Remember that in $\lambdadist$,
    residuals are either empty or singletons, so we will often skip the parentheses.
  \item $\name(\redexthree) = \name(\redexthree') = \lab_3$.
\end{itemize}
