
\begin{lemma}[Finite developments]
\llem{finite_developments}
Let $\redexset$ be a set of coinitial steps.
Then the length of every complete development of $\redexset$
is precisely the cardinality of $\redexset$.
In particular, developments are finite.
\end{lemma}
\begin{proof}
By induction on the cardinality of $\redexset$.
If $\redexset = \emptyset$, the only complete development of $\redexset$ is $\emptyDerivation$ and we are done.
Otherwise, if $\redseq$ is a complete development of $\redexset$, it is a non-empty derivation, \ie
$\redseq = \redex\redseqtwo$
where $\redex \in \redexset$
and such that $\redseqtwo$ is a complete development of $\redexset/\redex$.
Since residuals of distinct redexes have distinct names (and hence they are distinct)
we have that
$
  \redexset/\redex = \uplus_{\redextwo \in \redexset} (\redextwo/\redex)
$,
where $\uplus$ denotes the disjoint union of sets.
Moreover, $\#(\redextwo/\redex) = 1$ if and only if $\redex \neq \redextwo$
by \rlem{cardinality_of_the_set_of_residuals}, so:
\[
  \#(\redexset/\redex) =
  \Sigma_{\redextwo \in \redexset} \#(\redextwo/\redex) =
  \#(\redexset \setminus \set{\redex}) =
  \#(\redexset) - 1
\]
Hence by \ih the length of $\redseqtwo$ is $\#(\redexset) - 1$
and we conclude.
\end{proof}

\begin{proposition}
The distributive lambda-calculus is an Orthogonal Axiomatic Rewrite System
in the sense of Melli\`es.
\end{proposition}
\begin{proof}
There are four axioms to check:
\begin{enumerate}
\item {\bf Autoerasure.}
  Immediate from the cardinality of residuals lemma (\rlem{cardinality_of_the_set_of_residuals}).
\item {\bf Finite Residuals.}
  Immediate from the cardinality of residuals lemma (\rlem{cardinality_of_the_set_of_residuals}).
\item {\bf Finite Developments.}
  Proved in the Finite Developments lemma (\rlem{finite_developments}).
\item {\bf Semantic Orthogonality.}
  A consequence of the Strong Permutation property (\rprop{strong_permutation}).
\end{enumerate}
\end{proof}

\begin{definition}
A step $\redex$ \defn{belongs to} a coinitial derivation $\redseq$,
written $\redex \in \redseq$,
if and only if some residual of $\redex$ is contracted along $\redseq$. 
More precisely, $\redex \in \redseq$ if there exist $\redseq_1,\redex',\redseq_2$
such that $\redseq = \redseq_1\redex'\redseq_2$ and $\redex' \in \redex/\redseq_1$.
We write $\redex \not\in \redseq$ if it is not the case that $\redex \in \redseq$.
\end{definition}


\subsubsection{Examples}

Before going any further, we need to get some more residual theory definitions established.

The first is the concept of \defn{permutation equivalence}.
Permutation equivalence is a sense of equivalence for derivations.
Intuitively, 

