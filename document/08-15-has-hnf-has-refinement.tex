\TODO{fix this}

\begin{lemma}
\llem{hnf_iff_distnf}
Let $\tm \in \terms$ and $\tm' \in \termsdist$ such that $\tm \refinedby  \tm'$.
Then $\tm$ is in head normal form if and only if $\tm'$ is in head normal form.
\end{lemma}
\begin{proof}
  We will do each direction separately.
\begin{itemize}
  \item[]{\bf $\Rightarrow$.}
    We proceed by induction on $\tm$.
    \begin{enumerate}
    \item \Case{Variable, $\tm = \vartwo$}
      Then $\tm'$ is necessarily $\vartwo^\typ$.
    \item \Case{Abstraction, $\tm = \lam{\var_1}{\tmthree}$}
      In this case, as $\tm'$ refines $\tm$, $\tm' = \lamp{\lab}{\var_1}{\tmthree'}$,
        where $\tmthree' \refines \tmthree$.
      Note that $\tmthree$ is in head normal form, so by inductive hypothesis we get that $\tmthree'$ is too.
        Then $\tm'$ is in head normal form.
    \item \Case{Application, $\tm = \tmthree \tmfour$}
      Here, $\tm' = \tmthree' [\tmfour'_1, ..., \tmfour'_n]$,
        where $\tmthree'$ refines $\tmthree$, and $\tmfour'_i$ refines $\tmfour$ for every $i$.


      Note that as $\tm$ is in head normal form, then $\tmthree$ also is,
        because it is of the form $\vartwo \tm_1 \hdots \tm_m$.
        Then, by inductive hypothesis, we know that $\tmthree'$ is in head normal form, too.
        Furthermore, because of the
        shape that $\tmthree$ has, we know $\tmthree'$ does not have a lambda at the root
        (because if it had it would not refine $\tmthree$).
        So $\tmthree' = \vartwo \ls{\tm}_1 \hdots \ls{\tm}_m$ (with $m$ potentially $0$).

      All this allows us to arrive at the conclusion that
        $\vartwo \ls{\tm}_1 \hdots \ls{\tm}_m [\tmfour'_1, ..., \tmfour'_n]$, \ie $\tm'$, is in head normal form.
    \end{enumerate}

  \item[]{\bf $\Leftarrow$.} Analogous to $\Rightarrow$.
\end{itemize}
\end{proof}





\begin{lemma}[Backwards simulation]
\TODO{Arreglar o cambiar demo o lema}
\llem{backwards_simulation}
Let $\tm,\tmtwo \in \terms$ be lambda-terms and $\tmtwo' \in \termsdist$ be a distributive term such that:

\[
\xymatrix@C=1cm{
 \tm \ar[r]^{\beta} & \tmtwo \ar@{}[d]|*=0[@]{\rtimes} \\
                    & \tmtwo' \\
}
\]

Then there exists a term $\tm' \in \termsdist$ such that
\[
\xymatrix@C=1cm{
 \tm \ar@{}[d]|*=0[@]{\rtimes} \ar[r]^{\beta} & \tmtwo \ar@{}[d]|*=0[@]{\rtimes} \\
 \tm' \ar@{.>>}[r]^{\dist} & \tmtwo' \\
}
\]
Moreover, the proof is constructive.
\end{lemma}
\begin{proof}
By induction on $\tm$.
\begin{enumerate}
\item {\bf Variable (same), $\tm = \var$.} This case cannot happen, as $\tm$ has no redexes.
\item {\bf Abstraction, $\tm = \lam{\vartwo}{\tmthree}$.}
  In this case $\tmtwo = \lam{\vartwo}{\tmfour}$, where $\tmthree \to \tmfour$.
  Then $\tmtwo'$ must be of the form $\lamp{\lab}{\vartwo}{\tmfour'}$, where $\tmfour' \refines \tmfour$.
  So, using the inductive hypothesis we get that there is a term $\tmthree'$
    such that $\tmthree' \refines \tmthree$ and $\tmthree' \tomulti_\dist \tmfour'$.
  So, if we take $\tm'$ to be $\lamp{\lab}{\vartwo}{\tmthree'}$ we are done.
  \[
  \xymatrix@C=1cm{
    \lam{\vartwo}{\tmthree} \ar@{}[d]|*=0[@]{\rtimes} \ar[r]^{\beta} & \lam{\vartwo}{\tmfour} \ar@{}[d]|*=0[@]{\rtimes} \\
    \lamp{\lab}{\vartwo}{\tmthree'} \ar@{->>}[r]^{\dist} & \lamp{\lab}{\vartwo}{\tmfour'} \\
  }
  \]

\item {\bf Application, $\tm = \tmthree \tmfour$.}
  Here there are three subcases depending on where the redex is.
  \begin{enumerate}
    \item {\bf The redex is in $\tmthree$.} This case is straightforward using the inductive hypothesis, much like the {\bf Abstraction} case.
    \item {\bf The redex is in $\tmfour$.}
      We have that $\tmthree \tmfour \to \tmthree \overline{\tmfour}$,
      and that $\tmtwo' = \tmthree' [\overline{\tmfour}_1, \hdots, \overline{\tmfour}_n]$,
      where $\tmthree' \refines \tmthree$ and $\overline{\tmfour}_i \refines \overline{\tmfour}$.

      Note that for every $i$ we have that $\tmfour \to \overline{\tmfour} \refinedby \overline{\tmfour}_i$,
      so we can use the inductive hypothesis, which tells us that there is a $\tmfour_i$ that refines $\tmfour$
      and reduces to $\overline{\tmfour}_i$.
      So we get what we wanted
      \[
      \xymatrix@C=1cm{
        \tmthree \tmfour \ar@{}[d]|*=0[@]{\rtimes} \ar[r]^{\beta} & \tmthree \overline{\tmfour} \ar@{}[d]|*=0[@]{\rtimes} \\
        \tmthree'[\tmfour_1, \hdots, \tmfour_n] \ar@{->>}[r]^{\dist} & \tmthree'[\overline{\tmfour}_1, \hdots, \overline{\tmfour}_n] \\
      }
      \]
    \item {\bf The redex is at the head.}
      We have that $\tmthree = \lam{\var}{\overline{\tmthree}}$, and
        $\tmthree \tmfour \to \subs{\overline{\tmthree}}{\var}{\tmfour}$.
        As $\tmtwo' \refines \subs{\overline{\tmthree}}{\var}{\tmfour}$,
        by \rlem{refinement_substitution_inverse} we know that
        $\tmtwo' = \subs{\overline{\tmthree}'}{\var}{[\tmfour'_1, \hdots, \tmfour'_n]}$,
        where $\overline{\tmthree}' \refines \overline{\tmthree}$ and $\tmfour'_i \refines \tmfour$.

      Taking $\tm'$ to be $(\lamp{\lab}{\var}{\overline{\tmthree}'})[\tmfour'_1, \hdots, \tmfour'_n]$
      covers our needs, because it reduces to $\tmtwo'$ and refines $\tm$.
  \end{enumerate}
\end{enumerate}
\end{proof}

\bigskip

Let $\tm \in \terms$.
Then $\tm$ has a head normal form if and only if there exists $\tm' \in \termsdist$ such that $\tm \refinedby \tm'$.

\begin{proof} We prove both directions separately.
\begin{itemize}
  \item[] {\bf $\Rightarrow$.} If $\tm$ has a head normal form that means that there is a derivation $\tm \rtobeta \tmtwo$ such that $\tmtwo$ is a head normal form.
    As such, by \rlem{hnf_has_refinement}, it has a refinement, which we will call $\tmtwo'$.

    We will prove that if $\tm \rtobeta \tmtwo \refinedby \tmtwo'$ then
    there exists a $\tm'$ that refines $\tm$, and we will proceed by
    induction on the length of the derivation $\tm \rtobeta \tmtwo$.
    Note that this is enough
    \begin{enumerate}
      \item {\bf 0 steps, $\tm = \tmtwo$.} This case is trivial because we take $\tm'$ to be $\tmtwo'$.
      \item {\bf Inductive case, $\tm \rtobeta \tmthree \tobeta \tmtwo$.}
        Using backwards simulation (\rlem{backwards_simulation}), we know that
        there is a term $\tmthree'$ such that $\tmthree' \refines \tmthree$ and $\tmthree' \tomulti_\dist \tmtwo'$.

        Lastly, the inductive hypothesis tells us that there is a $\tm'$ that refines $\tm$.
        Our argument can be summed up in the following diagram,
        where the data labeled with $^1$ are what we know by hypothesis,
        the one labeled with $^2$ is the result of Backwards simulation~(\rlem{backwards_simulation}),
        and the one labeled with $^3$ is given by the inductive hypothesis.

        \[
        \xymatrix@C=1cm{
          \tm  \ar@{}[d]|*=0[@]{\rtimes}^{^3} \ar@{->>}[r]^{\beta^1}
            & \tmthree \ar@{}[d]|*=0[@]{\rtimes}^{^2} \ar[r]^{\beta^1}
            & \tmtwo \ar@{}[d]|*=0[@]{\rtimes}^{^1}  \\
          \tm' & \tmthree' \ar@{->>}[r]^{\dist} & \tmtwo' \\
        }
        \]
    \end{enumerate}
  \item[] {\bf $\Leftarrow$.} Let $\tm'$ such that $\tm' \refines \tm$.
    The term $\tm'$ has a normal form, which is in particular a head normal form.
    \TODO{esto hay que probarlo, pero capaz sale facil}

    We will prove that if $\tm \refinedby \tm' \tomulti_\dist \tmtwo'$ and $\tmtwo'$ is a normal form,
    then there exists a term $\tmtwo$ such that $\tm \rtobeta \tmtwo \refinedby \tmtwo'$.
    Note that this is enough, because if $\tmtwo'$ is in head normal form (as in our case),
    \rlem{hnf_iff_distnf} tells us that $\tmtwo$ is also in head normal form.

    Recall that, in the distributive lambda-calculus,
    the length of derivations starting from a fixed term
    is bounded~(\rcoro{bounded_length_of_derivations})
    as a corollary of the fact that the distributive lambda-calculus
    is strongly normalizing and finitely branching, by K\"onig's lemma.
    We proceed by induction on the length $n$ of the longest derivation going out from $\tm'$.
    \begin{enumerate}
      \item {\bf Base case, $n = 0$.} In this case $\tm' = \tmtwo'$,
        so we can take $\tmtwo$ to be equal to $\tm$ and we are done.
      \item {\bf Induction, $n > 0$.}
        Then there is at least one step:
        \[
        \xymatrix@C=1cm{
          \tm  \ar@{}[d]|*=0[@]{\rtimes}  & &      \\
          \tm' \ar@{->}[r]^{\dist} & \tmthree' \ar@{->>}[r]^{\dist} & \tmtwo' \\
        }
        \]
        Here, reverse simulation (\rprop{reverse_simulation}) tells us that there are $\tmthree$ and $\tmthree''$
        such that:
        \[
        \xymatrix@C=1cm{
          \tm  \ar@{}[d]|*=0[@]{\rtimes} \ar@{->}[rr]^{\beta} & & \tmthree  \ar@{}[d]|*=0[@]{\rtimes} \\
          \tm' \ar@{->}[r]^{\dist} & \tmthree' \ar@{->>}[r]^{\dist} & \tmthree'' \\
        }
        \]
        The longest derivation going out from $\tmthree''$ has $m$ steps.
        Note that $m < n$, for otherwise there would be a derivation $\tm' \todist \tmthree' \rtodist \tmthree'' \rtodist ...$ 
        of more than $n$ steps.

        Furthermore, as $\termsdist$ is confluent~(\rcoro{confluence}) and strongly normalizing~(\rprop{termination}),
        we have that $\tmthree'' \tomulti_\dist \tmtwo'$.
        Since $m < n$, we can use the inductive hypothesis
        on $\tmthree \refinedby \tmthree'' \tomulti_\dist \tmtwo'$, which closes the diagram:
        \[
        \xymatrix@C=1cm{
          \tm  \ar@{}[d]|*=0[@]{\rtimes} \ar@{->}[r]^{\beta} & \tmthree \ar@{}[d]|*=0[@]{\rtimes} \ar@{->>}[r]^{\beta} & \tmtwo \ar@{}[d]|*=0[@]{\rtimes} \\
          \tm' \ar@{->>}[r]^{\dist} & \tmthree'' \ar@{->>}[r]^{\dist} & \tmtwo' \\
        }
        \]
    \end{enumerate}
\end{itemize}
\end{proof}
