Before proving \rprop{has_hnf_has_refinement}, we need a few auxiliary results.

\begin{lemma}[$\todist$-normal forms refine head normal forms]
\llem{normal_forms_refine_hnfs}
Let $\tm' \in \termsdist$ be a $\todist$-normal form
and $\tm' \refines \tm$. Then $\tm$ is a head normal form.
\end{lemma}
\begin{proof}
Observe, by induction on $\tm'$
that if $\tm' \in \termsdist$ is a $\todist$-normal form, it must be
of the form
$
  \lamp{\lab_1}{\var_1}{\hdots\lamp{\lab_n}{\var_n} \vartwo^\typ \ls{\tmtwo}_1 \hdots \ls{\tmtwo}_m}
$,
as it cannot have a subterm of the form $(\lamp{\lab'}{\varthree}{\tmthree})\ls{\tmfour}$.
Then \[\lamp{\lab_1}{\var_1}{\hdots\lamp{\lab_n}{\var_n} \vartwo^\typ \ls{\tmtwo}_1 \hdots \ls{\tmtwo}_m} \refines \tm\].
So $\tm$ is of the form $\lam{\var_1}{\hdots\lam{\var_n}{\vartwo \tmtwo_1 \hdots \tmtwo_n}}$,
that is, $\tm$ is a head normal form.
\end{proof}

The following lemma is an adaptation of Subject~Expansion in~\cite{bucciarelli2017non}.
\begin{lemma}[Subject Expansion]
\llem{subject_expansion}
If $\tctx \vdash \conof{\subs{\tm}{\var}{\ls{\tmtwo}}} : \typ$ is derivable,
then $\tctx \vdash \conof{(\lamp{\lab}{\var}{\tm})\ls{\tmtwo}} : \typ$ is derivable.
\end{lemma}
\begin{proof}
The proof proceeds by induction on $\con$,
and the base case by induction on $\tm$, 
similar to the proof that substitution preserves typing in
the proof of Subject~Reduction (\rsec{substitution_preserves_typing}).
\end{proof}

Correctness is not necessarily preserved by $\todist$-expansion.
We need a stronger invariant, \defn{strong sequentiality},
that will be shown to be preserved by expansion
under appropiate conditions:

\begin{definition}[Subterms and free subterms]
The set of \defn{subterms} $\subterms{\tm}$ of a term $\tm$ is formally defined as follows:
\[
  \begin{array}{rcl}
    \subterms{\var^\typ} & \eqdef & \set{\var^\typ} \\
    \subterms{\lamp{\lab}{\var}{\tm}} & \eqdef & \set{\lamp{\lab}{\var}{\tm}} \cup \subterms{\tm} \\
    \subterms{\tm[\tmtwo_i]_{i=1}^n} & \eqdef & \set{\tm[\tmtwo_i]_{i=1}^n} \cup \subterms{\tm} \cup \cup_{i=1}^{n} \subterms{\tmtwo_i}
  \end{array}
\]
The set of \defn{free subterms} $\fsubterms{\tm}$ of a term $\tm$ is defined similarly,
except for the abstraction case, which requires that the subterm in question
do not include occurrences of bound variables:
\[
  \begin{array}{rcl}
    \fsubterms{\var^\typ} & \eqdef & \set{\var^\typ} \\
    \fsubterms{\lamp{\lab}{\var}{\tm}} & \eqdef & \set{\lamp{\lab}{\var}{\tm}} \cup \set{\tmthree \in \fsubterms{\tm} \ST \var \not\in \fv{\tmthree}} \\
    \fsubterms{\tm[\tmtwo_i]_{i=1}^n} & \eqdef & \set{\tm[\tmtwo_i]_{i=1}^n} \cup \fsubterms{\tm} \cup \cup_{i=1}^{n} \fsubterms{\tmtwo_i}
  \end{array}
\]
\end{definition}

\begin{definition}[Strong sequentiality]
A term $\tm$ is \defn{strongly sequential}
if it is correct and, moreover,
for every subterm $\tmtwo \in \subterms{\tm}$
and any two free subterms $\tmtwo_1,\tmtwo_2 \in \fsubterms{\tmtwo}$ lying at disjoint positions of $\tmtwo$,
the types of $\tmtwo_1$ and $\tmtwo_2$ have different external labels.
\end{definition}

\begin{example}
The following examples illustrate the notion of strong sequentiality:
\begin{enumerate}
\item The term $\tm = (\lamp{1}{\var}{\vartwo^{\alpha^2}})[\,]$ is strongly sequential.
      Note that $\tm$ and $\vartwo^{\alpha^2}$ have the same type, namely $\alpha^2$,
      but they do not occur at {\em disjoint} positions.
\item The term $\tm = (\lamp{1}{\var}{\var^{\alpha^2}})[\vartwo^{\alpha^2}]$ is strongly sequential.
      Note that $\var^{\alpha^2}$ and $\vartwo^{\alpha^2}$ both have type $\alpha^2$,
      but they are not simultaneously free subterms of the same subterm of $\tm$.
\item The term $\tm = \lamp{1}{\vartwo}{ \var^{[\alpha^2] \tolab{3} [\alpha^2] \tolab{4} \beta^5} [\vartwo^{\alpha^2}] [\varthree^{\alpha^2}]}$
      is not strongly sequential,
      since $\vartwo^{\alpha^2}$ and $\varthree^{\alpha^2}$
      have the same type and they are both free subterms of
      $\var^{[\alpha^2] \tolab{3} [\alpha^2] \tolab{4} \beta^5} [\vartwo^{\alpha^2}] [\varthree^{\alpha^2}] \in \subterms{\tm}$.
\end{enumerate}
\end{example}

\begin{lemma}[Refinement of a substitution: decomposition]
\llem{refinement_substitution_inverse}
If $\tmthree' \refines \subs{\tm}{\var}{\tmtwo}$
and $\tmthree'$ is strongly sequential,
then $\tmthree'$ is of the form $\subs{\tm'}{\var}{[\tmtwo'_i]_{i=1}^n}$.
Moreover, given a fresh label~$\lab$,
the term $(\lamp{\lab}{\var}{\tm'})[\tmtwo'_i]_{i=1}^{n}$ is strongly sequential,
$\tm' \refines \tm$ and $\tmtwo'_i \refines \tmtwo$ for all $i=1..n$.
\end{lemma}
\begin{proof}
By induction on $\tm$.
\begin{enumerate}
\item {\bf Variable (same), $\tm = \var$.}
  Then $\tmthree' \refines \tmtwo$.
  Let $\typ$ be the type of $\tmthree'$.
  Taking $\tm' := \var^\typ \refines \var$
  we have that $(\lamp{\lab}{\var}{\var^\typ})[\tmthree']$ is strongly sequential.
  Regarding strong sequentiality,
  observe that $\var^\typ$ and $\tmthree'$ have the same type, but they are not
  simultaneously the free subterms of any subterm of $(\lamp{\lab}{\var}{\var^\typ})[\tmthree']$.
\item {\bf Variable (different), $\tm = \vartwo \neq \var$.}
  Then $\tmthree' \refines \vartwo$, so $\tmthree'$ is of the form $\vartwo^\typ$.
  Taking $\tm' := \vartwo^\typ$ we have that
  $(\lamp{\lab}{\var}{\vartwo^\typ})[\,]$ is strongly sequential.
\item {\bf Abstraction, $\tm = \lam{\vartwo}{\tmfour}$.}
  Then $\tmthree' \refines \lam{\vartwo}{\subs{\tmfour}{\var}{\tmtwo}}$ 
  so $\tmthree'$ is of the form $\lamp{\lab'}{\vartwo}{\tmthree''}$
  where $\tmthree'' \refines \subs{\tmfour}{\var}{\tmtwo}$.
  By \ih, $\tmthree''$ is of the form $\subs{\tmfour'}{\var}{[\tmtwo'_i]_{i=1}^{n}}$
  where $(\lamp{\lab}{\var}{\tmfour'})[\tmtwo'_i]_{i=1}^{n}$ is strongly sequential,
  $\tmfour' \refines \tmfour$ and $\tmtwo'_i \refines \tmtwo$ for all $i=1..n$.
  Taking $\tm' := \lamp{\lab'}{\vartwo}{\tmfour'}$,
  we have that
  $\tm' = \lamp{\lab'}{\vartwo}{\tmfour'} \refines \lam{\vartwo}{\tmfour} = \tm$.
  Moreover, the term
  $(\lamp{\lab}{\var}{\tm'})[\tmtwo_i]_{i=1}^{n}
   = (\lamp{\lab}{\var}{\lamp{\labtwo}{\vartwo}{\tmfour'}})[\tmtwo_i]_{i=1}^{n}$
  is strongly sequential.
  Typability is a consequence of Subject Expansion~(\rlem{subject_expansion}).
  The remaining properties are:
  \begin{enumerate}
  \item {\em Uniquely labeled lambdas.}
    The multiset of labels decorating the lambdas of
    $(\lamp{\lab}{\var}{\lamp{\labtwo}{\vartwo}{\tmfour'}})[\tmtwo_i]_{i=1}^{n}$
    is given by
    $\Lambda((\lamp{\lab}{\var}{\lamp{\labtwo}{\vartwo}{\tmfour'}})[\tmtwo_i]_{i=1}^{n})
     = [\lab,\labtwo] + \Lambda(\tmfour') +_{i=1}^{n} \Lambda(\tmtwo_i)$.
    It suffices to check that this multiset has no repeats.
    The label $\lab$ is assumed to be fresh, so it occurs only once.
    By \ih, $\tmthree'' = \subs{\tmfour'}{\var}{[\tmtwo'_i]_{i=1}^{n}}$,
    so using \rlem{labels_over_lambdas_substitution} we have
    $\Lambda(\tmthree') =
     \Lambda(\lamp{\lab'}{\vartwo}{\tmthree'}) =
     [\lab'] + \Lambda(\tmthree'') =
     [\lab'] + \Lambda(\subs{\tmfour'}{\var}{[\tmtwo'_i]_{i=1}^{n}}) =
     [\lab'] + \Lambda(\tmfour') +_{i=1}^n \Lambda(\tmtwo'_i)$.
    Moreover, the term $\tmthree' = \lamp{\lab'}{\vartwo}{\tmthree''}$ is correct,
    so this multiset has no repeats.
  \item {\em Sequential contexts.}
    Let $\tmsix$ be a subterm of $(\lamp{\lab}{\var}{\lamp{\labtwo}{\vartwo}{\tmfour'}})[\tmtwo'_i]_{i=1}^{n}$.
    If $\tmsix$ is a subterm of $\tmfour'$ or a subterm of $\tmtwo'_i$ for some $i=1..n$ we conclude by \ih
    since $(\lamp{\lab}{\var}{\tmfour'})[\tmtwo'_i]_{i=1}^{n}$ is known to be correct.
    Moreover, if $\tctx \oplus \var : \mtyp \oplus \vartwo : \mtyptwo$ is the typing context for $\tmfour'$,
    the typing contexts of
    $\lamp{\labtwo}{\vartwo}{\tmfour'}$ and $\lamp{\lab}{\var}{\lamp{\labtwo}{\vartwo}{\tmfour'}}$
    are respectively $\tctx \oplus \var : \mtyp$ and $\tctx$, which are also sequential.
    Finally, if $\tctxtwo_i$ is the typing context for $\tmtwo'_i$, for each $i=1..n$,
    the typing context for $(\lamp{\lab}{\var}{\tmfour'})[\tmtwo'_i]_{i=1}^{n}$ 
    is of the form $\tctx +_{i=1}^n \tctxtwo_i + \vartwo : \mtyptwo$, and it is sequential by \ih.
    Hence the typing context for the whole term is $\tctx +_{i=1}^n \tctxtwo_i$, and it is sequential.
  \item {\em Sequential types.}
    \label{refinement_substitution_inverse__case_lambda_sequential_types}
    Let $\tmsix$ be a subterm $(\lamp{\lab}{\var}{\lamp{\labtwo}{\vartwo}{\tmfour'}})[\tmtwo'_i]_{i=1}^{n}$,
    and let $\mtypthree \tolab{\lab''} \typthree$ be a type that occurs in the typing context
    or the type of $\tmsix$.
    As in the previous case, we have that
    $\tctx \oplus \var : [\typtwo_i]_{i=1}^n \oplus \vartwo : \mtyptwo \vdash \tmfour' : \typ$ is derivable
    and $\tctxtwo_i \vdash \tmtwo'_i : \typtwo_i$ is derivable for all $i=1..n$.
    Moreover, they are correct by \ih, so if $\tmsix$ is a subterm of $\tmfour'$ or a subterm of some $\tmtwo'_i$,
    we are done.
    There are three cases left for $\tmsix$:
    \begin{enumerate}
    \item {\bf Case $\tmsix = \lamp{\labtwo}{\vartwo}{\tmfour'}$.}
      The typing context is $\tctx \oplus \var : [\typtwo_i]_{i=1}^n$ and the type $\mtyptwo \tolab{\labtwo} \typ$.
    \item {\bf Case $\tmsix = \lamp{\lab}{\var}{\lamp{\labtwo}{\vartwo}{\tmfour'}}$.}
      The typing context is $\tctx$ and the type $[\typtwo_i]_{i=1}^n \tolab{\lab} \mtyptwo \tolab{\labtwo} \typ$.
    \item {\bf Case $\tmsix = (\lamp{\lab}{\var}{\lamp{\labtwo}{\vartwo}{\tmfour'}})[\tmtwo_i]_{i=1}^{n}$.}
      The typing context is $\tctx$ and the type $\mtyptwo \tolab{\labtwo} \typ$.
    \end{enumerate}
    In all three cases, if $\mtypthree \tolab{\lab''} \typthree$ occurs in the typing context
    or the type of $\tmsix$, then $\mtypthree$ can be shown to be sequential using the \ih.
  \item {\em Strong sequentiality.}
    \label{refinement_substitution_inverse__case_lambda_strong_sequentiality}
    Let $\tmsix \in \subterms{(\lamp{\lab}{\var}{\lamp{\labtwo}{\vartwo}{\tmfour'}})[\tmtwo_i]_{i=1}^{n}}$
    be a subterm,
    and let $\tmsix_1,\tmsix_2 \in \fsubterms{\tmsix}$ be free subterms lying at disjoint positions of $\tmsix$.
    We argue that the types of $\tmsix_1$ and $\tmsix_2$ have different external labels.
    Consider the following five possibilities for $\tmsix_1$:
    \begin{enumerate}
    \item {\bf Case $\tmsix_1 = (\lamp{\lab}{\var}{\lamp{\labtwo}{\vartwo}{\tmfour'}})[\tmtwo_i]_{i=1}^{n}$.}
      Impossible since $\tmsix_2$ must be at a disjoint position.
    \item {\bf Case $\tmsix_1 = \lamp{\lab}{\var}{\lamp{\labtwo}{\vartwo}{\tmfour'}}$.}
      Then the external label of the type of $\tmsix_1$ is the label $\lab$, which is fresh,
      so it cannot coincide with the type of any other subterm.
    \item {\bf Case $\tmsix_1 = \lamp{\labtwo}{\vartwo}{\tmfour'}$.}
      Then $\tmsix_2$ must be a subterm of $\tmtwo_i$ for some $i=1..n$.
      Note that $\tmsix$ must be the whole term,
      and $n > 0$, so there is at least one free occurrence of $\var$ in $\lamp{\labtwo}{\vartwo}{\tmfour'}$.
      This means that $\tmsix_1 \not\in \fsubterms{\tmsix}$, so this case is impossible.
    \item {\bf Case $\tmsix_1$ is a subterm of $\tmfour'$.}
      If $\tmsix_2$ is also a subterm of $\tmfour'$, we conclude since by \ih
      $(\lamp{\lab}{\var}{\tmfour'})[\tmtwo'_i]_{i=1}^{n}$ is strongly sequential.
      Otherwise, $\tmsix_2$ is a subterm of $\tmtwo_i$ for some $i=1..n$,
      and we also conclude by \ih.
    \item {\bf Case $\tmsix_1$ is a subterm of $\tmtwo_i$ for some $i=1..n$.}
      If $\tmsix_2$ is a subterm of $\tmtwo_j$ for some $j=1..n$,
      we conclude since by \ih
      $(\lamp{\lab}{\var}{\tmfour'})[\tmtwo'_i]_{i=1}^{n}$ is strongly sequential.
      If $\tmsix_2$ is any other subterm, note that the symmetric case has already been considered
      in one of the previous cases.
    \end{enumerate}
  \end{enumerate}
\item {\bf Application, $\tm = \tmfour\tmfive$.}
  Then $\tmthree' \refines \subs{(\tmfour\tmfive)}{\var}{\tmtwo}$,
  so it is of the form $\tmthree'_0[\tmthree'_j]_{j=1}^m$
  where $\tmthree'_0 \refines \subs{\tmfour}{\var}{\tmtwo}$
  and $\tmthree'_j \refines \subs{\tmfive}{\var}{\tmtwo}$ for all $j=1..m$.
  By \ih we have that $\tmthree'_0$ is of the form
  $\subs{\tmfour'}{\var}{\ls{\tmtwo}_0}$
  and for all $j=1..m$ the term $\tmthree'_j$ is of the form
  $\subs{\tmfive'_j}{\var}{\ls{\tmtwo}_j}$,
  where $\tmfour' \refines \tmfour$
  and $\tmfive'_j \refines \tmfive$ for all $j=1..m$.
  Moreover,
  the length of the list $\ls{\tmtwo}_j$ is $k_j$
  and
  $\ls{\tmtwo}_j = [\tmtwo^{(j)}_i]_{i=1}^{k_j}$ for all $j=0..m$,
  and we have that $\tmtwo^{(j)}_i \refines \tmtwo$ for all $j=0..m,i=1..k_j$.
  By \ih we also know that
  $(\lamp{\lab}{\var}{\tmfour'})\ls{\tmtwo}_0$ is strongly sequential
  and $(\lamp{\lab}{\var}{\tmfive'_j})\ls{\tmtwo}_j$ is strongly sequential for all $j=1..m$.

  Let $\tm' := \tmfour'[\tmfive'_j]_{j=1}^{m}$,
  let $n := \sum_{j=0}^m k_j$,
  and let $[\tmtwo'_1,\hdots,\tmtwo'_n] := \sum_{j=0}^m \ls{\tmtwo}_j$.
  Note that $\tm' = \tmfour'[\tmfive'_j]_{j=1}^{m} \refines \tmfour\tmfive = \tm$
  and $\tmtwo'_i \refines \tmtwo$ for all $i=1..n$.

  Moreover, we have to check that $\tmthree' = \subs{\tm'}{\var}{[\tmtwo'_1,\hdots,\tmtwo'_n]}$.
  To prove this,
  note that $\subs{\tm'}{\var}{[\tmtwo'_1,\hdots,\tmtwo'_n]} = \subs{(\tmfour'[\tmfive'_j]_{j=1}^{m})}{\var}{[\tmtwo'_1,\hdots,\tmtwo'_n]}$.
  Suppose that the multiset $\tmlabel{[\tmtwo'_1,\hdots,\tmtwo'_n]}$ were sequential.
  Then the list of terms $[\tmtwo'_1,\hdots,\tmtwo'_n]$ would be partitioned
  as $(\ls{\tmthree}_0,\hdots,\ls{\tmthree}_m)$
  where $\ls{\tmthree}_j$ is a permutation of $\ls{\tmtwo}_j$ for all $j=0..m$,
  and we would have indeed:
  \[
    \begin{array}{rcll}
      \subs{\tm'}{\var}{[\tmtwo'_1,\hdots,\tmtwo'_n]}
    & = &
      \subs{(\tmfour'[\tmfive'_j]_{j=1}^{m})}{\var}{[\tmtwo'_1,\hdots,\tmtwo'_n]} \\
    & = & 
      \subs{\tmfour'}{\var}{\ls{\tmthree}_0}[\subs{\tmfive'_j}{\var}{\ls{\tmthree}_j}]_{j=1}^{m})
      \hfill\text{ by $\tmlabel{[\tmtwo'_1,\hdots,\tmtwo'_n]}$ sequential} \\
    & = & 
      \subs{\tmfour'}{\var}{\ls{\tmtwo}_0}[\subs{\tmfive'_j}{\var}{\ls{\tmtwo}_j}]_{j=1}^{m})
   \\
    & = & 
      \subs{\tmfour'}{\var}{\ls{\tmtwo}_0}[\subs{\tmfive'_j}{\var}{\ls{\tmtwo}_j}]_{j=1}^{m}) \hfill\text{by \rlem{substitution_permutation}} \\
    & = & 
      \tmthree'_0[\tmthree'_j]_{j=1}^{m} = \tmthree' \hfill\text{by \ih}
    \end{array}
  \]
  To see that $\tmlabel{[\tmtwo'_1,\hdots,\tmtwo'_n]}$ is sequential,
  note that for every $i \neq j$,
  the terms $\tmtwo'_i$ and $\tmtwo'_j$ are free subterms of $\tmthree'$
  and they lie at disjoint positions of $\tmthree'$.
  Since $\tmthree'$ is strongly sequential, the types of $\tmtwo'_i$ and $\tmtwo'_j$
  have different external labels. Hence $\tmlabel{[\tmtwo'_1,\hdots,\tmtwo'_n]}$ is sequential.

  To conclude, we are left to check that $(\lamp{\lab}{\var}{\tm'})[\tmtwo'_1,\hdots,\tmtwo'_n]$ is strongly sequential:
  \begin{enumerate}
  \item {\em Uniquely labeled lambdas.}
    The multiset of labels decorating the lambdas of $(\lamp{\lab}{\var}{\tm'})[\tmtwo'_1,\hdots,\tmtwo'_n]$
    is given by
    $\Lambda((\lamp{\lab}{\var}{\tm'})[\tmtwo'_1,\hdots,\tmtwo'_n]) =
     [\lab] + \Lambda(\tm') +_{i=1}^{n} \Lambda(\tmtwo'_i)$.
    It suffices to check that this multiset has no repeats.
    The label $\lab$ is assumed to be fresh, so it occurs only once.
    We have already argued that
    $\tmthree' = \subs{\tm'}{\var}{[\tmtwo'_1,\hdots,\tmtwo'_n]}$,
    so using \rlem{labels_over_lambdas_substitution} we have
    $\Lambda(\tmthree') = \Lambda(\tm') +_{i=1}^n \Lambda(\tmtwo'_i)$.
    Moreover $\tmthree'$ is correct, so this multiset has no repeats.
  \item {\em Sequential contexts.}
    Suppose that
    $\tctx_0 \oplus \var : \mtyp_0 \vdash \tmfour' : [\typthree_j]_{j=1}^m \tolab{\labtwo} \typ$
    is derivable,
    $\tctx_j \oplus \var : \mtyp_j \vdash \tmfive'_j : \typthree_j$
    is derivable for all $j=1..m$,
    and $\tctxtwo_i \vdash \tmtwo'_i : \typtwo_i$ is derivable for all $i=1..n$.
    Note that $\sum_{j=0}^{m} \mtyp_j = [\typtwo_i]_{i=1}^{n}$.

    Let $\tmsix$ be a subterm of
    $(\lamp{\lab}{\var}{\tmfour'[\tmfive'_j]_{j=1}^{m}})[\tmtwo'_1,\hdots,\tmtwo'_n]$.
    Consider four cases for $\tmsix$:

    \begin{enumerate}
    \item {\bf Case $\tmsix = (\lamp{\lab}{\var}{\tmfour'[\tmfive'_j]_{j=1}^{m}})[\tmtwo'_1,\hdots,\tmtwo'_n]$.}
      The typing context is $\sum_{j=0}^{m} \tctx_j + \sum_{i=1}^n \tctxtwo_i$.
      By Subject~Expansion~(\rlem{subject_expansion})
      the typing context of $\subs{(\tmfour'[\tmfive'_j]_{j=1}^{m})}{\var}{[\tmtwo'_1,\hdots,\tmtwo'_n]} = \tmthree'$
      is also $\sum_{j=0}^{m} \tctx_j + \sum_{i=1}^n \tctxtwo_i$
      and $\tmthree'$ is correct by hypothesis.
      So $\sum_{j=0}^{m} \tctx_j + \sum_{i=1}^n \tctxtwo_i$ is sequential.
    \item {\bf Case $\tmsix = \lamp{\lab}{\var}{\tmfour'[\tmfive'_j]_{j=1}^{m}}$.}
      The typing context is
      $\sum_{j=0}^{m} \tctx_j$,
      which is sequential because $\sum_{j=0}^{m} \tctx_j + \sum_{i=1}^n \tctxtwo_i$ is sequential.
    \item {\bf Case $\tmsix = \tmfour'[\tmfive'_j]_{j=1}^{m}$.}
      The typing context is
      $\sum_{j=0}^{m} \tctx_j \oplus \var : [\typtwo_i]_{i=1}^n$,
      which is sequential because $\sum_{j=0}^{m} \tctx_j$ is sequential
      and, moreover, $[\typtwo_i]_{i=1}^n = \tmlabel{[\typtwo'_1,\hdots,\typtwo'_n]}$
      which we have already shown to be sequential.
    \item {\bf Otherwise.}
      Then $\tmsix$ is a subterm of $\tmfour'$, a subterm of $\tmfive'_j$ for some $j=1..m$,
      or a subterm of some $\tmtwo'_i$ for some $i=1..n$.
      Then we conclude since by \ih $(\lamp{\lab}{\var}{\tmfour'})\ls{\tmtwo}_0$
      and all the $(\lamp{\lab}{\var}{\tmfive'_j})\ls{\tmtwo}_j$
      are strongly sequential.
    \end{enumerate}
  \item {\em Sequential types.}
    Let $\tmsix$ be a subterm of $(\lamp{\lab}{\var}{\tmfour'[\tmfive'_j]_{j=1}^{m}})[\tmtwo'_1,\hdots,\tmtwo'_n]$.
    We claim that if $\mtyptwo \tolab{\lab''} \typfour$ occurs in the context or in the type of $\tmsix$,
    then $\mtyptwo$ is sequential.
    The proof is similar as for subcase \label{refinement_substitution_inverse__case_lambda_sequential_types}.
  \item {\em Strong sequentiality.}
    Let $\tmsix \in \subterms{(\lamp{\lab}{\var}{\tmfour'[\tmfive'_j]_{j=1}^{m}})[\tmtwo'_1,\hdots,\tmtwo'_n]}$
    be a subterm,
    and let $\tmsix_1,\tmsix_2 \in \fsubterms{\tmsix}$ be free subterms lying at disjoint positions of $\tmsix$.
    We claim that the types of $\tmsix_1$ and $\tmsix_2$ have different external labels.
    The proof is similar as for subcase \label{refinement_substitution_inverse__case_lambda_strong_sequentiality}.
  \end{enumerate}
\end{enumerate}
\end{proof}

\begin{lemma}[Backwards Simulation]
\llem{backwards_simulation}
Let $\tm,\tmtwo \in \terms$ be $\lambda$-terms and
let $\tmtwo' \in \termsdist$ be a strongly sequential term such that
$\tm \tobeta \tmtwo$ and $\tmtwo' \refines \tmtwo$.
Then there exists a strongly sequential term $\tm' \in \termsdist$ such that:
\[
\xymatrix@R=.25cm{
 \tm \ar@{}[d]|*=0[@]{\rtimes} \ar[r]^{\beta} & \tmtwo \ar@{}[d]|*=0[@]{\rtimes} \\
 \tm' \ar@{.>>}[r]^{\dist} & \tmtwo' \\
}
\]
\end{lemma}
\begin{proof}
Let $\tm = \conof{(\lam{\var}{\tmthree})\tmfour} \tobeta \conof{\subs{\tmthree}{\var}{\tmfour}} = \tmtwo$.
The proof proceeds by induction on $\con$.
\begin{enumerate}
\item {\bf Empty, $\con = \conbase$.}
  By \rlem{refinement_substitution_inverse} we have that $\tmtwo'$
  is of the form $\subs{\tmthree'}{\var}{[\tmfour'_1,\hdots,\tmfour'_n]}$
  where $\tmthree' \refines \tmthree$ and $\tmfour'_i \refines \tmfour$ for all $i=1..n$.
  Moreover, taking $\lab$ to be a fresh label, $(\lamp{\lab}{\var}{\tmthree'})[\tmfour'_1,\hdots,\tmfour'_n]$ 
  is strongly sequential and
  $(\lamp{\lab}{\var}{\tmthree'})[\tmfour'_1,\hdots,\tmfour'_n] \refines (\lam{\var}{\tmthree})\tmfour$.
\item {\bf Under an abstraction, $\con = \lam{\var}{\con'}$.}
  Straightforward by \ih.
\item {\bf Left of an application, $\con = \con'\,\tmthree$}
  Straightforward by \ih.
\item {\bf Right of an application, $\con = \tmthree\,\con'$}
  Then $\tm = \tmthree\,\tmfour \tobeta \tmthree\,\tmfive = \tmtwo$
  where $\tmfour \tobeta \tmfive$ and $\tmtwo' \refines \tmtwo$.
  Then $\tmtwo'$ is of the form $\tmthree'[\tmfive'_1,\hdots,\tmfive'_n]$ where $\tmfive'_i \refines \tmfive$ for all $i=1..n$.
  By \ih, for all $i=1..n$ we have that there exist $\tmfour'_1,\hdots,\tmfour'_n$ such that:
  \[
  \xymatrix@R=.25cm{
   \tmfour \ar@{}[d]|*=0[@]{\rtimes} \ar[r]^{\beta} & \tmfive \ar@{}[d]|*=0[@]{\rtimes} \\
   \tmfour'_i \ar@{.>>}[r]^{\dist} & \tmfive'_i \\
  }
  \HS\HS\text{So we have:}
  \xymatrix@R=.25cm{
   \tmthree\,\tmfour \ar@{}[d]|*=0[@]{\rtimes} \ar[r]^{\beta} & \tmthree\tmfive \ar@{}[d]|*=0[@]{\rtimes} \\
   \tmthree'[\tmfour'_i]_{i=1}^{n} \ar@{.>>}[r]^{\dist} & \tmthree'[\tmfive'_i]_{i=1}^{n}  \\
  }
  \]
  Moreover, $\tmthree'[\tmfour'_i]_{i=1}^{n}$ is strongly sequential, which can be concluded
  from the facts that $\tmthree'[\tmfive'_i]_{i=1}^{n}$ is strongly sequential by hypothesis,
  $\tmfour'_i$ is strongly sequential for all $i=1..n$ by \ih,
  and $\tmfour'_i$ and $\tmfive'_i$ have the same types by Subject~Expansion~(\rlem{subject_expansion}).
\end{enumerate}
\end{proof}

To prove \rprop{has_hnf_has_refinement} recall that it states that the following are equivalent:
\begin{enumerate}
\item There exists $\tm' \in \termsdist$ such that $\tm' \refines \tm$.
\item There exists $\tm' \in \termsdist$ such that $\tm' \refines \tm$
      and $\tm' \todist^* \lamp{\lab_1}{\var_1}{\hdots\lamp{\lab_n}{\var_n}{\vartwo^\typ[]\hdots[]}}$.
\item There exists a head normal form $\tmtwo$ such that $\tm \tobeta^* \tmtwo$.
\end{enumerate}
Let us prove the chain of implications $1 \implies 3 \implies 2 \implies 1$:
\begin{itemize}
\item $(1 \implies 3)$
  Let $\tm' \refines \tm$.
  By Strong Normalization~(\rprop{termination}), reduce $\tm' \todist^* \tmtwo'$
  to normal form.
  We claim that there exists a term $\tmtwo$ such that $\tm \tobeta^* \tmtwo$ and $\tmtwo' \refines \tmtwo$.
  Observe that, since $\lambdadist$ is strongly normalizing~(\rprop{termination})
  and finitely branching, K\"onig's lemma ensures that there is a bound for the length of $\todist$-derivations
  going out from a term $\tm' \in \termsdist$.
  (Alternatively, according to the proof of Strong Normalization in~\rprop{termination},
  the bound may be explicitly taken to be the number of lambdas in $\tm'$).
  Call this bound the \defn{weight} of $\tm'$.

  We proceed by induction on the weight of $\tm'$.
  If the derivation is empty, we are done by taking $\tmtwo := \tm$.
  If the derivation is non-empty, it is of the form $\tm' \todist \tmthree' \todist^* \tmtwo'$.
  By Simulation~(\rprop{simulation}) there exist terms $\tmthree$ and $\tmthree''$
  such that $\tmthree' \todist^* \tmthree'' \refines \tmthree$
  and $\tm \tobeta \tmthree$. Since the $\lambdadist$-calculus is
  confluent~(\rprop{strong_permutation}) and $\tmtwo'$ is a normal form,
  we have that $\tmthree'' \rtodist \tmtwo'$.
  Note that the weight of $\tm'$ is strictly larger than the weight of $\tmthree''$,
  so by \ih there exists $\tmtwo$ such that $\tmthree \tobeta^* \tmtwo$
  and $\tmtwo' \refines \tmtwo$:
  \[
    \xymatrix@R=.5cm{
      \tm
        \ar@{}[d]|*=0[@]{\rtimes} 
        \ar[rr]^{\beta}
    &
    &
      \tmthree
        \ar@{}[d]|*=0[@]{\rtimes} 
        \ar@{->>}[r]^{\beta}
    &
      \tmtwo
        \ar@{}[d]|*=0[@]{\rtimes} 
    \\
      \tm'
        \ar[r]^{\dist}
    &
      \tmthree'
        \ar@{->>}[r]^{\dist}
        \ar@/_.75cm/@{->>}[rr]^{\dist}
    &
      \tmthree''
        \ar@{->>}[r]^{\dist}
    &
      \tmtwo'
    }
  \]
  Finally, since $\tmtwo'$ is a $\todist$-normal form and $\tmtwo' \refines \tmtwo$,
  \rlem{normal_forms_refine_hnfs}
  ensures that $\tmtwo$ is a head normal form, as required.
\item $(2 \implies 1)$ Obvious.
\item $(3 \implies 2)$
  Let $\tm \tobeta^* \tmtwo$ be a derivation to head normal form.
  We claim that there exists $\tm' \in \termsdist$ such that $\tm'$ is strongly sequential,
  and the normal form of $\tm'$ is of the form $\lamp{\lab_1}{\var_1}{\hdots\lamp{\lab_n}{\var_n}{\vartwo^\typ[]\hdots[]}}$.
  By induction on the length of the derivation $\tm \tobeta^* \tmtwo$.
  If the derivation is empty, $\tm = \tmtwo$ is a head normal form
  and we conclude by \rlem{hnf_has_refinement}, observing that the constructed term $\tm' \refines \tm$
  is strongly sequential.
  If the derivation is non-empty, conclude using the \ih and Backwards Simulation~(\rlem{backwards_simulation}).
\end{itemize}
