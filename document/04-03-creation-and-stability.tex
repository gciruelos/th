

In this section we will see two important lemmas that guarantee certain properties
of residues after steps or derivations.

The result that we will call full stability can be seen as a functorial extension
to the confluence result: confluence speaks about what happens to terms after
two coinitial derivations,
while full stability speaks about what happens to steps after two coinitial derivations.

\subsection*{Creation}

The first of this lemmas is creation.
We say that a step $\redex$ is \defn{created} by another step $\redextwo$ if
there is no step $\redex'$ coinitial with $\redextwo$ such that $\redex \in \redex' / \redextwo$.
What this means is that we were not able to do what $\redex$ does before executing $\redextwo$.
For example, in \rexample{derivation-ids-ex} we claimed
that the step $\redexthree$ is created by $\redex$, which indeed is.

Having a lemma that characterizes creation is relevant because in proofs,
it is common to have edge cases where a new redex is created.
A creation lemma can help handle those edge cases more easily.

The following lemma says that if $\redextwo$ creates $\redex$, then $\redextwo$ follows
one of three general forms.


\begin{lemma}[Creation]
\llem{creation}
There are three creation cases in the distributive lambda-calculus:
\begin{enumerate}
\item {\bf Creation case I.}
  $
    \con\of{ (\lamp{\lab}{\var}{\var^{\typ}})\,[\lamp{\labtwo}{\vartwo}{\tm}]\, \ls{\tmtwo}}
    \ \todist\ %
    \con\of{ (\lamp{\labtwo}{\vartwo}{\tm})\,\ls{\tmtwo} }
    \ \todist\ %
    \con\of{ \tm\sub{\vartwo}{\ls{\tmtwo}} }
  $.
\item {\bf Creation case II.}
  $
     \con\of{ (\lamp{\lab}{\var}{ \lamp{\labtwo}{\vartwo}{\tm} })\,\ls{\tmtwo}\,\ls{\tmthree} }
     \ \todist\ %
     \con\of{ (\lamp{\labtwo}{\vartwo}{\tm'})\,\ls{\tmthree} }
     \ \todist\ %
     \con\of{ \tm'\sub{\vartwo}{\ls{\tmthree}} }
  $, where:
  \[
     \begin{array}{rcl}
     \tm' & = & \tm\sub{\var}{\ls{\tmtwo}}
     \end{array}
  \]
\item {\bf Creation case III.}
  $
    \con_1\of{(\lamp{\lab}{\var}{\con_2\of{\var^{\typ}\,\ls{\tm}}})\ls{\tmtwo}}
    \ \todist\ %
    \con_1\of{\con'_2\of{(\lamp{\labtwo}{\vartwo}{\tmthree})\,\ls{\tm'}}}
    \ \todist\ %
    \con_1\of{\con'_2\of{\tmthree\sub{\vartwo}{\ls{\tm'}}}}
  $,
  where:
  \[
  \begin{array}{rcl}
    \con_2\sub{\var}{\ls{\tmtwo}}      & = & \con'_2 \\
    \var^{\typ}\sub{\var}{\ls{\tmtwo}} & = & \lamp{\labtwo}{\vartwo}{\tmthree} \\
    \ls{\tm}\sub{\var}{\ls{\tmtwo}}    & = & \ls{\tm'} \\
    \ls{\tmtwo}                        & = & [\ls{\tmtwo_1},\lamp{\labtwo}{\vartwo}{\tmthree},\ls{\tmtwo_2}] \\
  \end{array}
  \]
\end{enumerate}
\end{lemma}
\begin{proof}
Let $\redex : \conof{(\lamp{\lab}{\var}{\tm})\ls{\tmtwo}} \todist \conof{\tm\sub{\var}{\ls{\tmtwo}}}$ be a step,
and let $\redextwo : \conof{\tm\sub{\var}{\ls{\tmtwo}}} \todist \tmfive$ another step
such that $\redex$ creates $\redextwo$.
The redex contracted by the step $\redextwo$ is below a context $\con_1$,
so let $\conof{\tm\sub{\var}{\ls{\tmtwo}}} = \con_1\of{(\lamp{\labtwo}{\vartwo}{\tmthree})\ls{\tmfour}}$,
where $(\lamp{\labtwo}{\vartwo}{\tmthree})\ls{\tmfour}$ is the redex contracted by $\redextwo$.
We need consider three cases, depending on the relative positions of the holes of $\con$ and $\con_1$,
namely they may be disjoint,
$\con$ may be a prefix of $\con_1$,
or $\con_1$ may be a prefix of $\con$.
\SeeAppendixRef{creation_proof}
\end{proof}

\bigskip
\subsection*{Stability}

We also would like to prove a property that we call \emph{full stability},
which is a strong version of stability in the sense of Lévy \cite{levy_redex_stability}
(which can in turn be traced back to Berry's notion of stability).

Recall that L\'evy's stability states that for any $\redex \neq \redextwo$,
if $\redexthree_1$ and $\redexthree_2$ have a common descendant $\redexthree_3$, as in the figure below,
then they have a common ancestor $\redexthree_0$.
\[
  \xymatrix@C=.5cm@R=.5cm{
    &&&
  \\
    &
    & \ar@{.>}[u]^{\redexthree_0} \ar[dl]_{\redex} \ar[dr]^{\redextwo} &
  \\
    &
    \ar[l]_{\redexthree_1}
    \ar[dr]_{\redextwo/\redex} & & \ar[dl]^{\redex/\redextwo} \ar[r]^{\redexthree_2} &
  \\
    &
    & \ar[d]_{\redexthree_3} &
  \\
    &&&
  }
\]

In the figure above, $\redexthree_1$ and $\redexthree_2$ have a common descendant
$\redexthree_3$, which means that $\redexthree_3 \in \redexthree_1 / (\redextwo / \redex)$
and $\redexthree_3 \in \redexthree_2 / (\redex / \redextwo)$.
Informally, this means that $\redexthree_1$ and $\redexthree_2$ do the same work,
which means that that work is enabled by either performing $\redex$ or $\redextwo$.
The fact that a system is stable means that we could actually perform the work that
$\redexthree_1$ and $\redexthree_2$ do, without executing neither $\redex$ nor $\redextwo$
(by doing $\redexthree_0$).

Hence, what stability means is that steps are created essentially in a unique way.


\begin{lemma}[Stability]
\llem{basic_stability}
If $\redex,\redextwo$ are different coinitial steps such that
$\redex$ creates a step $\redexthree$,
then $\redex/\redextwo$ creates the step $\redexthree/(\redextwo/\redex)$.

More specifically:
\[
  \xymatrix@R=.5cm@C=2cm{
    \ar[r]^{\redex}
    \ar[d]_{\redextwo}
    &
    \ar[r]^{\redexthree}
    \ar[d]_{\redextwo/\redex}
    &
    \ar[d]_{\redextwo/\redex\redexthree}
  \\
    \ar[r]_{\redex/\redextwo}
    &
    \ar[r]_{\redexthree/(\redextwo/\redex)}
    &
  }
\]
\emph{Note:} It is easy to see that this is equivalent to stability in the sense of Lévy.
\end{lemma}
\begin{proof}
\SeeAppendixRef{basic_stability_proof}
Let $\redex : \conof{(\lamp{\lab}{\var}{\tm})\ls{\tmtwo}} \todist \conof{\subs{\tm}{\var}{\ls{\tmtwo}}}$,
let $\redextwo \neq \redex$ be a step coinitial to $\redex$,
and suppose that $\redex$ creates a step $\redexthree$.
By induction on the context $\con$ we argue that $\redex/\redextwo$ creates $\redexthree/(\redextwo/\redex)$.
\end{proof}

Next we prove \emph{full stability}, which is an extension to Lévy's notion of
stability, from steps to arbitrary reductions. Informally, it says that if we are able
to execute the same step after two completely different derivations,
then the step existed before
both reductions---\ie, a step is only created by a single reduction (modulo permutation
equivalence).

\begin{lemma}[Full stability]
\llem{full_stability}
Let $\redseq$ and $\redseqtwo$ be coinitial derivations such that
$\names(\redseq) \cap \names(\redseqtwo) = \emptyset$.
Let $\redexthree_1$, $\redexthree_2$, and $\redexthree_3$
be steps such that
$\redexthree_3 = \redexthree_1/(\redseqtwo/\redseq) = \redexthree_2/(\redseq/\redseqtwo)$.
Then there exists a step $\redexthree_0$ such that
$\redexthree_1 = \redexthree_0/\redseq$ and $\redexthree_2 = \redexthree_0/\redseqtwo$.
Diagrammatically:
\[
  \xymatrix@R=.5cm@C=.5cm{
    &&&
  \\
    &
    & \ar@{.>}[u]^{\redexthree_0} \ar@{->>}[ld]_{\redseq} \ar@{->>}[rd]^{\redseqtwo} &
  \\
    &
    \ar[l]_{\redexthree_1}
    \ar@{->>}[rd]_{\redseqtwo/\redseq} & & \ar@{->>}[ld]^{\redseq/\redseqtwo} \ar[r]^{\redexthree_2}
    &
  \\
    &
    &
    \ar[d]_{\redexthree_3}
    &
  \\
    &&&
  }
\]
\end{lemma}
\begin{proof}
We first prove the proposition in the particular case in which $\redseq$ is a single step, \ie $\redseq = \redex$
By induction on $\redseqtwo$:
\begin{enumerate}
\item {\bf Empty, $\redseqtwo = \emptyDerivation$.}
  Then $\redexthree_1 = \redexthree_3 = \redexthree_2/\redex$,
  so it suffices to take $\redexthree_0 := \redexthree_2$.
\item {\bf Non-empty, $\redseqtwo = \redextwo\,\redseqtwo'$.}
  Recall that permutation diagrams in the distributive lambda-calculus are {\em square},
  (\ie steps always have exactly one residual, except for the trivial case $\redex/\redex = \emptyset$,
  as was proved in \rlem{cardinality_of_the_set_of_residuals}).
  Since $\name(\redex) \not\in \names(\redextwo\redseqtwo')$,
  we know that $\redex$ is not any of the steps along $\redextwo\redseqtwo'$,
  so in particular $\redex/\redextwo$ and $\redex/\redextwo\redseqtwo'$
  are singletons, which means that the situation is the following:
  \[
    \xymatrix{
      &
      &
        \ar[ld]_{\redex} \ar[rd]^{\redextwo}
      &
    \\
      &
      \ar[l]_{\redexthree_1}
      \ar[rd]_{\redextwo/\redex}
      &
      &
        \ar[ld]_-{\redex/\redextwo} \ar@{->>}[rd]^{\redseqtwo'}
      & 
      &
    \\
      &
      &
        \ar@{->>}[rd]_-{\redseqtwo'/(\redex/\redextwo)}
      &
      &
        \ar@{->}[dl]^{\redex/\redextwo\redseqtwo'}
        \ar[r]^{\redexthree_2}
      &
    \\
      &
      &
      &
      \ar[d]_{\redexthree_3}
      &
    \\
      &
      &
      &
    }
  \]
  Observe that $\redexthree_3 = \redexthree_1/((\redextwo/\redex)(\redseqtwo'/(\redex/\redextwo)))$,
  so by taking $\redexthree'_1 := \redexthree_1/(\redextwo/\redex)$
  we have that $\redexthree_3 = \redexthree'_1/(\redseqtwo'/(\redex/\redextwo))$.
  By \ih on $\redseqtwo'$ we have that
  there exists a step $\redexthree'_2$ such that
  $\redexthree'_1 = \redexthree'_2/(\redex/\redextwo)$ and
  $\redexthree_2 = \redexthree'_2/\redseqtwo'$.

  To conclude, note that $\redex \neq \redextwo$ and
  $\redexthree'_1 = \redexthree_1/(\redextwo/\redex) = \redexthree_0/(\redex/\redextwo)$
  so by the Stability~lemma (\rlem{basic_stability})
  there must exist a step $\redexthree_0$ such that
  $\redexthree_1 = \redexthree_0/\redex$
  and $\redexthree'_2 = \redexthree_0/\redextwo$.
  Moreover $\redexthree_2 = \redexthree'_2/\redseqtwo' = \redexthree_0/\redextwo\redseqtwo'$,
  as required.
\end{enumerate}
Having established the previous claim, let us now prove the main statement of the
proposition by induction on $\redseq$.
\begin{enumerate}
\item {\bf Empty, $\redseq = \emptyDerivation$.}
  Then $\redexthree_2 = \redexthree_3 = \redexthree_1/\redseqtwo$ so by taking $\redexthree_0 := \redexthree_1$
  we conclude.
\item {\bf Non-empty, $\redseq = \redex\redseq'$.}
  First observe that, since $\names(\redex\redseq') \cap \names(\redseqtwo) = \emptyset$,
  we have that $\names(\redex) \cap \names(\redseqtwo) = \emptyset$ 
  and $\names(\redseq') \cap \names(\redseqtwo) = \emptyset$.
  The situation is the following:
  \[
    \xymatrix{
      &
      &
      &
        \ar[dl]_{\redex}
        \ar[dr]^{\redseqtwo}
      &
    \\
      &
      &
        \ar@{->>}[dl]_{\redseq'}
        \ar@{->>}[dr]^{\redseqtwo/\redex}
      &
      &
        \ar[dl]^{\redex/\redseqtwo}
        \ar[r]^{\redexthree_2}
      &
    \\
      &
        \ar[l]_{\redexthree_1}
        \ar@{->>}[dr]^{\redseqtwo/\redex\redseq'}
      &
      &
        \ar@{->>}[dl]^{\redseq'/(\redseqtwo/\redex)}
      &
    \\
      &
      &
        \ar[d]_{\redexthree_3}
      &
      &
    \\
      &
      &
      &
      &
    \\
    }
  \]
  Observe that $\redexthree_3 = \redexthree_2/((\redex/\redseqtwo)(\redseq'/(\redseqtwo/\redex)))$,
  so by taking $\redexthree'_2 := \redexthree_2/(\redex/\redseqtwo)$
  we have that $\redexthree_3 = \redexthree'_2/(\redseq'/(\redseqtwo/\redex))$.
  By \rlem{names_after_projection_along_a_step}
  we have that $\names(\redseqtwo/\redex) = \names(\redseqtwo) \setminus \set{\name(\redex)}$
  and $\name(\redex) \not\in \names(\redseqtwo)$,
  so $\names(\redseqtwo/\redex) = \names(\redseqtwo)$.
  In particular, $\names(\redseq') \cap \names(\redseqtwo/\redex) = \emptyset$
  so we may apply the \ih on $\redseq'$ to conclude that there exists
  a step $\redexthree'_1$ such that
  $\redexthree_1 = \redexthree'_1/\redseq'$
  and
  $\redexthree'_2 = \redexthree'_1/(\redseqtwo/\redex)$.

  To conclude, observe that $\redexthree'_2 = \redexthree'_1/(\redseqtwo/\redex) = \redexthree_2/(\redex/\redseqtwo)$,
  where $\name(\redex) \not\in \names(\redseqtwo)$,
  so by the previous claim we have that
  there is a step $\redexthree_0$ such that
  $\redexthree_0/\redex = \redexthree'_1$ and
  $\redexthree_0/\redseqtwo = \redexthree_2$.
  Moreover, $\redexthree_0/\redex\redseq' = \redexthree'_1/\redex = \redexthree_1$ 
  so we are done.
\end{enumerate}
\end{proof}
