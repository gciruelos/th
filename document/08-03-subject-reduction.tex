We are to prove that
if $\conof{(\lamp{\lab}{\var}{\tm})\ls{\tmtwo}}$ is correct
then $\conof{\subs{\tm}{\var}{\ls{\tmtwo}}}$ is correct,
and, moreover, that their unique typings are under the same typing context
and have the same type.
Also, we check that correctness is preserved.
We need a few auxiliary results:


\begin{lemma}
\llem{substitution_permutation}
If $\ls{\tmtwo}$ is a permutation of $\ls{\tmthree}$, then $\subs{\tm}{\var}{\ls{\tmtwo}} = \subs{\tm}{\var}{\ls{\tmthree}}$.
\end{lemma}
\begin{proof}
By induction on $\tm$.
\end{proof}

\begin{lemma}
\llem{correct_subterms}
If $\tm$ is correct, then any subterm of $\tm$ is correct.
\end{lemma}
\begin{proof}
Note, by definition of correctness~(\rdef{sequentiality_and_correctness}),
that if $\tm$ is a correct abstraction $\tm = \lamp{\lab}{\var}{\tmtwo}$ then $\tmtwo$ is correct,
and if $\tm$ is a correct application $\tm = \tmtwo[\tmthree_1,\hdots,\tmthree_n]$
then $\tmtwo$ and $\tmthree_1,\hdots,\tmthree_n$ are correct.
This allows us to conclude by induction on $\tm$.
\end{proof}

\begin{lemma}[Relevance]
\llem{relevance}
If $\tctx \vdash \tm : \typ$
and $\var \in \dom\tctx$ then $\var \in \fv{\tm}$.
\end{lemma}
\begin{proof}
By induction on $\tm$.
\end{proof}

\begin{definition}
$\Lambda(\tm)$ stands for the multiset of labels decorating the lambdas of $\tm$:
\[
  \begin{array}{rcl}
    \Lambda(\var^\typ) & \eqdef & [\,] \\
    \Lambda(\lamp{\lab}{\var}{\tm}) & \eqdef & [\lab] + \Lambda(\tm)  \\
    \Lambda(\tm[\tmtwo_i]_{i=1}^{n}) & \eqdef & \Lambda(\tm) +_{i=1}^{n} \Lambda(\tmtwo_i)  \\
  \end{array}
\]
\end{definition}
For example, $\Lambda((\lamp{1}{\var}{\var^{[\alpha^2] \tolab{3} \alpha^2}})[\lamp{3}{\var}{\var^{\alpha^2}}]) = [1,3]$.

\begin{lemma}
\llem{labels_over_lambdas_substitution}
Let $\tm,\tmtwo_1,\hdots,\tmtwo_n$ be correct terms.
Then
$\Lambda(\subs{\tm}{\var}{[\tmtwo_i]_{i=1}^{n}}) = \Lambda(\tm) +_{i=1}^n \Lambda(\tmtwo_i)$.
\end{lemma}
\begin{proof}
By induction on $\tm$.
\end{proof}

To prove \rlem{subject_reduction}, we first show that substitution preserves typing,
and then that it preserves correctness.

\subsubsection{Substitution preserves typing}
\lsec{substitution_preserves_typing}

  More precisely, let us show that
  if $\tctx \vdash \conof{(\lamp{\lab}{\var}{\tm})\ls{\tmtwo}} : \typ$ is derivable,
  then $\tctx \vdash \conof{\subs{\tm}{\var}{\ls{\tmtwo}}} : \typ$ is derivable.
  By induction on the context $\con$.
  \begin{enumerate}
  \item {\bf Empty, $\con = \conbase$.}
    \label{subject_reduction__case_base_empty_context}
    By induction on $\tm$.
    \begin{enumerate}
    \item {\bf Variable (same), $\tm = \var^\typ$, $\ls{\tmtwo} = [\tmtwo]$.}
      We have that $\var : [\typ] \vdash \var^\typ : \typ$
      and $\tctxtwo \vdash \tmtwo : \typtwo$ are derivable,
      so we are done.
    \item {\bf Variable (different), $\tm = \vartwo^\typ$, $\vartwo \neq \var$, $\ls{\tmtwo} = []$.}
      We have that $\vartwo : [\typ] \vdash \vartwo^\typ : \typ$ is derivable, so we are done.
    \item {\bf Abstraction, $\tm = \lamp{\lab}{\vartwo}{\tmthree}$.}
      Let $\tctx \oplus \var : [\typtwo_i]_{i=1}^n \vdash \lamp{\lab}{\var}{\tmthree} : \mtyp \tolab{\lab} \typ$
      be derivable
      and $\tctxtwo_i \vdash \tmtwo_i : \typtwo_i$ be derivable for all $i=1..n$.
      By inversion of the \indrulename{\toI} rule,
      we have that $\tctx \oplus \vartwo : \mtyp \oplus \var : [\typtwo_i]_{i=1}^n \vdash \tmthree : \typ$
      is derivable, so by \ih
      $(\tctx \oplus \vartwo : \mtyp) +_{i=1}^{n} \tctxtwo_i \vdash \subs{\tmthree}{\var}{[\tmtwo_i]_{i=1}^n} : \typ$ is derivable.
      Observe that $\vartwo \not\in \fv{\tmtwo_i}$ so $\vartwo \not\in \dom\tctxtwo_i$ by \rlem{relevance}.
      Hence the previous judgment can be written as
      $(\tctx +_{i=1}^{n} \tctxtwo_i) \oplus \vartwo : \mtyp \vdash \subs{\tmthree}{\var}{[\tmtwo_i]_{i=1}^n} : \typ$.
      Applying the \indrulename{\toI} rule we obtain
      $\tctx +_{i=1}^{n} \tctxtwo_i \vdash \subs{\lamp{\lab}{\vartwo}{\tmthree}}{\var}{[\tmtwo_i]_{i=1}^n} : \mtyp \tolab{\lab} \typ$
      as required.
    \item {\bf Application, $\tm = \tmthree[\tmfour_j]_{j=1}^m$.}
      Let $\tctx \oplus \var : [\typtwo_i]_{i=1}^n \vdash \tmthree[\tmfour_j]_{j=1}^m : \typ$
      be derivable
      and $\tctxtwo_i \vdash \tmtwo_i : \typtwo_i$ be derivable for all $i=1..n$.
      By inversion of the \indrulename{\toE} rule,
      the multiset of types $[\typtwo_i]_{i=1}^n$
      may be partitioned as $[\typtwo_i]_{i=1}^n = \sum_{j=0}^{m} \mtyp_j$,
      and
      the typing context $\tctx$
      may be partitioned as $\tctx = \sum_{j=0}^{m} \tctx_j$
      in such a way that
      $\tctx_0 \oplus \var : \mtyp_0 \vdash \tmthree : [\typthree_j]_{j=1}^m \tolab{\lab} \typ$
      is derivable
      and
      $\tctx_j \oplus \var : \mtyp_j \vdash \tmfour_j : \typthree_j$ is derivable for all $j=1..m$.
      Consider a partition $(\ls{\tmtwo}_0,\hdots,\ls{\tmtwo}_j)$
      of the list $\ls{\tmtwo}$ 
      such that for every $j=0..m$ we have $\tmlabel{\ls{\tmtwo}_j} = \mtyp_j$.
      Observe that this partition exists since $\tmlabel{\ls{\tmtwo}_0 + \hdots + \ls{\tmtwo}_j} = \tmlabel{\ls{\tmtwo}} = \sum_{j=0}^m \mtyp_j = [\typtwo_i]_{i=1}^n = \varlabel{\var}{\tm}$.

      Moreover, let $\tctxthree_j = \sum_{i : \tmtwo_i \in \ls{\tmtwo}_j} \tctxtwo_i$ for all $j=0..m$.
      By \ih we have that
      $\tctx_0 + \tctxthree_0 \vdash \subs{\tmthree}{\var}{\ls{\tmtwo}_0} : [\typthree_j]_{j=1}^m \tolab{\lab} \typ$
      is derivable
      and
      $\tctx_j + \tctxthree_j \vdash \subs{\tmfour_j}{\var}{\ls{\tmtwo}_j} : \typthree_j$
      is derivable for all $j=1..m$.
      Applying the \indrulename{\toE} rule we obtain
      that
      $\sum_{j=0}^m \tctx_j + \sum_{j=0}^m \tctxthree_j \vdash \subs{\tmthree[\tmfour_j]}{\var}{\sum_{j=0}^m \ls{\tmtwo}_j} : \typ$
      is derivable.
      By definition of
      $\tctx_0,\hdots,\tctx_m$ and
      $\tctxthree_0,\hdots,\tctxthree_m$
      this judgment equals
      $\tctx +_{i=1}^n \tctxtwo_i \vdash \subs{\tmthree[\tmfour_j]}{\var}{\sum_{j=0}^m \ls{\tmtwo}_j} : \typ$.
      By definition of $\ls{\tmtwo}_0,\hdots,\ls{\tmtwo}_m$
      and \rlem{substitution_permutation}
      this in turn equals
      $\tctx +_{i=1}^n \tctxtwo_i \vdash \subs{\tmthree[\tmfour_j]}{\var}{\ls{\tmtwo}} : \typ$,
      as required.
    \end{enumerate}
  \item {\bf Under an abstraction, $\con = \lamp{\labtwo}{\vartwo}{\con'}$.}
    Straightforward by \ih.
  \item {\bf Left of an application, $\con = \con'\ls{\tmthree}$.}
    Straightforward by \ih.
  \item {\bf Right of an application, $\con = \tmthree[\ls{\tmfour}_1,\con',\ls{\tmfour}_2]$.}
    Straightforward by \ih.
  \end{enumerate}

\subsubsection{Substitution preserves correctness}

  More precisely, let us show that
  if $\conof{(\lamp{\lab}{\var}{\tm})\ls{\tmtwo}}$ is correct
  then $\conof{\subs{\tm}{\var}{\ls{\tmtwo}}}$ is correct.
  By induction on $\con$:
  \begin{enumerate}
  \item {\bf Empty, $\con = \conbase$.}
    \label{subject_reduction__case_base_empty_context}
    Let $\ls{\tmtwo} = [\tmtwo_1,\hdots,\tmtwo_n]$.
    Observe that if $(\lamp{\lab}{\var}{\tm})[\tmtwo_1,\hdots,\tmtwo_n]$ is correct then:
    \begin{itemize}
    \item \condp{1} $\tctx \oplus \var : [\typtwo_i]_{i=1}^{n} \vdash \tm : \typ$ and $\tctxtwo_i \vdash \tmtwo_i : \typtwo_i$ are derivable for all $i=1..n$,
    \item \condp{2} $\tm,\tmtwo_1,\hdots,\tmtwo_n$ are correct,
    \item \condp{3} there are no free occurrences of $\var$ among $\tmtwo_1,\hdots,\tmtwo_n$,
    \item \condp{4} all the lambdas occurring in $\tm,\tmtwo_1,\hdots,\tmtwo_n$ have pairwise distinct labels,
    \item \condp{5} $\tctx +_{i=1}^n \tctxtwo_i$ is a sequential context.
    \end{itemize}
    Condition \condp{1} holds by inversion of the typing rules,
    condition \condp{2} holds by~\rlem{correct_subterms},
    condition \condp{3} holds by Barendregt's convention,
    and conditions \condp{4} and \condp{5} hold because the source term is supposed to be correct.

    By induction on $\tm$, we check that
    if $(\lamp{\lab}{\var}{\tm})[\tmtwo_1,\hdots,\tmtwo_n]$
    is correct, then
    $\subs{\tm}{\var}{\ls{\tmtwo}}$ is correct.
    \begin{enumerate}
    \item {\bf Variable (same), $\tm = \var^{\typ}$, $\ls{\tmtwo} = [\tmtwo]$.}
      Note that $\subs{\tm}{\var}{\ls{\tmtwo}} = \tmtwo$. Conclude by \condp{2}.
    \item {\bf Variable (different), $\tm = \vartwo^{\typ}$, $\ls{\tmtwo} = []$ with $\var \neq \vartwo$.}
      Note that $\subs{\tm}{\var}{\ls{\tmtwo}} = \vartwo^\typ$.
      Conclude by \condp{2}.
    \item {\bf Abstraction, $\tm = \lamp{\labtwo}{\vartwo}{\tmthree}$.}
      \label{subject_reduction__case_abstraction}
      Then $\typ = \mtyp \tolab{\labtwo} \typthree$ and
      by inversion $\tctx \oplus \vartwo : \mtyp \vdash \tmthree : \typthree$ is derivable.
      Note that $(\lamp{\lab}{\var}{\tmthree})\ls{\tmtwo}$ is correct,
      so by \ih $\subs{\tmthree}{\var}{\ls{\tmtwo}}$ is correct.
      The variable $\vartwo$ does not occur free in $\ls{\tmtwo}$,
      so $(\tctx \oplus \vartwo : \mtyp) +_{i=1}^{n} \tctxtwo_i = (\tctx +_{i=1}^{n} \tctxtwo_i) \oplus (\vartwo : \mtyp)$.
      Let us check that $\lamp{\labtwo}{\vartwo}{\subs{\tmthree}{\var}{\ls{\tmtwo}}}$ is correct:
      \begin{enumerate}
      \item {\em Uniquely labeled lambdas.}
        Let $\lab_1$ and $\lab_2$ be two labels decorating different lambdas of 
        $\lamp{\labtwo}{\vartwo}{\subs{\tmthree}{\var}{\ls{\tmtwo}}}$,
        and let us show that $\lab_1 \neq \lab_2$.
        There are two subcases, depending on whether one of the labels
        decorates the outermost lambda:
        \begin{enumerate}
        \item
          {\bf If $\lab_1$ or $\lab_2$ decorates the outermost lambda.}
          Suppose without loss of generality that $\lab_1 = \labtwo$ is the label decorating the outermost lambda.
          Then by \rlem{labels_over_lambdas_substitution},
          there are two cases: either $\lab_2$ decorates a lambda of $\tmthree$,
          or $\lab_2$ decorates a lambda of some term in the list $\ls{\tmtwo}$.
          If $\lab_2$ decorates a lambda of $\tmthree$, then $\lab_1 \neq \lab_2$
          since we knew that $\lamp{\labtwo}{\vartwo}{\tmthree}$ was a correct term by \condp{2}.
          If $\lab_2$ decorates a lambda of some term in the list $\ls{\tmtwo}$, then $\lab_1 \neq \lab_2$
          by condition \condp{4}.
        \item
          {\bf If $\lab_1$ and $\lab_2$ do not decorate the outermost lambda.}
          Then $\lab_1$ and $\lab_2$ decorate different lambdas of the term $\subs{\tmthree}{\var}{\ls{\tmtwo}}$,
          and we conclude by \ih.
        \end{enumerate}
      \item {\em Sequential contexts.}
        \label{subject_reduction__case_abstraction_sequential_contexts}
        Let $\tm'$ be a subterm of $\lamp{\labtwo}{\var}{\subs{\tmthree}{\var}{\ls{\tmtwo}}}$.
        If $\tm'$ is a subterm of $\subs{\tmthree}{\var}{\ls{\tmtwo}}$, we conclude by \ih.
        Otherwise $\tm'$ is the whole term and the context is $\tctx +_{i=1}^{n} \tctxtwo_i$,
        which is sequential by hypothesis \condp{5}.
      \item {\em Sequential types.}
        \label{subject_reduction__case_abstraction_sequential_types}
        Let $\tm'$ be a subterm of $\lamp{\labtwo}{\var}{\subs{\tmthree}{\var}{\ls{\tmtwo}}}$.
        If $\tm'$ is a subterm of $\subs{\tmthree}{\var}{\ls{\tmtwo}}$, we conclude by \ih.
        Otherwise $\tm'$ is the whole term. Then $\tctx +_{i=1}^{n} \tctxtwo_i \vdash \tm' : \typ$ is derivable,
        since we have already shown that substitution preserves typing.
        Let $\mtyptwo \tolab{\labthree} \typfour$ be
        a type such that $\mtyptwo \tolab{\labthree} \typfour \occursin \tctx +_{i=1}^{n} \tctxtwo_i$ or $\mtyptwo \tolab{\labthree} \typfour \occursin \typ$,
        and let us show that $\mtyptwo$ is sequential.
        In the first case, \ie if $\mtyptwo \tolab{\labthree} \typfour \occursin \tctx +_{i=1}^{n} \tctxtwo_i$ holds,
        then either $\mtyptwo \tolab{\labthree} \typfour \occursin \tctx$ or $\mtyptwo \tolab{\labthree} \typfour \occursin \tctxtwo_i$ for some $i=1..n$,
        and we have that $\mtyptwo$ is sequential because all the terms $\tm,\tmtwo_1,\hdots,\tmtwo_n$ are correct by \condp{2}.
        In the second case, \ie if $\mtyptwo \tolab{\labthree} \typfour \occursin \typ$ holds,
        then we have that $\mtyptwo$ is sequential because $\tm$ is correct by \condp{2}.
      \end{enumerate}
    \item {\bf Application, $\tm = \tmthree\,\ls{\tmfour}$.}
      \label{subject_reduction__case_application} 
      Let $\ls{\tmfour} = [\tmfour_1,\hdots,\tmfour_m]$.
      Note that $(\lamp{\lab}{\var}{\tmthree})\ls{\tmtwo}_0$ is correct
      and $(\lamp{\lab}{\var}{\tmfour_j})\ls{\tmtwo}_j$ is correct for all $j=1..m$,
      which means that we may apply the \ih in all these cases.
      Let us show that $\subs{\tmthree[\tmfour_j]_{j=1}^m}{\var}{\ls{\tmtwo}}$ is correct:
      \begin{enumerate}
      \item {\em Uniquely labeled lambdas.}
        Let $\lab_1$ and $\lab_2$ be two labels decorating different lambdas of
        $\subs{\tmthree[\tmfour_j]_{j=1}^m}{\var}{\ls{\tmtwo}}$, and let us show that $\lab_1 \neq \lab_2$.
        Observe that the term
        $\subs{\tmthree[\tmfour_j]_{j=1}^m}{\var}{\ls{\tmtwo}} =
         \subs{\tmthree}{\var}{\ls{\tmtwo}_0}[\subs{\tmfour_j}{\var}{\ls{\tmtwo}_j}]_{j=1}^m$
        has $m + 1$ immediate subterms, namely
        $\subs{\tmthree}{\var}{\ls{\tmtwo}_0}$
        and $\subs{\tmfour_j}{\var}{\ls{\tmtwo}_j}$ for each $j=1..m$.
        We consider two subcases, depending on whether $\lab_1$ and $\lab_2$
        decorate two lambdas in the same immediate subterm or in different immediate subterms.

        \begin{enumerate}
        \item {\bf The labels $\lab_1$ and $\lab_2$ decorate the same immediate subterm.}
          That is, $\lab_1$ and $\lab_2$ both decorate lambdas in $\subs{\tmthree}{\var}{\ls{\tmtwo}_0}$
          or both decorate lambdas in some $\subs{\tmfour_j}{\var}{\ls{\tmtwo}_j}$ for some $j=1..m$.
          Then we conclude, since both $\subs{\tmthree}{\var}{\ls{\tmtwo}_0}$ and
          the $\subs{\tmfour_j}{\var}{\ls{\tmtwo}_j}$ are correct by \ih.
        \item {\bf The labels $\lab_1$ and $\lab_2$ decorate different subterms.}
          Let $\tmfour_0 := \tmthree$.
          Then we have that
          $\lab_1$ decorates a lambda in $\subs{\tmfour_j}{\var}{\ls{\tmtwo}_j}$ for some $j=0..m$
          and
          $\lab_2$ decorates a lambda in $\subs{\tmfour_k}{\var}{\ls{\tmtwo}_k}$ for some $k=0..m$, $j \neq k$.
          By \rlem{labels_over_lambdas_substitution},
          $\lab_1$ decorates a lambda in $\tmfour_j$ or a lambda in a term of the list $\ls{\tmtwo}_j$,
          and similarly
          $\lab_2$ decorates a lambda in $\tmfour_k$ or a lambda in a term of the list $\ls{\tmtwo}_k$.
          This leaves four possibilities, which are all consequence of \condp{4}.
        \end{enumerate}
      \item {\em Sequential contexts.}
        Similar to item~\refcase{subject_reduction__case_abstraction_sequential_contexts}.
      \item {\em Sequential types.}
        Similar to item~\refcase{subject_reduction__case_abstraction_sequential_types}.
      \end{enumerate}
    \end{enumerate}
  \item {\bf Under an abstraction, $\con = \lamp{\labtwo}{\vartwo}{\con'}$.}
    Note that $\tctx \vdash \lamp{\labtwo}{\vartwo}{\con'\of{\subs{\tm}{\var}{\ls{\tmtwo}}}} : \mtyp \tolab{\labtwo} \typ$ is derivable.
    Let us check the three conditions to see that $\lamp{\labtwo}{\vartwo}{\con'\of{\subs{\tm}{\var}{\ls{\tmtwo}}}}$ is correct:
    \begin{enumerate}
    \item {\em Uniquely labeled lambdas.}
      Any two lambdas in $\con'\of{\subs{\tm}{\var}{\ls{\tmtwo}}}$ have different labels by \ih.
      We are left to check that $\labtwo$ does not decorate any lambda in $\con'\of{\subs{\tm}{\var}{\ls{\tmtwo}}}$.
      Let $\lab_1$ be a label that decorates a lambda in $\con'\of{\subs{\tm}{\var}{\ls{\tmtwo}}}$.
      Then we have that $\lab_1$ decorates a lambda in $\con'$,
      or it decorates a lambda in $\subs{\tm}{\var}{\ls{\tmtwo}}$.
      By what we proved in item~\refcase{subject_reduction__case_base_empty_context}
      this in turn means that it decorates a lambda in $\tm$ or a lambda in some of the terms of the list $\ls{\tmtwo}$.
      In any of these cases we have that $\lab_1 \neq \labtwo$
      since $\lamp{\labtwo}{\vartwo}{\con'\of{(\lamp{\lab}{\var}{\tm})\ls{\tmtwo}}}$ is correct.
    \item {\em Sequential contexts.}
      Let $\tm'$ be a subterm of $\lamp{\labtwo}{\vartwo}{\con'\of{\subs{\tm}{\var}{\ls{\tmtwo}}}}$
      and let us check that its typing context is sequential.
      If $\tm'$ is a subterm of $\con'\of{\subs{\tm}{\var}{\ls{\tmtwo}}}$ we conclude by \ih.
      We are left to check the property for $\tm'$ being the whole term, \ie that $\tctx$ is sequential.
      By \ih, $\tctx \oplus \vartwo : \mtyp$ is sequential, which implies that $\tctx$ is sequential.
    \item {\em Sequential types.}
      Let $\tm'$ be a subterm of $\lamp{\labtwo}{\vartwo}{\con'\of{\subs{\tm}{\var}{\ls{\tmtwo}}}}$
      and let us check that, if $\mtyptwo \tolab{\labthree} \typthree$ is any type occurring in the typing
      context or in the type of $\tm'$, then $\mtyptwo$ is sequential.
      If $\tm'$ is a subterm of $\con'\of{\subs{\tm}{\var}{\ls{\tmtwo}}}$ we conclude by \ih.
      We are left to check the property for $\tm'$ being the whole term.

      If $\mtyptwo \tolab{\labthree} \typthree \occursin \tctx$,
      then $\mtyptwo \tolab{\labthree} \typthree \occursin \tctx \oplus \vartwo:\mtyptwo$
      which is the type of $\con'\of{\subs{\tm}{\var}{\ls{\tmtwo}}}$,
      so by \ih $\mtyptwo$ is sequential.

      If $\mtyptwo \tolab{\labthree} \typthree \occursin \mtyp \tolab{\lab''} \typ$,
      there are three subcases:
      \begin{enumerate}
      \item If $\mtyptwo = \mtyp$, then note that $\mtyp$ is sequential
            because $\tctx \oplus \vartwo : \mtyp$ is the typing context of $\con'\of{\subs{\tm}{\var}{\ls{\tmtwo}}}$,
           which is sequential by \ih.
      \item If $\mtyptwo \tolab{\labthree} \typthree \occursin \typtwo$ where $\typtwo$ is one of the types of $\mtyp$,
            then $\mtyptwo \tolab{\labthree} \typthree \occursin \mtyp \tolab{\lab''} \typ$ which is the typing context
            of $\con'\of{\subs{\tm}{\var}{\ls{\tmtwo}}}$,
            and we conclude for this term has sequential types by \ih.
      \item If $\mtyptwo \tolab{\labthree} \typthree \occursin \typ$,
            note that $\typ$ is the type of $\con'\of{\subs{\tm}{\var}{\ls{\tmtwo}}}$,
            and we conclude for this term has sequential types by \ih.
      \end{enumerate}
    \end{enumerate}
  \item {\bf Left of an application, $\con = \con'\ls{\tmthree}$.}
    \label{subject_reduction__case_left_application}
    Note that $\tctx \vdash \con'\of{\subs{\tm}{\var}{\ls{\tmtwo}}} : [\typtwo_j]_{j=1}^m \tolab{\labtwo} \typ$
    is derivable.
    Moreover the list of arguments is of the form
    $\ls{\tmthree} = [\tmthree_1,\hdots,\tmthree_m]$ where all the $\tmthree_j$ are correct
    and $\tctxtwo_j \vdash \tmthree_j : \typtwo_j$ is derivable for all $j=1..m$.
    Then $\tctx +_{j=1}^m \tctxtwo_j \vdash \con'\of{\subs{\tm}{\var}{\ls{\tmtwo}}}[\tmthree_j]_{j=1}^{m} : \typ$
    is derivable.
    Let us check the three conditions to see that $\con'\of{\subs{\tm}{\var}{\ls{\tmtwo}}}[\tmthree_j]_{j=1}^{m}$ is correct:
    \begin{enumerate}
    \item {\em Uniquely labeled lambdas.}
      Let $\lab_1$ and $\lab_2$ be two labels decorating different lambdas in $\con'\of{\subs{\tm}{\var}{\ls{\tmtwo}}}[\tmthree_j]_{j=1}^{m}$.
      There are three subcases.
      \begin{enumerate}
      \item
        If $\lab_1$ and $\lab_2$ both decorate lambdas in the subterm $\con'\of{\subs{\tm}{\var}{\ls{\tmtwo}}}$
        then $\lab_1 \neq \lab_2$ since $\con'\of{\subs{\tm}{\var}{\ls{\tmtwo}}}$ is correct by \ih.
      \item
        If $\lab_1$ and $\lab_2$ both decorate lambdas somewhere in $[\tmthree_1,\hdots,\tmthree_m]$
        then $\lab_1 \neq \lab_2$ since $\con'\of{\tm}[\tmthree_1,\hdots,\tmthree_m]$ is correct by hypothesis.
      \item
        If $\lab_1$ decorates a lambda in $\con'\of{\subs{\tm}{\var}{\ls{\tmtwo}}}$
        and $\lab_2$ decorates a lambda in one of the terms $\tmthree_j$ for some $j=1..m$,
        then note that $\lab_1$ must either decorate a lambda in $\con'$ or a lambda in $\subs{\tm}{\var}{\ls{\tmtwo}}$.
        By what we proved in item~\refcase{subject_reduction__case_base_empty_context}
        this in turn means that it decorates a lambda in $\tm$ or a lambda in some of the terms of the list $\ls{\tmtwo}$.
        In any of these cases we have that $\lab_1 \neq \lab_2$
        since $\con'\of{(\lamp{\lab}{\var}{\tm})\ls{\tmtwo}}[\tmthree_j]_{j=1}^m$ is correct by hypothesis.
      \end{enumerate}
    \item {\em Sequential contexts.}
      Let $\tm'$ be a subterm of $\con'\of{\subs{\tm}{\var}{\ls{\tmtwo}}}[\tmthree_j]_{j=1}^m$
      and let us show that its typing context is sequential.
      If $\tm'$ is a subterm of $\con'\of{\subs{\tm}{\var}{\ls{\tmtwo}}}$ we conclude by \ih.
      If $\tm'$ is a subterm of one of the $\tmthree_j$ for some $j=1..m$, we conclude using that $\tmthree_j$ is correct by hypothesis.
      It remains to check that the whole term is correct, \ie that $\tctx +_{j=1}^m \tctxtwo_j$ is sequential.
      Observe that $\tctx +_{j=1}^m \tctxtwo_j$ is also the typing context of
      $\con'\of{(\lamp{\lab}{\var}{\tm})\ls{\tmtwo}}[\tmthree_j]_{j=1}^m$, which is correct by hypothesis.
    \item {\em Sequential types.}
      Let $\tm'$ be a subterm of $\con'\of{\subs{\tm}{\var}{\ls{\tmtwo}}}[\tmthree_j]_{j=1}^m$
      and let us show if $\mtyptwo \tolab{\labtwo} \typthree$ is a type
      that occurs in the typing context or in the type of $\tm'$, then $\mtyptwo$ is sequential.
      If $\tm'$ is a subterm of $\con'\of{\subs{\tm}{\var}{\ls{\tmtwo}}}$ we conclude by \ih.
      If $\tm'$ is a subterm of one of the $\tmthree_j$ for some $j=1..m$, we conclude using that $\tmthree_j$ is correct by hypothesis.
      We are left to check the property for $\tm'$ being the whole term.

      If $\mtyptwo \tolab{\labtwo} \typthree \occursin \tctx +_{j=1}^m \tctxtwo_j$,
      we conclude by observing
      that $\tctx +_{j=1}^m \tctxtwo_j$ is also the typing context of $\con'\of{(\lamp{\lab}{\var}{\tm})\ls{\tmtwo}}[\tmthree_j]_{j=1}^m$,
      which is correct by hypothesis, so it has sequential types.
      
      Similarly, if $\mtyptwo \tolab{\labtwo} \typthree \occursin \typ$,
      we conclude by observing
      that $\typ$ is also the type of $\con'\of{(\lamp{\lab}{\var}{\tm})\ls{\tmtwo}}[\tmthree_j]_{j=1}^m$.
    \end{enumerate}
  \item {\bf Right of an application, $\con = \tmthree[\ls{\tmfour}_1,\con',\ls{\tmfour}_2]$.}
    Similar to item~\refcase{subject_reduction__case_left_application}.
  \end{enumerate}

