
\chapter{Simulation of the $\lambda$-calculus}
\lcha{simulation}

In this chapter we will focus on establishing a correspondance between
the $\lambda$-calculus and the $\lambdadist$-calculus.
Our goal is to be able to prove that we can map derivations in one calculus
to derivations in the other.

First, we will learn how to relate terms from one calculus to terms in the other,
via what we call \emph{refinements}.
Then, we will look at how the derivation spaces of two related terms are connected.
After that we will learn that the fact that a term of the lambda calculus is
related to another in the distributive lambda calculus
says that it has a head normal form.
Finally, we will link and combine the residual theories of both calculi.


\begin{notation}
If $\tm$ is a term of the distributive lambda calculus,
we already defined $\ulbDerivDist{\tm}$ to be its derivation space,
which we proved is a distributive lattice.
$\termsdist$ is the set of all (correct) terms.

If $\tmtwo$ is a term of the pure lambda calculus,
we write $\ulbDerivLam{\tmtwo}$ to mean
the derivation space of $\tmtwo$, which
is an join semilattice.
$\terms$ is the set of all terms of the pure lambda calculus.
\end{notation}


\section{Refinements}
\lsec{refinements}

\begin{definition}[Refinement]
A lambda-term $\tm \in \terms$ is \defn{refined}
by a distributive term $\tm' \in \termsdist$,
written $\tm \refinedby \tm'$ according to the following inductive definition:
\[
  \indrule{r-var}{
  }{
    \var \refinedby \var^\typ
  }
  \indrule{r-lam}{
    \tm \refinedby \tm'
  }{
    \lam{\var}{\tm} \refinedby \lamp{\lab}{\var}{\tm'}
  }
  \indrule{r-app}{
    \tm \refinedby \tm'
    \HS
    \tmtwo \refinedby \tmtwo_i \text{ for all $i=1..n$}
  }{
    \tm\,\tmtwo \refinedby \tm'[\tmtwo_1,\hdots,\tmtwo_n]
  }
\]
We write $\tm' \refines \tm$ if $\tm \refinedby \tm'$.
Refinement is also generalized to contexts, by declaring that $\conbase \refines \conbase$.
\end{definition}

Note that a term can have more than one refinement.
For example, the following terms refine $(\lam{\var}{\var\var})\vartwo$:
\[
  \begin{array}{cc}
  (\lamp{1}{\var}{\var^{[\,] \tolab{2} \alpha^3}[\,]})[\vartwo^{[\,] \tolab{2} \alpha^3}]
  &
  (\lamp{1}{\var}{\var^{[\alpha^2,\beta^3] \tolab{4} \gamma^5}[\var^{\alpha^2},\var^{\beta^3}]})[\vartwo^{[\alpha^2,\beta^3] \tolab{4} \gamma^5},\vartwo^{\alpha^2},\vartwo^{\beta^3}]
  \\
  (\lamp{1}{\var}{\var^{[\alpha^2] \tolab{3} \beta^4}[\var^{\alpha^2}]})[\vartwo^{[\alpha^2] \tolab{3} \beta^4},\vartwo^{\alpha^2}]
  &
  (\lamp{1}{\var}{\var^{[\alpha^2] \tolab{3} \beta^4}[\var^{\alpha^2}]})[\vartwo^{\alpha^2}, \vartwo^{[\alpha^2] \tolab{3} \beta^4}]
  \end{array}
\]

Also, a term may have no refinements---for example,
$\Omega = (\lam{\var}{\var \var}) (\lam{\var}{\var \var})$.
We leave this as an exercise to the reader.

The fact that a term doesn't have a refinement doesn't imply that sub-terms or
terms that contain it as a sub-term don't have refinements.
For example $\lam{\var}{\var \var}$ is refinable by
$\lamp{\lab}{\var}{\var^{[] \to \alpha}} []$,
and $(\lam{\var}{a}) \Omega$
is refinable by $(\lamp{\lab}{\var}{a^{\alpha}}) []$.




\section{Simulation}
Having defined the correspondance between terms we
have to somehow show how that correspondance is compatible with the notions
of reduction in both calculi.

For this, we puth forth two simulation results, one that deals with simulating steps
of the $\lambda$-calculus in the $\lambdadist$-calculus, and another that does the
reverse.

\begin{proposition}[Simulation]
\lprop{simulation}
Let $\tm,\tmtwo \in \terms$ be lambda-terms and $\tm' \in \termsdist$ be a distributive term such that:
\[
  \tm' \refines \tm \tobeta \tmtwo
\]
then there is a distributive term $\tmtwo' \in \termsdist$ such that:
\[
  \tm' \rtodist \tmtwo' \refines \tmtwo
\]
Diagramatically:
\[
\xymatrix@C=1cm{
 \tm \ar@{}[d]|*=0[@]{\rtimes} \ar[r]^{\beta} & \tmtwo \ar@{}[d]|*=0[@]{\rtimes} \\
 \tm' \ar@{.>>}[r]^{\dist} & \tmtwo' \\
% \tm'    \ar@{.>>}[d]^{\dist} & \refines & \tm \ar[d]^{\beta} \\
% \tmtwo'                      & \refines & \tmtwo             \\
}
\]
\end{proposition}
\begin{proof}
\SeeAppendixRef{simulation_proof} By case analysis.
The proof is constructive, and the resulting derivation $\tm' \rtodist \tmtwo'$ is
a multistep (it is a complete development of a set of coinitial steps).
\end{proof}

\begin{example}
The following are simulations of the step $\var\,((\lam{\var}{\var})\vartwo) \tobeta \var\,\vartwo$
with $\todist$-steps:
\[
  {\small
    \xymatrix@R=.25cm@C=.5cm{
      \var\,((\lam{\var}{\var})\vartwo)
      \ar[r]^-{\beta}
      \ar@{}[d]|*=0[@]{\rtimes}
      &
      \var\,\vartwo
      \ar@{}[d]|*=0[@]{\rtimes}
    \\
      \var^{[] \tolab{1} \alpha^2}\,[]
      \ar@{=}[r]
      &
      \var^{[] \tolab{1} \alpha^2}\,[]
    }
    \HS
    \xymatrix@R=.25cm@C=.5cm{
      \var\,((\lam{\var}{\var})\vartwo)
      \ar[r]^-{\beta}
      \ar@{}[d]|*=0[@]{\rtimes}
      &
      \var\,\vartwo
      \ar@{}[d]|*=0[@]{\rtimes}
    \\
      \var^{[\alpha^1] \tolab{2} \beta^3}\,[(\lamp{4}{\var}{\var^{\alpha^1}})[\vartwo^{\alpha^1}]]
      \ar[r]^-{\dist}
      &
      \var^{[\alpha^1] \tolab{2} \beta^3}\,[\vartwo^{\alpha^1}]
    }
  }
\]
\[
  {\small
    \xymatrix@R=.25cm@C=.5cm{
      \var\,((\lam{\var}{\var})\vartwo)
      \ar[r]^-{\beta}
      \ar@{}[d]|*=0[@]{\rtimes}
      &
      \var\,\vartwo
      \ar@{}[d]|*=0[@]{\rtimes}
    \\
      \var^{[\alpha^1,\beta^2] \tolab{3} \gamma^4}\,[
        (\lamp{5}{\var}{\var^{\alpha^1}})[\vartwo^{\alpha^1}],
        (\lamp{6}{\var}{\var^{\beta^2}})[\vartwo^{\beta^2}]
      ]
      \ar@{->>}[r]^-{\dist}
      &
      \var^{[\alpha^1,\beta^2] \tolab{3} \gamma^4}\,[
        \vartwo^{\alpha^1},
        \vartwo^{\beta^2}
      ]
    }
  }
\]
\end{example}

\bigskip

Now we want to go the other way: we have $\tm' \refines \tm$ and $\tm' \tomdist \tmtwo'$.
We would like to get a term $\tmtwo$ such that $\tm \tomulti \tmtwo \refinedby \tmtwo'$.
But that is not possible in general, to see that consider the following counterexample:

\[
  {\small
    \xymatrix@R=.25cm@C=.5cm{
      (\lam{\var}{\var \var}) ((\lam{\vartwo}{\vartwo}) a)
      \ar@{->>}[r]^-{\beta}
      \ar@{}[d]|*=0[@]{\rtimes}
      &
      \tmtwo
      \ar@{}[d]|*=0[@]{\rtimes}
    \\
      (\lamp{1}{\var}{\var^{[\alpha] \to \beta} [\var^{\alpha}]})
          [(\lamp{2}{\vartwo}{\vartwo^{[\alpha] \to \beta}}) a^{[\alpha] \to \beta},
           (\lam{3}{\vartwo}{\vartwo^\alpha}) a^\alpha]
      \ar@{->}[r]^-{\dist}
      &
      (\lamp{1}{\var}{\var^{[\alpha] \to \beta} [\var^{\alpha}]})
          [a^{[\alpha] \to \beta},
           (\lamp{3}{\vartwo}{\vartwo^\alpha}) a^\alpha]
    }
  }
\]

Note that if $\tmtwo$ is to be refined by
  \[(\lamp{1}{\var}{\var^{[\alpha] \to \beta} [\var^{\alpha}]})
          [a^{[\alpha] \to \beta},
           (\lamp{3}{\vartwo}{\vartwo^\alpha}) a^\alpha]\]
Then it must be an application, $\tmtwo = \tmtwo_1 \tmtwo_2$, such that $\tmtwo_2$
is refined by both $a^{[\alpha] \to \beta}$ and $(\lam{\vartwo}{\vartwo^\alpha}) a^\alpha$---
but that is impossible, because the first says $\tmtwo_2$ should be a variable, and the second
says it should be an application.
So what we need to reduce the term in the lower right part of the diagram until
it refines some term $\tmtwo$.

The exact statement for what we want is the proposition that follows.

\begin{proposition}[Reverse simulation]
\lprop{reverse_simulation}
Let $\tm',\tmtwo' \in \termsdist$ be distributive-terms and let $\tm \in \terms$ be a lambda-term such that:
\[
  \tm \refinedby \tm' \todist \tmtwo'
\]
then there is a distributive term $\tmtwo'' \in \termsdist$ and a lambda-term $\tmtwo \in \terms$ such that:
\[
  \tm \tobeta \tmtwo \refinedby \tmtwo''
\]
and the step $\tm' \todist \tmtwo'$ is contained in the multistep $\tm' \tomdist \tmtwo''$.
Diagramatically:
\[
\xymatrix@C=.25cm{
\tm' \ar[d]^{\dist} \ar@{.>>}@/_1cm/[dd]^{\dist} & \refines  & \tm    \ar@{.>}[dd]^{\beta} \\
\tmtwo' \ar@{.>>}[d]^{\dist}                     &            &                             \\
\tmtwo''                                         & \refines  & \tmtwo                      \\
}
\]
\end{proposition}
\begin{proof}
\SeeAppendixRef{reverse_simulation_proof} By induction on $\tm'$. The proof is constructive.
\end{proof}

\begin{example}
In the example above,
we needed to apply one more step in order to be able to close the diagram:

{\small
\[
    \xymatrix@C=.3cm{
      (\lamp{1}{\var}{\var^{[\alpha] \to \beta} [\var^{\alpha}]})
          [(\lamp{2}{\vartwo}{\vartwo^{[\alpha] \to \beta}}) a^{[\alpha] \to \beta},
           (\lamp{3}{\vartwo}{\vartwo^\alpha}) a^\alpha]
      \ar@{->}[d]^-{\dist}
      & \refines
      &
      (\lam{\var}{\var \var}) ((\lam{\vartwo}{\vartwo}) a)
      \ar@{->}[dd]^-{\beta}
      \\
      (\lamp{1}{\var}{\var^{[\alpha] \to \beta} [\var^{\alpha}]})
          [a^{[\alpha] \to \beta},
           (\lamp{3}{\vartwo}{\vartwo^\alpha}) a^\alpha]
      \ar@{->}[d]^-{\dist}
      &
      &
    \\
      (\lam{1}{\var}{\var^{[\alpha] \to \beta} [\var^{\alpha}]})
          [a^{[\alpha] \to \beta}, a^\alpha]
      &
      \refines
      &
      (\lam{\var}{\var \var}) a
    }
\]
}
\end{example}


\section{Head normal forms}
When we defined what a refinement is, we saw that some terms had refinements and other
did not, but the reason behind that was not clear.

In this section we will see that terms that can be refined are exactly the terms
that have a normal form.
This is coherent with the fact that typability in system~$\mathcal{W}$ characterizes
head normalization \cite[Corollary 5.5]{bucciarelli2017non}.

\begin{definition}[Head normal forms]
A term $\tm \in \terms$ of the lambda-calculus is a \defn{head normal form}
if it is of the form:
%there exist an integer $n \geq 0$, a sequence of $n$ variables $\var_1,\hdots,\var_n$,
%a variable $\vartwo$,
%an integer $m \geq 0$, and a sequence of $m$ terms $\tm_1,\hdots,\tm_m \in \terms$ such that:
\[
  \tm = \lam{\var_1}{\hdots\lam{\var_n}{\vartwo\,\tm_1\,\hdots\,\tm_m}}
\]
Note that $\vartwo$ might be or not be among the $\var_1,\hdots,\var_n$.
Similarly, a term $\tm \in \termsdist$ of the distributive lambda-calculus is a \defn{head normal form}
if it is of the form:
%there exist an integer $n \geq 0$,
%labels $\lab_1,\hdots,\lab_n$,
%variables $\var_1,\hdots,\var_n$,
%a variable $\vartwo$, a type $\typ$,
%an integer $m \geq 0$, and a sequence of $m$ lists of terms $\ls{\tm}_1,\hdots,\ls{\tm}_m \in \ls{\termsdist}$
%such that:
\[
  \tm = \lamp{\lab_1}{\var_1}{\hdots\lamp{\lab_n}{\var_n}{\vartwo^{\typ}\,\ls{\tm}_1\,\hdots\,\ls{\tm}_m}}
\]
\end{definition}

We say that a term \emph{has} a head normal form if it can be reduced to a head normal form.
And the claim that we are putting forward is that a term can be refined if and only if it has
a head normal form.

Note that for terms that \emph{are} head normal forms this is easy. The idea is that
if \[\lam{\var_1}{\hdots\lam{\var_n}{\vartwo\,\tm_1\,\hdots\,\tm_m}} \in \terms\]
then it is refined by
\[\lamp{\lab_1}{\var_1}{\hdots\lamp{\lab_n}{\var_n}{\vartwo^{[] \tolab{1} \hdots \tolab{n} \alpha^1}\,[]\,\hdots\,[]}} \in \termsdist\]
if $\var_i \neq \vartwo$ for every $i$ (otherwise the type of $\vartwo$ is slightly different,
but the term has the same shape).


\begin{lemma}
\llem{hnf_has_refinement}
Let $\tm \in \terms$ such that it is in head normal form.
 Then there exists $\tm' \in \termsdist$ such that $\tm \refinedby \tm'$.
\end{lemma}
\begin{proof}
Suppose $\tm \in \terms$ is in head normal form.
Then $\tm = \lam{\var_1}{\hdots\lam{\var_n}{\vartwo\,\tm_1\,\hdots\,\tm_m}}$.

We will prove that there exists a correct $\tctx \vdash \tm' : \typtwo$ such that $\tm' \refines \tm$.
To make the proof easier, we are going to prove something stronger:
  (a) that there exists such $\tm'$,
  (b) that for every application in $\tm'$ the argument list is empty, and
  (c) that if $\var_i = \vartwo$ for any $i$, $\tctx$ will be empty,
      otherwise it will be a singleton of the form $\set{\vartwo : [\typ]}$.

We proceed by induction on the pair $(n, m)$, that is, by induction on
  $\mathbb{N}^2$ with lexicographic order.
\end{proof}

\begin{proposition}[Refinability characterizes head normalization]
\lprop{has_hnf_has_refinement}
The following are equivalent:
\begin{enumerate}
  \item There exists $\tm' \in \termsdist$ such that $\tm' \refines \tm$.
  \item There exists $\tm' \in \termsdist$ such that $\tm' \refines \tm$
    and $\tm' \tomdist
         \lamp{\lab_1}{\var_1}{\hdots \lamp{\lab_n}{\var_n}{\vartwo^\typ}{[] \hdots []}}$.
  \item There exists a head normal form $\tmtwo$ such that $\tm \tomulti_\beta \tmtwo$, \ie
    $\tm$ has a head normal form.
\end{enumerate}
\end{proposition}
\begin{proof}
\SeeAppendixRef{has_hnf_has_refinement_proof}
$(1 \implies 3)$ relies on Simulation (\rprop{simulation}).
$(2 \implies 1)$ is obvious.
$(3 \implies 2)$ uses that head normal forms are refinable, plus a subject expansion lemma.
Subject expansion requires formulating a strongeer property than correctness,
invariant by $\todist$-expansion (\ie by $\todist^{-1}$).
\end{proof}


Note that, in general, there is no need to refine a head normal form using empty lists for
all parameters.
A head normal form has a corresponding B\"om tree, from which one
can obtain an \defn{approximant},
$\lam{\var_1}{\hdots\lam{\var_n}{\Omega\,\Omega\,\hdots\,\Omega}}$
\cite{Barendregt:1984,dezani-ciancaglini-bohm-approximant}.
A finite approximant is essentially any finite prefix of the B\"om tree, given by the grammar
$A ::= \Omega |\lam{\var_1}{\hdots\lam{\var_n}{A_1\,A_2\,\hdots\,A_m}}$.
$(2 \implies 3)$ from \rprop{has_hnf_has_refinement} may be generalized to arbitrary
finite approximations. We did not delve into this theory.





\section{Simulation residuals}

Simulation~(\rprop{simulation})
ensures that every step $\tm \tobeta \tmtwo$ can be simulated
in $\lambdadist$ starting from a term $\tm' \refines \tm$.
Actually, a finer relationship can be established between the
derivation spaces $\ulbDerivLam{\tm}$ and $\ulbDerivDist{\tm'}$.
For this, we introduce the idea of \defn{simulation residual}.

The input data of \rprop{simulation} consists of the term $\tm$,
the step $\tm \tobeta \tmtwo$, and the refinement $\tm' \refines \tm$.
Similarly as for the usual notion of residual,
we define notation to denote the output data of \rprop{simulation},
namely the multistep $\tm' \tomdist \tmtwo'$, and the refinement $\tmtwo' \refines \tmtwo$.

\begin{definition}[Simulation residuals]
\ldef{simulation_residuals_def}
Let $\tm' \refines \tm$ and let $\redex : \tm \tobeta \tmtwo$ be a step.
The constructive proof of Simulation~(\rprop{simulation})
associates the $\tobeta$-step $\redex$ to a possibly empty set of $\todist$-steps
$\{\redex_1,\hdots,\redex_n\}$ all of which start from $\tm'$.
We write $\redex/\tm' \eqdef \set{\redex_1,\hdots,\redex_n}$,
and we call $\redex_1,\hdots,\redex_n$ the \defn{simulation residuals of $\redex$ after $\tm'$}.
All the complete developments of $\redex/\tm'$ have a common target,
which we denote by $\tm'/\redex$,
called the {\em simulation residual of $\tm'$ after $\redex$}.
\end{definition}

\begin{remark}
If $\tm' \refines \tm$ and $\redex : \tm \to_\beta \tmtwo$,
then $\tm' / \redex \in \termsdist$
  and $\src(\redex_i) = \tm'$ for all $\redex_i \in \redex / \tm'$.
\end{remark}

Recall that, by abuse of notation, $\redex/\tm'$ stands
for some complete development of the set $\redex/\tm'$.
By Simulation~(\rprop{simulation}),
the following diagram always holds given $\tm' \refines \tm \tobeta \tmtwo$:
\[
  \xymatrix@R=.5cm@C=2cm{
    \tm
      \ar[r]^{\beta}_{\redex}
      \ar@{}[d]|*=0[@]{\rtimes}
  &
    \tmtwo
      \ar@{}[d]|*=0[@]{\rtimes}
  \\
    \tm'
      \ar@{->>}[r]^{\dist}_{\redex/\tm'}
  &
    \tm'/\redex
  }
\]

\begin{example}[Examples of simulation residuals]
Let $\redex : \var\,((\lam{\var}{\var})\vartwo) \tobeta \var\,\vartwo$.
\begin{enumerate}
\setcounter{enumi}{-1}
\item If $\tm' = \var^{[] \tolab{1} \alpha^2}\,[\,]$, then $\redex/\tm' = \emptyset$.
\item If $\tm' = \var^{[\alpha^1] \tolab{2} \beta^3}\,[(\lamp{4}{\var}{\var^{\alpha^1}})[\vartwo^{\alpha^1}]]$, then $\redex/\tm' = \set{\redex'}$,
      where
      $\redex' : \tm' \todist \var^{[\alpha^1] \tolab{2} \beta^3}\,[\vartwo^{\alpha^1}]$.
\item If $\tm' = (\var^{[\alpha^1,\beta^2] \tolab{3} \gamma^4}\,[
                    (\lamp{5}{\var}{\var^{\alpha^1}})[\vartwo^{\alpha^1}],
                    (\lamp{6}{\var}{\var^{\beta^2}})[\vartwo^{\beta^2}]
                  ])$,
      then
      $\redex/\tm' = \set{\redex_1,\redex_2}$
      where
      $\redex_1 : \tm' \todist \tm'_1$
      and
      $\redex_2 : \tm' \todist \tm'_2$,
      where $
      \tm'_1 = \var^{[\alpha^1,\beta^2] \tolab{3} \gamma^4}\,[
                 \vartwo^{\alpha^1},
                 (\lamp{6}{\var}{\var^{\beta^2}})[\vartwo^{\beta^2}]
               ]
      $ and $
      \tm'_2 = \var^{[\alpha^1,\beta^2] \tolab{3} \gamma^4}\,[
                 (\lamp{5}{\var}{\var^{\alpha^1}})[\vartwo^{\alpha^1}],
                 \vartwo^{\beta^2}
               ]
      $.
      Moreover $\tm'/\redex = \tm'_3$,
      where $
        \tm'_3 = \var^{[\alpha^1,\beta^2] \tolab{3} \gamma^4}\,[
                   \vartwo^{\alpha^1},
                   \vartwo^{\beta^2}
                 ]
      $.
\end{enumerate}
\end{example}

\bigskip

Simulation residuals $\tm'/\redex$ and $\redex/\tm'$ are defined when $\redex$ is a single step.
Mimicking the standard extension of residuals for derivations,
we extend simulation residuals for derivations,
\ie $\tm'/\redseq$ and $\redseq/\tm'$ as follows.

\begin{definition}[Simulation residuals extended to derivations]
If $\redseq : \tm \rtobeta \tmtwo$ is a derivation and $\tm' \in \termsdist$ is
a correct term such that $\tm' \refines \tm$
then the \defn{simulation residual of $\tm'$ after $\redseq$} is defined
by induction on $\redseq$ as a term $\tm'/\redseq \in \termsdist$ such that $\tm'/\redseq \refines \tmtwo$:
\begin{itemize}
\item $\tm'/\emptyDerivation \eqdef \tm'$
\item $\tm'/(\redex\redseqtwo) \eqdef (\tm'/\redex)/\redseqtwo$
\end{itemize}
The \defn{simulation residual of $\redseq$ after $\tm'$} is defined
by induction on $\redseq$ as a derivation $\redseq/\tm'$ such that $\redseq/\tm' : \tm' \rtodist \tm'/\redseq$:
\begin{itemize}
\item $\emptyDerivation/\tm' \eqdef \emptyDerivation$
\item $(\redex\redseqtwo)/\tm' \eqdef (\redex/\tm')(\redseqtwo/(\tm'/\redex))$ \\
      where, in the right-hand side,
      the expression $\redex/\tm'$ is an abuse of notation,
      and it technically stands for
      a (canonical) complete development of the multistep $\redex/\tm'$.
\end{itemize}
The diagram for the inductive case is:
\[
\xymatrix@C=1cm{
 \tm \ar@{}[d]|*=0[@]{\rtimes} \ar[rr]^{\redex} && \ar@{}[d]|*=0[@]{\rtimes} \ar[rr]^{\redseqtwo} && \tmtwo \ar@{}[d]|*=0[@]{\rtimes} \\
 \tm' \ar@{.>>}[rr]^{\redex/\tm'} && \tm'/\redex \ar@{.>>}[rr]^{\sigma/(\tm'/\redex)} && \tm'/\redex\redseqtwo \ar@{=}[r] & (\tm'/\redex)/\redseqtwo \\
}
\]
\end{definition}

This extension of the definition works as expected for compositions,
as shown by the following lemma.

\begin{lemma}[Simulation residuals and composition]
\llem{simulation_residuals_composition}
Let $\redseq,\redseqtwo$ be composable derivations and let $\tm' \refines \src(\redseq)$.
Then:
\begin{enumerate}
\item $\tm'/\redseq\redseqtwo = (\tm'/\redseq)/\redseqtwo$
\item $\redseq\redseqtwo/\tm' = (\redseq/\tm')(\redseqtwo/(\tm'/\redseq))$
\end{enumerate}
\end{lemma}
\begin{proof}
Item 1. is by induction on $\redseq$.
The interesting case is when $\redseq$ is non-empty, \ie $\redseq = \redex\redseqthree$.
Then:
  \[
          \tm'/\redex\redseqthree\redseqtwo
    =     (\tm'/\redex)/\redseqthree\redseqtwo
    \eqih ((\tm'/\redex)/\redseqthree)/\redseqtwo
    =     (\tm'/\redex\redseqthree)/\redseqtwo
  \]
as required.

Item 2. is by induction on $\redseq$.
The interesting case is when $\redseq$ is non-empty, \ie $\redseq = \redex\redseqthree$.
Then:
  \[
          \redex\redseqthree\redseqtwo/\tm'
    =     (\redex/\tm')(\redseqthree\redseqtwo/(\tm'/\redex))
    \eqih (\redex/\tm')(\redseqthree/(\tm'/\redex))(\redseqtwo/((\tm'/\redex)/\redseqthree))
    =     (\redex\redseqthree/\tm')(\redseqtwo/(\tm'/\redex\redseqthree))
  \]
as required.
\end{proof}

The result can be summarized by the following diagram.

\[
\xymatrix@C=2cm{
 \tm \ar@{}[d]|*=0[@]{\rtimes} \ar[r]^{\redseq}
  & \tmtwo \ar@{}[d]|*=0[@]{\rtimes} \ar[r]^{\redseqtwo}
  & \tmthree \ar@{}[d]|*=0[@]{\rtimes} \\
 \tm' \ar@{.>>}[r]^{\redseq/\tm'}
  & \tm'/\redseq \ar@{.>>}[r]^{\sigma/(\tm'/\redseq)}
  & (\tm'/\redseq) / \redseqtwo
 \\
}
\]

\subsection*{Cube lemma}

The following result resembles the usual Cube Lemma\footnote{
  Recall that the usual Cube Lemma states
  that $(\redseq/\redseqtwo)/(\redseqthree/\redseqtwo) \permeq (\redseq/\redseqthree)/(\redseqtwo/\redseqthree)$;
  see \eg~\cite[Lemma~12.2.6]{Barendregt:1984}.
}.
It is basically a coherence result that relates different notions of residuals
and checks they work together as expected.

We first enunciate a basic cube lemma that deals with single steps,
which we later on will generalize to arbitrary derivations.

But before that let us explain what the cube lemma does.
The situation it speaks about is as shown in the following diagram.

\[
\xymatrix@C=1cm{
  & \tm
  \ar[dl]_{\redex} \ar[dr]^{\redextwo}
  &
  & \refinedby
  &
  & \tm'
  \ar@{->>}[dl]_{\redex / \tm'} \ar@{->>}[dr]^{\redextwo / \tm'}
  &
  \\
  \ar@{->>}[dr]_{\redextwo / \redex}
  &
  &
  \ar@{->>}[dl]^{\redex / \redextwo}
  &
  &
  \tm' / \redex
  \ar@{->>}[dr]
  &
  &
  \tm' / \redextwo
  \ar@{->>}[dl]^{\alpha}
  \\
  &
  &
  &
  &
  &
  t'/(R \sqcup S)
  &
  \\
}
\]

In the diagram before, $\alpha$ is a complete development of some set of steps
$\redexset_\alpha$.
The cube lemma reconciles the two possible ways of writing $\redexset_\alpha$.
See that $\redexset_\alpha$ can be seen as projecting $\redex / \redextwo$
on the term $\tm' / \redextwo$, which would suggest
$\redexset_\alpha = (\redex / \redextwo) / (\tm' / \redextwo)$.
On the other hand, if we only concentrate on the right hand side of the
diagram, $\redexset_\alpha$ may be seen as simply the residual of $\redex / \tm'$
after $\redextwo / \tm'$, \ie, $(\redex / \tm') / (\redextwo / \tm')$.
The basic cube lemma states that this two different ways of seing $\redexset_\alpha$
yield the same set of steps.

\begin{lemma}[Basic cube lemma for simulation residuals]
\llem{cube_lemma_for_simulation_residuals}
Let $\redex : \tm \to \tmtwo$ and $\redextwo : \tm \to \tmthree$ be coinitial steps,
and let $\tm' \in \termsdist$ be a correct term such that $\tm' \refines \tm$.
Then the following equality between sets of coinitial steps holds:
\[
  (\redex/\tm')/(\redextwo/\tm') = (\redex/\redextwo)/(\tm'/\redextwo)
\]
Observe that there are four notions of residual involved here:
\[
  (\redex\ /^{(1)}\ \tm')\ /^{(2)}\ (\redextwo\ /^{(1)}\ \tm') = (\redex\ /^{(3)}\ \redextwo)\ /^{(1)}\ (\tm'\ /^{(4)}\ \redextwo)
\]
\begin{enumerate}
\item Set of simulation residuals of a $\beta$-step relative to a correct term.
\item Set of residuals of a $\dist$-step after a $\dist$-step.
\item Set of residuals of a $\beta$-step after a $\beta$-step.
\item Simulation residual of a correct term after a $\beta$-step.
\end{enumerate}
\end{lemma}
\begin{proof}
If $\redex = \redextwo$ then it is easy to see that the proposition holds, so we can assume that $\redex \neq \redextwo$.
  Also, note that it is enough to see that $\names((\redex/\tm')/(\redextwo/\tm')) = \names(((\redex/\redextwo)/(\tm'/\redextwo))$, as we will do that in some cases.
We proceed by induction on $\tm$.
\SeeAppendixRef{cube_lemma_for_simulation_residuals_proof}
\end{proof}

The following result relates complete developments from both worlds. It will be
useful to prove the generalized version of the cube lemma, and other intermediate results.

\begin{lemma}[Simulation residual of a development]
\llem{simulation_residual_of_a_development}
Let $\redexset$ be a set of coinitial steps,
let $\redseq$ be a complete development of $\redexset$,
and let $\tm' \in \termsdist$ be a correct term such that $\tm' \refines \src(\redseq)$.
Then $\redseq/\tm'$ is a complete development of $\redexset/\tm'$.
\end{lemma}
\begin{proof}
By induction on depth of $\redexset$, \ie the maximum length of any development of $\redexset$.
\begin{enumerate}
  \item \Case{$n = 0$} Necessarily, $\redseq = \emptyDerivation$ and $\redexset = \emptyset$.
    Then $\redseq / \tm' = \emptyDerivation$, which is a (complete) development of an empty set of steps,
    like $\redexset / \tm'$.
  \item \Case{Inductive case} In this case, the longest development of $\redexset$ is non-trivial, so
    $\redexset$ must be non-empty, let it be $\set{\redex_1, \hdots, \redex_{n+1}}$
    and let $\redseq$ be a complete development of it.
    Without loss of generality, we may assume that $\redex_{n+1}$ is the first step executed by $\redseq$,
    followed by a complete development of
    $\redexset' = \set{\redex_1 / \redex_{n+1}, \hdots, \redex_n / \redex_{n+1}}$, which we will call $\redseq'$.
    Note that the longest development of $\redexset'$ (let's call it $\redseqtwo$) is strictly shorter
    than the longest development of $\redexset$ because $\redex_{n+1} \redseqtwo$
    is a complete development of $\redexset$.

    By \ih, $\redseq' / (\tm' / \redex_{n+1})$ is a complete development of
    $\bigcup_{i=1}^n \frac{\redex_i / \redex_{n+1}}{\tm' / \redex_{n+1}}$,
    which by \rlem{cube_lemma_for_simulation_residuals} is equal to
    $\bigcup_{i=1}^n \frac{\redex_i / \tm'}{\redex_{n+1} / \tm'}$.
    Then, $(\redex_{n+1} / \tm') (\redseq' / (\tm' / \redex_{n+1}))$ is a complete development of
    $\redexset / \tm'$ (because by definition it is $\set{\redex_1 / \tm', \hdots, \redex_{n+1} / \tm'}$).

    But note that
    $\redseq / \tm' =
        \frac{\redex_{n+1}\redseq'}{\tm'} =
        \frac{\redex_{n+1}}{\tm'} \frac{\redseq'}{\tm' / \redex_{n+1}}$ so we are done.
\end{enumerate}
\end{proof}
\bigskip

Having shown the previous lemma, we are able to
prove that the operation of projecting derivations from the pure
$\lambda$-calculus to the $\lambdadist$-calculus is compatible with
permutation equivalence.

\begin{proposition}[Compatibility] % of simulation residuals and permutation equivalence]
\lprop{compatibility_of_simulation_residuals_and_permutation_equivalence}
Let $\redseq \permeq \redseqtwo$ be permutation equivalent derivations in the $\lambda$-calculus,
and let $\tm' \in \termsdist$ be a correct term such that $\tm' \refines \src(\redseq)$.
Then:
\begin{enumerate}
\item $\tm'/\redseq = \tm'/\redseqtwo$
\item $\redseq/\tm' \permeq \redseqtwo/\tm'$
\end{enumerate}
\end{proposition}
\begin{proof}
\SeeAppendixRef{compatibility_of_simulation_residuals_and_permutation_equivalence_proof}
We proceed by induction on the proof that $\redseq \permeq \redseqtwo$.
\end{proof}

\begin{corollary}[Simulation residuals and prefixes]
\lcoro{simulation_residuals_and_prefixes}
If $\redseq \permle \redseqtwo$
then $\redseq/\tm' \permle \redseqtwo/\tm'$.
\end{corollary}
\begin{proof}
Since $\redseq \permle \redseqtwo$ we have that $\redseq\redseqthree \permeq \redseqtwo$
for some derivation $\redseqthree$.
Then by item 2 of \rprop{compatibility_of_simulation_residuals_and_permutation_equivalence},
$\redseq\redseqthree/\tm' \permeq \redseqtwo/\tm'$,
so $(\redseq/\tm')(\redseqthree/(\tm'/\redseq)) \permeq \redseqtwo/\tm'$.
This implies that $\redseq/\tm' \permle \redseqtwo/\tm'$ as required.
\end{proof}

The cube lemma may be generalized for the case in which $\redex$ and $\redextwo$ are arbitrary derivations:

\begin{lemma}[Generalized cube lemma for simulation residuals]
\llem{generalized_cube_lemma}
Let $\redseq : \tm \rtobeta \tmtwo$ and $\redseqtwo : \tm \rtobeta \tmthree$ be coinitial derivations,
and let $\tm' \in \termsdist$ be a correct term such that $\tm' \refines \tm$.
Then the following equality between derivations holds:
\[
  (\redseq/\tm')/(\redseqtwo/\tm') \permeq (\redseq/\redseqtwo)/(\tm'/\redseqtwo)
\]
\end{lemma}
\begin{proof}
By induction on $\redseq$ using \rlem{cube_lemma_for_simulation_residuals}.
%\begin{pabloenv}
%The interesting case is when $\redseq$ is non-empty, \ie $\redseq = \redex\redseqthree$. 
%\end{pabloenv}
We will prove this result in two steps. The first step will be to prove it in the case where $\redseq$ is just one step, \ie, \[(\redex / \tm') / (\redseqtwo / \tm') \permeq (\redex / \redseqtwo) / (\tm' / \redseqtwo). \tag{2} \label{eq:cubelemma_partial_result}\]
We proceed by induction on $\redseqtwo$.
\begin{enumerate}
  \item {\bf $\redseqtwo = \emptyDerivation$.}
    In this case $\redex / \emptyDerivation = \set{\redex}$, $\tm' / \emptyDerivation = \tm'$, and $\emptyDerivation / \tm' = \emptyDerivation$,
    so we end up with the equality we wanted, $\redex / \tm' = \redex / \tm'$.
  \item {\bf $\redseqtwo = \redextwo \redseqtwo'$.}
    First, $\redex / (\redextwo \redseqtwo') = (\redex / \redextwo) / \redseqtwo'
            = \set{\redexthree / \redseqtwo' : \redexthree \in \redex / \redextwo}$.
    Also, $\tm' / \redseqtwo = (\tm' / \redextwo) / \redseqtwo'$.
    Then, the right hand side of the equation is
    $((\redex / \redextwo) / \redseqtwo') / ((\tm' / \redextwo) / \redseqtwo')$,
    which in reality is the set
    $\set{(\redexthree / \redseqtwo') / ((\tm' / \redextwo) / \redseqtwo') : \redexthree \in \redex / \redextwo}$,
    which by inductive hypothesis equals (modulo permutation equivalence) the set
    $\set{(\redexthree / (\tm' / \redextwo)) / (\redseqtwo' / (\tm' / \redextwo)) : \redexthree \in \redex / \redextwo}$,
    \ie the set, $((\redex / \redextwo) / (\tm' / \redextwo)) / (\redseqtwo' / (\tm' / \redextwo))$.

    In turn, this set, by \rlem{cube_lemma_for_simulation_residuals}, equals the set
    $((\redex / \tm') / (\redextwo / \tm')) / (\redseqtwo' / (\tm' / \redextwo))$.

    On the other hand, the left side of the equation is $(\redex / \tm') / ((\redextwo \redseqtwo') / \tm')$,
    which by definition of simulation residual equals
    $(\redex / \tm') / ((\redextwo / \tm') (\redseqtwo' / (\tm' / \redextwo)))$.
    That last derivation equals the canonical development of the
    right hand side of the equation because of the general
    rule $\alpha / (\beta \gamma) \permeq (\alpha / \beta) / \gamma$.
\end{enumerate}

Now we proceed to prove the full proposition, and we will proceed by inducion on $\redseq$.
\begin{enumerate}
  \item {\bf $\redseq = \emptyDerivation$.} In this case both sides of the equation are the empty derivation.
  \item {\bf $\redseq = \redex \redseq'$.} This case can be proven with a series of equalities,
    in which we will abuse notation when necessary, writing $\redex / \redseqtwo$ to mean a complete canonical development of the actual residual set.
    \[
      \begin{array}{rcll}
      ((\redex \redseq')/\redseqtwo)/(\tm'/\redseqtwo)
        & = & \frac{(\redex \redseq')/\redseqtwo}{\tm'/\redseqtwo} \\
        & = & \frac{(\redex / \redseqtwo) (\redseq' / (\redseqtwo / \redex))}{\tm'/\redseqtwo} & \text{ by definition of residual} \\
        & = & \frac{\redex / \redseqtwo}{\tm' / \redseqtwo} \frac{\redseq' / (\redseqtwo / \redex)}{(\tm' / \redseqtwo) / (\redex / \redseqtwo)} & \text{ by definition of simulation residual} \\
      & \permeq & \frac{\redex / \redseqtwo}{\tm' / \redseqtwo} \frac{\redseq' / (\redseqtwo / \redex)}{(\tm' / \redex) / (\redseqtwo / \redex)} & \text{ by the previous partial result } (\ref{eq:cubelemma_partial_result}) \\
        & \permeq & \frac{\redex / \redseqtwo}{\tm' / \redseqtwo} \frac{\redseq' / (\tm' / \redex)}{(\redseqtwo / \redex) / (\tm' / \redex)} & \text{ by inductive hypothesis} \\
        & \permeq & \frac{\redex / \tm'}{\redseqtwo / \tm'} \frac{\redseq' / (\tm' / \redex)}{(\redseqtwo / \redex) / (\tm' / \redex)} & \text{ by the previous partial result } (\ref{eq:cubelemma_partial_result}) \\
        & = & \frac{(\redex / \tm') (\redseq' / (\tm' / \redex))}{\redseqtwo / \tm'} & \text{ by definition of residual}\\
        & = & \frac{(\redex \redseq') / \tm'}{\redseqtwo / \tm'} & \text{ by definition of simulation residual}\\
      \end{array}
    \]
\end{enumerate}
\end{proof}


\begin{corollary}[Algebraic Simulation]
\lcoro{algebraic_simulation}
\lcoro{simulation_residuals_and_prefixes}
Let $\tm' \refines \tm$.
Then the mapping $\ulbDerivLam{\tm} \to \ulbDerivDist{\tm'}$
given by $\cls{\redseq} \mapsto \cls{\redseq/\tm'}$ is a morphism of upper
semilattices.
\end{corollary}
\begin{proof}
We need to check that the morphism is monotonic and that it preserves joins.
First, if $\redseq \permle \redseqtwo$ 
then $\redseq\redseqthree \permeq \redseqtwo$ for some $\redseqthree$.
So $\redseq/\tm \permle (\redseq/\tm)(\redseqthree/(\tm/\redseq)) = \redseq\redseqthree/\tm \permeq \redseqtwo/\tm$
by Compatibility~(\rprop{compatibility_of_simulation_residuals_and_permutation_equivalence}).

Secondly,
$
  (\redseq \sqcup \redseqtwo)/\tm =
  \redseq(\redseqtwo/\redseq)/\tm =
  (\redseq/\tm)((\redseqtwo/\redseq)/(\tm/\redseq)) \permeq
  (\redseq/\tm)((\redseqtwo/\tm)/(\redseq/\tm)) =
  (\redseq/\tm) \sqcup (\redseqtwo/\tm)
$.
\end{proof}

\begin{example}
\lexample{algebraic_simulation_example}
Let $I = \lam{\var}{\var}$
and $\Delta = (\lamp{5}{\var}{\var^{\alpha^2}})[\varthree^{\alpha^2}]$
and let us write $\hat{\vartwo}$ for $\vartwo^{[\alpha^2] \tolab{3} [\,] \tolab{4} \beta^5}$.
The refinement $\tm' := (\lamp{1}{\var}{\hat{\vartwo}[\var^{\alpha^2}][\,]})[\Delta] \refines (\lam{\var}{\vartwo\var\var})(I\varthree)$
induces a morphism between the upper semilattices represented by the following reduction graphs:
\[
  \xymatrix@R=.3cm@C=.3cm{
  &
    (\lam{\var}{\vartwo\var\var})(I\varthree)
    \ar@/_.25cm/[dl]_-{\redex_1}
    \ar@/^.25cm/[dr]^-{\redextwo}
  &
  &
  \\
    \vartwo(I z)(I z)
    \ar[d]_-{\redextwo_{11}}
    \ar[r]^-{\redextwo_{21}}
  &
    \vartwo(I\varthree)\varthree
    \ar[d]^{\redextwo_{12}}
  &
    (\lam{\var}{\vartwo\var\var})\varthree
    \ar@/^.25cm/[dl]^-{\redex_2}
  \\
    \vartwo \varthree (I\varthree)
    \ar[r]_-{\redextwo_{22}}
  &
    \vartwo \varthree \varthree
  }
  \xymatrix@R=.3cm@C=.3cm{
  &
    (\lamp{1}{\var}{\hat{\vartwo}[\var^{\alpha^2}][\,]})[\Delta]
    \ar@/_.25cm/[dl]_-{\redex'_1}
    \ar@/^.25cm/[dr]^-{\redextwo'}
  \\
    \hat{\vartwo}[\Delta][\,]
    \ar@/_.25cm/[dr]_-{\redextwo'_1}
  &
  &
    \hspace{-1cm}(\lamp{1}{\var}{\hat{\vartwo}[\var^{\alpha^2}][\,]})[\varthree^{\alpha^2}]
    \ar@/^.25cm/[dl]^-{\redex'_2}
  \\
  &
    \hat{\vartwo}[\varthree^{\alpha^2}][\,]
  }
\]
For example
$(\redex_1 \sqcup \redextwo)/\tm' = (\redex_1\redextwo_{11}\redextwo_{22})/\tm' = \redex'_1\redextwo'_1 = \redex'_1 \sqcup \redextwo' = \redex_1/\tm' \sqcup \redextwo/\tm'$.
Note that the step $\redextwo_{22}$ is erased by the simulation: $\redextwo_{22}/(\hat{\vartwo}[\varthree^{\alpha^2}][\,]) = \emptyset$.
Intuitively, $\redextwo_{22}$ is ``garbage'' with respect to the refinement $\hat{\vartwo}[\varthree^{\alpha^2}][\,]$,
because it lies inside an {\em untyped} argument.
\end{example}

\begin{example}
We can do a larger example. Consider the term
$(\lam{\var}{\var \var}) ((\lam{\vartwo}{\vartwo a}) (\lam{\varthree}{\varthree}))$,
which is refined by
$(\lamp{1}{\var}{\var []}) [(\lamp{2}{\vartwo}{\vartwo [a]}) [\lamp{3}{\varthree}{\varthree}]]$, among others
\footnote{
The fully labeled term is
$
  (\lamp{1}{\var}{\var^{[] \tolab{5} \alpha^4} []})
    [(\lamp{2}{
      \vartwo^{([] \tolab{5} \alpha^4) \tolab{3} ([] \tolab{5} \alpha^4)}
      }{\vartwo [a^{[] \tolab{5} \alpha^4}]})
        [\lamp{3}{\varthree}{\varthree^{[] \tolab{5} \alpha^4}}]
    ].
$
}.
Let's take a look at the derivation space of the distributive term. We use the labels of the
steps to name the steps ($\redex_n, \redex_n', \redex_n'', \hdots$ are steps that have the name $n$).

\[
  \xymatrix{
  (\lamp{1}{\var}{\var []}) [(\lamp{2}{\vartwo}{\vartwo [a]}) [\lamp{3}{\varthree}{\varthree}]]
  \ar[r]^-{\redex_1}
  \ar[d]^-{\redex_2}
  &
  (\lamp{2}{\vartwo}{\vartwo [a]}) [\lamp{3}{\varthree}{\varthree}]] []
  \ar[d]^-{\redex_2'}
  \\
  (\lamp{1}{\var}{\var []}) [(\lamp{3}{\varthree}{\varthree}) [a]]
  \ar[r]^-{\redex_1'}
  \ar[d]^-{\redex_3}
  &
  (\lamp{3}{\varthree}{\varthree}) [a] []
  \ar[d]^-{\redex_3'}
  \\
  (\lamp{1}{\var}{\var []}) [a]
  \ar[r]^-{\redex_1''}
  &
  a []
  }
\]

Next, the derivation space of the pure lambda term.
We name the steps such that a derivation below with a step $\redextwo_i^n$ maps to
the derivation with $\redex_i^n$ above. The steps with no name are mapped to the empty
derivation above.
{\tiny
\[
\xymatrix@R=.3cm@C=.1cm{
  &
  &
  & (\lam{\var}{\var \var})((\lam{\vartwo}{\vartwo a}) (\lam{\varthree}{\varthree}))
  \ar[dl]^-{\redextwo_1}
  \ar[drr]^-{\redextwo_2}
  &
  &
  \\
  &
  & ((\lam{\vartwo}{\vartwo a}) (\lam{\varthree}{\varthree})) ((\lam{\vartwo}{\vartwo a}) (\lam{\varthree}{\varthree}))
  \ar[dl]^-{\redextwo_2'}
  \ar[dr]
  &
  &
  & (\lam{\var}{\var \var}) ((\lam{\varthree}{\varthree}) a)
  \ar[ddd]^-{\redextwo_3}
  \ar@/^.5cm/[ddlll]^{\redextwo_1'}
  \\
  & ((\lam{\varthree}{\varthree}) a) ((\lam{\vartwo}{\vartwo a}) (\lam{\varthree}{\varthree}))
  \ar[dl]^-{\redextwo_3'}
  \ar[dr]
  &
  & ((\lam{\vartwo}{\vartwo a}) (\lam{\varthree}{\varthree})) ((\lam{\varthree}{\varthree}) a)
  \ar[dl]_-{\redextwo_2'}
  \ar[dr]
  &
  &
  \\
  a ((\lam{\vartwo}{\vartwo a}) (\lam{\varthree}{\varthree}))
  \ar[dr]
  &
  & ((\lam{\varthree}{\varthree}) a) ((\lam{\varthree}{\varthree}) a)
  \ar[dl]^-{\redextwo_3'}
  \ar[dr]
  &
  & ((\lam{\vartwo}{\vartwo a}) (\lam{\varthree}{\varthree})) a
  \ar[dl]^-{\redextwo_2'}
  &
  \\
  & a ((\lam{\varthree}{\varthree}) a)
  \ar@/_.3cm/[drr]
  &
  & ((\lam{\varthree}{\varthree}) a) a
  \ar[d]^-{\redextwo_3'}
  &
  & (\lam{\var}{\var \var}) a
  \ar@/^.3cm/[dll]^-{\redextwo_1''}
  \\
  &
  &
  & a a
  &
  &
  \\
}
\]
}

Note that, for example, the derivation $\redextwo_2 \redextwo_3 \redextwo_1''$ maps to
the derivation $\redex_2 \redex_3 \redex_1''$.


Notice how the unlabeled steps in the second diagram correspond to the empty derivation
in the first one.
That means something: the unlabeled
steps are not necessary if we want to arrive to the head normal form $a \Omega$,
which is ``casually'' the head normal form represented by $a \emptylset$.

We will study this phenomenon more carefully in the next chapter.
\end{example}

