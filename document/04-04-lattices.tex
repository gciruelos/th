
\begin{lemma}[Projections are decreasing]
\llem{projections_are_decreasing}
Let $\redex \in \redseq$. Then $\lengthof{\redseq} = 1 + \lengthof{\redseq/\redex}$.
\end{lemma}
\begin{proof}
Observe that $\redex \permle \redseq$ by \rlem{characterization_of_belonging}.
So $\redseq \permeq \redex(\redseq/\redex)$, which gives us that:
\[
  \begin{array}{rcll}
  \lengthof{\redseq} & = & \#\names(\redseq) & \text{ by \rcoro{length_of_derivation_is_number_of_distinct_names}} \\
                     & = & \#\names(\redex(\redseq/\redex)) & \text{ by \rcoro{permutation_equivalence_in_terms_of_names}, since $\redseq \permeq \redex(\redseq/\redex)$} \\
                     & = & \#(\names(\redex) \uplus \names(\redseq/\redex) & \text{ by \rlem{names_concatenation_disjoint_union}} \\
                     & = & 1 + \#\names(\redseq/\redex) \\
                     & = & 1 + \lengthof{\redseq/\redex} & \text{ by \rcoro{length_of_derivation_is_number_of_distinct_names}} \\
  \end{array}
\]
\end{proof}

\begin{proposition}[Meet of derivations]
\lprop{meet_of_derivations}
Let $\redseq,\redseqtwo$ be coinitial derivations in the distributive lambda-calculus.
Then there exists an infimum for $\redseq,\redseqtwo$ with respect to the prefix order $\permle$.
We write $\redseq \sqcap \redseqtwo$ for the infimum of $\set{\redseq,\redseqtwo}$ obtained by this
construction.
\end{proposition}
\begin{proof}
If $\redseq$ and $\redseqtwo$ are derivations, we say that a step $\redex$
is a \defn{common} (to $\redseq$ and $\redseqtwo$) whenever $\redex \in \redseq$ and $\redex \in \redseqtwo$.
Define $\redseq \sqcap \redseqtwo$ as follows,
by induction on the length of $\redseq$:
\[
  \redseq \sqcap \redseqtwo \eqdef
    \begin{cases}
    \emptyDerivation & \text{if there are no common steps to $\redseq$ and $\redseqtwo$} \\
    \redex((\redseq/\redex) \sqcap (\redseqtwo/\redex)) & \text{if the step $\redex$ is common to $\redseq$ and $\redseqtwo$ } \\
    \end{cases}
\]
In the second case of the definition, there might be more than one $\redex$
common to $\redseq$ and $\redseqtwo$.
We suppose that one of them is chosen deterministically but make no further assumptions.
To see that this recursive construction is well-defined,
note that the length of $\redseq/\redex$ is lesser than the length of $\redseq$
by the fact that projections are decreasing (\rlem{projections_are_decreasing}).
To conclude the construction, we show that $\redseq \sqcap \redseqtwo$ is an infimum, \ie a greatest lower bound:
\begin{enumerate}
\item {\bf Lower bound.}
  Let us show that $\redseq \sqcap \redseqtwo \permle \redseq$
  by induction on the length of $\redseq$.
  There are two subcases, depending on whether
  there is a step common to $\redseq$ and $\redseqtwo$.

  If there is no common step,
  then $\redseq \sqcap \redseqtwo = \emptyDerivation$
  trivially verifies $\redseq \sqcap \redseqtwo \permle \redseq$.

  On the other hand, if there is a common step, we have by definition that
  $\redseq \sqcap \redseqtwo = \redex((\redseq/\redex) \sqcap (\redseqtwo/\redex))$
  where $\redex$ is common to $\redseq$ and $\redseqtwo$.
  Recall that projections are decreasing (\rlem{projections_are_decreasing})
  so $\lengthof{\redseq} > \lengthof{\redseq/\redex}$.
  This allows us to apply the \ih and conclude:
  \[
  \begin{array}{rcll}
    \redseq \sqcap \redseqtwo & =       & \redex((\redseq/\redex) \sqcap (\redseqtwo/\redex)) & \text{ by definition} \\
                              & \permle & \redex(\redseq/\redex)   & \text{ since by \ih $(\redseq/\redex) \sqcap (\redseqtwo/\redex) \permle \redseq/\redex$} \\
                              & \permeq & \redseq(\redex/\redseq)  \\
                              & =       & \redseq                  & \text{ since $\redex \permle \redseq$ by \rlem{characterization_of_belonging}.} \\
  \end{array}
  \]
  Showing that $\redseq \sqcap \redseqtwo \permle \redseqtwo$ is symmetric,
  by induction on the length of $\redseqtwo$.
\item {\bf Greatest lower bound.}
  Let $\redseqthree$ be a lower bound for $\set{\redseq, \redseqtwo}$,
  \ie $\redseqthree \permle \redseq$ and $\redseqthree \permle \redseqtwo$,
  and let us show that $\redseqthree \permle \redseq \sqcap \redseqtwo$.
  We proceed by induction on the length of $\redseq$.
  There are two subcases, depending on whether
  there is a step common to $\redseq$ and $\redseqtwo$.

  If there is no common step,
  we claim that $\redseqthree$ must be empty.
  Otherwise we would have that $\redseqthree = \redexthree\redseqthree' \permle \redseq$
  so in particular $\redexthree \permle \redseq$ and $\redexthree \in \redseq$ by \rlem{characterization_of_belonging}.
  Similarly, $\redexthree \in \redseqtwo$ so
  $\redexthree$ is a step common to $\redseq$ and $\redseqtwo$,
  which is a contradiction.
  We obtain that $\redseqthree$ is empty,
  so trivially $\redseqthree = \emptyDerivation \permle \redseq \sqcap \redseqtwo$.

  On the other hand, if there is a common step, we have by definition that
  $\redseq \sqcap \redseqtwo = \redex((\redseq/\redex) \sqcap (\redseqtwo/\redex))$
  where $\redex$ is common to $\redseq$ and $\redseqtwo$.
  Moreover, since $\redseqthree \permle \redseq$ and $\redseqthree \permle \redseqtwo$,
  by projecting along $\redex$ we know that
  $\redseqthree/\redex \permle \redseq/\redex$ and $\redseqthree/\redex \permle \redseqtwo/\redex$.
  So:
  \[
    \begin{array}{rcll}
      \redseqthree & \permle & \redseqthree(\redex/\redseqthree) \\
                   & \permeq & \redex(\redseqthree/\redex) \\
                   & \permle & \redex((\redseq/\redex) \sqcap (\redseqtwo/\redex)) & \text{ since by \ih $\redseqthree/\redex \permle (\redseq/\redex) \sqcap (\redseqtwo/\redex)$} \\
                   & =       & \redseq \sqcap \redseqtwo & \text{ by definition}
    \end{array}
  \]
\end{enumerate}
\end{proof}

\begin{remark}
The infimum of $\set{\redseq,\redseqtwo}$ is unique modulo permutation equivalence,
\ie if $\redseqthree$ is an infimum for $\set{\redseq,\redseqtwo}$ then $\redseqthree \permeq \redseq \sqcap \redseqtwo$.
\end{remark}



\begin{lemma}[Properties of disjoint derivations]
\llem{properties_of_disjoint_derivations}
Let $\redseq,\redseqtwo$ be coinitial derivations.
Then the following are equivalent:
\begin{enumerate}
\item $\names(\redseq) \cap \names(\redseqtwo) = \emptyset$.
\item $\redseq \sqcap \redseqtwo = \emptyDerivation$.
\item There is no step common to $\redseq$ and $\redseqtwo$.
\end{enumerate}
In this case we say that $\redseq$ and $\redseqtwo$ are \defn{disjoint}.
\end{lemma}
\begin{proof}
The implication $(1 \implies 2)$ is immediate since if we suppose that $\redseq \sqcap \redseqtwo$ is non-empty
then the first step of $\redseq \sqcap \redseqtwo$ is a step $\redexthree$
such that $\redexthree \in \redseq$ and $\redexthree \in \redseqtwo$.
By Characterization of belonging~(\rlem{characterization_of_belonging}),
this means that $\name(\redexthree) \in \names(\redseq) \cap \names(\redseqtwo)$,
contradicting the fact that $\name(\redexthree)$ and $\names(\redseq)$ are disjoint.

The implication
$(2 \implies 3)$ is immediate by definition of $\redseq \sqcap \redseqtwo$.

Let us check that the implication $(3 \implies 1)$ holds.
By the contrapositive, suppose that $\names(\redseq)$ and $\names(\redseqtwo)$
are not disjoint, and let us show that there is a step common to $\redseq$ and $\redseqtwo$.
Since $\names(\redseq) \cap \names(\redseqtwo) \neq \emptyset$,
we know that the derivation $\redseq$ can be written as $\redseq = \redseq_1 \redex \redseq_2$
where $\name(\redex) \in \names(\redseqtwo)$.
Without loss of generality we may suppose
that $\redex$ is the first step in $\redseq$ with that property,
\ie that $\names(\redseq_1) \cap \names(\redseqtwo) = \emptyset$.
Moreover, let us write $\redseqtwo$ as $\redseqtwo = \redseqtwo_1 \redextwo \redseqtwo_2$ 
where $\name(\redex) = \name(\redextwo)$.

Observe that the name of $\redex$ does not appear anywhere along the sequence of steps $\redseq_1$,
\ie that $\name(\redex) \not\in \names(\redseq_1)$, as a consequence of the
fact that no names are ever repeated in any sequence of steps (\rlem{names_concatenation_disjoint_union}).
This implies that $\name(\redextwo) \not\in \names(\redseq_1/\redseqtwo_1)$.
Indeed:
\[
  \name(\redextwo)
  = \name(\redex)
  \not\in \names(\redseq_1)
  \supseteq \names(\redseq_1) \setminus \names(\redseqtwo_1)
  =^{(\text{\rlem{names_after_projection_along_a_step}})} \names(\redseq_1/\redseqtwo_1)
\]
This means that $\redextwo$ is not erased by the derivation $\redseq_1/\redseqtwo_1$.
More precisely, $\redextwo/(\redseq_1/\redseqtwo_1)$ is a singleton.

Symmetrically, $\redex/(\redseqtwo_1/\redseq_1)$ is a singleton.
Moreover, $\name(\redextwo/(\redseq_1/\redseqtwo_1)) = \name(\redextwo) = \name(\redex) = \name(\redex/(\redseqtwo_1/\redseq_1))$ 
so we have that $\redextwo/(\redseq_1/\redseqtwo_1) = \redex/(\redseqtwo_1/\redseq_1)$.
The situation is the following, where $\names(\redseq_1) \cap \names(\redseqtwo_1) = \emptyset$:
\[
  \xymatrix{
    &
    &
    & \ar@{->>}[ld]_{\redseq_1} \ar@{->>}[rd]^{\redseqtwo_1} &
  \\
    &
    \ar@{->>}[l]_{\redseq_2}
    &
    \ar[l]_{\redex}
    \ar@{->>}[rd]_{\redseqtwo_1/\redseq_1} & & \ar@{->>}[ld]^{\redseq_1/\redseqtwo_1} \ar[r]^{\redextwo}
    &
    \ar@{->>}[r]^{\redseqtwo_2}
    &
  \\
    &
    &
    &
    \ar[d]_{\redex/(\redseqtwo_1/\redseq_1) = \redextwo/(\redseq_1/\redseqtwo_1)}
    &
  \\
    &
    &&&
  }
\]
By Full stability~(\rlem{full_stability}) this means that there exists a step $\redexthree$
such that $\redexthree/\redseq_1 = \redex$ and $\redexthree/\redseqtwo_1 = \redextwo$.
Then $\redexthree \in \redseq_1\redex\redseq_2 = \redseq$
and also $\redexthree \in \redseqtwo_1\redextwo\redseqtwo_2 = \redseqtwo$
so $\redexthree$ is common to $\redseq$ and $\redseqtwo$,
by which we conclude.
\end{proof}


\begin{lemma}
\llem{names_of_meet_included_in_names}
Let $\redseq$ and $\redseqtwo$ be coinitial derivations.
Then $\names(\redseq \sqcap \redseqtwo) \subseteq \names(\redseq)$.
\end{lemma}
\begin{proof}
By induction on the length of $\redseq \cap \redseqtwo$:
\begin{enumerate}
\item {\bf Empty, $\redseq \sqcap \redseqtwo = \emptyDerivation$.}
      Then $\names(\redseq \cap \redseqtwo) = \emptyset \subseteq \names(\redseq)$ is immediate.
\item {\bf Non-empty, $\redseq \sqcap \redseqtwo = \redexthree(\redseq/\redexthree \sqcap \redseqtwo/\redexthree)$,
           where $\redexthree$ is a step common to $\redseq$ and $\redseqtwo$.}
      Then since $\redexthree$ is common to $\redseq$ and $\redseqtwo$,
      we have that $\name(\redexthree) \in \names(\redseq)$.
      Moreover, by \ih $\names(\redseq/\redexthree \sqcap \redseqtwo/\redexthree) \subseteq \names(\redseq/\redexthree)$.
      So:
      \[
        \begin{array}{rcll}
        \names(\redseq \cap \redseqtwo) & = & \set{\name(\redexthree)} \cup \names(\redseq/\redexthree \sqcap \redseqtwo/\redexthree) \\
                                        & \subseteq & \names(\redseq) \cup \names(\redseq/\redexthree) \\
                                        & = & \names(\redseq) \cup (\names(\redseq) \setminus \set{\name(\redexthree)} & \text{by \rlem{names_after_projection_along_a_step}} \\
                                        & = & \names(\redseq)
        \end{array}
      \]
      as required.
\end{enumerate}
\end{proof}


\begin{proposition}[Names of join and meet]
\lprop{names_of_join_and_meet}
The following hold:
\begin{enumerate}
\item $\names(\redseq \sqcup \redseqtwo) = \names(\redseq) \cup \names(\redseqtwo)$
\item $\names(\redseq \sqcap \redseqtwo) = \names(\redseq) \cap \names(\redseqtwo)$
\end{enumerate}
\end{proposition}
\begin{proof}
Item 1 is easy resorting to the definition of $\sqcup$
and \rlem{names_after_projection_along_a_step}:
\[
  \begin{array}{rcll}
  \names(\redseq \sqcup \redseqtwo) & = & \names(\redseq(\redseqtwo/\redseq)) & \text{ by definition of $\sqcup$} \\
                                    & = & \names(\redseq) \cup \names(\redseqtwo/\redseq) \\
                                    & = & \names(\redseq) \cup (\names(\redseqtwo) \setminus \names(\redseq)) & \text{ by \rlem{names_after_projection_along_a_step}} \\
                                    & = & \names(\redseq) \cup \names(\redseqtwo) \\
  \end{array}
\]
For item 2., 
the inclusion $\names(\redseq \sqcap \redseqtwo) \subseteq \names(\redseq) \cap \names(\redseqtwo)$
is an immediate consequence of \rlem{names_of_meet_included_in_names}.
For other inclusion, namely to show  $\names(\redseq) \cap \names(\redseqtwo) \subseteq \names(\redseq \sqcap \redseqtwo)$,
we first prove the following claim:\medskip

{\bf Claim.} $\names(\redseq/(\redseq \sqcap \redseqtwo)) \cap \names(\redseqtwo/(\redseq \sqcap \redseqtwo) = \emptyset$.
{\em Proof of the claim.}
By \rlem{properties_of_disjoint_derivations} it suffices to show that
$(\redseq/(\redseq \sqcap \redseqtwo)) \sqcap (\redseqtwo/(\redseq \sqcap \redseqtwo)) = \emptyDerivation$.
By contradiction, suppose that there is a step $\redexthree$ common to
the derivations
$\redseq/(\redseq \sqcap \redseqtwo)$ and $\redseqtwo/(\redseq \sqcap \redseqtwo)$.
Then the derivation $(\redseq \sqcap \redseqtwo)\redexthree$ is a lower bound for $\set{\redseq,\redseqtwo}$,
\ie
$(\redseq \sqcap \redseqtwo)\redexthree \permle \redseq$
and
$(\redseq \sqcap \redseqtwo)\redexthree \permle \redseqtwo$.
Since $\redseq \sqcap \redseqtwo$ is the greatest lower bound for $\set{\redseq,\redseqtwo}$,
we have that $(\redseq \sqcap \redseqtwo)\redexthree \permle \redseq \sqcap \redseqtwo$.
But this implies that $\redexthree \permle \emptyDerivation$, which is a contradiction.
This concludes the proof of the claim.
\medskip

Note that $\redseq \sqcap \redseqtwo \permle \redseq$,
so we have that $\redseq \permeq (\redseq \sqcap \redseqtwo)(\redseq/(\redseq \sqcap \redseqtwo))$,
and this in turn implies that
$\names(\redseq) = \names((\redseq \sqcap \redseqtwo)(\redseq/(\redseq \sqcap \redseqtwo)))$
by \rcoro{permutation_equivalence_in_terms_of_names}.
Symmetrically,
$\names(\redseqtwo) = \names((\redseq \sqcap \redseqtwo)(\redseqtwo/(\redseq \sqcap \redseqtwo)))$.
Then:
\[
  \begin{array}{rcll}
    &&  \names(\redseq) \cap \names(\redseqtwo) \\
  & = & \names((\redseq \sqcap \redseqtwo)(\redseq/(\redseq \sqcap \redseqtwo))) \cap \names((\redseq \sqcap \redseqtwo)(\redseqtwo/(\redseq \sqcap \redseqtwo))) \\
  & = & \left(\names(\redseq \sqcap \redseqtwo) \cup \names(\redseq/(\redseq \sqcap \redseqtwo))\right) \cap \left(\names(\redseq \sqcap \redseqtwo) \cup \names(\redseqtwo/(\redseq \sqcap \redseqtwo))\right) \\
     && \text{ by \rremark{names_concatenation_union}} \\
  & = &
        \names(\redseq \sqcap \redseqtwo) \cup 
        \left( \names(\redseq/(\redseq \sqcap \redseqtwo)) \cap \names(\redseqtwo/(\redseq \sqcap \redseqtwo)) \right) \\
  && \text{ since $(A \cup B) \cap (A \cup C) = A \cup (B \cap C)$ for arbitrary sets $A,B,C$} \\
  & = &
        \names(\redseq \sqcap \redseqtwo) \\
  && \text{ since $\left( \names(\redseq/(\redseq \sqcap \redseqtwo)) \cap \names(\redseqtwo/(\redseq \sqcap \redseqtwo)) \right) = \emptyset$ by the previous claim} \\
  \end{array}
\]
This concludes the proof.
\end{proof}

\begin{theorem}[Derivations modulo $\permeq$ form a distributive lattice]
Let $\tm \in \termsdist$ be a correct term.
Then derivations modulo $\permeq$ form a lattice $\derivlattice{\tm}$.
More precisely, let $X$ be the set of derivations in the distributive lambda-calculus going out from $\tm$,
modulo permutation equivalence:
\[
  X \eqdef \frac{\set{\redseq \ST \src(\redseq) = \tm}}{\permeq}
\]
Let moreover $\classof{\redseq}$ denote the equivalence class of a derivation $\redseq$
modulo $\permeq$.
Then $\derivlattice{\tm} \eqdef (X, \leq, \land, \lor)$ is a distributive lattice, where:
\[
  \begin{array}{rcl}
  \classof{\redseq} \leq \classof{\redseqtwo}  & \iffdef & \redseq \permle \redseqtwo \\
  \classof{\redseq} \land \classof{\redseqtwo} & \iffdef & \redseq \sqcap \redseqtwo \\
  \classof{\redseq} \lor \classof{\redseqtwo}  & \iffdef & \redseq \sqcup \redseqtwo \\
  \end{array}
\]
\end{theorem}
\begin{proof}
It is straightforward to check that $\leq$ is a partial order,
and that $\classof{\redseq} \land \classof{\redseqtwo}$ (resp. $\classof{\redseq} \lor \classof{\redseqtwo}$)
is the infimum (resp. supremum) of $\set{\classof{\redseq},\classof{\redseqtwo}}$.

To see that it is distributive, let us first prove the first distributive law:
$(\classof{\redseq} \land \classof{\redseqtwo}) \lor \classof{\redseqthree} =
(\classof{\redseq} \lor \classof{\redseqthree}) \land (\classof{\redseqtwo} \lor \classof{\redseqthree})$.
Let $\redseq,\redseqtwo,\redseqthree$ be arbitrary coinitial derivations.
The following equality holds trivially, since
$(A \cap B) \cup C = (A \cup C) \cap (A \cup B)$ is valid for arbitrary sets $A, B, C$:
  \[
    (\names(\redseq) \cap \names(\redseqtwo)) \cup \names(\redseqthree)
    =
    (\names(\redseq) \cup \names(\redseqthree)) \cap (\names(\redseqtwo) \cup \names(\redseqthree))
  \]
By \rprop{names_of_join_and_meet} this entails:
  \[
    \names((\redseq \sqcap \redseqtwo) \sqcup \redseqthree)
    =
    \names((\redseq \sqcup \redseqthree) \sqcap (\redseqtwo \sqcup \redseqthree))
  \]
By \rcoro{permutation_equivalence_in_terms_of_names} this in turn implies that:
  \[
    (\redseq \sqcap \redseqtwo) \sqcup \redseqthree
    \permeq
    (\redseq \sqcup \redseqthree) \sqcap (\redseqtwo \sqcup \redseqthree)
  \]
So by definition of $\land,\lor$ we obtain:
  \[
    (\classof{\redseq} \land \classof{\redseqtwo}) \lor \classof{\redseqthree}
    =
    (\classof{\redseq} \lor \classof{\redseqthree}) \land (\classof{\redseqtwo} \lor \classof{\redseqthree})
  \]
The other distributive law, namely
$(\classof{\redseq} \lor \classof{\redseqtwo}) \land \classof{\redseqthree} =
(\classof{\redseq} \land \classof{\redseqthree}) \lor (\classof{\redseqtwo} \land \classof{\redseqthree})$
is proved analogously.
\end{proof}

\begin{remark}
The function $\names$ that takes a derivation and returns a set of labels
is well-defined for permutation-equivalence classes,
as a consequence of \rcoro{permutation_equivalence_in_terms_of_names}:
\[
  \names([\redseq]) \eqdef \names(\redseq)
\]
\end{remark}

\begin{theorem}[The lattice of derivations is representable as a ring of sets]
If $\tm \in \termsdist$ is a correct term,
then $\names : \derivlattice{\tm} \to \powerset(\labelset)$ is a monomorphism of lattices,
where $\derivlattice{\tm}$ is the lattice of derivations of $\tm$
and $\powerset(\labelset)$ is the lattice whose elements are sets of labels
ordered by inclusion, with set intersection and set union as
the meet and join operators.
\end{theorem}
\begin{proof}
We are to show that $\names$ is monotonic, that it preserves meets and joins,
and finally that it is a monomorphism:
\begin{itemize}
\item {\bf Monotonic.}
  If $\classof{\redseq} \leq \classof{\redseqtwo}$
  then $\names(\redseq) \subseteq \names(\redseqtwo)$.
  This has been proved in \rprop{prefixes_as_subsets}.
\item {\bf Preserves meets.}
  $\names(\classof{\redseq} \land \classof{\redseqtwo}) =
   \names(\redseq) \cap \names(\redseqtwo)$
  by \rprop{names_of_join_and_meet}.
\item {\bf Preserves joins.}
  $\names(\classof{\redseq} \lor \classof{\redseqtwo}) =
   \names(\redseq) \cup \names(\redseqtwo)$
  by \rprop{names_of_join_and_meet}.
\item {\bf Monomorphism.}
  It suffices to show that $\names$ is injective.
  Indeed, 
  suppose that $\names(\classof{\redseq}) = \names(\classof{\redseqtwo})$.
  By \rcoro{permutation_equivalence_in_terms_of_names}
  we have that $\redseq \permeq \redseqtwo$,
  so
  $\classof{\redseq} = \classof{\redseqtwo}$.
\end{itemize}
\end{proof}

