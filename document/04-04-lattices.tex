\lsec{lattices_and_derivation_spaces}
In this section we are going to consider the derivation spaces of terms in
the $\lambdadist$-calculus as lattices and prove properties about those lattices.

In a general orthogonal rewrite system,
the \defn{space of derivations} of a term $\tm$ is the set
$\set{\redseq : \src(\redseq) = \tm}$ (usually modulo permutation equivalence),
together with the order relation $\permle$.
Recall that
$\redseq \permle \redseqtwo$ if and only if $\redseq / \redseqtwo = \emptyDerivation$.

In the preliminaries we defined a \defn{lattice} to be partially ordered set with two operations:
join and meet. Given two elements of the lattice, their join is their least upper bound,
and their join is their greatest lower bound. Notice that those elements have to be unique.

As a consequence of
and axiomatic results~\cite{thesismellies},
derivation spaces of terms of orthogonal abstract rewrite systems are
\emph{upper semilattices}. This means that they have joins, but not necessarily meets.

Joins for these systems are given essentially by their confluence property:
the join (or least upper bound)
of two reductions $\redseq$, $\redseqtwo$ is $\redseq (\redseqtwo / \redseq)$
(or equivalently $\redseqtwo (\redseq / \redseqtwo)$).
This can be rephrased in category theory by saying that the derivation space of
terms of orthogonal abstract rewrite systems have push-outs.

\begin{example} Laneve's counterexample from the introduction
showed that the meet of derivations is not well defined for the pure lambda calculus
(which is a orthogonal abstract rewrite system \cite{thesismellies}).
Let $\Omega = (\lam{\var}{\var\var})\lam{\var}{\var\var}$,
and consider the reduction space of
$(\lam{\var}{(\lam{\vartwo}{a})(\var \Omega)}) (\lam{\varthree}{b})$
where the steps $U_i$ contract $\Omega$:

\[
  \xymatrix{
    &
    (\lam{\var}{(\lam{\vartwo}{a})(\var \Omega)}) (\lam{\varthree}{b})
    \ar_{U_1}[d]
    \ar@/_1cm/_{S}[dddl]
    \ar@/^1cm/^{R}[dddr]
    &
  \\
    &
    (\lam{\var}{(\lam{\vartwo}{a})(\var \Omega)}) (\lam{\varthree}{b})
    \ar@/_1cm/[ddl] \ar@/^1cm/[ddr]
    \ar_{U_2}[d]
    &
  \\
    &
    (\lam{\var}{(\lam{\vartwo}{a})(\var \Omega)}) (\lam{\varthree}{b})
    \ar@/_1cm/[dl] \ar@/^1cm/[dr]
    \ar_{U_3}[d]
    &
  \\
    (\lam{\vartwo}{a})((\lam{\varthree}{b}) \Omega)
    \ar_{T}[d]
    &
    \vdots
    &
    (\lam{\var}{a})(\lam{\varthree}{b})
  \\
    (\lam{\vartwo}{a}) b
    &
    &
    &
  }
\]

As we pointed out in the introduction,
$ST$ and $R$ do not have a greatest lower bound: $ST$ and $R$ are not
comparable, and neither are $S$ and $R$,
and for all $n \in \mathbb{N}$ we have that $U_1 \hdots U_n \sqsubseteq R$
and $U_1 \hdots U_n \sqsubseteq ST$ so the meet $R \sqcap ST$ does not exist.
\end{example}

\bigskip

Let us write $\ulbDerivDist{\tm}$ for the set of derivations of
$\tm$ in the $\lambdadist$-calculus,
modulo permutation equivalence.
Because of Orthogonality~(\rprop{orthogonality})
the set $\ulbDerivDist{\tm}$ armed with $\permle$ and $\sqcup$ is an upper semilattice.
Following, we prove that moreover the space $\ulbDerivDist{\tm}$
is a {\em distributive lattice}.

\begin{definition}
A \defn{distributive lattice} is a lattice $(A, \leq, \vee, \wedge)$
such that for all $a,b,c \in A$,
$a \wedge (b \vee c) = (a \wedge b) \vee (a \wedge c)$ and
$a \vee (b \wedge c) = (a \vee b) \wedge (a \vee c)$.
That is, each operation is distributive over the other.
\end{definition}

\begin{example}
For example, if $X$ is a set, then $(\mathcal{P}(X), \subseteq, \cup, \cap)$,
its powerset,
is a distributive lattice.
\end{example}

If we want to prove that $\ulbDerivDist{\tm}$ is a distributive lattice,
we need to prove that it is a lattice and that it satisfies the distributivity property.
We know that $\ulbDerivDist{\tm}$ is a join semilattice (or upper semilattice), so
to prove that it is a lattice we need to define a meet that works.

\subsection*{Meets}

The informal idea behind the definition below
is that the meet of two derivations $\redseq$ and $\redseqtwo$
will be the work they both have performed.

\begin{proposition}[Meet of derivations]
\lprop{meet_of_derivations}
Let $\redseq,\redseqtwo$ be coinitial derivations in the distributive lambda-calculus.
Then there exists an infimum for $\redseq,\redseqtwo$ with respect to the prefix order $\permle$.
We write $\redseq \sqcap \redseqtwo$ for the infimum of $\set{\redseq,\redseqtwo}$ obtained by this
construction.
\end{proposition}
\begin{proof}
If $\redseq$ and $\redseqtwo$ are derivations, we say that a step $\redex$
is a \defn{common} (to $\redseq$ and $\redseqtwo$) whenever $\redex \in \redseq$ and $\redex \in \redseqtwo$.
Define $\redseq \sqcap \redseqtwo$ as follows,
by induction on the length of $\redseq$:
\[
  \redseq \sqcap \redseqtwo \eqdef
    \begin{cases}
    \emptyDerivation & \text{if there are no common steps to $\redseq$ and $\redseqtwo$} \\
    \redex((\redseq/\redex) \sqcap (\redseqtwo/\redex)) & \text{if the step $\redex$ is common to $\redseq$ and $\redseqtwo$ } \\
    \end{cases}
\]
In the second case of the definition, there might be more than one $\redex$
common to $\redseq$ and $\redseqtwo$.
We suppose that one of them is chosen deterministically but make no further assumptions.
To see that this recursive construction is well-defined,
note that the length of $\redseq/\redex$ is lesser than the length of $\redseq$
by the fact that projections are decreasing (\rlem{projections_are_decreasing}).
To conclude the construction, we show that $\redseq \sqcap \redseqtwo$ is an infimum, \ie a greatest lower bound:
\begin{enumerate}
\item {\bf Lower bound.}
  Let us show that $\redseq \sqcap \redseqtwo \permle \redseq$
  by induction on the length of $\redseq$.
  There are two sub-cases, depending on whether
  there is a step common to $\redseq$ and $\redseqtwo$.

  If there is no common step,
  then $\redseq \sqcap \redseqtwo = \emptyDerivation$
  trivially verifies $\redseq \sqcap \redseqtwo \permle \redseq$.

  On the other hand, if there is a common step, we have by definition that
  $\redseq \sqcap \redseqtwo = \redex((\redseq/\redex) \sqcap (\redseqtwo/\redex))$
  where $\redex$ is common to $\redseq$ and $\redseqtwo$.
  Recall that projections are decreasing (\rlem{projections_are_decreasing})
  so $\lengthof{\redseq} > \lengthof{\redseq/\redex}$.
  This allows us to apply the \ih and conclude:
  \[
  \begin{array}{rcll}
    \redseq \sqcap \redseqtwo & =       & \redex((\redseq/\redex) \sqcap (\redseqtwo/\redex)) & \text{ by definition} \\
                              & \permle & \redex(\redseq/\redex)   & \text{ since by \ih $(\redseq/\redex) \sqcap (\redseqtwo/\redex) \permle \redseq/\redex$} \\
                              & \permeq & \redseq(\redex/\redseq)  \\
                              & =       & \redseq                  & \text{ since $\redex \permle \redseq$ by \rlem{characterization_of_belonging}.} \\
  \end{array}
  \]
  Showing that $\redseq \sqcap \redseqtwo \permle \redseqtwo$ is symmetric,
  by induction on the length of $\redseqtwo$.
\item {\bf Greatest lower bound.}
  Let $\redseqthree$ be a lower bound for $\set{\redseq, \redseqtwo}$,
  \ie $\redseqthree \permle \redseq$ and $\redseqthree \permle \redseqtwo$,
  and let us show that $\redseqthree \permle \redseq \sqcap \redseqtwo$.
  We proceed by induction on the length of $\redseq$.
  There are two sub-cases, depending on whether
  there is a step common to $\redseq$ and $\redseqtwo$.

  If there is no common step,
  we claim that $\redseqthree$ must be empty.
  Otherwise we would have that $\redseqthree = \redexthree\redseqthree' \permle \redseq$
  so in particular $\redexthree \permle \redseq$ and $\redexthree \in \redseq$ by \rlem{characterization_of_belonging}.
  Similarly, $\redexthree \in \redseqtwo$ so
  $\redexthree$ is a step common to $\redseq$ and $\redseqtwo$,
  which is a contradiction.
  We obtain that $\redseqthree$ is empty,
  so trivially $\redseqthree = \emptyDerivation \permle \redseq \sqcap \redseqtwo$.

  On the other hand, if there is a common step, we have by definition that
  $\redseq \sqcap \redseqtwo = \redex((\redseq/\redex) \sqcap (\redseqtwo/\redex))$
  where $\redex$ is common to $\redseq$ and $\redseqtwo$.
  Moreover, since $\redseqthree \permle \redseq$ and $\redseqthree \permle \redseqtwo$,
  by projecting along $\redex$ we know that
  $\redseqthree/\redex \permle \redseq/\redex$ and $\redseqthree/\redex \permle \redseqtwo/\redex$.
  So:
  \[
    \begin{array}{rcll}
      \redseqthree & \permle & \redseqthree(\redex/\redseqthree) \\
                   & \permeq & \redex(\redseqthree/\redex) \\
                   & \permle & \redex((\redseq/\redex) \sqcap (\redseqtwo/\redex)) & \text{ since by \ih $\redseqthree/\redex \permle (\redseq/\redex) \sqcap (\redseqtwo/\redex)$} \\
                   & =       & \redseq \sqcap \redseqtwo & \text{ by definition}
    \end{array}
  \]
\end{enumerate}
\end{proof}

\begin{remark}
The infimum of $\set{\redseq,\redseqtwo}$ is unique modulo permutation equivalence,
\ie if $\redseqthree$ is an infimum for $\set{\redseq,\redseqtwo}$ then $\redseqthree \permeq \redseq \sqcap \redseqtwo$.
\end{remark}

\subsection*{Names of join and meet}

Names of derivations will be a helpful tool when computing
meets and joins of derivations, as the next proposition
allows us to easily get them by
taking the intersection or union of sets.

\begin{proposition}[Names of join and meet]
\lprop{names_of_join_and_meet}
The following hold:
\begin{enumerate}
\item $\names(\redseq \sqcup \redseqtwo) = \names(\redseq) \cup \names(\redseqtwo)$
\item $\names(\redseq \sqcap \redseqtwo) = \names(\redseq) \cap \names(\redseqtwo)$
\end{enumerate}
\end{proposition}
\begin{proof}
Item 1 is easy resorting to the definition of $\sqcup$
and \rlem{names_after_projection_along_a_step}:
\[
  \begin{array}{rcll}
  \names(\redseq \sqcup \redseqtwo) & = & \names(\redseq(\redseqtwo/\redseq)) & \text{ by definition of $\sqcup$} \\
                                    & = & \names(\redseq) \cup \names(\redseqtwo/\redseq) \\
                                    & = & \names(\redseq) \cup (\names(\redseqtwo) \setminus \names(\redseq)) & \text{ by \rlem{names_after_projection_along_a_step}} \\
                                    & = & \names(\redseq) \cup \names(\redseqtwo) \\
  \end{array}
\]
%For item 2., 
%the inclusion $\names(\redseq \sqcap \redseqtwo) \subseteq \names(\redseq) \cap \names(\redseqtwo)$
%is an immediate consequence of \rlem{names_of_meet_included_in_names}.
%For other inclusion, namely to show  $\names(\redseq) \cap \names(\redseqtwo) \subseteq \names(\redseq \sqcap \redseqtwo)$,
For item 2., we first prove the following claim:\medskip

{\bf Claim.} $\names(\redseq/(\redseq \sqcap \redseqtwo)) \cap \names(\redseqtwo/(\redseq \sqcap \redseqtwo) = \emptyset$.
{\em Proof of the claim.}
By \rlem{properties_of_disjoint_derivations} it suffices to show that
$(\redseq/(\redseq \sqcap \redseqtwo)) \sqcap (\redseqtwo/(\redseq \sqcap \redseqtwo)) = \emptyDerivation$.
By contradiction, suppose that there is a step $\redexthree$ common to
the derivations
$\redseq/(\redseq \sqcap \redseqtwo)$ and $\redseqtwo/(\redseq \sqcap \redseqtwo)$.
Then the derivation $(\redseq \sqcap \redseqtwo)\redexthree$ is a lower bound for $\set{\redseq,\redseqtwo}$,
\ie
$(\redseq \sqcap \redseqtwo)\redexthree \permle \redseq$
and
$(\redseq \sqcap \redseqtwo)\redexthree \permle \redseqtwo$.
Since $\redseq \sqcap \redseqtwo$ is the greatest lower bound for $\set{\redseq,\redseqtwo}$,
we have that $(\redseq \sqcap \redseqtwo)\redexthree \permle \redseq \sqcap \redseqtwo$.
But this implies that $\redexthree \permle \emptyDerivation$, which is a contradiction.
This concludes the proof of the claim.
\medskip

Note that $\redseq \sqcap \redseqtwo \permle \redseq$,
so we have that $\redseq \permeq (\redseq \sqcap \redseqtwo)(\redseq/(\redseq \sqcap \redseqtwo))$,
and this in turn implies that
$\names(\redseq) = \names((\redseq \sqcap \redseqtwo)(\redseq/(\redseq \sqcap \redseqtwo)))$
by \rcoro{permutation_equivalence_in_terms_of_names}.
Symmetrically,
$\names(\redseqtwo) = \names((\redseq \sqcap \redseqtwo)(\redseqtwo/(\redseq \sqcap \redseqtwo)))$.
Then:
\[
  \begin{array}{rcll}
    &&  \names(\redseq) \cap \names(\redseqtwo) \\
  & = & \names((\redseq \sqcap \redseqtwo)(\redseq/(\redseq \sqcap \redseqtwo))) \cap \names((\redseq \sqcap \redseqtwo)(\redseqtwo/(\redseq \sqcap \redseqtwo))) \\
  & = & \left(\names(\redseq \sqcap \redseqtwo) \cup \names(\redseq/(\redseq \sqcap \redseqtwo))\right) \cap \left(\names(\redseq \sqcap \redseqtwo) \cup \names(\redseqtwo/(\redseq \sqcap \redseqtwo))\right) \\
     && \text{ by \rremark{names_concatenation_union}} \\
  & = &
        \names(\redseq \sqcap \redseqtwo) \cup 
        \left( \names(\redseq/(\redseq \sqcap \redseqtwo)) \cap \names(\redseqtwo/(\redseq \sqcap \redseqtwo)) \right) \\
  && \text{ since $(A \cup B) \cap (A \cup C) = A \cup (B \cap C)$ for arbitrary sets $A,B,C$} \\
  & = &
        \names(\redseq \sqcap \redseqtwo) \\
  && \text{ since $\left( \names(\redseq/(\redseq \sqcap \redseqtwo)) \cap \names(\redseqtwo/(\redseq \sqcap \redseqtwo)) \right) = \emptyset$ by the previous claim} \\
  \end{array}
\]
This concludes the proof.
\end{proof}

\subsection*{Distributive lattice}

\begin{theorem}[Derivations modulo $\permeq$ form a distributive lattice]
\lthm{lattice_is_distributive}
Let $\tm \in \termsdist$ be a correct term.
Then derivations modulo $\permeq$ form a lattice $\ulbDerivDist{\tm}$.
More precisely, let $X$ be the set of derivations in the distributive lambda-calculus going out from $\tm$,
modulo permutation equivalence:
\[
  X \eqdef \frac{\set{\redseq \ST \src(\redseq) = \tm}}{\permeq}
\]
Let moreover $\classof{\redseq}$ denote the equivalence class of a derivation $\redseq$
modulo $\permeq$.
Then $\ulbDerivDist{\tm} \eqdef (X, \leq, \land, \lor)$ is a distributive lattice, where:
\[
  \begin{array}{rcl}
  \classof{\redseq} \leq \classof{\redseqtwo}  & \iffdef & \redseq \permle \redseqtwo \\
  \classof{\redseq} \land \classof{\redseqtwo} & \iffdef & \redseq \sqcap \redseqtwo \\
  \classof{\redseq} \lor \classof{\redseqtwo}  & \iffdef & \redseq \sqcup \redseqtwo \\
  \end{array}
\]
\end{theorem}
\begin{proof}
It is straightforward to check that $\leq$ is a partial order,
and that $\classof{\redseq} \land \classof{\redseqtwo}$ (resp. $\classof{\redseq} \lor \classof{\redseqtwo}$)
is the infimum (resp. supremum) of $\set{\classof{\redseq},\classof{\redseqtwo}}$.

To see that it is distributive, let us first prove the first distributive law:
$(\classof{\redseq} \land \classof{\redseqtwo}) \lor \classof{\redseqthree} =
(\classof{\redseq} \lor \classof{\redseqthree}) \land (\classof{\redseqtwo} \lor \classof{\redseqthree})$.
Let $\redseq,\redseqtwo,\redseqthree$ be arbitrary coinitial derivations.
The following equality holds trivially, since
$(A \cap B) \cup C = (A \cup C) \cap (A \cup B)$ is valid for arbitrary sets $A, B, C$:
  \[
    (\names(\redseq) \cap \names(\redseqtwo)) \cup \names(\redseqthree)
    =
    (\names(\redseq) \cup \names(\redseqthree)) \cap (\names(\redseqtwo) \cup \names(\redseqthree))
  \]
By \rprop{names_of_join_and_meet} this entails:
  \[
    \names((\redseq \sqcap \redseqtwo) \sqcup \redseqthree)
    =
    \names((\redseq \sqcup \redseqthree) \sqcap (\redseqtwo \sqcup \redseqthree))
  \]
By \rcoro{permutation_equivalence_in_terms_of_names} this in turn implies that:
  \[
    (\redseq \sqcap \redseqtwo) \sqcup \redseqthree
    \permeq
    (\redseq \sqcup \redseqthree) \sqcap (\redseqtwo \sqcup \redseqthree)
  \]
So by definition of $\land,\lor$ we obtain:
  \[
    (\classof{\redseq} \land \classof{\redseqtwo}) \lor \classof{\redseqthree}
    =
    (\classof{\redseq} \lor \classof{\redseqthree}) \land (\classof{\redseqtwo} \lor \classof{\redseqthree})
  \]
The other distributive law, namely
$(\classof{\redseq} \lor \classof{\redseqtwo}) \land \classof{\redseqthree} =
(\classof{\redseq} \land \classof{\redseqthree}) \lor (\classof{\redseqtwo} \land \classof{\redseqthree})$
is proved analogously.
\end{proof}

\begin{remark}
The function $\names$ that takes a derivation and returns a set of labels
is well-defined for permutation-equivalence classes,
as a consequence of \rcoro{permutation_equivalence_in_terms_of_names}:
\[
  \names([\redseq]) \eqdef \names(\redseq)
\]
\end{remark}

\begin{theorem}[The lattice of derivations is representable as a ring of sets]
\lthm{lattice_ring_of_sets}
If $\tm \in \termsdist$ is a correct term,
then $\names : \ulbDerivDist{\tm} \to \powerset(\labelset)$ is a monomorphism of lattices,
where $\ulbDerivDist{\tm}$ is the lattice of derivations of $\tm$
and $\powerset(\labelset)$ is the lattice whose elements are sets of labels
ordered by inclusion, with set intersection and set union as
the meet and join operators.
\end{theorem}
\begin{proof}
We are to show that $\names$ is monotonic, that it preserves meets and joins,
and finally that it is a monomorphism:
\begin{itemize}
\item {\bf Monotonic.}
  If $\classof{\redseq} \leq \classof{\redseqtwo}$
  then $\names(\redseq) \subseteq \names(\redseqtwo)$.
  This has been proved in \rprop{prefixes_as_subsets}.
\item {\bf Preserves meets.}
  $\names(\classof{\redseq} \land \classof{\redseqtwo}) =
   \names(\redseq) \cap \names(\redseqtwo)$
  by \rprop{names_of_join_and_meet}.
\item {\bf Preserves joins.}
  $\names(\classof{\redseq} \lor \classof{\redseqtwo}) =
   \names(\redseq) \cup \names(\redseqtwo)$
  by \rprop{names_of_join_and_meet}.
\item {\bf Monomorphism.}
  It suffices to show that $\names$ is injective.
  Indeed, 
  suppose that $\names(\classof{\redseq}) = \names(\classof{\redseqtwo})$.
  By \rcoro{permutation_equivalence_in_terms_of_names}
  we have that $\redseq \permeq \redseqtwo$,
  so
  $\classof{\redseq} = \classof{\redseqtwo}$.
\end{itemize}
\end{proof}

\begin{example}
If we consider the derivation space from \rexample{derivation-ids-ex},
we see that we have 8 derivations modulo permutation equivalence,
representable by the powerset of the set of labels $\set{1, 2, 3}$.
Some examples:
\begin{itemize}
  \item $\emptyset$ represents the empty derivation, $\emptyDerivation$.
  \item $\set{1}$ represents $\redex$.
  \item $\set{1, 2, 3}$ represents $[\redex \redextwo' \redexthree'']$,
    which is equal to $[\redex \redexthree \redextwo'']$ and $[\redextwo \redex' \redexthree']$.
  \item As $\set{1}$ represents $\redex$ and $\set{3}$ represents $\redextwo$, and $\set{1} \cup \set{3} = \set{1, 3}$, we know that $\redex \sqcup \redextwo = [\redex \redextwo']$.
\end{itemize}
\end{example}

