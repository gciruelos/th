
We want to prove that if $\var \neq \vartwo$ and $\var \not\in \fv{\ls{\tmthree}}$, then

\[
  \subs{\subs{\tm}{\var}{\ls{\tmtwo}}}{\vartwo}{\ls{\tmthree}}
  =
  \subs{
    \subs{\tm}{\vartwo}{\ls{\tmthree}_1}
  }{\var}{
    \subs{\ls{\tmtwo}}{\vartwo}{\ls{\tmthree}_2}
  }
\]
where $\ls{\tmthree}$ is a permutation of $\ls{\tmthree}_1 + \ls{\tmthree}_2$.

Recall that the equation above requires that both sides be defined;
in particular
$\lengthof{\ls{\tmthree}_1} = \numof{\vartwo}{\tm}$
and $\lengthof{\ls{\tmthree}_2} = \numof{\vartwo}{\ls{\tmtwo}}$.
When read from left to right, the equation means that given $\ls{\tmthree}$,
it can always be written as some permutation of $\ls{\tmthree}_1 + \ls{\tmthree}_2$.
When read from right to left, it means that given $\ls{\tmthree}_1$ and $\ls{\tmthree}_2$,
taking $\ls{\tmthree}$ to be any permutation of $\ls{\tmthree}_1 + \ls{\tmthree}_2$
verifies the equality.

\begin{proof}
By induction on $\tm$.
\begin{enumerate}
\item \Case{Variable (same, first case) $\tm = \var$}
Then $\ls{\tmtwo} = [\tmtwo]$ and $\ls{\tmthree}_1 = \emptylset$,
so $\ls{\tmthree}_2 = \ls{\tmthree}$.

Firstly,
\begin{equation*}\begin{split}
  \subs{\subs{\var}{\var}{\ls{\tmtwo}}}{\vartwo}{\ls{\tmthree}} &
                                        = \subs{\tmtwo}{\vartwo}{\ls{\tmthree}} \\
\end{split}\end{equation*}

Also,
\begin{equation*}\begin{split}
  \subs{\subs{\var}{\vartwo}{[]}}{\var}{\subs{[\tmtwo]}{\vartwo}{\ls{\tmthree}}}
                   & = \subs{\var}{\var}{\subs{[\tmtwo]}{\vartwo}{\ls{\tmthree}}} \\
                   & = \subs{\var}{\var}{[\subs{\tmtwo}{\vartwo}{\ls{\tmthree}}]} \\
                   & = \subs{\tmtwo}{\vartwo}{\ls{\tmthree}} \\
\end{split}\end{equation*}

\item \Case{Variable (same, second case) $\tm = \vartwo$}
Then $\ls{\tmtwo} = \emptylset$, so $\ls{\tmthree} = [\tmthree]$.

Firstly,
\begin{equation*}\begin{split}
  \subs{\subs{\vartwo}{\var}{\emptylset}}{\vartwo}{[\tmthree]}
                   & = \subs{\vartwo}{\vartwo}{[\tmthree]} \\
                   & = \tmthree \\
\end{split}\end{equation*}

Also,
\begin{equation*}\begin{split}
  \subs{\subs{\vartwo}{\vartwo}{[\tmthree]}}{\var}{\subs{\emptylset}{\vartwo}{\emptylset}}
                   & = \subs{\tmthree}{\var}{\subs{\emptylset}{\vartwo}{\emptylset}} \\
                   & = \subs{\tmthree}{\var}{\emptylset} \\
                   & = \tmthree \\
\end{split}\end{equation*}

\item \Case{Variable (different) $\tm = \varthree$}
Then $\ls{\tmtwo} = \emptylset$ and $\ls{\tmthree} = \emptylset$.

Firstly,
\begin{equation*}\begin{split}
  \subs{\subs{\varthree}{\var}{\emptylset}}{\vartwo}{\emptylset}
                   & = \subs{\varthree}{\vartwo}{\emptylset} \\
                   & = \varthree \\
\end{split}\end{equation*}

Also,
\begin{equation*}\begin{split}
  \subs{\subs{\varthree}{\vartwo}{\emptylset}}{\var}{\subs{\emptylset}{\vartwo}{\emptylset}}
                   & = \subs{\varthree}{\var}{\subs{\emptylset}{\vartwo}{\emptylset}} \\
                   & = \subs{\varthree}{\var}{\emptylset} \\
                   & = \varthree \\
\end{split}\end{equation*}

\item \Case{Abstraction, $\tm = \lamp{\lab}{\varthree}{\tmfour}$}
\begin{equation*}\begin{split}
  \subs{\subs{(\lamp{\lab}{\varthree}{\tmfour})}{\var}{\ls{\tmtwo}}}{\vartwo}{\ls{\tmthree}}
     & = \subs{(\lamp{\lab}{\varthree}{\subs{\tmfour}{\var}{\ls{\tmtwo}}})}{\vartwo}{\ls{\tmthree}} \\
     & = \lamp{\lab}{\varthree}{\subs{\subs{\tmfour}{\var}{\ls{\tmtwo}}}{\vartwo}{\ls{\tmthree}}} \\
     & \eqih \lamp{\lab}{\varthree}{\subs{\subs{\tmfour}
                                               {\vartwo}{\ls{\tmthree}_1}}
                                         {\var}{\subs{\ls{\tmtwo}}
                                                     {\vartwo}{\ls{\tmthree}_2}}} \\
     & = \subs{(\lamp{\lab}{\varthree}{\subs{\tmfour}
                                            {\vartwo}{\ls{\tmthree}_1}})}
              {\var}{\subs{\ls{\tmtwo}}{\vartwo}{\ls{\tmthree}_2}} \\
     & = \subs{\subs{(\lamp{\lab}{\varthree}{\tmfour})}
                                            {\vartwo}{\ls{\tmthree}_1}}
              {\var}{\subs{\ls{\tmtwo}}{\vartwo}{\ls{\tmthree}_2}} \\
\end{split}\end{equation*}
\item \Case{Application, $\tm = \tmfour [\tmfour_1,\hdots,\tmfour_n]$}

\begin{equation*}\begin{split}
  \subs{\subs{(\tmfour [\tmfour_1,\hdots,\tmfour_n])}{\var}{\ls{\tmtwo}}}{\vartwo}{\ls{\tmthree}}
     & = \subs{(\subs{\tmfour}{\var}{\ls{\tmtwo}_0} [\subs{\tmfour_i}{\var}{\ls{\tmtwo}_i}]_{i=0}^n)}
              {\vartwo}{\ls{\tmthree}} \hspace{1cm}\\
\end{split}\end{equation*}
\begin{equation*}\begin{split}
\hspace{2cm} & = \subs{\subs{\tmfour}{\var}{\ls{\tmtwo}_0}}{\vartwo}{\ls{\tmthree}_0}
               [\subs{\subs{\tmfour_i}{\var}{\ls{\tmtwo}_i}}{\vartwo}{\ls{\tmthree}_i}]_{i=0}^n \\
     & \eqih \subs{\subs{\tmfour}{\vartwo}{\ls{\tmthree}_{0,1}}}{\var}{\subs{\ls{\tmtwo}_0}
                  {\vartwo}{\ls{\tmthree}_{0,2}}}
            [\subs{\subs{\tmfour_i}{\vartwo}{\ls{\tmthree}_{i,1}}}{\var}{\subs{\ls{\tmtwo}_i}
                  {\vartwo}{\ls{\tmthree}_{i,2}}}]_{i=0}^n \\
     & = \subs{(\subs{\tmfour}{\vartwo}{\ls{\tmthree}_{0,1}}
                     [\subs{\tmfour_i}{\vartwo}{\ls{\tmthree}_{i,1}}]_{i=0}^n)}
              {\var}{\sum_{i=0}^n \subs{\ls{\tmtwo}_i}{\vartwo}{\ls{\tmthree}_{i,2}}} \\
\end{split}\end{equation*}
\noindent Because $\sum_{i=0}^n \ls{\tmtwo}_i$ is a permutation of $\ls{\tmtwo}$ and
we can perform the substitution of each $\ls{\tmthree}_{i,2}$ simultaneously, defining
$\ls{\tmthreevariant}_2$ as $\sum_{i=0}^n \ls{\tmthree}_{i,2}$, the last term can be rewritten to:

\[
  \subs{(\subs{\tmfour}{\vartwo}{\ls{\tmthree}_{0,1}}
           [\subs{\tmfour_i}{\vartwo}{\ls{\tmthree}_{i,1}}]_{i=0}^n)}
       {\var}{\subs{\ls{\tmtwo}}{\vartwo}{\ls{\tmthreevariant}_2}} \\
\]

\noindent Lastly, if we define $\ls{\tmthreevariant}_1$ to be $\sum_{i=0}^n \ls{\tmthree}_{i,1}$,
we get the desired term:

\[
  \subs{\subs{(\tmfour [\tmfour_1, \hdots, \tmfour_n])}
             {\vartwo}{\ls{\tmthreevariant}_1}}
       {\var}{\subs{\ls{\tmtwo}}{\vartwo}{\ls{\tmthreevariant}_2}} \\
\]
To conclude observe that $\ls{\tmthreevariant}_1 + \ls{\tmthreevariant}_2$ is indeed a permutation of $\ls{\tmthree}$;
indeed:
\[
  \begin{array}{rcll}
  \ls{\tmthreevariant}_1 + \ls{\tmthreevariant}_2
  & = & (\sum_{i=0}^n \ls{\tmthree}_{i,1}) + (\sum_{i=0}^n \ls{\tmthree}_{i,2}) \\
  & \approx & \sum_{i=0}^n \ls{\tmthree}_{i,1} + \ls{\tmthree}_{i,2} \\
  & \approx & \sum_{i=0}^n \ls{\tmthree}_i & \text{by \ih on each index $i=0..n$} \\
  & = & \ls{\tmthree}
  \end{array}
\]
\end{enumerate}
\end{proof}
