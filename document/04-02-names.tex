
The fact that in $\lambdadist$ names of steps
are given by labels of lambdas, which by correctness are pairwise different,
and that the calculus has no deletion nor duplication will
make names of steps suitable to name derivations (\ie series of steps).

Furthermore, giving names to derivations will make them very easy to analyze, as we
will see in this section.

\begin{definition}[Set of names of a derivation]
If $\redseq$ is a derivation, its set of names is
\[
  \names(\redseq) \eqdef \set{\name(\redex) \ST \exists \redseq_1,\redseq_2.\ \redseq = \redseq_1\redex\redseq_2}
\]
\end{definition}

A more precise way to say that our calculus has no duplication nor deletion is the following lemma.

\begin{lemma}[Permanence]
\llem{redex_permanence}
If $\redex \not\in \redseq$ then $\redex/\redseq$ is a singleton.
\end{lemma}
\begin{proof}
By induction on $\redseq$. The base case is trivial, so let $\redseq = \redextwo\redseqtwo$. 
Note that $\redex \neq \redextwo$, so by \rlem{cardinality_of_the_set_of_residuals}
we have that $\redex/\redextwo = \redextwo'$,
where $\redextwo' \not\in \redseqtwo$.
So $\redex/\redextwo\redseqtwo = \redextwo'/\redseqtwo$ is a singleton by \ih.
\end{proof}

\subsection*{Belonging}

There are several notions that can define the fact that a step $\redex$ is performed in
a derivation $\redseq$ (where $\redex$ and $\redseq$ are coinitial).

One of this notions is the one that says that whatever $\redex$ does,
is also done by $\redseq$, \ie, $\redex / \redseq = \emptyset$:
there is no more $\redex$-related work to do. This is usually written as
$\redex \permle \redseq$. This notion can be easily extended to derivations:
$\redseqtwo \permle \redseq$ if $\redseqtwo / \redseq = \emptyDerivation$.

Another (weaker) notion asks that \emph{some} of the work that $\redex$ does be
done in $\redseq$. This is the notion of belonging.
A step $\redex$ \defn{belongs to} a coinitial derivation $\redseq$,
written $\redex \in \redseq$,
if and only if some residual of $\redex$ is contracted along $\redseq$. 
More precisely, $\redex \in \redseq$ if there exist $\redseq_1,\redex',\redseq_2$
such that $\redseq = \redseq_1\redex'\redseq_2$ and $\redex' \in \redex/\redseq_1$.
We write $\redex \not\in \redseq$ if it is not the case that $\redex \in \redseq$.

As it turns out, these two notions are equivalent in our calculus, because
there is no deletion nor duplication of redexes.

\begin{lemma}[Characterization of belonging]
\llem{characterization_of_belonging}
Let $\redex$ be a step and $\redseq$ a coinitial derivation
in the distributive lambda-calculus.
Then the following are equivalent:
\begin{enumerate}
\item $\redex \in \redseq$,
\item $\redex \permle \redseq$,
\item $\name(\redex) \in \names(\redseq)$.
\end{enumerate}
{\em Note:} the hypothesis that $\redex$ and $\redseq$ are coinitial is crucial.
In particular, $(1)$ and $(2)$ by definition only hold when $\redex$ and $\redseq$ are coinitial,
while $(3)$ might hold even if $\redex$ and $\redseq$ are not coinitial.
\end{lemma}
\begin{proof}
$(1 \Rightarrow 2)$
  Let $\redseq = \redseq_1\redextwo\redseqtwo_2$ where $\redextwo$ is a residual of $\redex$.
  Suppose moreover, without loss of generality,
  that $\redseq_1$ is minimal, \ie that $\redex \not\in \redseq_1$.
  By Permanence (\rlem{redex_permanence}) $\redex/\redseq_1$ is a singleton,
  so $\redex/\redseq_1 = \redextwo$.
  This means that $\redex/\redseq_1\redextwo\redseq_2 = \emptyset$,
  so indeed $\redex \permle \redseq_1\redextwo\redseq_2$.

$(2 \Rightarrow 3)$
  By induction on $\redseq$.
  If $\redseq$ is empty, the implication is vacuously true,
  so let $\redseq = \redextwo\redseqtwo$
  and consider two subcases, depending on whether $\redex = \redextwo$.
  If $\redex = \redextwo$, then indeed the first step of $\redseq = \redex\redseqtwo$
  has the same name as $\redex$.
  On the other hand, if $\redex \neq \redextwo$, 
  then by \rlem{cardinality_of_the_set_of_residuals}
  we have that $\redex/\redextwo = \redex'$,
  where $\name(\redex) = \name(\redex')$.
  Note that $\redex \permle \redextwo\redseqtwo$ 
  so $\redex/\redextwo = \redex' \permle \redseqtwo$.
  By applying the \ih we obtain that there must be a step in $\redseqtwo$
  whose name is $\name(\redex) = \name(\redex')$, and we are done.

$(3 \Rightarrow 1)$
  By induction on $\redseq$.
  If $\redseq$ is empty, the implication is vacuously true,
  so let $\redseq = \redextwo\redseqtwo$
  and consider two subcases, depending on whether $\name(\redex) = \name(\redextwo)$.
  If $\name(\redex) = \name(\redextwo)$ then $\redex$ and $\redextwo$ must be the same step,
  since terms are correct, which means that labels decorating lambdas are pairwise distinct.
  Hence $\redex \in \redseq = \redex\redseqtwo$.
  On the other hand, if $\name(\redex) \neq \name(\redextwo)$, then
  $\redex \neq \redextwo$ so by \rlem{cardinality_of_the_set_of_residuals}
  we have that $\redex/\redextwo = \redex'$,
  where $\name(\redex) = \name(\redex')$.
  By hypothesis, there is a step in the derivation $\redseq = \redextwo\redseqtwo$ whose name
  is $\name(\redex)$, and it is not $\redextwo$, so there must be at least one step in the
  derivation $\redseqtwo$ whose name is is $\name(\redex) = \name(\redex')$.
  By \ih $\redex' \in \redseqtwo$
  and then, since $\redex'$ is a residual of $\redex$,
  we conclude that $\redex \in \redextwo\redseqtwo$,
  as required.
\end{proof}

\bigskip

\subsection*{Name of a reduction}

In this subsection we will see why the naming scheme we proposed for steps and
reductions is useful. Recall that the names of a reduction are the set of names of the steps that form it.
What follows is a list of different results that characterize properties
of reductions in terms of its name.

It is immediate to note that
when composing two derivations $\redseq, \redseqtwo$,
the set of names of $\redseq\redseqtwo$
results from the union of the names of $\redseq$ and $\redseqtwo$:

\begin{remark}
\lremark{names_concatenation_union}
$\names(\redseq\redseqtwo) = \names(\redseq) \cup \names(\redseqtwo)$
\end{remark}

Indeed, a stronger results holds. Namely, the union is disjoint:

\begin{lemma}
\llem{names_concatenation_disjoint_union}
$\names(\redseq\redseqtwo) = \names(\redseq) \uplus \names(\redseqtwo)$
\end{lemma}
\begin{proof}
Let $\redseq : \tm \rtodist \tmtwo$.
By induction on $\redseq$ we argue that the set of labels decorating the lambdas in $\tmtwo$
is disjoint from $\names(\redseq)$.
The base case is immediate, so let $\redseq = \redexthree\redseqthree$.
The step $\redexthree$ is of the form:
\[
  \redexthree \HS:\HS
  \tm =
  \conof{(\lamp{\lab}{\var}{\tmthree})\ls{\tmfour}}
  \todist
  \conof{\subs{\tmthree}{\var}{\ls{\tmfour}}}
\]
hence the target of $\redexthree$ has no lambdas decorated with the label $\lab$.
Moreover, the derivation $\redseqthree$ is of the form:
\[
  \redseqthree \HS:\HS
  \conof{\subs{\tmthree}{\var}{\ls{\tmfour}}}
  \rtodist
  \tmtwo
\]
and by \ih the set of labels decorating the lambdas in $\tmtwo$
is disjoint from $\names(\redseqthree)$.
As a consequence,
the set of labels decorating the lambdas in $\tmtwo$
is disjoint from both $\names(\redseqthree)$
and $\set{\lab}$.
This completes the proof.
\end{proof}

\begin{corollary}[The length of a derivation equals the number of distinct names along it]
\lcoro{length_of_derivation_is_number_of_distinct_names}
If $\redseq$ is a derivation in the distributive lambda-calculus
and $\lengthof{\redseq}$ denotes the length of $\redseq$,
then
$\lengthof{\redseq} = \#(\names(\redseq))$.
\end{corollary}
\begin{proof}
Let $\redseq = \redex_1\hdots\redex_n$.
Then by \rlem{names_concatenation_disjoint_union},
$\names(\redseq) = \biguplus_{i=1}^{n} \set{\name(\redex_i)}$
and
$\#(\names(\redseq)) = n$, as required.
\end{proof}

The names of a derivation are coherent and work well
with all the usual residual theory definitions: projections, prefix order, and permutation
equivalence.

\begin{lemma}[Names after a projection]
\llem{names_after_projection_along_a_step}
If $\redseq$ and $\redseqtwo$ are coinitial derivations, then
$\names(\redseq/\redseqtwo) = \names(\redseq) \setminus \names(\redseqtwo)$
\end{lemma}
\begin{proof}
First we claim that $\names(\redseq/\redex) = \names(\redseq) \setminus \set{\name(\redex)}$.
We proceed by induction on $\redseq$.
The base case is immediate, so let $\redseq = \redextwo\redseqtwo$
and consider two subcases, depending on whether $\redex = \redextwo$.
If $\redex = \redextwo$,
then:
\[
  \begin{array}{rcl}
  \names(\redseq/\redex) & = & \names(\redex\redseqtwo/\redex) \\
                         & = & \names(\redseqtwo) \\
                         & = & \names(\redex\redseqtwo) \setminus \set{\name(\redex)} \\
  \end{array}
\]
On the other hand if $\redex \neq \redextwo$,
then by \rlem{cardinality_of_the_set_of_residuals}
we have that $\redex/\redextwo = \redex'$
where $\name(\redex) = \name(\redex')$
and, similarly,
$\redextwo/\redex = \redextwo'$,
where $\name(\redextwo) = \name(\redextwo')$. Hence:
\[
  \begin{array}{rcll}
  \names(\redseq/\redex) & = & \names(\redextwo\redseqtwo/\redex) \\
                         & = & \names((\redextwo/\redex)(\redseqtwo/(\redex/\redextwo))) \\
                         & = & \names(\redextwo/\redex) \cup \names(\redseqtwo/(\redex/\redextwo)) \\
                         & = & \names(\redextwo') \cup \names(\redseqtwo/\redex') \\
                         & = & \names(\redextwo') \cup (\names(\redseqtwo) \setminus \set{\name(\redex')} & \text{ by \ih} \\
                         & = & \names(\redextwo) \cup (\names(\redseqtwo) \setminus \set{\name(\redex)}) \\
                         & = & (\names(\redextwo) \cup \names(\redseqtwo)) \setminus \set{\name(\redex)} & \text{since $\name(\redex) \neq \name(\redextwo)$} \\
                         & = & \names(\redextwo\redseqtwo) \setminus \set{\name(\redex)} \\
                         & = & \names(\redseq) \setminus \set{\name(\redex)} \\
  \end{array}
\]
which completes the claim.
To see that $\names(\redseq/\redseqtwo) = \names(\redseq) \setminus \names(\redseqtwo)$
for an arbitrary derivation $\redseqtwo$,
proceed by induction on $\redseqtwo$.
If $\redseqtwo$ is empty it is trivial, so
consider the case in which $\redseqtwo = \redex\redseqthree$.
Then:
\[
  \begin{array}{rcll}
  \names(\redseq/\redex\redseqthree) & = & \names((\redseq/\redex)/\redseqthree) \\
                                     & = & \names(\redseq/\redex) \setminus \names(\redseqthree) & \text{ by \ih} \\
                                     & = & (\names(\redseq) \setminus \set{\name(\redex)}) \setminus \names(\redseqthree) & \text{ by the previous claim} \\
                                     & = & \names(\redseq) \setminus (\set{\name(\redex)} \cup \names(\redseqthree)) \\
                                     & = & \names(\redseq) \setminus \names(\redex\redseqthree) \\
  \end{array}
\]
as required.
\end{proof}

\begin{proposition}[Prefixes as subsets]
\lprop{prefixes_as_subsets}
Let $\redseq,\redseqtwo$ be coinitial derivations in the distributive lambda-calculus.
Then
$\redseq \permle \redseqtwo$ if and only if $\names(\redseq) \subseteq \names(\redseqtwo)$.
\end{proposition}
\begin{proof}
By induction on $\redseq$.
The base case is immediate since $\emptyDerivation \permle \redseqtwo$
and $\emptyset \subseteq \names(\redseqtwo)$ both hold.
So let $\redseq = \redexthree\redseqthree$. First note that
the following equivalence holds:
\begin{equation}
\leqn{prefixes_as_subsets:1}
    \redexthree\redseqthree \permle \redseqtwo
    \HS\iff\HS
    \redexthree \permle \redseqtwo \ \land\ \redseqthree \permle \redseqtwo/\redexthree
\end{equation}
Indeed:
\begin{itemize}
\item $(\Rightarrow)$
    Suppose that $\redexthree\redseqthree \permle \redseqtwo$.
    Then, on one hand, $\redexthree \permle \redexthree\redseqthree \permle \redseqtwo$.
    On the other hand, projection is monotonic, so
    $\redseqthree = \redexthree\redseqthree/\redexthree \permle \redseqtwo/\redexthree$.
\item $(\Leftarrow)$
    Since $\redseqthree \permle \redseqtwo/\redexthree$
    we have that $\redexthree\redseqthree \permle \redexthree(\redseqtwo/\redexthree) \permeq \redseqtwo(\redexthree/\redseqtwo) = \redseqtwo$
    since $\redexthree/\redseqtwo = \emptyDerivation$.
\end{itemize}
So we have that:
\[
  \begin{array}{rcll}
    \redexthree\redseqthree \permle \redseqtwo
    & \iff &
    \redexthree \permle \redseqtwo \ \land\ \redseqthree \permle \redseqtwo/\redexthree
      & \text{ by \reqn{prefixes_as_subsets:1}} \\
    & \iff &
    \name(\redexthree) \in \names(\redseqtwo) \ \land\ \redseqthree \permle \redseqtwo/\redexthree
      & \text{ by \rlem{characterization_of_belonging}} \\
    & \iff &
    \name(\redexthree) \in \names(\redseqtwo) \ \land\ \names(\redseqthree) \subseteq \names(\redseqtwo/\redexthree)
      & \text{ by \ih} \\
    & \iff &
    \name(\redexthree) \in \names(\redseqtwo) \ \land\ \names(\redseqthree) \subseteq \names(\redseqtwo) \setminus \set{\name(\redexthree)}
      & \text{ by \rlem{names_after_projection_along_a_step}} \\
    & \iff &
    \names(\redexthree\redseqthree) \subseteq \names(\redseqtwo)
  \end{array}
\]
To justify the very last equivalence, the $(\Rightarrow)$ direction is immediate.
For the $(\Leftarrow)$ direction,
the difficulty is ensuring that
$\names(\redseqthree) \subseteq \names(\redseqtwo) \setminus \set{\name(\redexthree)}$
from the fact that
$\names(\redexthree\redseqthree) \subseteq \names(\redseqtwo)$.
To see this it suffices to observe that by \rlem{names_concatenation_disjoint_union},
$\names(\redexthree\redseqthree)$
is the {\em disjoint} union
$\names(\redexthree) \uplus \names(\redseqthree)$,
which means that
$\name(\redexthree) \not\in \names(\redseqthree)$.

\end{proof}

\begin{corollary}[Permutation equivalence in terms of names]
\lcoro{permutation_equivalence_in_terms_of_names}
Let $\redseq,\redseqtwo$ be coinitial derivations in the distributive lambda-calculus.
Then
$\redseq \permeq \redseqtwo$ if and only if $\names(\redseq) = \names(\redseqtwo)$.
\end{corollary}
\begin{proof}
Immediate since
\[
  \begin{array}{rcll}
  \redseq \permeq \redseqtwo & \iff & \redseq \permle \redseqtwo \ \land\ \redseqtwo \permle \redseq \\
                             & \iff & \names(\redseq) \subseteq \names(\redseqtwo) \ \land\ \names(\redseqtwo) \subseteq \names(\redseq)
                                    & \text{ by \rprop{prefixes_as_subsets}}\\
                             & \iff & \names(\redseq) = \names(\redseqtwo)
  \end{array}
\]
\end{proof}
