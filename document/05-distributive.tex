\chapter{A distributive $\lambda$-calculus}
\lcha{lambdadist}

In this chapter we present a
{\em distributive $\lambda$-calculus} ($\lambdadist$),
and we prove some basic properties it enjoys.

Terms of the $\lambdadist$-calculus are typing derivations of a non-idempotent intersection type
system, written using proof term syntax.
The underlying type system is a variant of
system~$\mathcal{W}$ of \cite{bucciarelli2014inhabitation,bucciarelli2017non},
the main difference being that $\lambdadist$
uses {\em labels} and a suitable invariant on terms,
to ensure that the formal parameters of all functions
are in 1--1 correspondence with the actual arguments that they receive.


\section{Types}

We will now present the type system we will work with, which as we said, is a variant of the system presented
in \cite{bucciarelli2017non}.

\begin{definition}[Types and typing contexts]
Let $\labelset = \set{e,e',e'',\hdots}$ be a denumerable set of labels.
The sets of \defn{types}, ranged over by $\typ,\typtwo,\typthree,\hdots$,
and \defn{finite sets of types}, ranged over by $\mtyp,\mtyptwo,\mtypthree,\hdots$,
are given mutually inductively by the following abstract syntax:
\[
  \typ ::= \basetyp^{\lab} \mid \mtyp \tolab{\lab} \typ
\]
\[
  \mtyp ::= \lset{\typ_i}_{i=1}^{n} \HS \text{for some $n \geq 0$}
\]
In a type like $\basetyp^{\lab}$ and $\mtyp \tolab{\lab} \typ$,
the label $\lab$ is called the \defn{external label}.
\defn{Typing contexts}, or contexts for short,
ranged over by $\tctx,\tctxtwo,\tctxthree,\hdots$ are (total) functions from variables to finite sets of types.
We write $\dom\tctx$ for the set of variables $\var$ such that $\tctx(\var) \neq \lset{}$.
We write $\emptyContext$ for the context such that $\emptyContext(\var) = \lset{}$ for every variable $\var$.
The notation $\tctx + \tctxtwo$ stands for the \defn{sum of contexts}, defined as follows:
\[
    (\tctx + \tctxtwo)(\var) \eqdef \tctx(\var) + \tctxtwo(\var)
\]
The notation $\tctx \oplus \tctxtwo$ stands for the \defn{disjoint sum of contexts},
\ie it stands for $\tctx + \tctxtwo$ provided $\dom\tctx \cap \dom\tctxtwo = \emptyset$.
We also write $\tctx +_{i=1}^{n} \tctxtwo_i$ for $\tctx + \sum_{i=1}^{n} \tctxtwo_i$.
Moreover, $\var : \mtyp$ denotes the context such that $(\var : \mtyp)(\var) = \mtyp$
and $\dom(\var : \mtyp) = \set{\var}$.
\end{definition}

Remark that the (only) difference with the system~$\mathcal{W}$ is that we include labels.
The other, related, difference will appear when we define the terms of the calculus.

More in general, note that it will not be possible for a term to have multiple types,
like the name \textit{intersection type system} would suggest.
Rather, what happens is that function terms will receive a parameter that can be interpreted
as having several types.


\section{Syntax}


\begin{definition}[Distributive type system]
The set of \defn{distributive terms},
ranged over by ($\tm,\tmtwo,\tmthree,\hdots$) is given by the following abstract syntax:
\[
  \tm ::= \var^{\typ} \mid \lamp{\lab}{\var}{\tm} \mid \tm\,\ls{\tm}
\]
Typing rules are defined inductively as follows.
\[
  \indrule{var}{
  }{
    \var : \lset{\typ} \vdash \var^\typ : \typ
  }
  \indrule{\toI}{
    \tctx \oplus \var : \mtyp \vdash \tm : \typtwo
  }{
    \tctx \vdash \lamp{\lab}{\var}{\tm} : \mtyp \tolab{\lab} \typtwo
  }
\]
\[
  \indrule{\toE}{
    \tctx \vdash \tm : \lset{\typtwo_1,\hdots,\typtwo_n} \tolab{\lab} \typ
    \HS
    \left( \tctxtwo_i \vdash \tmtwo_i : \typtwo_i \right)_{i=1}^{n}
  }{
    \tctx +_{i=1}^{n} \tctxtwo_i \vdash \tm[\tmtwo_1,\hdots,\tmtwo_n] : \typ
  }
\]
Moreover, we introduce a judgment of the form
$[\tctx_1,\hdots,\tctx_n] \vdash [\tm_1,\hdots,\tm_n] : [\typ_1,\hdots,\typ_n]$
with the following rule:
\[
  \indrule{t-multi}{
    \tctx_i \vdash \tm_i : \typ_i \text{ for all $i=1..n$}
  }{
    [\tctx_1,\hdots,\tctx_n] \vdash [\tm_1,\hdots,\tm_n] : [\typ_1,\hdots,\typ_n]
  }
\]
\end{definition}

For example, using integer labels,
$\vdash \lamp{1}{\var}{\var^{[\alpha^2,\alpha^3] \tolab{4} \beta^5}[\var^{\alpha^3},\var^{\alpha^2}]}
: [[\alpha^2,\alpha^3] \tolab{4} \beta^5, \alpha^2, \alpha^3] \tolab{1} \beta^5$
is a derivable judgment.
For another example,
$\var : [] \tolab{1} \alpha^2 \vdash \var^{[] \tolab{1} \alpha^2} [] : \alpha^2$
is a derivable judgment.

Note that writing all labels is rather cumbersome, so we will omit or simplify them when possible.

\subsection{Correctness}
Observe that the definition we gave has a fatal problem:
we cannot uniquely associate arguments with variables in the body of the lambdas.

For example, consider the term
$(\lamp{1}{x}{y^{[\alpha^2, \alpha^2] \tolab{3} \alpha^4} [x^{\alpha^2}, x^{\alpha^2}]})
[a^{\alpha^2}, b^{\alpha^2}]$.
We don't know which parameter to associate which each $x$-- which parameter goes in the first $x$, $a$ or $b$?

To solve that problem we introduce an invariant that will ensure
that problem does not manifest.
We will call that invariant \defn{correctness},
and we will consider $\lambdadist$ to be the system of all correct terms.

\begin{definition}[Correct term]
A typable term $\tm$ is \defn{correct} if the three following conditions hold:
\begin{itemize}
\item {\bf Unique lambdas.}
      Lambdas in a term are decorated with pairwise distinct labels.
\item {\bf Sequential contexts.}
      For every judgment $\tctx \vdash \tmtwo : \typ$ in the type derivation of $\tm$,
      and for every variable $\var$, the set $\tctx(\var)$ is sequential.
\item {\bf Sequential types.}
      For every subtype $\mtyp \tolab{\lab} \typ$ ocurring somewhere in the derivation of $\tm$,
      the set $\mtyp$ is sequential.
\end{itemize}
The set of correct terms is written $\lambdadist$.
\end{definition}


Having defined the types and the terms of out system
sheds some light on the resource management capabilities that we claimed it will enjoy:
note that we can track very precisely how (\ie with which type) a term will be used or
a bounded variable will be evaluated.
This will prove very useful to analyze the $\lambda$-calculus.

\section{Basic Properties}

\section{Termination}

\section{Confluence}

\section{Simulation}
\lsec{simulation}

\section{Residual Theory}

\section{Simulation Residuals}

