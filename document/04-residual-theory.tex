\chapter{Residual theory}
\lcha{residual_theory}
In this section we will develop the theory of residuals of the $\lambdadist$-calculus.
We do this as it will help us to prove that derivation spaces of $\lambdadist$
terms have a very simple structure. This together with the fact that
the $\lambdadist$-calculus
can simulate the $\lambda$-calculus in a strong sense (as we will learn in the next chapter)
will make the $\lambdadist$-calculus a very useful tool
to analyze the pure $\lambda$-calculus.


The basic concepts of general rewriting theory were outlined in
the preliminaries (\rsec{rewriting_theory_preliminaries}),
and we will now concentrate in the specifics of our calculus.

Informally, the residual of a step after another is what is left of a step after executing another one; it is a set of steps.

\begin{example} If we consider the term of the pure lambda calculus:
\[(\lam{\var}{\vartwo})((\lam{\varthree}{\varthree}) w)\]
Then there are two steps we can perform (we shall call them $\redex$ and $\redextwo$).
\begin{align*}
  \redex &:
    (\lam{\var}{\vartwo})((\lam{\varthree}{\varthree}) w) \to y \\
  \redextwo &:
    (\lam{\var}{\vartwo})((\lam{\varthree}{\varthree}) w) \to (\lam{\var}{\vartwo}) w
\end{align*}
The residual of $\redex$ after $\redextwo$ is the step $(\lam{\var}{\vartwo}) w \to w$.
On the other hand, the residual of $\redextwo$ after $\redex$ is the empty set:
$\redextwo$ was erased by $\redex$.
\end{example}

It is interesting to develop a theory of residuals for $\lambdadist$ because
residuals allow us to trace a redex through the reduction of a term.
Given that the purpose of $\lambdadist$ is to be able to track
how resources are used in $\lambda$-terms, it is crucial to
be able to learn how these resources interact during a reduction.

As it turns out, the theory of residuals of $\lambdadist$ will prove to be powerful
enough to represent meaningful information, but simple enough to have a
comprehensible structure.


\begin{definition}
If $\redex : \tm \todistl{\lab} \tm'$ is a step in the distributive lambda
calculus, then:
\[
  \begin{array}{rcll}
    \src(\redex) & \eqdef & \tm & \text{ is the \defn{source} of $\redex$} \\
    \tgt(\redex) & \eqdef & \tm' & \text{ is the \defn{target} of $\redex$} \\
    \name(\redex) & \eqdef & \lab & \text{ is the \defn{name} of $\redex$}
  \end{array}
\]
Two steps $\redex$ and $\redextwo$
are \defn{coinitial} if $\src(\redex) = \src(\redextwo)$
and \defn{cofinal} if $\tgt(\redex) = \tgt(\redextwo)$.
\end{definition}

\begin{definition}[Residuals in the distributive lambda-calculus]
\ldef{residuals}
Given coinitial steps $\redex, \redextwo$,
the set $\redex/\redextwo$ of \defn{residuals of $\redex$ after $\redextwo$} is defined as follows:
\[
  \redex/\redextwo = \set{\redex' \ST \src(\redex') = \tgt(\redextwo) \text{ and } \name(\redex) = \name(\redex')}
\]
\end{definition}

\begin{remark}
Recall that the name of a step is the label that decorates the lambda reduced by the step,
and that in correct terms all lambdas have pairwise distinct labels.
Given that our calculus has no duplication or erasure (as per the following lemma),
names of steps will be useful to name reductions.
\end{remark}


\begin{lemma}[Cardinality of the set of residuals]
\llem{cardinality_of_the_set_of_residuals}
\[
  \#(\redex/\redextwo) = \begin{cases}
                         0 & \text{ if $\redex = \redextwo$} \\
                         1 & \text{ otherwise}
                         \end{cases}
\]
\end{lemma}
\begin{proof}
Recall that, by definition, lambdas in a correct term have pairwise distinct labels.
Consider first the case when $\redex = \redextwo$.
Then
$
  \redex = \redextwo : \conof{(\lamp{\lab}{\var}{\tm})\ls{\tmtwo}} \todistl{\lab} \conof{\subs{\tm}{\var}{\ls{\tmtwo}}}
$.
There is only one lambda decorated with $\lab$ in the source,
so there are no lambdas decorated with $\lab$ in the target.
Hence $\redex/\redextwo = \emptyset$.

On the other hand if, $\redex \neq \redextwo$,
by the Strong Permutation property (\rprop{strong_permutation})
there exists a step $\redex' \in \redex/\redextwo$
with the same name as $\redex$.
There are no other lambdas decorated with $\lab$ in the target.
Hence $\redex/\redextwo = \set{\redex'}$.
\end{proof}


\section{Orthogonality of $\lambdadist$}

Recall from the preliminaries that some abstract rewriting systems have the property of being
\emph{orthogonal}.
Being orthtogonal entails a myriad of properties and results that make working
with orthogonal rewrite systems very easy.
Informally, in an orthogonal rewrite system residuals behave and have the properties
that one would expect, some of them are summarized in the table that follows.

{\footnotesize
\[
\begin{array}{|c|c|c|}
\hline
\begin{array}{r@{\hspace{0.2cm}}c@{\hspace{0.2cm}}l}
  \emptyDerivation\,\redseq & = & \redseq
\\
  \redseq\,\emptyDerivation & = & \redseq
\\
  \emptyDerivation/\redseq & = & \emptyDerivation
\\
  \redseq/\emptyDerivation & = & \redseq
\\
  \redseq/\redseqtwo\redseqthree & = & (\redseq/\redseqtwo)/\redseqthree
\\
  \redseq\redseqtwo/\redseqthree & = & (\redseq/\redseqthree)(\redseqtwo/(\redseqthree/\redseq))
\\
  \redseq/\redseq & = & \emptyDerivation
\end{array}
&
\begin{array}{r@{\hspace{0.2cm}}c@{\hspace{0.2cm}}l}
  \redseq \permle \redseqtwo & \iffdef & \redseq/\redseqtwo = \emptyDerivation
\\
  \redseq \permeq \redseqtwo & \iffdef & \redseq \permle \redseqtwo \land \redseqtwo \permle \redseq
\\
  \redseq \sqcup \redseqtwo  & \eqdef & \redseq(\redseqtwo/\redseq) 
\\
  \redseq \permeq \redseqtwo & \implies & \redseqthree/\redseq = \redseqthree/\redseqtwo
\\
  \redseq \permle \redseqtwo & \iff & \exists \redseqthree.\ \redseq\redseqthree \permeq \redseqtwo
\\
  \redseq \permle \redseqtwo & \iff & \redseq \sqcup \redseqtwo \permeq \redseqtwo
\end{array}
&
\begin{array}{r@{\hspace{0.2cm}}c@{\hspace{0.2cm}}l}
  \redseq \permle \redseqtwo & \implies & \redseq/\redseqthree \permle \redseqtwo/\redseqthree
\\
  \redseq \permle \redseqtwo & \iff & \redseqthree\redseq \permle \redseqthree\redseqtwo
\\
  \redseq \sqcup \redseqtwo  & \permeq & \redseqtwo \sqcup \redseq
\\
  (\redseq \sqcup \redseqtwo) \sqcup \redseqthree & = & \redseq \sqcup (\redseqtwo \sqcup \redseqthree)
\\
  \redseq & \permle & \redseq \sqcup \redseqtwo
\\
  (\redseq \sqcup \redseqtwo)/\redseqthree & = & (\redseq/\redseqthree) \sqcup (\redseqtwo/\redseqthree) \\
\end{array}
\\
\hline
\end{array}
\]
}

As we stated in the preliminaries, one must check four axioms in order to prove that
a given system (in this case $\lambdadist$) is orthogonal.

Let's see why the axiom called Finite Developments is interesting.
As stated in \cite{thesismellies}, the \emph{Axiome FD}, or finite developments axiom,
asks that for every set of coinitial steps $\redexset$, then all developments of $\redexset$
are finite.

This axiom is important because suppose we have two coinitial steps $\redex$ and $\redextwo$,
such that $\redex : \tm \to \tmtwo$. Then $\redextwo / \redex$ is a set of steps with
the same source ($\tmtwo$). The idea is that if we had $\tm$ and wanted to ``execute'' both
$\redex$ and $\redextwo$, if we didn't have finite developments, then in particular
we don't have finite complete developments, so a reduction that tries to
execute what's left of $\redextwo$ after $\redex$ may not finish.

The $\lambdadist$-calculus not only enjoys of finite developments, but we can
give the exact length a complete development of a set of steps will have.

\begin{lemma}[Finite developments]
\llem{finite_developments}
Let $\redexset$ be a set of coinitial steps.
Then the length of every complete development of $\redexset$
is precisely the cardinality of $\redexset$.
In particular, developments are finite.
\end{lemma}
\begin{proof}
By induction on the cardinality of $\redexset$.
If $\redexset = \emptyset$, the only complete development of $\redexset$ is $\emptyDerivation$ and we are done.
Otherwise, if $\redseq$ is a complete development of $\redexset$, it is a non-empty derivation, \ie
$\redseq = \redex\redseqtwo$
where $\redex \in \redexset$
and such that $\redseqtwo$ is a complete development of $\redexset/\redex$.
Since residuals of distinct redexes have distinct names (and hence they are distinct)
we have that
$
  \redexset/\redex = \uplus_{\redextwo \in \redexset} (\redextwo/\redex)
$,
where $\uplus$ denotes the disjoint union of sets.
Moreover, $\#(\redextwo/\redex) = 1$ if and only if $\redex \neq \redextwo$
by \rlem{cardinality_of_the_set_of_residuals}, so:
\[
  \#(\redexset/\redex) =
  \Sigma_{\redextwo \in \redexset} \#(\redextwo/\redex) =
  \#(\redexset \setminus \set{\redex}) =
  \#(\redexset) - 1
\]
Hence by \ih the length of $\redseqtwo$ is $\#(\redexset) - 1$
and we conclude.
\end{proof}

\begin{proposition}
The distributive lambda-calculus is an Orthogonal Axiomatic Rewrite System
in the sense of Melli\`es.
\end{proposition}
\begin{proof}
There are four axioms to check:
\begin{enumerate}
\item {\bf Autoerasure.}
  Immediate from the cardinality of residuals lemma (\rlem{cardinality_of_the_set_of_residuals}).
\item {\bf Finite Residuals.}
  Immediate from the cardinality of residuals lemma (\rlem{cardinality_of_the_set_of_residuals}).
\item {\bf Finite Developments.}
  Proved in the Finite Developments lemma (\rlem{finite_developments}).
\item {\bf Semantic Orthogonality.}
  A consequence of the Strong Permutation property (\rprop{strong_permutation}).
\end{enumerate}
\end{proof}



\bigskip
\begin{example}
\lexample{derivation-ids-ex}
Next we present an example that will be useful to illustrate
some of the upcoming propositions. Consider the following term:

\[
  (\lamp{\lab_1}{\var}{\var^{[\alpha] \to \alpha} [\var^\alpha])}
    [\lamp{\lab_2}{\vartwo}{\vartwo^\alpha},
    (\lamp{\lab_3}{\varthree}{\varthree^\alpha}) [w^\alpha]].
\]

What follows is its derivation space, \ie,
all possible derivations that have the previous term as a source.

{\footnotesize
\xymatrix{
  &
  (\lamp{\lab_1}{\var}{\var^{[\alpha] \to \alpha} [\var^\alpha])}
    [\lamp{\lab_2}{\vartwo}{\vartwo^\alpha},
    (\lamp{\lab_3}{\varthree}{\varthree^\alpha}) [w^\alpha]]
    \ar@[->][rd]^{\lab_3}_{\redextwo}
    \ar@[->][ld]_{\lab_1}^{\redex}
  & \\
  (\lamp{\lab_2}{\vartwo}{\vartwo^\alpha})
   ((\lamp{\lab_3}{\varthree}{\varthree^\alpha}) [w^\alpha])
   \ar@[->][rd]_{\lab_3}^{\redextwo'}
   \ar@[->][dd]_{\lab_2}^{\redexthree}
   &
   &
  (\lamp{\lab_1}{\var}{\var^{[\alpha] \to \alpha} [\var^\alpha])}
    [\lamp{\lab_2}{\vartwo}{\vartwo^\alpha},
    w^\alpha]
   \ar@[->][ld]^{\lab_1}_{\redex'} \\
  &
  (\lamp{\lab_2}{\vartwo}{\vartwo^\alpha})  [w^\alpha]
  \ar@[->][d]_{\lab_2}^{\redexthree'}
  & \\
  (\lamp{\lab_3}{\varthree}{\varthree^\alpha})  [w^\alpha]
  \ar@[->][r]_{\lab_3}^{\redextwo''}
  &
  w^\alpha
  & \\}
}
\end{example}

Some facts about these derivations:

\begin{itemize}
  \item $\redex$ creates $\redexthree$: we cannot perform $\redexthree$ until we performed $\redex$. We will expand on the topic of creation later.
  \item $\redex \redextwo' \permeq \redextwo \redex'$. Remember that two derivations $\redseq$,
    $\redseqtwo$ are permutation equivalent if $\redseq \permle \redseqtwo \permle \redseq$,
    \ie if they perform the same amount of work.
  \item $\redextwo / \redex = \set{\redex''}$ and
    $\redextwo / \redex \redexthree = \set{\redextwo''}$. Remember that in $\lambdadist$,
    residuals are either empty or singletons, so we will often skip the parentheses.
  \item $\name(\redexthree) = \name(\redexthree') = \lab_3$.
\end{itemize}


\section{Names and labels}

The fact that in $\lambdadist$ names of steps
are given by labels of lambdas, which by correctness are pairwise different,
and that the calculus has no deletion nor duplication will
make names of steps suitable to name derivations (\ie series of steps).

Furthermore, giving names to derivations will make them very easy to analyze, as we
will see in this section.

\begin{definition}[Set of names of a derivation]
If $\redseq$ is a derivation, its set of names is
\[
  \names(\redseq) \eqdef \set{\name(\redex) \ST \exists \redseq_1,\redseq_2.\ \redseq = \redseq_1\redex\redseq_2}
\]
\end{definition}

A more precise way to say that our calculus has no duplication nor deletion is the following lemma.

\begin{lemma}[Permanence]
\llem{redex_permanence}
If $\redex \not\in \redseq$ then $\redex/\redseq$ is a singleton.
\end{lemma}
\begin{proof}
By induction on $\redseq$. The base case is trivial, so let $\redseq = \redextwo\redseqtwo$. 
Note that $\redex \neq \redextwo$, so by \rlem{cardinality_of_the_set_of_residuals}
we have that $\redex/\redextwo = \redextwo'$,
where $\redextwo' \not\in \redseqtwo$.
So $\redex/\redextwo\redseqtwo = \redextwo'/\redseqtwo$ is a singleton by \ih.
\end{proof}

\subsection*{Belonging}

There are several notions that can define the fact that a step $\redex$ is performed in
a derivation $\redseq$ (where $\redex$ and $\redseq$ are coinitial).

One of this notions is the one that says that whatever $\redex$ does,
is also done by $\redseq$, \ie, $\redex / \redseq = \emptyset$:
there is no more $\redex$-related work to do. This is usually written as
$\redex \permle \redseq$. This notion can be easily extended to derivations:
$\redseqtwo \permle \redseq$ if $\redseqtwo / \redseq = \emptyDerivation$.

Another (weaker) notion asks that \emph{some} of the work that $\redex$ does be
done in $\redseq$. This is the notion of belonging.
A step $\redex$ \defn{belongs to} a coinitial derivation $\redseq$,
written $\redex \in \redseq$,
if and only if some residual of $\redex$ is contracted along $\redseq$. 
More precisely, $\redex \in \redseq$ if there exist $\redseq_1,\redex',\redseq_2$
such that $\redseq = \redseq_1\redex'\redseq_2$ and $\redex' \in \redex/\redseq_1$.
We write $\redex \not\in \redseq$ if it is not the case that $\redex \in \redseq$.

As it turns out, these two notions are equivalent in our calculus, because
there is no deletion nor duplication of redexes.

\begin{lemma}[Characterization of belonging]
\llem{characterization_of_belonging}
Let $\redex$ be a step and $\redseq$ a coinitial derivation
in the distributive lambda-calculus.
Then the following are equivalent:
\begin{enumerate}
\item $\redex \in \redseq$,
\item $\redex \permle \redseq$,
\item $\name(\redex) \in \names(\redseq)$.
\end{enumerate}
{\em Note:} the hypothesis that $\redex$ and $\redseq$ are coinitial is crucial.
In particular, $(1)$ and $(2)$ by definition only hold when $\redex$ and $\redseq$ are coinitial,
while $(3)$ might hold even if $\redex$ and $\redseq$ are not coinitial.
\end{lemma}
\begin{proof}
$(1 \Rightarrow 2)$
  Let $\redseq = \redseq_1\redextwo\redseqtwo_2$ where $\redextwo$ is a residual of $\redex$.
  Suppose moreover, without loss of generality,
  that $\redseq_1$ is minimal, \ie that $\redex \not\in \redseq_1$.
  By Permanence (\rlem{redex_permanence}) $\redex/\redseq_1$ is a singleton,
  so $\redex/\redseq_1 = \redextwo$.
  This means that $\redex/\redseq_1\redextwo\redseq_2 = \emptyset$,
  so indeed $\redex \permle \redseq_1\redextwo\redseq_2$.

$(2 \Rightarrow 3)$
  By induction on $\redseq$.
  If $\redseq$ is empty, the implication is vacuously true,
  so let $\redseq = \redextwo\redseqtwo$
  and consider two subcases, depending on whether $\redex = \redextwo$.
  If $\redex = \redextwo$, then indeed the first step of $\redseq = \redex\redseqtwo$
  has the same name as $\redex$.
  On the other hand, if $\redex \neq \redextwo$, 
  then by \rlem{cardinality_of_the_set_of_residuals}
  we have that $\redex/\redextwo = \redex'$,
  where $\name(\redex) = \name(\redex')$.
  Note that $\redex \permle \redextwo\redseqtwo$ 
  so $\redex/\redextwo = \redex' \permle \redseqtwo$.
  By applying the \ih we obtain that there must be a step in $\redseqtwo$
  whose name is $\name(\redex) = \name(\redex')$, and we are done.

$(3 \Rightarrow 1)$
  By induction on $\redseq$.
  If $\redseq$ is empty, the implication is vacuously true,
  so let $\redseq = \redextwo\redseqtwo$
  and consider two subcases, depending on whether $\name(\redex) = \name(\redextwo)$.
  If $\name(\redex) = \name(\redextwo)$ then $\redex$ and $\redextwo$ must be the same step,
  since terms are correct, which means that labels decorating lambdas are pairwise distinct.
  Hence $\redex \in \redseq = \redex\redseqtwo$.
  On the other hand, if $\name(\redex) \neq \name(\redextwo)$, then
  $\redex \neq \redextwo$ so by \rlem{cardinality_of_the_set_of_residuals}
  we have that $\redex/\redextwo = \redex'$,
  where $\name(\redex) = \name(\redex')$.
  By hypothesis, there is a step in the derivation $\redseq = \redextwo\redseqtwo$ whose name
  is $\name(\redex)$, and it is not $\redextwo$, so there must be at least one step in the
  derivation $\redseqtwo$ whose name is is $\name(\redex) = \name(\redex')$.
  By \ih $\redex' \in \redseqtwo$
  and then, since $\redex'$ is a residual of $\redex$,
  we conclude that $\redex \in \redextwo\redseqtwo$,
  as required.
\end{proof}

\bigskip

\subsection*{Name of a reduction}

In this subsection we will see why the naming scheme we proposed for steps and
reductions is useful. Recall that the names of a reduction are the set of names of the steps that form it.
What follows is a list of different results that characterize properties
of reductions in terms of its name.

It is immediate to note that
when composing two derivations $\redseq, \redseqtwo$,
the set of names of $\redseq\redseqtwo$
results from the union of the names of $\redseq$ and $\redseqtwo$:

\begin{remark}
\lremark{names_concatenation_union}
$\names(\redseq\redseqtwo) = \names(\redseq) \cup \names(\redseqtwo)$
\end{remark}

Indeed, a stronger results holds. Namely, the union is disjoint:

\begin{lemma}
\llem{names_concatenation_disjoint_union}
$\names(\redseq\redseqtwo) = \names(\redseq) \uplus \names(\redseqtwo)$
\end{lemma}
\begin{proof}
Let $\redseq : \tm \rtodist \tmtwo$.
By induction on $\redseq$ we argue that the set of labels decorating the lambdas in $\tmtwo$
is disjoint from $\names(\redseq)$.
The base case is immediate, so let $\redseq = \redexthree\redseqthree$.
The step $\redexthree$ is of the form:
\[
  \redexthree \HS:\HS
  \tm =
  \conof{(\lamp{\lab}{\var}{\tmthree})\ls{\tmfour}}
  \todist
  \conof{\subs{\tmthree}{\var}{\ls{\tmfour}}}
\]
hence the target of $\redexthree$ has no lambdas decorated with the label $\lab$.
Moreover, the derivation $\redseqthree$ is of the form:
\[
  \redseqthree \HS:\HS
  \conof{\subs{\tmthree}{\var}{\ls{\tmfour}}}
  \rtodist
  \tmtwo
\]
and by \ih the set of labels decorating the lambdas in $\tmtwo$
is disjoint from $\names(\redseqthree)$.
As a consequence,
the set of labels decorating the lambdas in $\tmtwo$
is disjoint from both $\names(\redseqthree)$
and $\set{\lab}$.
This completes the proof.
\end{proof}

\begin{corollary}[The length of a derivation equals the number of distinct names along it]
\lcoro{length_of_derivation_is_number_of_distinct_names}
If $\redseq$ is a derivation in the distributive lambda-calculus
and $\lengthof{\redseq}$ denotes the length of $\redseq$,
then
$\lengthof{\redseq} = \#(\names(\redseq))$.
\end{corollary}
\begin{proof}
Let $\redseq = \redex_1\hdots\redex_n$.
Then by \rlem{names_concatenation_disjoint_union},
$\names(\redseq) = \biguplus_{i=1}^{n} \set{\name(\redex_i)}$
and
$\#(\names(\redseq)) = n$, as required.
\end{proof}

The names of a derivation are coherent and work well
with all the usual residual theory definitions: projections, prefix order, and permutation
equivalence.

\begin{lemma}[Names after a projection]
\llem{names_after_projection_along_a_step}
If $\redseq$ and $\redseqtwo$ are coinitial derivations, then
$\names(\redseq/\redseqtwo) = \names(\redseq) \setminus \names(\redseqtwo)$
\end{lemma}
\begin{proof}
First we claim that $\names(\redseq/\redex) = \names(\redseq) \setminus \set{\name(\redex)}$.
We proceed by induction on $\redseq$.
The base case is immediate, so let $\redseq = \redextwo\redseqtwo$
and consider two subcases, depending on whether $\redex = \redextwo$.
If $\redex = \redextwo$,
then:
\[
  \begin{array}{rcl}
  \names(\redseq/\redex) & = & \names(\redex\redseqtwo/\redex) \\
                         & = & \names(\redseqtwo) \\
                         & = & \names(\redex\redseqtwo) \setminus \set{\name(\redex)} \\
  \end{array}
\]
On the other hand if $\redex \neq \redextwo$,
then by \rlem{cardinality_of_the_set_of_residuals}
we have that $\redex/\redextwo = \redex'$
where $\name(\redex) = \name(\redex')$
and, similarly,
$\redextwo/\redex = \redextwo'$,
where $\name(\redextwo) = \name(\redextwo')$. Hence:
\[
  \begin{array}{rcll}
  \names(\redseq/\redex) & = & \names(\redextwo\redseqtwo/\redex) \\
                         & = & \names((\redextwo/\redex)(\redseqtwo/(\redex/\redextwo))) \\
                         & = & \names(\redextwo/\redex) \cup \names(\redseqtwo/(\redex/\redextwo)) \\
                         & = & \names(\redextwo') \cup \names(\redseqtwo/\redex') \\
                         & = & \names(\redextwo') \cup (\names(\redseqtwo) \setminus \set{\name(\redex')} & \text{ by \ih} \\
                         & = & \names(\redextwo) \cup (\names(\redseqtwo) \setminus \set{\name(\redex)}) \\
                         & = & (\names(\redextwo) \cup \names(\redseqtwo)) \setminus \set{\name(\redex)} & \text{since $\name(\redex) \neq \name(\redextwo)$} \\
                         & = & \names(\redextwo\redseqtwo) \setminus \set{\name(\redex)} \\
                         & = & \names(\redseq) \setminus \set{\name(\redex)} \\
  \end{array}
\]
which completes the claim.
To see that $\names(\redseq/\redseqtwo) = \names(\redseq) \setminus \names(\redseqtwo)$
for an arbitrary derivation $\redseqtwo$,
proceed by induction on $\redseqtwo$.
If $\redseqtwo$ is empty it is trivial, so
consider the case in which $\redseqtwo = \redex\redseqthree$.
Then:
\[
  \begin{array}{rcll}
  \names(\redseq/\redex\redseqthree) & = & \names((\redseq/\redex)/\redseqthree) \\
                                     & = & \names(\redseq/\redex) \setminus \names(\redseqthree) & \text{ by \ih} \\
                                     & = & (\names(\redseq) \setminus \set{\name(\redex)}) \setminus \names(\redseqthree) & \text{ by the previous claim} \\
                                     & = & \names(\redseq) \setminus (\set{\name(\redex)} \cup \names(\redseqthree)) \\
                                     & = & \names(\redseq) \setminus \names(\redex\redseqthree) \\
  \end{array}
\]
as required.
\end{proof}

\begin{proposition}[Prefixes as subsets]
\lprop{prefixes_as_subsets}
Let $\redseq,\redseqtwo$ be coinitial derivations in the distributive lambda-calculus.
Then
$\redseq \permle \redseqtwo$ if and only if $\names(\redseq) \subseteq \names(\redseqtwo)$.
\end{proposition}
\begin{proof}
By induction on $\redseq$.
The base case is immediate since $\emptyDerivation \permle \redseqtwo$
and $\emptyset \subseteq \names(\redseqtwo)$ both hold.
So let $\redseq = \redexthree\redseqthree$. First note that
the following equivalence holds:
\begin{equation}
\leqn{prefixes_as_subsets:1}
    \redexthree\redseqthree \permle \redseqtwo
    \HS\iff\HS
    \redexthree \permle \redseqtwo \ \land\ \redseqthree \permle \redseqtwo/\redexthree
\end{equation}
Indeed:
\begin{itemize}
\item $(\Rightarrow)$
    Suppose that $\redexthree\redseqthree \permle \redseqtwo$.
    Then, on one hand, $\redexthree \permle \redexthree\redseqthree \permle \redseqtwo$.
    On the other hand, projection is monotonic, so
    $\redseqthree = \redexthree\redseqthree/\redexthree \permle \redseqtwo/\redexthree$.
\item $(\Leftarrow)$
    Since $\redseqthree \permle \redseqtwo/\redexthree$
    we have that $\redexthree\redseqthree \permle \redexthree(\redseqtwo/\redexthree) \permeq \redseqtwo(\redexthree/\redseqtwo) = \redseqtwo$
    since $\redexthree/\redseqtwo = \emptyDerivation$.
\end{itemize}
So we have that:
\[
  \begin{array}{rcll}
    \redexthree\redseqthree \permle \redseqtwo
    & \iff &
    \redexthree \permle \redseqtwo \ \land\ \redseqthree \permle \redseqtwo/\redexthree
      & \text{ by \reqn{prefixes_as_subsets:1}} \\
    & \iff &
    \name(\redexthree) \in \names(\redseqtwo) \ \land\ \redseqthree \permle \redseqtwo/\redexthree
      & \text{ by \rlem{characterization_of_belonging}} \\
    & \iff &
    \name(\redexthree) \in \names(\redseqtwo) \ \land\ \names(\redseqthree) \subseteq \names(\redseqtwo/\redexthree)
      & \text{ by \ih} \\
    & \iff &
    \name(\redexthree) \in \names(\redseqtwo) \ \land\ \names(\redseqthree) \subseteq \names(\redseqtwo) \setminus \set{\name(\redexthree)}
      & \text{ by \rlem{names_after_projection_along_a_step}} \\
    & \iff &
    \names(\redexthree\redseqthree) \subseteq \names(\redseqtwo)
  \end{array}
\]
To justify the very last equivalence, the $(\Rightarrow)$ direction is immediate.
For the $(\Leftarrow)$ direction,
the difficulty is ensuring that
$\names(\redseqthree) \subseteq \names(\redseqtwo) \setminus \set{\name(\redexthree)}$
from the fact that
$\names(\redexthree\redseqthree) \subseteq \names(\redseqtwo)$.
To see this it suffices to observe that by \rlem{names_concatenation_disjoint_union},
$\names(\redexthree\redseqthree)$
is the {\em disjoint} union
$\names(\redexthree) \uplus \names(\redseqthree)$,
which means that
$\name(\redexthree) \not\in \names(\redseqthree)$.

\end{proof}

\begin{corollary}[Permutation equivalence in terms of names]
\lcoro{permutation_equivalence_in_terms_of_names}
Let $\redseq,\redseqtwo$ be coinitial derivations in the distributive lambda-calculus.
Then
$\redseq \permeq \redseqtwo$ if and only if $\names(\redseq) = \names(\redseqtwo)$.
\end{corollary}
\begin{proof}
Immediate since
\[
  \begin{array}{rcll}
  \redseq \permeq \redseqtwo & \iff & \redseq \permle \redseqtwo \ \land\ \redseqtwo \permle \redseq \\
                             & \iff & \names(\redseq) \subseteq \names(\redseqtwo) \ \land\ \names(\redseqtwo) \subseteq \names(\redseq)
                                    & \text{ by \rprop{prefixes_as_subsets}}\\
                             & \iff & \names(\redseq) = \names(\redseqtwo)
  \end{array}
\]
\end{proof}


\section{Stability (and creation)}


In this section we will see two important lemmas that guarantee certain properties
of residues after steps or derivations.

The result that we will call full stability can be seen as a functorial extension
to the confluence result: confluence speaks about what happens to terms after
two coinitial derivations,
while full stability speaks about what happens to steps after two coinitial derivations.

\subsection*{Creation}

The first of this lemmas is creation.
We say that a step $\redex$ is \defn{created} by another step $\redextwo$ if
there is no step $\redex'$ coinitial with $\redextwo$ such that $\redex \in \redex' / \redextwo$.
What this means is that we were not able to do what $\redex$ does before executing $\redextwo$.
For example, in \rexample{derivation-ids-ex} we claimed
that the step $\redexthree$ is created by $\redex$, which indeed is.

Having a lemma that characterizes creation is relevant because in proofs,
it is common to have edge cases where a new redex is created.
A creation lemma can help handle those edge cases more easily.

The following lemma says that if $\redextwo$ creates $\redex$, then $\redextwo$ follows
one of three general forms.


\begin{lemma}[Creation]
\llem{creation}
There are three creation cases in the distributive lambda-calculus:
\begin{enumerate}
\item {\bf Creation case I.}
  $
    \con\of{ (\lamp{\lab}{\var}{\var^{\typ}})\,[\lamp{\labtwo}{\vartwo}{\tm}]\, \ls{\tmtwo}}
    \ \todist\ %
    \con\of{ (\lamp{\labtwo}{\vartwo}{\tm})\,\ls{\tmtwo} }
    \ \todist\ %
    \con\of{ \tm\sub{\vartwo}{\ls{\tmtwo}} }
  $.
\item {\bf Creation case II.}
  $
     \con\of{ (\lamp{\lab}{\var}{ \lamp{\labtwo}{\vartwo}{\tm} })\,\ls{\tmtwo}\,\ls{\tmthree} }
     \ \todist\ %
     \con\of{ (\lamp{\labtwo}{\vartwo}{\tm'})\,\ls{\tmthree} }
     \ \todist\ %
     \con\of{ \tm'\sub{\vartwo}{\ls{\tmthree}} }
  $, where:
  \[
     \begin{array}{rcl}
     \tm' & = & \tm\sub{\var}{\ls{\tmtwo}}
     \end{array}
  \]
\item {\bf Creation case III.}
  $
    \con_1\of{(\lamp{\lab}{\var}{\con_2\of{\var^{\typ}\,\ls{\tm}}})\ls{\tmtwo}}
    \ \todist\ %
    \con_1\of{\con'_2\of{(\lamp{\labtwo}{\vartwo}{\tmthree})\,\ls{\tm'}}}
    \ \todist\ %
    \con_1\of{\con'_2\of{\tmthree\sub{\vartwo}{\ls{\tm'}}}}
  $,
  where:
  \[
  \begin{array}{rcl}
    \con_2\sub{\var}{\ls{\tmtwo}}      & = & \con'_2 \\
    \var^{\typ}\sub{\var}{\ls{\tmtwo}} & = & \lamp{\labtwo}{\vartwo}{\tmthree} \\
    \ls{\tm}\sub{\var}{\ls{\tmtwo}}    & = & \ls{\tm'} \\
    \ls{\tmtwo}                        & = & [\ls{\tmtwo_1},\lamp{\labtwo}{\vartwo}{\tmthree},\ls{\tmtwo_2}] \\
  \end{array}
  \]
\end{enumerate}
\end{lemma}
\begin{proof}
Let $\redex : \conof{(\lamp{\lab}{\var}{\tm})\ls{\tmtwo}} \todist \conof{\tm\sub{\var}{\ls{\tmtwo}}}$ be a step,
and let $\redextwo : \conof{\tm\sub{\var}{\ls{\tmtwo}}} \todist \tmfive$ another step
such that $\redex$ creates $\redextwo$.
The redex contracted by the step $\redextwo$ is below a context $\con_1$,
so let $\conof{\tm\sub{\var}{\ls{\tmtwo}}} = \con_1\of{(\lamp{\labtwo}{\vartwo}{\tmthree})\ls{\tmfour}}$,
where $(\lamp{\labtwo}{\vartwo}{\tmthree})\ls{\tmfour}$ is the redex contracted by $\redextwo$.
We need consider three cases, depending on the relative positions of the holes of $\con$ and $\con_1$,
namely they may be disjoint,
$\con$ may be a prefix of $\con_1$,
or $\con_1$ may be a prefix of $\con$.
\SeeAppendixRef{creation_proof}
\end{proof}

\bigskip
\subsection*{Stability}

We also would like to prove a property that we call \emph{full stability},
which is a strong version of stability in the sense of Lévy \cite{levy_redex_stability}
(which can in turn be traced back to Berry's notion of stability).

Recall that L\'evy's stability states that for any $\redex \neq \redextwo$,
if $\redexthree_1$ and $\redexthree_2$ have a common descendant $\redexthree_3$, as in the figure below,
then they have a common ancestor $\redexthree_0$.
\[
  \xymatrix@C=.5cm@R=.5cm{
    &&&
  \\
    &
    & \ar@{.>}[u]^{\redexthree_0} \ar[dl]_{\redex} \ar[dr]^{\redextwo} &
  \\
    &
    \ar[l]_{\redexthree_1}
    \ar[dr]_{\redextwo/\redex} & & \ar[dl]^{\redex/\redextwo} \ar[r]^{\redexthree_2} &
  \\
    &
    & \ar[d]_{\redexthree_3} &
  \\
    &&&
  }
\]

In the figure above, $\redexthree_1$ and $\redexthree_2$ have a common descendant
$\redexthree_3$, which means that $\redexthree_3 \in \redexthree_1 / (\redextwo / \redex)$
and $\redexthree_3 \in \redexthree_2 / (\redex / \redextwo)$.
Informally, this means that $\redexthree_1$ and $\redexthree_2$ do the same work,
which means that that work is enabled by either performing $\redex$ or $\redextwo$.
The fact that a system is stable means that we could actually perform the work that
$\redexthree_1$ and $\redexthree_2$ do, without executing neither $\redex$ nor $\redextwo$
(by doing $\redexthree_0$).

Hence, what stability means is that steps are created essentially in a unique way.


\begin{lemma}[Stability]
\llem{basic_stability}
If $\redex,\redextwo$ are different coinitial steps such that
$\redex$ creates a step $\redexthree$,
then $\redex/\redextwo$ creates the step $\redexthree/(\redextwo/\redex)$.

More specifically:
\[
  \xymatrix@R=.5cm@C=2cm{
    \ar[r]^{\redex}
    \ar[d]_{\redextwo}
    &
    \ar[r]^{\redexthree}
    \ar[d]_{\redextwo/\redex}
    &
    \ar[d]_{\redextwo/\redex\redexthree}
  \\
    \ar[r]_{\redex/\redextwo}
    &
    \ar[r]_{\redexthree/(\redextwo/\redex)}
    &
  }
\]
\emph{Note:} It is easy to see that this is equivalent to stability in the sense of Lévy.
\end{lemma}
\begin{proof}
\SeeAppendixRef{basic_stability_proof}
Let $\redex : \conof{(\lamp{\lab}{\var}{\tm})\ls{\tmtwo}} \todist \conof{\subs{\tm}{\var}{\ls{\tmtwo}}}$,
let $\redextwo \neq \redex$ be a step coinitial to $\redex$,
and suppose that $\redex$ creates a step $\redexthree$.
By induction on the context $\con$ we argue that $\redex/\redextwo$ creates $\redexthree/(\redextwo/\redex)$.
\end{proof}

Next we prove \emph{full stability}, which is an extension to Lévy's notion of
stability, from steps to arbitrary reductions. Informally, it says that if we are able
to execute the same step after two completely different derivations,
then the step existed before
both reductions---\ie, a step is only created by a single reduction (modulo permutation
equivalence).

\begin{lemma}[Full stability]
\llem{full_stability}
Let $\redseq$ and $\redseqtwo$ be coinitial derivations such that
$\names(\redseq) \cap \names(\redseqtwo) = \emptyset$.
Let $\redexthree_1$, $\redexthree_2$, and $\redexthree_3$
be steps such that
$\redexthree_3 = \redexthree_1/(\redseqtwo/\redseq) = \redexthree_2/(\redseq/\redseqtwo)$.
Then there exists a step $\redexthree_0$ such that
$\redexthree_1 = \redexthree_0/\redseq$ and $\redexthree_2 = \redexthree_0/\redseqtwo$.
Diagrammatically:
\[
  \xymatrix@R=.5cm@C=.5cm{
    &&&
  \\
    &
    & \ar@{.>}[u]^{\redexthree_0} \ar@{->>}[ld]_{\redseq} \ar@{->>}[rd]^{\redseqtwo} &
  \\
    &
    \ar[l]_{\redexthree_1}
    \ar@{->>}[rd]_{\redseqtwo/\redseq} & & \ar@{->>}[ld]^{\redseq/\redseqtwo} \ar[r]^{\redexthree_2}
    &
  \\
    &
    &
    \ar[d]_{\redexthree_3}
    &
  \\
    &&&
  }
\]
\end{lemma}
\begin{proof}
We first prove the proposition in the particular case in which $\redseq$ is a single step, \ie $\redseq = \redex$
By induction on $\redseqtwo$:
\begin{enumerate}
\item {\bf Empty, $\redseqtwo = \emptyDerivation$.}
  Then $\redexthree_1 = \redexthree_3 = \redexthree_2/\redex$,
  so it suffices to take $\redexthree_0 := \redexthree_2$.
\item {\bf Non-empty, $\redseqtwo = \redextwo\,\redseqtwo'$.}
  Recall that permutation diagrams in the distributive lambda-calculus are {\em square},
  (\ie steps always have exactly one residual, except for the trivial case $\redex/\redex = \emptyset$,
  as was proved in \rlem{cardinality_of_the_set_of_residuals}).
  Since $\name(\redex) \not\in \names(\redextwo\redseqtwo')$,
  we know that $\redex$ is not any of the steps along $\redextwo\redseqtwo'$,
  so in particular $\redex/\redextwo$ and $\redex/\redextwo\redseqtwo'$
  are singletons, which means that the situation is the following:
  \[
    \xymatrix{
      &
      &
        \ar[ld]_{\redex} \ar[rd]^{\redextwo}
      &
    \\
      &
      \ar[l]_{\redexthree_1}
      \ar[rd]_{\redextwo/\redex}
      &
      &
        \ar[ld]_-{\redex/\redextwo} \ar@{->>}[rd]^{\redseqtwo'}
      & 
      &
    \\
      &
      &
        \ar@{->>}[rd]_-{\redseqtwo'/(\redex/\redextwo)}
      &
      &
        \ar@{->}[dl]^{\redex/\redextwo\redseqtwo'}
        \ar[r]^{\redexthree_2}
      &
    \\
      &
      &
      &
      \ar[d]_{\redexthree_3}
      &
    \\
      &
      &
      &
    }
  \]
  Observe that $\redexthree_3 = \redexthree_1/((\redextwo/\redex)(\redseqtwo'/(\redex/\redextwo)))$,
  so by taking $\redexthree'_1 := \redexthree_1/(\redextwo/\redex)$
  we have that $\redexthree_3 = \redexthree'_1/(\redseqtwo'/(\redex/\redextwo))$.
  By \ih on $\redseqtwo'$ we have that
  there exists a step $\redexthree'_2$ such that
  $\redexthree'_1 = \redexthree'_2/(\redex/\redextwo)$ and
  $\redexthree_2 = \redexthree'_2/\redseqtwo'$.

  To conclude, note that $\redex \neq \redextwo$ and
  $\redexthree'_1 = \redexthree_1/(\redextwo/\redex) = \redexthree_0/(\redex/\redextwo)$
  so by the Stability~lemma (\rlem{basic_stability})
  there must exist a step $\redexthree_0$ such that
  $\redexthree_1 = \redexthree_0/\redex$
  and $\redexthree'_2 = \redexthree_0/\redextwo$.
  Moreover $\redexthree_2 = \redexthree'_2/\redseqtwo' = \redexthree_0/\redextwo\redseqtwo'$,
  as required.
\end{enumerate}
Having established the previous claim, let us now prove the main statement of the
proposition by induction on $\redseq$.
\begin{enumerate}
\item {\bf Empty, $\redseq = \emptyDerivation$.}
  Then $\redexthree_2 = \redexthree_3 = \redexthree_1/\redseqtwo$ so by taking $\redexthree_0 := \redexthree_1$
  we conclude.
\item {\bf Non-empty, $\redseq = \redex\redseq'$.}
  First observe that, since $\names(\redex\redseq') \cap \names(\redseqtwo) = \emptyset$,
  we have that $\names(\redex) \cap \names(\redseqtwo) = \emptyset$ 
  and $\names(\redseq') \cap \names(\redseqtwo) = \emptyset$.
  The situation is the following:
  \[
    \xymatrix{
      &
      &
      &
        \ar[dl]_{\redex}
        \ar[dr]^{\redseqtwo}
      &
    \\
      &
      &
        \ar@{->>}[dl]_{\redseq'}
        \ar@{->>}[dr]^{\redseqtwo/\redex}
      &
      &
        \ar[dl]^{\redex/\redseqtwo}
        \ar[r]^{\redexthree_2}
      &
    \\
      &
        \ar[l]_{\redexthree_1}
        \ar@{->>}[dr]^{\redseqtwo/\redex\redseq'}
      &
      &
        \ar@{->>}[dl]^{\redseq'/(\redseqtwo/\redex)}
      &
    \\
      &
      &
        \ar[d]_{\redexthree_3}
      &
      &
    \\
      &
      &
      &
      &
    \\
    }
  \]
  Observe that $\redexthree_3 = \redexthree_2/((\redex/\redseqtwo)(\redseq'/(\redseqtwo/\redex)))$,
  so by taking $\redexthree'_2 := \redexthree_2/(\redex/\redseqtwo)$
  we have that $\redexthree_3 = \redexthree'_2/(\redseq'/(\redseqtwo/\redex))$.
  By \rlem{names_after_projection_along_a_step}
  we have that $\names(\redseqtwo/\redex) = \names(\redseqtwo) \setminus \set{\name(\redex)}$
  and $\name(\redex) \not\in \names(\redseqtwo)$,
  so $\names(\redseqtwo/\redex) = \names(\redseqtwo)$.
  In particular, $\names(\redseq') \cap \names(\redseqtwo/\redex) = \emptyset$
  so we may apply the \ih on $\redseq'$ to conclude that there exists
  a step $\redexthree'_1$ such that
  $\redexthree_1 = \redexthree'_1/\redseq'$
  and
  $\redexthree'_2 = \redexthree'_1/(\redseqtwo/\redex)$.

  To conclude, observe that $\redexthree'_2 = \redexthree'_1/(\redseqtwo/\redex) = \redexthree_2/(\redex/\redseqtwo)$,
  where $\name(\redex) \not\in \names(\redseqtwo)$,
  so by the previous claim we have that
  there is a step $\redexthree_0$ such that
  $\redexthree_0/\redex = \redexthree'_1$ and
  $\redexthree_0/\redseqtwo = \redexthree_2$.
  Moreover, $\redexthree_0/\redex\redseq' = \redexthree'_1/\redex = \redexthree_1$ 
  so we are done.
\end{enumerate}
\end{proof}


\section{Lattices and derivation spaces}

\begin{lemma}[Projections are decreasing]
\llem{projections_are_decreasing}
Let $\redex \in \redseq$. Then $\lengthof{\redseq} = 1 + \lengthof{\redseq/\redex}$.
\end{lemma}
\begin{proof}
Observe that $\redex \permle \redseq$ by \rlem{characterization_of_belonging}.
So $\redseq \permeq \redex(\redseq/\redex)$, which gives us that:
\[
  \begin{array}{rcll}
  \lengthof{\redseq} & = & \#\names(\redseq) & \text{ by \rcoro{length_of_derivation_is_number_of_distinct_names}} \\
                     & = & \#\names(\redex(\redseq/\redex)) & \text{ by \rcoro{permutation_equivalence_in_terms_of_names}, since $\redseq \permeq \redex(\redseq/\redex)$} \\
                     & = & \#(\names(\redex) \uplus \names(\redseq/\redex) & \text{ by \rlem{names_concatenation_disjoint_union}} \\
                     & = & 1 + \#\names(\redseq/\redex) \\
                     & = & 1 + \lengthof{\redseq/\redex} & \text{ by \rcoro{length_of_derivation_is_number_of_distinct_names}} \\
  \end{array}
\]
\end{proof}

\begin{proposition}[Meet of derivations]
\lprop{meet_of_derivations}
Let $\redseq,\redseqtwo$ be coinitial derivations in the distributive lambda-calculus.
Then there exists an infimum for $\redseq,\redseqtwo$ with respect to the prefix order $\permle$.
We write $\redseq \sqcap \redseqtwo$ for the infimum of $\set{\redseq,\redseqtwo}$ obtained by this
construction.
\end{proposition}
\begin{proof}
If $\redseq$ and $\redseqtwo$ are derivations, we say that a step $\redex$
is a \defn{common} (to $\redseq$ and $\redseqtwo$) whenever $\redex \in \redseq$ and $\redex \in \redseqtwo$.
Define $\redseq \sqcap \redseqtwo$ as follows,
by induction on the length of $\redseq$:
\[
  \redseq \sqcap \redseqtwo \eqdef
    \begin{cases}
    \emptyDerivation & \text{if there are no common steps to $\redseq$ and $\redseqtwo$} \\
    \redex((\redseq/\redex) \sqcap (\redseqtwo/\redex)) & \text{if the step $\redex$ is common to $\redseq$ and $\redseqtwo$ } \\
    \end{cases}
\]
In the second case of the definition, there might be more than one $\redex$
common to $\redseq$ and $\redseqtwo$.
We suppose that one of them is chosen deterministically but make no further assumptions.
To see that this recursive construction is well-defined,
note that the length of $\redseq/\redex$ is lesser than the length of $\redseq$
by the fact that projections are decreasing (\rlem{projections_are_decreasing}).
To conclude the construction, we show that $\redseq \sqcap \redseqtwo$ is an infimum, \ie a greatest lower bound:
\begin{enumerate}
\item {\bf Lower bound.}
  Let us show that $\redseq \sqcap \redseqtwo \permle \redseq$
  by induction on the length of $\redseq$.
  There are two subcases, depending on whether
  there is a step common to $\redseq$ and $\redseqtwo$.

  If there is no common step,
  then $\redseq \sqcap \redseqtwo = \emptyDerivation$
  trivially verifies $\redseq \sqcap \redseqtwo \permle \redseq$.

  On the other hand, if there is a common step, we have by definition that
  $\redseq \sqcap \redseqtwo = \redex((\redseq/\redex) \sqcap (\redseqtwo/\redex))$
  where $\redex$ is common to $\redseq$ and $\redseqtwo$.
  Recall that projections are decreasing (\rlem{projections_are_decreasing})
  so $\lengthof{\redseq} > \lengthof{\redseq/\redex}$.
  This allows us to apply the \ih and conclude:
  \[
  \begin{array}{rcll}
    \redseq \sqcap \redseqtwo & =       & \redex((\redseq/\redex) \sqcap (\redseqtwo/\redex)) & \text{ by definition} \\
                              & \permle & \redex(\redseq/\redex)   & \text{ since by \ih $(\redseq/\redex) \sqcap (\redseqtwo/\redex) \permle \redseq/\redex$} \\
                              & \permeq & \redseq(\redex/\redseq)  \\
                              & =       & \redseq                  & \text{ since $\redex \permle \redseq$ by \rlem{characterization_of_belonging}.} \\
  \end{array}
  \]
  Showing that $\redseq \sqcap \redseqtwo \permle \redseqtwo$ is symmetric,
  by induction on the length of $\redseqtwo$.
\item {\bf Greatest lower bound.}
  Let $\redseqthree$ be a lower bound for $\set{\redseq, \redseqtwo}$,
  \ie $\redseqthree \permle \redseq$ and $\redseqthree \permle \redseqtwo$,
  and let us show that $\redseqthree \permle \redseq \sqcap \redseqtwo$.
  We proceed by induction on the length of $\redseq$.
  There are two subcases, depending on whether
  there is a step common to $\redseq$ and $\redseqtwo$.

  If there is no common step,
  we claim that $\redseqthree$ must be empty.
  Otherwise we would have that $\redseqthree = \redexthree\redseqthree' \permle \redseq$
  so in particular $\redexthree \permle \redseq$ and $\redexthree \in \redseq$ by \rlem{characterization_of_belonging}.
  Similarly, $\redexthree \in \redseqtwo$ so
  $\redexthree$ is a step common to $\redseq$ and $\redseqtwo$,
  which is a contradiction.
  We obtain that $\redseqthree$ is empty,
  so trivially $\redseqthree = \emptyDerivation \permle \redseq \sqcap \redseqtwo$.

  On the other hand, if there is a common step, we have by definition that
  $\redseq \sqcap \redseqtwo = \redex((\redseq/\redex) \sqcap (\redseqtwo/\redex))$
  where $\redex$ is common to $\redseq$ and $\redseqtwo$.
  Moreover, since $\redseqthree \permle \redseq$ and $\redseqthree \permle \redseqtwo$,
  by projecting along $\redex$ we know that
  $\redseqthree/\redex \permle \redseq/\redex$ and $\redseqthree/\redex \permle \redseqtwo/\redex$.
  So:
  \[
    \begin{array}{rcll}
      \redseqthree & \permle & \redseqthree(\redex/\redseqthree) \\
                   & \permeq & \redex(\redseqthree/\redex) \\
                   & \permle & \redex((\redseq/\redex) \sqcap (\redseqtwo/\redex)) & \text{ since by \ih $\redseqthree/\redex \permle (\redseq/\redex) \sqcap (\redseqtwo/\redex)$} \\
                   & =       & \redseq \sqcap \redseqtwo & \text{ by definition}
    \end{array}
  \]
\end{enumerate}
\end{proof}

\begin{remark}
The infimum of $\set{\redseq,\redseqtwo}$ is unique modulo permutation equivalence,
\ie if $\redseqthree$ is an infimum for $\set{\redseq,\redseqtwo}$ then $\redseqthree \permeq \redseq \sqcap \redseqtwo$.
\end{remark}



\begin{lemma}[Properties of disjoint derivations]
\llem{properties_of_disjoint_derivations}
Let $\redseq,\redseqtwo$ be coinitial derivations.
Then the following are equivalent:
\begin{enumerate}
\item $\names(\redseq) \cap \names(\redseqtwo) = \emptyset$.
\item $\redseq \sqcap \redseqtwo = \emptyDerivation$.
\item There is no step common to $\redseq$ and $\redseqtwo$.
\end{enumerate}
In this case we say that $\redseq$ and $\redseqtwo$ are \defn{disjoint}.
\end{lemma}
\begin{proof}
The implication $(1 \implies 2)$ is immediate since if we suppose that $\redseq \sqcap \redseqtwo$ is non-empty
then the first step of $\redseq \sqcap \redseqtwo$ is a step $\redexthree$
such that $\redexthree \in \redseq$ and $\redexthree \in \redseqtwo$.
By Characterization of belonging~(\rlem{characterization_of_belonging}),
this means that $\name(\redexthree) \in \names(\redseq) \cap \names(\redseqtwo)$,
contradicting the fact that $\name(\redexthree)$ and $\names(\redseq)$ are disjoint.

The implication
$(2 \implies 3)$ is immediate by definition of $\redseq \sqcap \redseqtwo$.

Let us check that the implication $(3 \implies 1)$ holds.
By the contrapositive, suppose that $\names(\redseq)$ and $\names(\redseqtwo)$
are not disjoint, and let us show that there is a step common to $\redseq$ and $\redseqtwo$.
Since $\names(\redseq) \cap \names(\redseqtwo) \neq \emptyset$,
we know that the derivation $\redseq$ can be written as $\redseq = \redseq_1 \redex \redseq_2$
where $\name(\redex) \in \names(\redseqtwo)$.
Without loss of generality we may suppose
that $\redex$ is the first step in $\redseq$ with that property,
\ie that $\names(\redseq_1) \cap \names(\redseqtwo) = \emptyset$.
Moreover, let us write $\redseqtwo$ as $\redseqtwo = \redseqtwo_1 \redextwo \redseqtwo_2$ 
where $\name(\redex) = \name(\redextwo)$.

Observe that the name of $\redex$ does not appear anywhere along the sequence of steps $\redseq_1$,
\ie that $\name(\redex) \not\in \names(\redseq_1)$, as a consequence of the
fact that no names are ever repeated in any sequence of steps (\rlem{names_concatenation_disjoint_union}).
This implies that $\name(\redextwo) \not\in \names(\redseq_1/\redseqtwo_1)$.
Indeed:
\[
  \name(\redextwo)
  = \name(\redex)
  \not\in \names(\redseq_1)
  \supseteq \names(\redseq_1) \setminus \names(\redseqtwo_1)
  =^{(\text{\rlem{names_after_projection_along_a_step}})} \names(\redseq_1/\redseqtwo_1)
\]
This means that $\redextwo$ is not erased by the derivation $\redseq_1/\redseqtwo_1$.
More precisely, $\redextwo/(\redseq_1/\redseqtwo_1)$ is a singleton.

Symmetrically, $\redex/(\redseqtwo_1/\redseq_1)$ is a singleton.
Moreover, $\name(\redextwo/(\redseq_1/\redseqtwo_1)) = \name(\redextwo) = \name(\redex) = \name(\redex/(\redseqtwo_1/\redseq_1))$ 
so we have that $\redextwo/(\redseq_1/\redseqtwo_1) = \redex/(\redseqtwo_1/\redseq_1)$.
The situation is the following, where $\names(\redseq_1) \cap \names(\redseqtwo_1) = \emptyset$:
\[
  \xymatrix{
    &
    &
    & \ar@{->>}[ld]_{\redseq_1} \ar@{->>}[rd]^{\redseqtwo_1} &
  \\
    &
    \ar@{->>}[l]_{\redseq_2}
    &
    \ar[l]_{\redex}
    \ar@{->>}[rd]_{\redseqtwo_1/\redseq_1} & & \ar@{->>}[ld]^{\redseq_1/\redseqtwo_1} \ar[r]^{\redextwo}
    &
    \ar@{->>}[r]^{\redseqtwo_2}
    &
  \\
    &
    &
    &
    \ar[d]_{\redex/(\redseqtwo_1/\redseq_1) = \redextwo/(\redseq_1/\redseqtwo_1)}
    &
  \\
    &
    &&&
  }
\]
By Full stability~(\rlem{full_stability}) this means that there exists a step $\redexthree$
such that $\redexthree/\redseq_1 = \redex$ and $\redexthree/\redseqtwo_1 = \redextwo$.
Then $\redexthree \in \redseq_1\redex\redseq_2 = \redseq$
and also $\redexthree \in \redseqtwo_1\redextwo\redseqtwo_2 = \redseqtwo$
so $\redexthree$ is common to $\redseq$ and $\redseqtwo$,
by which we conclude.
\end{proof}


\begin{lemma}
\llem{names_of_meet_included_in_names}
Let $\redseq$ and $\redseqtwo$ be coinitial derivations.
Then $\names(\redseq \sqcap \redseqtwo) \subseteq \names(\redseq)$.
\end{lemma}
\begin{proof}
By induction on the length of $\redseq \cap \redseqtwo$:
\begin{enumerate}
\item {\bf Empty, $\redseq \sqcap \redseqtwo = \emptyDerivation$.}
      Then $\names(\redseq \cap \redseqtwo) = \emptyset \subseteq \names(\redseq)$ is immediate.
\item {\bf Non-empty, $\redseq \sqcap \redseqtwo = \redexthree(\redseq/\redexthree \sqcap \redseqtwo/\redexthree)$,
           where $\redexthree$ is a step common to $\redseq$ and $\redseqtwo$.}
      Then since $\redexthree$ is common to $\redseq$ and $\redseqtwo$,
      we have that $\name(\redexthree) \in \names(\redseq)$.
      Moreover, by \ih $\names(\redseq/\redexthree \sqcap \redseqtwo/\redexthree) \subseteq \names(\redseq/\redexthree)$.
      So:
      \[
        \begin{array}{rcll}
        \names(\redseq \cap \redseqtwo) & = & \set{\name(\redexthree)} \cup \names(\redseq/\redexthree \sqcap \redseqtwo/\redexthree) \\
                                        & \subseteq & \names(\redseq) \cup \names(\redseq/\redexthree) \\
                                        & = & \names(\redseq) \cup (\names(\redseq) \setminus \set{\name(\redexthree)} & \text{by \rlem{names_after_projection_along_a_step}} \\
                                        & = & \names(\redseq)
        \end{array}
      \]
      as required.
\end{enumerate}
\end{proof}


\begin{proposition}[Names of join and meet]
\lprop{names_of_join_and_meet}
The following hold:
\begin{enumerate}
\item $\names(\redseq \sqcup \redseqtwo) = \names(\redseq) \cup \names(\redseqtwo)$
\item $\names(\redseq \sqcap \redseqtwo) = \names(\redseq) \cap \names(\redseqtwo)$
\end{enumerate}
\end{proposition}
\begin{proof}
Item 1 is easy resorting to the definition of $\sqcup$
and \rlem{names_after_projection_along_a_step}:
\[
  \begin{array}{rcll}
  \names(\redseq \sqcup \redseqtwo) & = & \names(\redseq(\redseqtwo/\redseq)) & \text{ by definition of $\sqcup$} \\
                                    & = & \names(\redseq) \cup \names(\redseqtwo/\redseq) \\
                                    & = & \names(\redseq) \cup (\names(\redseqtwo) \setminus \names(\redseq)) & \text{ by \rlem{names_after_projection_along_a_step}} \\
                                    & = & \names(\redseq) \cup \names(\redseqtwo) \\
  \end{array}
\]
For item 2., 
the inclusion $\names(\redseq \sqcap \redseqtwo) \subseteq \names(\redseq) \cap \names(\redseqtwo)$
is an immediate consequence of \rlem{names_of_meet_included_in_names}.
For other inclusion, namely to show  $\names(\redseq) \cap \names(\redseqtwo) \subseteq \names(\redseq \sqcap \redseqtwo)$,
we first prove the following claim:\medskip

{\bf Claim.} $\names(\redseq/(\redseq \sqcap \redseqtwo)) \cap \names(\redseqtwo/(\redseq \sqcap \redseqtwo) = \emptyset$.
{\em Proof of the claim.}
By \rlem{properties_of_disjoint_derivations} it suffices to show that
$(\redseq/(\redseq \sqcap \redseqtwo)) \sqcap (\redseqtwo/(\redseq \sqcap \redseqtwo)) = \emptyDerivation$.
By contradiction, suppose that there is a step $\redexthree$ common to
the derivations
$\redseq/(\redseq \sqcap \redseqtwo)$ and $\redseqtwo/(\redseq \sqcap \redseqtwo)$.
Then the derivation $(\redseq \sqcap \redseqtwo)\redexthree$ is a lower bound for $\set{\redseq,\redseqtwo}$,
\ie
$(\redseq \sqcap \redseqtwo)\redexthree \permle \redseq$
and
$(\redseq \sqcap \redseqtwo)\redexthree \permle \redseqtwo$.
Since $\redseq \sqcap \redseqtwo$ is the greatest lower bound for $\set{\redseq,\redseqtwo}$,
we have that $(\redseq \sqcap \redseqtwo)\redexthree \permle \redseq \sqcap \redseqtwo$.
But this implies that $\redexthree \permle \emptyDerivation$, which is a contradiction.
This concludes the proof of the claim.
\medskip

Note that $\redseq \sqcap \redseqtwo \permle \redseq$,
so we have that $\redseq \permeq (\redseq \sqcap \redseqtwo)(\redseq/(\redseq \sqcap \redseqtwo))$,
and this in turn implies that
$\names(\redseq) = \names((\redseq \sqcap \redseqtwo)(\redseq/(\redseq \sqcap \redseqtwo)))$
by \rcoro{permutation_equivalence_in_terms_of_names}.
Symmetrically,
$\names(\redseqtwo) = \names((\redseq \sqcap \redseqtwo)(\redseqtwo/(\redseq \sqcap \redseqtwo)))$.
Then:
\[
  \begin{array}{rcll}
    &&  \names(\redseq) \cap \names(\redseqtwo) \\
  & = & \names((\redseq \sqcap \redseqtwo)(\redseq/(\redseq \sqcap \redseqtwo))) \cap \names((\redseq \sqcap \redseqtwo)(\redseqtwo/(\redseq \sqcap \redseqtwo))) \\
  & = & \left(\names(\redseq \sqcap \redseqtwo) \cup \names(\redseq/(\redseq \sqcap \redseqtwo))\right) \cap \left(\names(\redseq \sqcap \redseqtwo) \cup \names(\redseqtwo/(\redseq \sqcap \redseqtwo))\right) \\
     && \text{ by \rremark{names_concatenation_union}} \\
  & = &
        \names(\redseq \sqcap \redseqtwo) \cup 
        \left( \names(\redseq/(\redseq \sqcap \redseqtwo)) \cap \names(\redseqtwo/(\redseq \sqcap \redseqtwo)) \right) \\
  && \text{ since $(A \cup B) \cap (A \cup C) = A \cup (B \cap C)$ for arbitrary sets $A,B,C$} \\
  & = &
        \names(\redseq \sqcap \redseqtwo) \\
  && \text{ since $\left( \names(\redseq/(\redseq \sqcap \redseqtwo)) \cap \names(\redseqtwo/(\redseq \sqcap \redseqtwo)) \right) = \emptyset$ by the previous claim} \\
  \end{array}
\]
This concludes the proof.
\end{proof}

\begin{theorem}[Derivations modulo $\permeq$ form a distributive lattice]
Let $\tm \in \termsdist$ be a correct term.
Then derivations modulo $\permeq$ form a lattice $\derivlattice{\tm}$.
More precisely, let $X$ be the set of derivations in the distributive lambda-calculus going out from $\tm$,
modulo permutation equivalence:
\[
  X \eqdef \frac{\set{\redseq \ST \src(\redseq) = \tm}}{\permeq}
\]
Let moreover $\classof{\redseq}$ denote the equivalence class of a derivation $\redseq$
modulo $\permeq$.
Then $\derivlattice{\tm} \eqdef (X, \leq, \land, \lor)$ is a distributive lattice, where:
\[
  \begin{array}{rcl}
  \classof{\redseq} \leq \classof{\redseqtwo}  & \iffdef & \redseq \permle \redseqtwo \\
  \classof{\redseq} \land \classof{\redseqtwo} & \iffdef & \redseq \sqcap \redseqtwo \\
  \classof{\redseq} \lor \classof{\redseqtwo}  & \iffdef & \redseq \sqcup \redseqtwo \\
  \end{array}
\]
\end{theorem}
\begin{proof}
It is straightforward to check that $\leq$ is a partial order,
and that $\classof{\redseq} \land \classof{\redseqtwo}$ (resp. $\classof{\redseq} \lor \classof{\redseqtwo}$)
is the infimum (resp. supremum) of $\set{\classof{\redseq},\classof{\redseqtwo}}$.

To see that it is distributive, let us first prove the first distributive law:
$(\classof{\redseq} \land \classof{\redseqtwo}) \lor \classof{\redseqthree} =
(\classof{\redseq} \lor \classof{\redseqthree}) \land (\classof{\redseqtwo} \lor \classof{\redseqthree})$.
Let $\redseq,\redseqtwo,\redseqthree$ be arbitrary coinitial derivations.
The following equality holds trivially, since
$(A \cap B) \cup C = (A \cup C) \cap (A \cup B)$ is valid for arbitrary sets $A, B, C$:
  \[
    (\names(\redseq) \cap \names(\redseqtwo)) \cup \names(\redseqthree)
    =
    (\names(\redseq) \cup \names(\redseqthree)) \cap (\names(\redseqtwo) \cup \names(\redseqthree))
  \]
By \rprop{names_of_join_and_meet} this entails:
  \[
    \names((\redseq \sqcap \redseqtwo) \sqcup \redseqthree)
    =
    \names((\redseq \sqcup \redseqthree) \sqcap (\redseqtwo \sqcup \redseqthree))
  \]
By \rcoro{permutation_equivalence_in_terms_of_names} this in turn implies that:
  \[
    (\redseq \sqcap \redseqtwo) \sqcup \redseqthree
    \permeq
    (\redseq \sqcup \redseqthree) \sqcap (\redseqtwo \sqcup \redseqthree)
  \]
So by definition of $\land,\lor$ we obtain:
  \[
    (\classof{\redseq} \land \classof{\redseqtwo}) \lor \classof{\redseqthree}
    =
    (\classof{\redseq} \lor \classof{\redseqthree}) \land (\classof{\redseqtwo} \lor \classof{\redseqthree})
  \]
The other distributive law, namely
$(\classof{\redseq} \lor \classof{\redseqtwo}) \land \classof{\redseqthree} =
(\classof{\redseq} \land \classof{\redseqthree}) \lor (\classof{\redseqtwo} \land \classof{\redseqthree})$
is proved analogously.
\end{proof}

\begin{remark}
The function $\names$ that takes a derivation and returns a set of labels
is well-defined for permutation-equivalence classes,
as a consequence of \rcoro{permutation_equivalence_in_terms_of_names}:
\[
  \names([\redseq]) \eqdef \names(\redseq)
\]
\end{remark}

\begin{theorem}[The lattice of derivations is representable as a ring of sets]
If $\tm \in \termsdist$ is a correct term,
then $\names : \derivlattice{\tm} \to \powerset(\labelset)$ is a monomorphism of lattices,
where $\derivlattice{\tm}$ is the lattice of derivations of $\tm$
and $\powerset(\labelset)$ is the lattice whose elements are sets of labels
ordered by inclusion, with set intersection and set union as
the meet and join operators.
\end{theorem}
\begin{proof}
We are to show that $\names$ is monotonic, that it preserves meets and joins,
and finally that it is a monomorphism:
\begin{itemize}
\item {\bf Monotonic.}
  If $\classof{\redseq} \leq \classof{\redseqtwo}$
  then $\names(\redseq) \subseteq \names(\redseqtwo)$.
  This has been proved in \rprop{prefixes_as_subsets}.
\item {\bf Preserves meets.}
  $\names(\classof{\redseq} \land \classof{\redseqtwo}) =
   \names(\redseq) \cap \names(\redseqtwo)$
  by \rprop{names_of_join_and_meet}.
\item {\bf Preserves joins.}
  $\names(\classof{\redseq} \lor \classof{\redseqtwo}) =
   \names(\redseq) \cup \names(\redseqtwo)$
  by \rprop{names_of_join_and_meet}.
\item {\bf Monomorphism.}
  It suffices to show that $\names$ is injective.
  Indeed, 
  suppose that $\names(\classof{\redseq}) = \names(\classof{\redseqtwo})$.
  By \rcoro{permutation_equivalence_in_terms_of_names}
  we have that $\redseq \permeq \redseqtwo$,
  so
  $\classof{\redseq} = \classof{\redseqtwo}$.
\end{itemize}
\end{proof}






