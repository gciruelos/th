\documentclass{beamer}
%
% Choose how your presentation looks.
%
% For more themes, color themes and font themes, see:
% http://deic.uab.es/~iblanes/beamer_gallery/index_by_theme.html
%
\mode<presentation>
{
  \usetheme{default}      % or try Darmstadt, Madrid, Warsaw, ...
  \usecolortheme{default} % or try albatross, beaver, crane, ...
  \usefonttheme{default}  % or try serif, structurebold, ...
  \setbeamertemplate{navigation symbols}{}
  \setbeamertemplate{caption}[numbered]
}

\usepackage[spanish]{babel}
\usepackage[utf8x]{inputenc}
\usepackage{amsfonts}
\usepackage[all]{xy}
\usepackage{mathtools}
\usepackage{array,amsmath,amssymb,amsthm}
\usepackage{xcolor}

\newtheorem{defes}{Definición}
\newtheorem{teoes}{Teorema}
\newtheorem{proes}{Proposición}



\definecolor{darkGreen}{rgb}{0,0.5,0}
\definecolor{darkBlue}{rgb}{0,0,1}
\newcommand{\cArith}[1]{{\color{darkGreen} #1}}
\newcommand{\cLam}[1]{{\color{darkBlue} #1}}
\newcommand{\uline}[1]{\underline{#1}}

\title{Factorizando espacios de derivación \\ a través de tipos intersección}
\author{Gonzalo Ciruelos \\ {\footnotesize Director: Pablo Barenbaum}}
\institute{Facultad de Ciencias Exactas y Naturales \\ Universidad de Buenos Aires}
\date{28 de junio de 2018}

%%% Inductive rules with prooftree
\newcommand{\indrulename}[1]{\texttt{#1}}
\newcommand{\indrule}[3]{
\ensuremath{
\begin{array}{c}
  \prooftree #2
    \justifies #3
    \thickness=0.05em
    \using \indrulename{#1}
  \endprooftree
\end{array}}}

\renewcommand{\theenumi}{\arabic{enumi}}
\renewcommand{\theenumii}{\arabic{enumii}}
\renewcommand{\theenumiii}{\arabic{enumiii}}
\renewcommand{\theenumiv}{\arabic{enumiv}}
%%
\renewcommand{\labelenumi}{\arabic{enumi}.}
\renewcommand{\labelenumii}{\arabic{enumi}.\arabic{enumii}}
\renewcommand{\labelenumiii}{\arabic{enumi}.\arabic{enumii}.\arabic{enumiii}}
\renewcommand{\labelenumiv}{\arabic{enumi}.\arabic{enumii}.\arabic{enumiii}.\arabic{enumiv}}
%%
\makeatletter
\renewcommand\p@enumii{\theenumi.}
\renewcommand\p@enumiii{\theenumi.\theenumii.}
\renewcommand\p@enumiv{\theenumi.\theenumii.\theenumiii.}
\makeatother

%%% Theorem environments

%  \newtheoremstyle{break}% name
%    {}%         Space above, empty = `usual value'
%    {}%         Space below
%    {\itshape}% Body font
%    {}%         Indent amount (empty = no indent, \parindent = para indent)
%    {\bfseries}% Thm head font
%    {.}%        Punctuation after thm head
%    {\newline}% Space after thm head: \newline = linebreak
%    {}%         Thm head spec
%  
%  \theoremstyle{break}
\newtheorem{dummythm}{dummythm}
\newtheorem{lemma}[dummythm]{Lemma}
\newtheorem{convention}[dummythm]{Convention}
\newtheorem{proposition}[dummythm]{Proposition}
\newtheorem{theorem}[dummythm]{Theorem}
\newtheorem{example}[dummythm]{Example}
\newtheorem{corollary}[dummythm]{Corollary}

\theoremstyle{definition}
\newtheorem{definition}[dummythm]{Definition}

\theoremstyle{remark}
\newtheorem{remark}[dummythm]{Remark}
\newtheorem{notation}[dummythm]{Notation}

\usepackage{chngcntr}
\counterwithin{lemma}{chapter}
\counterwithin{convention}{chapter}
\counterwithin{proposition}{chapter}
\counterwithin{theorem}{chapter}
\counterwithin{example}{chapter}
\counterwithin{corollary}{chapter}
\counterwithin{definition}{chapter}
\counterwithin{remark}{chapter}
\counterwithin{notation}{chapter}


% El \expandafter\uppercase es un hack para que anden las refs
% en los headings.
% https://tex.stackexchange.com/questions/13885/reference-undefined-in-section-with-amsbook
\newcommand{\llem}[1]{\expandafter\uppercase\expandafter{\label{lemma:#1}}}
\newcommand{\rlem}[1]{Lemma~\expandafter\uppercase\expandafter{\ref{lemma:#1}}}
\newcommand{\ldef}[1]{\expandafter\uppercase\expandafter{\label{def:#1}}}
\newcommand{\rdef}[1]{Definition~\expandafter\uppercase\expandafter{\ref{def:#1}}}
\newcommand{\lprop}[1]{\expandafter\uppercase\expandafter{\label{prop:#1}}}
\newcommand{\rprop}[1]{Proposition~\expandafter\uppercase\expandafter{\ref{prop:#1}}}
\newcommand{\lthm}[1]{\expandafter\uppercase\expandafter{\label{thm:#1}}}
\newcommand{\rthm}[1]{Theorem~\expandafter\uppercase\expandafter{\ref{thm:#1}}}
\newcommand{\lremark}[1]{\label{remark:#1}}
\newcommand{\rremark}[1]{Remark~\ref{remark:#1}}
\newcommand{\lcoro}[1]{\label{coro:#1}}
\newcommand{\rcoro}[1]{Corollary~\ref{coro:#1}}
\newcommand{\lsec}[1]{\expandafter\uppercase\expandafter{\label{section:#1}}}
\newcommand{\rsec}[1]{Section~\expandafter\uppercase\expandafter{\ref{section:#1}}}
\newcommand{\lcha}[1]{\label{ch:#1}}
\newcommand{\rcha}[1]{Chapter~\ref{ch:#1}}
\newcommand{\lconvention}[1]{\label{convention:#1}}
\newcommand{\rconvention}[1]{Convention~\ref{convention:#1}}
\newcommand{\lexample}[1]{\label{example:#1}}
\newcommand{\rexample}[1]{Example~\ref{example:#1}}
\newcommand{\leqn}[1]{\label{eqn:#1}}
\newcommand{\reqn}[1]{(\ref{eqn:#1})}
\newcommand{\lfig}[1]{\label{fig:#1}}
\newcommand{\rfig}[1]{Figure~\ref{fig:#1}}
%
\newcommand{\lpart}[1]{\label{part:#1}}
\newcommand{\rlempart}[2]{Lemma~\ref{lemma:#1}~(\ref{part:#2})}


%%%%%%
\newcommand{\defn}[1]{{\bf #1}}
\newcommand{\eg}{{\em e.g.}\xspace}
\newcommand{\ie}{{\em i.e.}\xspace}
\newcommand{\ih}{{\em i.h.}\xspace}
\newcommand{\ST}{\ |\ }
\newcommand{\HS}{\hspace{.5cm}}
\newcommand{\HStight}{\hspace{.2cm}}
\newcommand{\VS}{\vspace{.5cm}}
\renewcommand{\emptyset}{\varnothing}
\newcommand{\set}[1]{\{#1\}}
\newcommand{\Nat}{\mathbb{N}}
\newcommand{\eqdef}{\overset{\mathrm{def}}{=}}
\newcommand{\iffdef}{\overset{\mathrm{def}}{\iff}}
\newcommand{\eqih}{\overset{\mathrm{h.i.}}{=}}
\newcommand{\TODO}[1]{\textcolor{red}{TODO: #1}}

\newcommand{\Case}[1]{{\bf #1.}}

\newcommand{\tm}{t}
\newcommand{\tmtwo}{s}
\newcommand{\tmthree}{u}
\newcommand{\tmthreevariant}{v}
\newcommand{\tmfour}{r}
\newcommand{\tmfive}{p}
\newcommand{\tmsix}{q}

\newcommand{\var}{x}
\newcommand{\vartwo}{y}
\newcommand{\varthree}{z}

\newcommand{\basetyp}{\alpha}
\newcommand{\basetyptwo}{\beta}
\newcommand{\basetypthree}{\gamma}

\newcommand{\labelset}{\mathscr{L}}
\newcommand{\lab}{\ell}
\newcommand{\labtwo}{\lab'}
\newcommand{\labthree}{\lab''}

\newcommand{\conbase}{\Box}
\newcommand{\of}[1]{\langle#1\rangle}
\newcommand{\con}{\mathtt{C}}
\newcommand{\contwo}{\con'}
\newcommand{\conthree}{\con''}
\newcommand{\conof}[1]{\con\of{#1}}
\newcommand{\contwoof}[1]{\contwo\of{#1}}
\newcommand{\conhat}{\hat{\con}}

\newcommand{\Anon}{\mathtt{X}}

\newcommand{\typ}{\tau}
\newcommand{\typtwo}{\sigma}
\newcommand{\typthree}{\rho}
\newcommand{\typfour}{\phi}

\newcommand{\lset}[1]{[#1]}
\newcommand{\lsetenum}[3]{[#1_#2, \hdots, #1_#3]}
\newcommand{\lsetcomp}[4]{[#1_#2]_{#2=#3}^{#4}}
\newcommand{\mset}[1]{[\![#1]\!]}
\newcommand{\emptylset}{\lset{\,}}

\newcommand{\mtyp}{\mathcal{M}}
\newcommand{\mtyptwo}{\mathcal{N}}
\newcommand{\mtypthree}{\mathcal{P}}

\newcommand{\emptyContext}{\varnothing}
\newcommand{\tctx}{\Gamma}
\newcommand{\tctxtwo}{\Delta}
\newcommand{\tctxthree}{\Theta}
\newcommand{\tctxfour}{\Psi}
\newcommand{\card}[1]{|#1|}
\DeclareMathOperator{\dom}{dom}

\newcommand{\lam}[2]{\lambda #1. #2}
\newcommand{\lamp}[3]{\lambda^{#1} #2. #3}
\newcommand{\subs}[3]{#1\{#2 := #3\}}
\newcommand{\sub}[2]{\{\!\!\{#1 := #2\}\!\!\}}

\newcommand{\totwo}{\Rightarrow}
\newcommand{\tolab}[1]{\overset{#1}{\to}}
\newcommand{\tomulti}{\mathrel{\twoheadrightarrow}}
% \newcommand{\tomulti}{\mathrel{\mbox{\ensuremath{\relbar\!\!\!\!\relbar\!\!\!\!\!\!\overset{}{\relbar\!\!\!\!\circ}\!\!\!\!\relbar\!\!\!\!\rightarrow}}}}

\newcommand{\tobeta}{\mathrel{\to_\beta}}
\newcommand{\rtobeta}{\mathrel{\twoheadrightarrow_\beta}}
\newcommand{\toabeta}[1]{\mathrel{\xrightarrow{#1}_\beta}}
\newcommand{\todist}{\mathrel{\to_\dist}}
\newcommand{\rtodist}{\mathrel{\twoheadrightarrow_\dist}}
\newcommand{\todistl}[1]{\mathrel{\xrightarrow{#1}_\dist}}
\newcommand{\tomdist}{\mathrel{\tomulti_\dist}}
\newcommand{\todistih}{\overset{\mathrm{(h.i.)}}{\todist}}

\newcommand{\prooft}{\Phi}

\newcommand{\emptyDerivation}{\epsilon}
\newcommand{\redex}{R}
\newcommand{\redextwo}{S}
\newcommand{\redexthree}{T}
\newcommand{\name}{\mathsf{name}}
\newcommand{\names}{\mathsf{names}}
\newcommand{\src}{\mathsf{src}}
\newcommand{\tgt}{\mathsf{tgt}}
\newcommand{\redexset}{\mathcal{M}}
\newcommand{\redexsettwo}{\mathcal{N}}
\newcommand{\redseq}{\rho}
\newcommand{\redseqtwo}{\sigma}
\newcommand{\redseqthree}{\tau}
\newcommand{\nameof}[1]{\mathsf{name}(#1)}

\newcommand{\arsA}{\mathscr{A}}
\newcommand{\Objs}{\mathsf{Objs}}
\newcommand{\Derivs}{\mathsf{Derivs}}
\newcommand{\permeq}{\equiv}
\newcommand{\permle}{\sqsubseteq}

%%
\newcommand{\conv}{\Pi}
\newcommand{\convtwo}{\Phi}
\newcommand{\convthree}{\Psi}
\newcommand{\convid}{\mathsf{Id}}
\newcommand{\perm}{\pi}
\newcommand{\permtwo}{\phi}
\newcommand{\permconv}[2]{{{#1}\choose{#2}}}

\newcommand{\convto}{\Rightarrow}
\newcommand{\splitt}[3]{\mathsf{split}^{#1}_{#2}(#3)}
\newcommand{\emptyList}{\epsilon}
\newcommand{\cons}{\cdot}
\newcommand{\dist}{\mathtt{\#}}
\newcommand{\terms}{\mathcal{T}^{\lambda}}
\newcommand{\termsdist}{\mathcal{T}^{\dist}}
\newcommand{\numof}[2]{\#_{#1}(#2)}
\newcommand{\lengthof}[1]{|#1|}
\newcommand{\fv}[1]{\mathsf{fv}({#1})}
\newcommand{\ls}[1]{\vec{#1}}

\newcommand{\refinedby}{\mathrel{\rtimes}}
\newcommand{\refines}{\mathrel{\ltimes}}

%\newcommand{\tmlabel}[1]{\ell(#1)}
%\newcommand{\varlabel}[2]{\ell_{#1}(#2)}

\newcommand{\anon}{\mathtt{X}}

\newcommand{\derivlattice}[1]{\mathcal{D}_{#1}}
\newcommand{\classof}[1]{{[#1]}}
\newcommand{\powerset}{\mathcal{P}}

\newcommand{\sieve}{\downarrow}

\newcommand{\Ob}{\mathsf{Ob}}
\newcommand{\Hom}{\mathsf{Hom}}
\newcommand{\id}{\mathsf{id}}
\newcommand{\Cat}{\mathscr{C}}
\newcommand{\CatTwo}{\mathscr{D}}
\newcommand{\USL}{\mathsf{USL}}
\newcommand{\USLB}{\mathsf{USLB}}
\newcommand{\cls}[1]{{[#1]}}
\newcommand{\bigcls}[1]{{\left[#1\right]}}
\newcommand{\grothy}[2]{\int_{#1}{#2}}
\newcommand{\ulbFree}[2]{\mathbb{F}(#1, #2)}
\newcommand{\ulbGarbage}[2]{\mathbb{G}(#1, #2)}
\newcommand{\ulbF}{\mathcal{F}}
\newcommand{\ulbG}{\mathcal{G}}
\newcommand{\leqF}{\unlhd}
\newcommand{\landF}{\!\vartriangle\!}
\newcommand{\lorF}{\triangledown}
\newcommand{\leqG}{\permle}
\newcommand{\identity}{\mathsf{id}}

\newcommand{\toI}{\ensuremath{\to_{\mathrm{I}}}}
\newcommand{\toE}{\ensuremath{\to_{\mathrm{E}}}}
\newcommand{\lambdadist}{\lambda^\dist}
\newcommand{\pt}[2]{#2 \hookrightarrow #1}
\newcommand{\ptF}[2]{#2 \hookrightarrow_{\mathcal{F}} #1}


% pub macros
\newcommand{\SeeAppendix}{$\spadesuit$\xspace}
\newcommand{\SeeAppendixRef}[1]{\hyperref[#1]{\SeeAppendix}}
\newcommand{\OR}{\ \mid\ }
\newcommand{\ulbDeriv}[1]{\mathbb{D}(#1)}
\newcommand{\ulbDerivLam}[1]{\mathbb{D}^{\lambda}(#1)}
\newcommand{\ulbDerivDist}[1]{\mathbb{D}^{\dist}(#1)}

\newcommand{\Obj}{\mathsf{Obj}}
\newcommand{\Stp}{\mathsf{Stp}}
\newcommand{\lpos}{\mathsf{pos}}

\newcommand{\cf}{{\em cf.}\xspace}

\newcommand{\Poset}{\mathsf{Poset}}

\newcommand{\totwocell}{\leq}

\newcommand{\occursin}{\preceq}

\newcommand{\itemname}[1]{\textcolor{darkgray}{\sffamily\bfseries{#1}}}
\newcommand{\refcase}[1]{\itemname{\ref{#1}}}
\newcommand{\condp}[1]{\textrm{\texttt{[c#1]}}}

\newcommand{\tmlabel}[1]{\mathtt{T}(#1)}
\newcommand{\varlabel}[2]{\mathtt{T}_{#1}(#2)}

\newcommand{\lefttodist}{\mathrel{\leftarrow_\dist}}

\newcommand{\subterms}[1]{\mathsf{sub}(#1)}
\newcommand{\fsubterms}[1]{\mathsf{sub}^{\circ}(#1)}



\begin{document}

\begin{frame}
  \titlepage
\end{frame}

% Uncomment these lines for an automatically generated outline.
%\begin{frame}{Outline}
%  \tableofcontents
%\end{frame}

\section{Espacios de derivación}




\begin{frame}{Espacios de derivación}

\textbf{Aritmética elemental y pares ordenados}
\[
\begin{array}{rclc}
  \cArith{\uline{n} + \uline{m}} & \to & \cArith{\uline{n + m}} & \\
  \cArith{\uline{n} \cdot \tm  } & \to & \underbrace{\cArith{\tm + \tm + ... + \tm}}_{n \text{ veces}} & \text{ si } n > 0 \\
  \cArith{\uline{0} \cdot \tm  } & \to & \cArith{\uline{0}} & \\
  & & & \\
  & & & \\
  \cArith{(\tm, \tmtwo)} & \to & \cArith{(\tm', \tmtwo)} & \text{ si } \cArith{\tm} \to \cArith{\tm'} \\
  \cArith{(\tm, \tmtwo)} & \to & \cArith{(\tm, \tmtwo')} & \text{ si } \cArith{\tmtwo} \to \cArith{\tmtwo'} \\
\end{array}
\]


\end{frame}

\begin{frame}{Espacios de derivación}

$
\ulbDeriv{\cArith{\uline{1} + \uline{1}}} =
  \bigg(\cArith{\uline{1} + \uline{1}} \to \cArith{\uline{2}}\bigg)
$

$\ulbDeriv{\cArith{\uline{0} \cdot (\uline{5} + \uline{5})}} =
\xymatrix@C=0.3cm{
  \cArith{\uline{0} \cdot (\uline{5} + \uline{5})} \ar[rr] \ar[dr] &  & \cArith{\uline{0} \cdot \uline{10}} \ar[dl] \\
  & \cArith{\uline{0}}   &
}
$
{\footnotesize
$\ulbDeriv{\cArith{(\uline{1} + \uline{1}, \uline{0} \cdot (\uline{5} + \uline{5}))}} =
\xymatrix@C=0.2cm{
  \cArith{(\uline{1} + \uline{1}, \uline{0} \cdot (\uline{5} + \uline{5}))} \ar[rr] \ar[dr] \ar[dd]&  & \cArith{(\uline{1} + \uline{1}, \uline{0} \cdot \uline{10})} \ar[dd] \ar[dl] \\
  & \cArith{(\uline{1} + \uline{1}, \uline{0})}  \ar[dd] & \\
  \cArith{(\uline{2}, \uline{0} \cdot (\uline{5} + \uline{5}))} \ar[rr]|!{[r]}\hole \ar[dr] &  & \cArith{(\uline{2}, \uline{0} \cdot \uline{10})} \ar[dl] \\
  & \cArith{(\uline{2}, \uline{0})}   & \\
}$
}

$\ulbDeriv{\cArith{(A,B)}} \simeq \ulbDeriv{\cArith{A}} \times \ulbDeriv{\cArith{B}}$

\end{frame}

\begin{frame}{El $\lambda$-cálculo}

\textbf{Términos del $\lambda$-cálculo}

\[
  t ::= x\  |\ t\, t \ |\ \lam{x}{t}
\]

Por ejemplo,
\begin{itemize}
 \item[] $\lam{x}{x}$
 \item[] $(\lam{x}{z})\ y$
 \item[] $(\lam{x}{\lam{y}{x}})\ z\ w$ = $((\lam{x}{(\lam{y}{x})})\ z)\ w$
\end{itemize}

\vskip 0.5cm

\textbf{Variables libres}
\[
\begin{array}{rclc}
  \fv{x} & \eqdef & x & \\
  \fv{u\, v} & \eqdef & \fv{u} \cup \fv{v} & \\
  \fv{\lam{x}{u}} & \eqdef & \fv{u} - \set{x} & \\
\end{array}
\]
\end{frame}


\begin{frame}{El $\lambda$-cálculo}

\textbf{$\beta$-reducción}

\[
  (\lam{x}{t})\, s \to_\beta \subs{t}{x}{s}
\]
\vskip 0.5cm


\textbf{...donde sustituir significa}
\[
\begin{array}{rclc}
  \subs{x}{x}{s} & \eqdef & s & \\
  \subs{y}{x}{s} & \eqdef & y & \\
  \subs{(u\, v)}{x}{s} & \eqdef & \subs{u}{x}{s}\, \subs{v}{x}{s} & \\
  \subs{(\lam{y}{u})}{x}{s} & \eqdef & \lam{y}{\subs{u}{x}{s}} & \text{ si } x\neq y \text{ e } y \not\in \fv{u} \\
\end{array}
\]

\textbf{Por ejemplo}

\[(\lam{x}{x}) ((\lam{y}{y}) z) \to (\lam{y}{y}) z \to z\]


\end{frame}

\begin{frame}{Espacios de derivación en el $\lambda$-cálculo}
\[\ulbDeriv{\cLam{(\lam{x}{x}) ((\lam{y}{y}) z)}} =
\xymatrix@C=0.3cm{
\cLam{(\lam{x}{x}) ((\lam{y}{y}) z)} \ar[d] \ar[r] & \cLam{(\lam{y}{y}) z} \ar[d] \\
\cLam{(\lam{x}{x}) z}  \ar[r] & \cLam{z}
}
\]
\end{frame}

\begin{frame}{Espacios de derivación en el $\lambda$-cálculo -- Problemas}
\textbf{Creación}
\[\ulbDeriv{\cLam{(\lam{x}{x x}) (\lam{x}{x x})}} =
\xymatrix@C=0.3cm{
\cLam{(\lam{x}{x x}) (\lam{x}{x x})} \ar[d] \\
\cLam{(\lam{x}{x x}) (\lam{x}{x x})} \ar[d] \\
\cLam{(\lam{x}{x x}) (\lam{x}{x x})} \ar[d] \\
\vdots  \\
}
\]
$\ulbDeriv{\cLam{(\lam{x}{x x}) (\lam{x}{x x})}}$ es infinito, mientras que
$\ulbDeriv{\cLam{(\lam{x}{x x})}}$ es finito.
\end{frame}

\begin{frame}{Espacios de derivación en el $\lambda$-cálculo -- Problemas}
\textbf{Duplicación}
\[\ulbDeriv{\cLam{(\lam{x}{x x})\ (I\, z)}} =
\xymatrix@C=0.3cm{
& & \cLam{(\lam{x}{x x})\ (I\, z)} \ar[dr] \ar[dl] & \\
& \cLam{(I\, z)\ (I\, z)} \ar[dr] \ar[dl] & & \cLam{(\lam{x}{x x})\ z} \ar@/^2.0pc/[ddll]\\
\cLam{z\ (I\, z)} \ar[dr] & & \cLam{(I\, z)\ z} \ar[dl] & \\
 & \cLam{z\ z} & & \\
}
\]

{\footnotesize
$\ulbDeriv{\cLam{(\lam{x}{x x})\ (I\, z)}}
\not\simeq
\ulbDeriv{\cLam{(\lam{x}{x x})\ \conbase}} \times \ulbDeriv{\cLam{I\, z}}
=
\underbrace{\bigg(\cLam{(\lam{x}{x x})\ \conbase} \to \cLam{\conbase \, \conbase}\bigg) \times \bigg(\cLam{I\, z} \to \cLam{z}\bigg)}_{\xymatrix@=0.2cm{
& \bullet \ar[dr] \ar[dl] & \\
\bullet \ar[dr] & & \bullet \ar[dl] \\
& \bullet &}}$
}
\end{frame}

\begin{frame}{Espacios de derivación en el $\lambda$-cálculo -- Problemas}
\textbf{Borrado}
\[\ulbDeriv{\cLam{(\lam{x}{y})\ (I\, z)}} =
\xymatrix@C=0.3cm{
\cLam{(\lam{x}{y})\ (I\, z)} \ar[rr] \ar[dr] & & \cLam{(\lam{x}{y})\ z} \ar[dl]\\
& \cLam{y} & \\
}
\]

{\footnotesize
$\ulbDeriv{\cLam{(\lam{x}{y})\ (I\, z)}}
\not\simeq
\ulbDeriv{\cLam{(\lam{x}{y})\ \conbase}} \times \ulbDeriv{\cLam{I\, z}}
=
\underbrace{\bigg(\cLam{(\lam{x}{y})\ \conbase} \to \cLam{y}\bigg) \times \bigg(\cLam{I\, z} \to \cLam{z}\bigg)}_{\xymatrix@=0.2cm{
& \bullet \ar[dr] \ar[dl] & \\
\bullet \ar[dr] & & \bullet \ar[dl] \\
& \bullet &}}$
}
\end{frame}

\begin{frame}{Espacios de derivación en el $\lambda$-cálculo}

\begin{defes}[Equivalencia por permutaciones]
Decimos que dos secuencias de reducción $\rho, \sigma$ son equivalentes por permutación si $\rho/\sigma = \sigma/\rho = \epsilon$.
Lo escribimos como $\rho \permeq \sigma$.
\end{defes}

\begin{defes}[Espacio de derivación]
Si $\cLam{\tm}$ es un término, $\ulbDeriv{\cLam{\tm}}$ es el conjunto de
\defn{secuencias de reducción} desde $\cLam{\tm}$:
\[
  \set{\rho \ | \ \rho : \cLam{t} \to^* \cLam{s} \text{ es una secuencia de pasos de reescritura}} \bigg/ \permeq
\]
\end{defes}

Para darle más estructura, proveemos al conjunto de un orden.

\begin{defes}[Orden del prefijo]
\[[\rho] \permle [\sigma]   \iffdef \rho / \sigma = \epsilon\]
\end{defes}
\end{frame}

\section{Some \LaTeX{} Examples}

\subsection{Tables and Figures}

\begin{frame}{Tables and Figures}

\begin{itemize}
\item Use \texttt{tabular} for basic tables --- see Table~\ref{tab:widgets}, for example.
\item You can upload a figure (JPEG, PNG or PDF) using the files menu. 
\item To include it in your document, use the \texttt{includegraphics} command (see the comment below in the source code).
\end{itemize}

% Commands to include a figure:
%\begin{figure}
%\includegraphics[width=\textwidth]{your-figure's-file-name}
%\caption{\label{fig:your-figure}Caption goes here.}
%\end{figure}

\begin{table}
\centering
\begin{tabular}{l|r}
Item & Quantity \\\hline
Widgets & 42 \\
Gadgets & 13
\end{tabular}
\caption{\label{tab:widgets}An example table.}
\end{table}

\end{frame}

\subsection{Mathematics}

\begin{frame}{Readable Mathematics}

Let $X_1, X_2, \ldots, X_n$ be a sequence of independent and identically distributed random variables with $\text{E}[X_i] = \mu$ and $\text{Var}[X_i] = \sigma^2 < \infty$, and let
$$S_n = \frac{X_1 + X_2 + \cdots + X_n}{n}
      = \frac{1}{n}\sum_{i}^{n} X_i$$
denote their mean. Then as $n$ approaches infinity, the random variables $\sqrt{n}(S_n - \mu)$ converge in distribution to a normal $\mathcal{N}(0, \sigma^2)$.

\end{frame}

\end{document}


