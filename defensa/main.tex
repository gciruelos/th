\documentclass{beamer}
%
% Choose how your presentation looks.
%
% For more themes, color themes and font themes, see:
% http://deic.uab.es/~iblanes/beamer_gallery/index_by_theme.html
%
\mode<presentation>
{
  \usetheme{default}      % or try Darmstadt, Madrid, Warsaw, ...
  \usecolortheme{default} % or try albatross, beaver, crane, ...
  \usefonttheme{default}  % or try serif, structurebold, ...
  \setbeamertemplate{navigation symbols}{}
  \setbeamertemplate{caption}[numbered]
}

\usepackage[spanish]{babel}
\usepackage[utf8x]{inputenc}
\usepackage{amsfonts}
\usepackage[all]{xy}
\usepackage{mathtools}
\usepackage{array,amsmath,amssymb,amsthm}
\usepackage{xcolor}
\usepackage{listings}

\theoremstyle{definition}
\newtheorem{defes}{Definición}
\newtheorem{teoes}{Teorema}
\newtheorem{lemes}{Lema}
\newtheorem{proes}{Proposición}
\newtheorem{comes}{Comentario}
\newtheorem{cores}{Corolario}



\definecolor{darkGreen}{rgb}{0,0.5,0}
\definecolor{darkBlue}{rgb}{0,0,1}
\definecolor{lightBlue}{rgb}{1,0.2,0.6}
\definecolor{normalGrey}{rgb}{0.7,0.7,0.7}
\newcommand{\cArith}[1]{{\color{darkGreen} #1}}
\newcommand{\cLam}[1]{{\color{darkBlue} #1}}
\newcommand{\cProto}[1]{{\color{lightBlue} #1}}
\newcommand{\cDist}[1]{{\color{red} #1}}
\newcommand{\cGrey}[1]{{\color{normalGrey} #1}}
\newcommand{\uline}[1]{\underline{#1}}
\newcommand{\clambdadist}{\cDist{\lambdadist}}
\newcommand{\clambda}{\cLam{\lambda}}
\newcommand{\cdist}{\cDist{\dist}}
\newcommand{\cLab}[1]{{\color{black} #1}}

\title{Factorizando espacios de derivación \\ a través de tipos intersección}
\author{Gonzalo Ciruelos \\ Director: Pablo Barenbaum}
\institute{Facultad de Ciencias Exactas y Naturales \\ Universidad de Buenos Aires}
\date{28 de junio de 2018}

%%% Inductive rules with prooftree
\newcommand{\indrulename}[1]{\texttt{#1}}
\newcommand{\indrule}[3]{
\ensuremath{
\begin{array}{c}
  \prooftree #2
    \justifies #3
    \thickness=0.05em
    \using \indrulename{#1}
  \endprooftree
\end{array}}}

\renewcommand{\theenumi}{\arabic{enumi}}
\renewcommand{\theenumii}{\arabic{enumii}}
\renewcommand{\theenumiii}{\arabic{enumiii}}
\renewcommand{\theenumiv}{\arabic{enumiv}}
%%
\renewcommand{\labelenumi}{\arabic{enumi}.}
\renewcommand{\labelenumii}{\arabic{enumi}.\arabic{enumii}}
\renewcommand{\labelenumiii}{\arabic{enumi}.\arabic{enumii}.\arabic{enumiii}}
\renewcommand{\labelenumiv}{\arabic{enumi}.\arabic{enumii}.\arabic{enumiii}.\arabic{enumiv}}
%%
\makeatletter
\renewcommand\p@enumii{\theenumi.}
\renewcommand\p@enumiii{\theenumi.\theenumii.}
\renewcommand\p@enumiv{\theenumi.\theenumii.\theenumiii.}
\makeatother

%%% Theorem environments

%  \newtheoremstyle{break}% name
%    {}%         Space above, empty = `usual value'
%    {}%         Space below
%    {\itshape}% Body font
%    {}%         Indent amount (empty = no indent, \parindent = para indent)
%    {\bfseries}% Thm head font
%    {.}%        Punctuation after thm head
%    {\newline}% Space after thm head: \newline = linebreak
%    {}%         Thm head spec
%  
%  \theoremstyle{break}
\newtheorem{dummythm}{dummythm}
\newtheorem{lemma}[dummythm]{Lemma}
\newtheorem{convention}[dummythm]{Convention}
\newtheorem{proposition}[dummythm]{Proposition}
\newtheorem{theorem}[dummythm]{Theorem}
\newtheorem{example}[dummythm]{Example}
\newtheorem{corollary}[dummythm]{Corollary}

\theoremstyle{definition}
\newtheorem{definition}[dummythm]{Definition}

\theoremstyle{remark}
\newtheorem{remark}[dummythm]{Remark}
\newtheorem{notation}[dummythm]{Notation}

% El \expandafter\uppercase es un hack para que anden las refs
% en los headings.
% https://tex.stackexchange.com/questions/13885/reference-undefined-in-section-with-amsbook
\newcommand{\llem}[1]{\expandafter\uppercase\expandafter{\label{lemma:#1}}}
\newcommand{\rlem}[1]{Lemma~\expandafter\uppercase\expandafter{\ref{lemma:#1}}}
\newcommand{\ldef}[1]{\expandafter\uppercase\expandafter{\label{def:#1}}}
\newcommand{\rdef}[1]{Definition~\expandafter\uppercase\expandafter{\ref{def:#1}}}
\newcommand{\lprop}[1]{\expandafter\uppercase\expandafter{\label{prop:#1}}}
\newcommand{\rprop}[1]{Proposition~\expandafter\uppercase\expandafter{\ref{prop:#1}}}
\newcommand{\lthm}[1]{\expandafter\uppercase\expandafter{\label{thm:#1}}}
\newcommand{\rthm}[1]{Theorem~\expandafter\uppercase\expandafter{\ref{thm:#1}}}
\newcommand{\lremark}[1]{\label{remark:#1}}
\newcommand{\rremark}[1]{Remark~\ref{remark:#1}}
\newcommand{\lcoro}[1]{\label{coro:#1}}
\newcommand{\rcoro}[1]{Corollary~\ref{coro:#1}}
\newcommand{\lsec}[1]{\label{section:#1}}
\newcommand{\rsec}[1]{Section~\ref{section:#1}}
\newcommand{\lcha}[1]{\label{ch:#1}}
\newcommand{\rcha}[1]{Chapter~\ref{ch:#1}}
\newcommand{\lconvention}[1]{\label{convention:#1}}
\newcommand{\rconvention}[1]{Convention~\ref{convention:#1}}
\newcommand{\lexample}[1]{\label{example:#1}}
\newcommand{\rexample}[1]{Ex.~\ref{example:#1}}
\newcommand{\leqn}[1]{\label{eqn:#1}}
\newcommand{\reqn}[1]{(\ref{eqn:#1})}
\newcommand{\lfig}[1]{\label{fig:#1}}
\newcommand{\rfig}[1]{Figure~\ref{fig:#1}}
%
\newcommand{\lpart}[1]{\label{part:#1}}
\newcommand{\rlempart}[2]{Lemma~\ref{lemma:#1}~(\ref{part:#2})}


%%%%%%
\newcommand{\defn}[1]{{\bf #1}}
\newcommand{\eg}{{\em e.g.}\xspace}
\newcommand{\ie}{{\em i.e.}\xspace}
\newcommand{\ih}{{\em i.h.}\xspace}
\newcommand{\ST}{\ |\ }
\newcommand{\HS}{\hspace{.5cm}}
\newcommand{\HStight}{\hspace{.2cm}}
\newcommand{\VS}{\vspace{.5cm}}
\renewcommand{\emptyset}{\varnothing}
\newcommand{\set}[1]{\{#1\}}
\newcommand{\Nat}{\mathbb{N}}
\newcommand{\eqdef}{\overset{\mathrm{def}}{=}}
\newcommand{\iffdef}{\overset{\mathrm{def}}{\iff}}
\newcommand{\eqih}{\overset{\mathrm{h.i.}}{=}}
\newcommand{\TODO}[1]{\textcolor{red}{TODO: #1}}

\newcommand{\Case}[1]{{\bf #1.}}

\newcommand{\tm}{t}
\newcommand{\tmtwo}{s}
\newcommand{\tmthree}{u}
\newcommand{\tmthreevariant}{v}
\newcommand{\tmfour}{r}
\newcommand{\tmfive}{p}
\newcommand{\tmsix}{q}

\newcommand{\var}{x}
\newcommand{\vartwo}{y}
\newcommand{\varthree}{z}

\newcommand{\basetyp}{\alpha}
\newcommand{\basetyptwo}{\beta}
\newcommand{\basetypthree}{\gamma}

\newcommand{\labelset}{\mathscr{L}}
\newcommand{\lab}{\ell}
\newcommand{\labtwo}{\lab'}
\newcommand{\labthree}{\lab''}

\newcommand{\conbase}{\Box}
\newcommand{\of}[1]{\langle#1\rangle}
\newcommand{\con}{\mathtt{C}}
\newcommand{\contwo}{\con'}
\newcommand{\conthree}{\con''}
\newcommand{\conof}[1]{\con\of{#1}}
\newcommand{\contwoof}[1]{\contwo\of{#1}}
\newcommand{\conhat}{\hat{\con}}

\newcommand{\Anon}{\mathtt{X}}

\newcommand{\typ}{\tau}
\newcommand{\typtwo}{\sigma}
\newcommand{\typthree}{\rho}
\newcommand{\typfour}{\phi}

\newcommand{\lset}[1]{[#1]}
\newcommand{\lsetenum}[3]{[#1_#2, \hdots, #1_#3]}
\newcommand{\lsetcomp}[4]{[#1_#2]_{#2=#3}^{#4}}
\newcommand{\mset}[1]{[\![#1]\!]}
\newcommand{\emptylset}{\lset{\,}}

\newcommand{\mtyp}{\mathcal{M}}
\newcommand{\mtyptwo}{\mathcal{N}}
\newcommand{\mtypthree}{\mathcal{P}}

\newcommand{\emptyContext}{\varnothing}
\newcommand{\tctx}{\Gamma}
\newcommand{\tctxtwo}{\Delta}
\newcommand{\tctxthree}{\Theta}
\newcommand{\tctxfour}{\Psi}
\newcommand{\card}[1]{|#1|}
\DeclareMathOperator{\dom}{dom}

\newcommand{\lam}[2]{\lambda #1. #2}
\newcommand{\lamp}[3]{\lambda^{#1} #2. #3}
\newcommand{\subs}[3]{#1\{#2 := #3\}}
\newcommand{\sub}[2]{\{\!\!\{#1 := #2\}\!\!\}}

\newcommand{\totwo}{\Rightarrow}
\newcommand{\tolab}[1]{\overset{#1}{\to}}
\newcommand{\tomulti}{\mathrel{\twoheadrightarrow}}
% \newcommand{\tomulti}{\mathrel{\mbox{\ensuremath{\relbar\!\!\!\!\relbar\!\!\!\!\!\!\overset{}{\relbar\!\!\!\!\circ}\!\!\!\!\relbar\!\!\!\!\rightarrow}}}}

\newcommand{\tobeta}{\mathrel{\to_\beta}}
\newcommand{\rtobeta}{\mathrel{\twoheadrightarrow_\beta}}
\newcommand{\toabeta}[1]{\mathrel{\xrightarrow{#1}_\beta}}
\newcommand{\todist}{\mathrel{\to_\dist}}
\newcommand{\rtodist}{\mathrel{\twoheadrightarrow_\dist}}
\newcommand{\todistl}[1]{\mathrel{\xrightarrow{#1}_\dist}}
\newcommand{\tomdist}{\mathrel{\tomulti_\dist}}
\newcommand{\todistih}{\overset{\mathrm{(h.i.)}}{\todist}}

\newcommand{\prooft}{\Phi}

\newcommand{\emptyDerivation}{\epsilon}
\newcommand{\redex}{R}
\newcommand{\redextwo}{S}
\newcommand{\redexthree}{T}
\newcommand{\name}{\mathsf{name}}
\newcommand{\names}{\mathsf{names}}
\newcommand{\src}{\mathsf{src}}
\newcommand{\tgt}{\mathsf{tgt}}
\newcommand{\redexset}{\mathcal{M}}
\newcommand{\redexsettwo}{\mathcal{N}}
\newcommand{\redseq}{\rho}
\newcommand{\redseqtwo}{\sigma}
\newcommand{\redseqthree}{\tau}
\newcommand{\nameof}[1]{\mathsf{name}(#1)}

\newcommand{\arsA}{\mathscr{A}}
\newcommand{\Objs}{\mathsf{Objs}}
\newcommand{\Derivs}{\mathsf{Derivs}}
\newcommand{\permeq}{\equiv}
\newcommand{\permle}{\sqsubseteq}

%%
\newcommand{\conv}{\Pi}
\newcommand{\convtwo}{\Phi}
\newcommand{\convthree}{\Psi}
\newcommand{\convid}{\mathsf{Id}}
\newcommand{\perm}{\pi}
\newcommand{\permtwo}{\phi}
\newcommand{\permconv}[2]{{{#1}\choose{#2}}}

\newcommand{\convto}{\Rightarrow}
\newcommand{\splitt}[3]{\mathsf{split}^{#1}_{#2}(#3)}
\newcommand{\emptyList}{\epsilon}
\newcommand{\cons}{\cdot}
\newcommand{\dist}{\mathtt{\#}}
\newcommand{\terms}{\mathcal{T}}
\newcommand{\termsdist}{\mathcal{T}^{\dist}}
\newcommand{\numof}[2]{\#_{#1}(#2)}
\newcommand{\lengthof}[1]{|#1|}
\newcommand{\fv}[1]{\mathsf{fv}({#1})}
\newcommand{\ls}[1]{\vec{#1}}

\newcommand{\refinedby}{\mathrel{\rtimes}}
\newcommand{\refines}{\mathrel{\ltimes}}

%\newcommand{\tmlabel}[1]{\ell(#1)}
%\newcommand{\varlabel}[2]{\ell_{#1}(#2)}

\newcommand{\anon}{\mathtt{X}}

\newcommand{\derivlattice}[1]{\mathcal{D}_{#1}}
\newcommand{\classof}[1]{{[#1]}}
\newcommand{\powerset}{\mathcal{P}}

\newcommand{\sieve}{\downarrow}

\newcommand{\Ob}{\mathsf{Ob}}
\newcommand{\Hom}{\mathsf{Hom}}
\newcommand{\id}{\mathsf{id}}
\newcommand{\Cat}{\mathscr{C}}
\newcommand{\CatTwo}{\mathscr{D}}
\newcommand{\USL}{\mathsf{USL}}
\newcommand{\USLB}{\mathsf{USLB}}
\newcommand{\cls}[1]{{[#1]}}
\newcommand{\bigcls}[1]{{\left[#1\right]}}
\newcommand{\grothy}[2]{\int_{#1}{#2}}
\newcommand{\ulbFree}[2]{\mathbb{F}(#1, #2)}
\newcommand{\ulbGarbage}[2]{\mathbb{G}(#1, #2)}
\newcommand{\ulbF}{\mathcal{F}}
\newcommand{\ulbG}{\mathcal{G}}
\newcommand{\leqF}{\unlhd}
\newcommand{\lorF}{\triangledown}
\newcommand{\leqG}{\permle}
\newcommand{\identity}{\mathsf{id}}

\newcommand{\toI}{\ensuremath{\to_{\mathrm{I}}}}
\newcommand{\toE}{\ensuremath{\to_{\mathrm{E}}}}
\newcommand{\lambdadist}{\lambda^\dist}
\newcommand{\pt}[2]{#1//#2}
\newcommand{\ptF}[2]{#1/_{\mathcal{F}}/#2}


% pub macros
\newcommand{\SeeAppendix}{$\spadesuit$\xspace}
\newcommand{\SeeAppendixRef}[1]{\hyperref[#1]{\SeeAppendix}}
\newcommand{\OR}{\ \mid\ }
\newcommand{\ulbDeriv}[1]{\mathbb{D}(#1)}
\newcommand{\ulbDerivLam}[1]{\mathbb{D}^{\lambda}(#1)}
\newcommand{\ulbDerivDist}[1]{\mathbb{D}^{\dist}(#1)}

\newcommand{\Obj}{\mathsf{Obj}}
\newcommand{\Stp}{\mathsf{Stp}}

\newcommand{\cf}{{\em cf.}\xspace}

\newcommand{\Poset}{\mathsf{Poset}}

\newcommand{\totwocell}{\leq}

\newcommand{\occursin}{\preceq}

\newcommand{\itemname}[1]{\textcolor{darkgray}{\sffamily\bfseries{#1}}}
\newcommand{\refcase}[1]{\itemname{\ref{#1}}}
\newcommand{\condp}[1]{\textrm{\texttt{[c#1]}}}

\newcommand{\tmlabel}[1]{\mathtt{T}(#1)}
\newcommand{\varlabel}[2]{\mathtt{T}_{#1}(#2)}

\newcommand{\lefttodist}{\mathrel{\leftarrow_\dist}}

% \renewcommand{proof}{\tiny}{\normalsize}

\message{<Paul Taylor's Proof Trees, 2 August 1996>}
%% Build proof tree for Natural Deduction, Sequent Calculus, etc.
%% WITH SHORTENING OF PROOF RULES!
%% Paul Taylor, begun 10 Oct 1989
%% *** THIS IS ONLY A PRELIMINARY VERSION AND THINGS MAY CHANGE! ***
%%
%% 2 Aug 1996: fixed \mscount and \proofdotnumber
%%
%%      \prooftree
%%              hyp1            produces:
%%              hyp2
%%              hyp3            hyp1    hyp2    hyp3
%%      \justifies              -------------------- rulename
%%              concl                   concl
%%      \thickness=0.08em
%%      \shiftright 2em
%%      \using
%%              rulename
%%      \endprooftree
%%
%% where the hypotheses may be similar structures or just formulae.
%%
%% To get a vertical string of dots instead of the proof rule, do
%%
%%      \prooftree                      which produces:
%%              [hyp]
%%      \using                                  [hyp]
%%              name                              .
%%      \proofdotseparation=1.2ex                 .name
%%      \proofdotnumber=4                         .
%%      \leadsto                                  .
%%              concl                           concl
%%      \endprooftree
%%
%% Within a prooftree, \[ and \] may be used instead of \prooftree and
%% \endprooftree; this is not permitted at the outer level because it
%% conflicts with LaTeX. Also,
%%      \Justifies
%% produces a double line. In LaTeX you can use \begin{prooftree} and
%% \end{prootree} at the outer level (however this will not work for the inner
%% levels, but in any case why would you want to be so verbose?).
%%
%% All of of the keywords except \prooftree and \endprooftree are optional
%% and may appear in any order. They may also be combined in \newcommand's
%% eg "\def\Cut{\using\sf cut\thickness.08em\justifies}" with the abbreviation
%% "\prooftree hyp1 hyp2 \Cut \concl \endprooftree". This is recommended and
%% some standard abbreviations will be found at the end of this file.
%%
%% \thickness specifies the breadth of the rule in any units, although
%% font-relative units such as "ex" or "em" are preferable.
%% It may optionally be followed by "=".
%% \proofrulebreadth=.08em or \setlength\proofrulebreadth{.08em} may also be
%% used either in place of \thickness or globally; the default is 0.04em.
%% \proofdotseparation and \proofdotnumber control the size of the
%% string of dots
%%
%% If proof trees and formulae are mixed, some explicit spacing is needed,
%% but don't put anything to the left of the left-most (or the right of
%% the right-most) hypothesis, or put it in braces, because this will cause
%% the indentation to be lost.
%%
%% By default the conclusion is centered wrt the left-most and right-most
%% immediate hypotheses (not their proofs); \shiftright or \shiftleft moves
%% it relative to this position. (Not sure about this specification or how
%% it should affect spreading of proof tree.)
%
% global assignments to dimensions seem to have the effect of stretching
% diagrams horizontally.
%
%%==========================================================================

\def\introrule{{\cal I}}\def\elimrule{{\cal E}}%%
\def\andintro{\using{\land}\introrule\justifies}%%
\def\impelim{\using{\Rightarrow}\elimrule\justifies}%%
\def\allintro{\using{\forall}\introrule\justifies}%%
\def\allelim{\using{\forall}\elimrule\justifies}%%
\def\falseelim{\using{\bot}\elimrule\justifies}%%
\def\existsintro{\using{\exists}\introrule\justifies}%%

%% #1 is meant to be 1 or 2 for the first or second formula
\def\andelim#1{\using{\land}#1\elimrule\justifies}%%
\def\orintro#1{\using{\lor}#1\introrule\justifies}%%

%% #1 is meant to be a label corresponding to the discharged hypothesis/es
\def\impintro#1{\using{\Rightarrow}\introrule_{#1}\justifies}%%
\def\orelim#1{\using{\lor}\elimrule_{#1}\justifies}%%
\def\existselim#1{\using{\exists}\elimrule_{#1}\justifies}

%%==========================================================================

\newdimen\proofrulebreadth \proofrulebreadth=.05em
\newdimen\proofdotseparation \proofdotseparation=1.25ex
\newdimen\proofrulebaseline \proofrulebaseline=2ex
\newcount\proofdotnumber \proofdotnumber=3
\let\then\relax
\def\hfi{\hskip0pt plus.0001fil}
\mathchardef\squigto="3A3B
%
% flag where we are
\newif\ifinsideprooftree\insideprooftreefalse
\newif\ifonleftofproofrule\onleftofproofrulefalse
\newif\ifproofdots\proofdotsfalse
\newif\ifdoubleproof\doubleprooffalse
\let\wereinproofbit\relax
%
% dimensions and boxes of bits
\newdimen\shortenproofleft
\newdimen\shortenproofright
\newdimen\proofbelowshift
\newbox\proofabove
\newbox\proofbelow
\newbox\proofrulename
%
% miscellaneous commands for setting values
\def\shiftproofbelow{\let\next\relax\afterassignment\setshiftproofbelow\dimen0 }
\def\shiftproofbelowneg{\def\next{\multiply\dimen0 by-1 }%
\afterassignment\setshiftproofbelow\dimen0 }
\def\setshiftproofbelow{\next\proofbelowshift=\dimen0 }
\def\setproofrulebreadth{\proofrulebreadth}

%=============================================================================
\def\prooftree{% NESTED ZERO (\ifonleftofproofrule)
%
% first find out whether we're at the left-hand end of a proof rule
\ifnum  \lastpenalty=1
\then   \unpenalty
\else   \onleftofproofrulefalse
\fi
%
% some space on left (except if we're on left, and no infinity for outermost)
\ifonleftofproofrule
\else   \ifinsideprooftree
        \then   \hskip.5em plus1fil
        \fi
\fi
%
% begin our proof tree environment
\bgroup% NESTED ONE (\proofbelow, \proofrulename, \proofabove,
%               \shortenproofleft, \shortenproofright, \proofrulebreadth)
\setbox\proofbelow=\hbox{}\setbox\proofrulename=\hbox{}%
\let\justifies\proofover\let\leadsto\proofoverdots\let\Justifies\proofoverdbl
\let\using\proofusing\let\[\prooftree
\ifinsideprooftree\let\]\endprooftree\fi
\proofdotsfalse\doubleprooffalse
\let\thickness\setproofrulebreadth
\let\shiftright\shiftproofbelow \let\shift\shiftproofbelow
\let\shiftleft\shiftproofbelowneg
\let\ifwasinsideprooftree\ifinsideprooftree
\insideprooftreetrue
%
% now begin to set the top of the rule (definitions local to it)
\setbox\proofabove=\hbox\bgroup$\displaystyle % NESTED TWO
\let\wereinproofbit\prooftree
%
% these local variables will be copied out:
\shortenproofleft=0pt \shortenproofright=0pt \proofbelowshift=0pt
%
% flags to enable inner proof tree to detect if on left:
\onleftofproofruletrue\penalty1
}

%=============================================================================
% end whatever box and copy crucial values out of it
\def\eproofbit{% NESTED TWO
%
% various hacks applicable to hypothesis list 
\ifx    \wereinproofbit\prooftree
\then   \ifcase \lastpenalty
        \then   \shortenproofright=0pt  % 0: some other object, no indentation
        \or     \unpenalty\hfil         % 1: empty hypotheses, just glue
        \or     \unpenalty\unskip       % 2: just had a tree, remove glue
        \else   \shortenproofright=0pt  % eh?
        \fi
\fi
%
% pass out crucial values from scope
\global\dimen0=\shortenproofleft
\global\dimen1=\shortenproofright
\global\dimen2=\proofrulebreadth
\global\dimen3=\proofbelowshift
\global\dimen4=\proofdotseparation
\global\count255=\proofdotnumber
%
% end the box
$\egroup  % NESTED ONE
%
% restore the values
\shortenproofleft=\dimen0
\shortenproofright=\dimen1
\proofrulebreadth=\dimen2
\proofbelowshift=\dimen3
\proofdotseparation=\dimen4
\proofdotnumber=\count255
}

%=============================================================================
\def\proofover{% NESTED TWO
\eproofbit % NESTED ONE
\setbox\proofbelow=\hbox\bgroup % NESTED TWO
\let\wereinproofbit\proofover
$\displaystyle
}%
%
%=============================================================================
\def\proofoverdbl{% NESTED TWO
\eproofbit % NESTED ONE
\doubleprooftrue
\setbox\proofbelow=\hbox\bgroup % NESTED TWO
\let\wereinproofbit\proofoverdbl
$\displaystyle
}%
%
%=============================================================================
\def\proofoverdots{% NESTED TWO
\eproofbit % NESTED ONE
\proofdotstrue
\setbox\proofbelow=\hbox\bgroup % NESTED TWO
\let\wereinproofbit\proofoverdots
$\displaystyle
}%
%
%=============================================================================
\def\proofusing{% NESTED TWO
\eproofbit % NESTED ONE
\setbox\proofrulename=\hbox\bgroup % NESTED TWO
\let\wereinproofbit\proofusing
\kern0.3em$
}

%=============================================================================
\def\endprooftree{% NESTED TWO
\eproofbit % NESTED ONE
% \dimen0 =     length of proof rule
% \dimen1 =     indentation of conclusion wrt rule
% \dimen2 =     new \shortenproofleft, ie indentation of conclusion
% \dimen3 =     new \shortenproofright, ie
%                space on right of conclusion to end of tree
% \dimen4 =     space on right of conclusion below rule
  \dimen5 =0pt% spread of hypotheses
% \dimen6, \dimen7 = height & depth of rule
%
% length of rule needed by proof above
\dimen0=\wd\proofabove \advance\dimen0-\shortenproofleft
\advance\dimen0-\shortenproofright
%
% amount of spare space below
\dimen1=.5\dimen0 \advance\dimen1-.5\wd\proofbelow
\dimen4=\dimen1
\advance\dimen1\proofbelowshift \advance\dimen4-\proofbelowshift
%
% conclusion sticks out to left of immediate hypotheses
\ifdim  \dimen1<0pt
\then   \advance\shortenproofleft\dimen1
        \advance\dimen0-\dimen1
        \dimen1=0pt
%       now it sticks out to left of tree!
        \ifdim  \shortenproofleft<0pt
        \then   \setbox\proofabove=\hbox{%
                        \kern-\shortenproofleft\unhbox\proofabove}%
                \shortenproofleft=0pt
        \fi
\fi
%
% and to the right
\ifdim  \dimen4<0pt
\then   \advance\shortenproofright\dimen4
        \advance\dimen0-\dimen4
        \dimen4=0pt
\fi
%
% make sure enough space for label
\ifdim  \shortenproofright<\wd\proofrulename
\then   \shortenproofright=\wd\proofrulename
\fi
%
% calculate new indentations
\dimen2=\shortenproofleft \advance\dimen2 by\dimen1
\dimen3=\shortenproofright\advance\dimen3 by\dimen4
%
% make the rule or dots, with name attached
\ifproofdots
\then
        \dimen6=\shortenproofleft \advance\dimen6 .5\dimen0
        \setbox1=\vbox to\proofdotseparation{\vss\hbox{$\cdot$}\vss}%
        \setbox0=\hbox{%
                \advance\dimen6-.5\wd1
                \kern\dimen6
                $\vcenter to\proofdotnumber\proofdotseparation
                        {\leaders\box1\vfill}$%
                \unhbox\proofrulename}%
\else   \dimen6=\fontdimen22\the\textfont2 % height of maths axis
        \dimen7=\dimen6
        \advance\dimen6by.5\proofrulebreadth
        \advance\dimen7by-.5\proofrulebreadth
        \setbox0=\hbox{%
                \kern\shortenproofleft
                \ifdoubleproof
                \then   \hbox to\dimen0{%
                        $\mathsurround0pt\mathord=\mkern-6mu%
                        \cleaders\hbox{$\mkern-2mu=\mkern-2mu$}\hfill
                        \mkern-6mu\mathord=$}%
                \else   \vrule height\dimen6 depth-\dimen7 width\dimen0
                \fi
                \unhbox\proofrulename}%
        \ht0=\dimen6 \dp0=-\dimen7
\fi
%
% set up to centre outermost tree only
\let\doll\relax
\ifwasinsideprooftree
\then   \let\VBOX\vbox
\else   \ifmmode\else$\let\doll=$\fi
        \let\VBOX\vcenter
\fi
% this \vbox or \vcenter is the actual output:
\VBOX   {\baselineskip\proofrulebaseline \lineskip.2ex
        \expandafter\lineskiplimit\ifproofdots0ex\else-0.6ex\fi
        \hbox   spread\dimen5   {\hfi\unhbox\proofabove\hfi}%
        \hbox{\box0}%
        \hbox   {\kern\dimen2 \box\proofbelow}}\doll%
%
% pass new indentations out of scope
\global\dimen2=\dimen2
\global\dimen3=\dimen3
\egroup % NESTED ZERO
\ifonleftofproofrule
\then   \shortenproofleft=\dimen2
\fi
\shortenproofright=\dimen3
%
% some space on right and flag we've just made a tree
\onleftofproofrulefalse
\ifinsideprooftree
\then   \hskip.5em plus 1fil \penalty2
\fi
}

%==========================================================================
% IDEAS
% 1.    Specification of \shiftright and how to spread trees.
% 2.    Spacing command \m which causes 1em+1fil spacing, over-riding
%       exisiting space on sides of trees and not affecting the
%       detection of being on the left or right.
% 3.    Hack using \@currenvir to detect LaTeX environment; have to
%       use \aftergroup to pass \shortenproofleft/right out.
% 4.    (Pie in the sky) detect how much trees can be "tucked in"
% 5.    Discharged hypotheses (diagonal lines).


\begin{document}

\begin{frame}
  \titlepage
\end{frame}

% Uncomment these lines for an automatically generated outline.
%\begin{frame}{Outline}
%  \tableofcontents
%\end{frame}

\section{Espacios de derivación}




\begin{frame}{Espacios de derivación}

\textbf{Aritmética elemental y pares ordenados}
\[
\begin{array}{rclc}
  \cArith{\uline{n} + \uline{m}} & \to & \cArith{\uline{n + m}} & \\
  \cArith{\uline{n} \cdot \tm  } & \to & \underbrace{\cArith{\tm + \tm + ... + \tm}}_{n \text{ veces}} & \text{ si } n > 0 \\
  \cArith{\uline{0} \cdot \tm  } & \to & \cArith{\uline{0}} & \\
  & & & \\
  & & & \\
  \cArith{(\tm, \tmtwo)} & \to & \cArith{(\tm', \tmtwo)} & \text{ si } \cArith{\tm} \to \cArith{\tm'} \\
  \cArith{(\tm, \tmtwo)} & \to & \cArith{(\tm, \tmtwo')} & \text{ si } \cArith{\tmtwo} \to \cArith{\tmtwo'} \\
\end{array}
\]


\end{frame}

\begin{frame}{Espacios de derivación}

$
\ulbDeriv{\cArith{\uline{1} + \uline{1}}} =
  \bigg(\cArith{\uline{1} + \uline{1}} \to \cArith{\uline{2}}\bigg)
$

$\ulbDeriv{\cArith{\uline{0} \cdot (\uline{5} + \uline{5})}} =
\xymatrix@C=0.3cm{
  \cArith{\uline{0} \cdot (\uline{5} + \uline{5})} \ar[rr] \ar[dr] &  & \cArith{\uline{0} \cdot \uline{10}} \ar[dl] \\
  & \cArith{\uline{0}}   &
}
$
{\footnotesize
$\ulbDeriv{\cArith{(\uline{1} + \uline{1}, \uline{0} \cdot (\uline{5} + \uline{5}))}} =
\xymatrix@C=0.2cm{
  \cArith{(\uline{1} + \uline{1}, \uline{0} \cdot (\uline{5} + \uline{5}))} \ar[rr] \ar[dr] \ar[dd]&  & \cArith{(\uline{1} + \uline{1}, \uline{0} \cdot \uline{10})} \ar[dd] \ar[dl] \\
  & \cArith{(\uline{1} + \uline{1}, \uline{0})}  \ar[dd] & \\
  \cArith{(\uline{2}, \uline{0} \cdot (\uline{5} + \uline{5}))} \ar[rr]|!{[r]}\hole \ar[dr] &  & \cArith{(\uline{2}, \uline{0} \cdot \uline{10})} \ar[dl] \\
  & \cArith{(\uline{2}, \uline{0})}   & \\
}$
}

$\ulbDeriv{\cArith{(A,B)}} \simeq \ulbDeriv{\cArith{A}} \times \ulbDeriv{\cArith{B}}$

\end{frame}

\begin{frame}{El cálculo-$\clambda$}

\textbf{Términos del cálculo-$\clambda$}

\[
  \cLam{t} ::= \cLam{x}\  |\ \cLam{t\, t} \ |\ \cLam{\lam{x}{t}}
\]

Por ejemplo,
\begin{itemize}
  \item[] $\cLam{\lam{x}{x}}$
  \item[] $\cLam{(\lam{x}{z})\ y}$
  \item[] $\cLam{(\lam{x}{\lam{y}{x}})\ z\ w}$ = $\cLam{((\lam{x}{(\lam{y}{x})})\ z)\ w}$
\end{itemize}

\vskip 0.5cm

\textbf{Variables libres}
\[
\begin{array}{rclc}
  \fv{\cLam{x}} & \eqdef & \cLam{x} & \\
  \fv{\cLam{u\, v}} & \eqdef & \fv{\cLam{u}} \cup \fv{\cLam{v}} & \\
  \fv{\cLam{\lam{x}{u}}} & \eqdef & \fv{\cLam{u}} - \set{\cLam{x}} & \\
\end{array}
\]
\end{frame}


\begin{frame}{El cálculo-$\clambda$}

\textbf{$\beta$-reducción}

\[
  \cLam{(\lam{x}{t})\, s} \to_\cLam{\beta} \cLam{\subs{t}{x}{s}}
\]
\vskip 0.5cm


\textbf{...donde sustituir significa}
\[
\begin{array}{rclc}
  \cLam{\subs{x}{x}{s}}            & \eqdef & \cLam{s} & \\
  \cLam{\subs{y}{x}{s}}            & \eqdef & \cLam{y} & \\
  \cLam{\subs{(u\, v)}{x}{s}}      & \eqdef & \cLam{\subs{u}{x}{s}\, \subs{v}{x}{s}} & \\
  \cLam{\subs{(\lam{y}{u})}{x}{s}} & \eqdef & \cLam{\lam{y}{\subs{u}{x}{s}}} & \text{ si } \cLam{x}\neq \cLam{y} \text{ e } \cLam{y} \not\in \fv{\cLam{s}} \\
\end{array}
\]

\textbf{Por ejemplo}

  \[\cLam{(\lam{x}{x}) ((\lam{y}{y}) z)} \to \cLam{(\lam{y}{y}) z} \to \cLam{z}\]


\end{frame}

\begin{frame}{Espacios de derivación en el cálculo-$\clambda$}
\[\ulbDeriv{\cLam{(\lam{x}{x}) ((\lam{y}{y}) z)}} =
\xymatrix@C=0.8cm{
\cLam{(\lam{x}{x}) ((\lam{y}{y}) z)} \ar[d] \ar[r] & \cLam{(\lam{y}{y}) z} \ar[d] \\
\cLam{(\lam{x}{x}) z}  \ar[r] & \cLam{z}
}
\]
\end{frame}

\begin{frame}{Espacios de derivación en el cálculo-$\clambda$ -- Problemas}
\textbf{Creación}
\[\ulbDeriv{\cLam{(\lam{x}{x x}) (\lam{x}{x x})}} =
\xymatrix@C=0.3cm{
\cLam{(\lam{x}{x x}) (\lam{x}{x x})} \ar[d] \\
\cLam{(\lam{x}{x x}) (\lam{x}{x x})} \ar[d] \\
\cLam{(\lam{x}{x x}) (\lam{x}{x x})} \ar[d] \\
\vdots  \\
}
\]
$\ulbDeriv{\cLam{(\lam{x}{x x}) (\lam{x}{x x})}}$ es infinito, mientras que
$\ulbDeriv{\cLam{(\lam{x}{x x})}}$ es finito.
\end{frame}

\begin{frame}{Espacios de derivación en el cálculo-$\clambda$ -- Problemas}
\textbf{Duplicación}
\[\ulbDeriv{\cLam{(\lam{x}{x x})\ (I\, z)}} =
\xymatrix@C=0.3cm{
& & \cLam{(\lam{x}{x x})\ (I\, z)} \ar[dr] \ar[dl] & \\
& \cLam{(I\, z)\ (I\, z)} \ar[dr] \ar[dl] & & \cLam{(\lam{x}{x x})\ z} \ar@/^2.0pc/[ddll]\\
\cLam{z\ (I\, z)} \ar[dr] & & \cLam{(I\, z)\ z} \ar[dl] & \\
 & \cLam{z\ z} & & \\
}
\]

{\footnotesize
$\ulbDeriv{\cLam{(\lam{x}{x x})\ (I\, z)}}
\not\simeq
\ulbDeriv{\cLam{(\lam{x}{x x})\ \conbase}} \times \ulbDeriv{\cLam{I\, z}}
=
\underbrace{\bigg(\cLam{(\lam{x}{x x})\ \conbase} \to \cLam{\conbase \, \conbase}\bigg) \times \bigg(\cLam{I\, z} \to \cLam{z}\bigg)}_{\xymatrix@=0.2cm{
& \bullet \ar[dr] \ar[dl] & \\
\bullet \ar[dr] & & \bullet \ar[dl] \\
& \bullet &}}$
}
\end{frame}

\begin{frame}{Espacios de derivación en el cálculo-$\clambda$ -- Problemas}
\textbf{Borrado}
\[\ulbDeriv{\cLam{(\lam{x}{y})\ (I\, z)}} =
\xymatrix@C=0.3cm{
\cLam{(\lam{x}{y})\ (I\, z)} \ar[rr] \ar[dr] & & \cLam{(\lam{x}{y})\ z} \ar[dl]\\
& \cLam{y} & \\
}
\]

{\footnotesize
$\ulbDeriv{\cLam{(\lam{x}{y})\ (I\, z)}}
\not\simeq
\ulbDeriv{\cLam{(\lam{x}{y})\ \conbase}} \times \ulbDeriv{\cLam{I\, z}}
=
\underbrace{\bigg(\cLam{(\lam{x}{y})\ \conbase} \to \cLam{y}\bigg) \times \bigg(\cLam{I\, z} \to \cLam{z}\bigg)}_{\xymatrix@=0.2cm{
& \bullet \ar[dr] \ar[dl] & \\
\bullet \ar[dr] & & \bullet \ar[dl] \\
& \bullet &}}$
}
\end{frame}

\begin{frame}{Residuos en el cálculo-$\clambda$}
\begin{defes}
Sean $R$, $S$ dos pasos coiniciales. Definimos el residuo de $R$ después de $S$ como lo que queda
de el paso $R$ después de hacer $S$: es un conjunto de pasos que salen del target de $S$.
Lo escribimos como $R/S$.

Formalmente, se puede definir con posiciones o etiquetas.
\end{defes}
\vskip 0.2cm
La definición se puede extender para derivaciones: $\rho/\sigma$ es lo que
queda de $\rho$ después hacer hacer $\sigma$.

\vskip 0.7cm

Para ordenar las derivaciones que salen de un mismo término, usamos el orden del prefijo.

\begin{defes}[Orden del prefijo]
\[[\rho] \permle [\sigma]   \iffdef \rho / \sigma = \epsilon\]
\end{defes}
\end{frame}

\begin{frame}{Espacios de derivación en el cálculo-$\clambda$}
\begin{defes}[Equivalencia por permutaciones]
  Decimos que dos secuencias de reducción $\rho, \sigma$ son \defn{equivalentes por permutación} si $\rho/\sigma = \epsilon$ y $\sigma/\rho = \epsilon$.
Lo escribimos como $\rho \permeq \sigma$.
\end{defes}

\vskip 0.7cm

\begin{defes}[Espacio de derivación]
Si $\cLam{\tm}$ es un término, $\ulbDeriv{\cLam{\tm}}$ es el conjunto de
\defn{secuencias de reducción} desde $\cLam{\tm}$:
\[
  \set{\rho \ | \ \rho : \cLam{t} \to^* \cLam{s} \text{ es una secuencia de pasos de reescritura}} \bigg/ \permeq
\]
\end{defes}
\end{frame}

\begin{frame}{Espacios de derivación en el cálculo-$\clambda$}

\begin{defes}[Reticulado]
  Un \defn{reticulado} es un conjunto parcialmente ordenado $(A, \leq)$ en el cual
todo par de elementos tiene un supremo (mínima cota superior)
y un ínfimo (máxima cota inferior).
\end{defes}

\vskip 0.5cm

\begin{teoes}[J.-J. Lévy]
En el cálculo-$\clambda$, $\ulbDeriv{\cLam{t}}$ forma un \defn{semi-reticulado con supremos},
donde:
\[[\rho] \sqcup [\sigma] \eqdef [\rho (\sigma / \rho)] \]
\noindent $\ulbDeriv{\cLam{t}}$ no necesariamente es un reticulado.
\end{teoes}

\end{frame}


\begin{frame}{Objetivo}
Queremos \textbf{entender los espacios de derivación del cálculo-$\lambda$.}
\vskip 0.2cm
Creemos que \textbf{explicitar el manejo de recursos puede ser útil.}
\vskip 0.5cm

$\ulbDeriv{\cProto{(\lam{x}{x x}) [I\, z, I\, z]}} = $
\[\xymatrix@=0.3cm{
& \cProto{(\lam{x}{x x}) [I\, z, I\, z]} \ar[rr] \ar[dd]|!{[d]}\hole \ar[dl]
& & \cProto{(\lam{x}{x x}) [z, I\, z]} \ar[dd] \ar[dl]
\\
\cProto{(\lam{x}{x x}) [I\, z, z]} \ar[rr]\ar[dd]
& & \cProto{(\lam{x}{x x}) [z, z]} \ar[dd]
\\
& \cProto{(I\, z)\ (I\, z)} \ar[rr]|!{[r]}\hole \ar[dl]
& & \cProto{z\ (I\, z)} \ar[dl]
\\
\cProto{(I\, z)\ z} \ar[rr]
& & \cProto{z\ z}
}\]
\end{frame}


\section{El cálculo-$\lambda$ distributivo}


\begin{frame}[fragile]{El cálculo-$\lambda$ distributivo ($\clambdadist$)}
\begin{itemize}
\item En sistemas de tipos intersección, distintas ocurrencias ligadas de una misma variable
pueden tener distintos tipos.
\item La no-idempotencia de la intersección nos da la capacidad de manejar cómo se usan los recursos.
\item Nos basamos en el sistema $\mathcal{W}$, un sistema de tipos intersección no-idempotente.
\end{itemize}

\textbf{Idea:}

\begin{lstlisting}[language=Python]
def f(x):
  return x * x + x(100)
\end{lstlisting}

El parámetro se usa con dos tipos distintos.
En consecuencia, el tipo de $f$ es
$[\text{Int}, \text{Int}, \text{Int} \to \text{Int}] \to \text{Int}$.
\end{frame}

\begin{frame}{El cálculo-$\lambda$ distributivo ($\clambdadist$)}
\begin{defes}[$\clambdadist$ ``na\"if'']
\textbf{Sintaxis}
{\footnotesize
\[\arraycolsep=4pt
\begin{array}{llllllll}
\text{Términos} & \cDist{t} & ::= & \cDist{x^A}\ |\ \cDist{t \, \ls{t}}\ |\ \cDist{\lam{x}{t}} & \text{Listas de términos} & \cDist{\ls{t}} & ::= & \cDist{[t_1, ..., t_n]} \\
\text{Tipos} & \cDist{A} & ::= & \cDist{\alpha}\ |\ \cDist{\mathcal{M} \to A} & \text{Multiconjuntos de tipos} & \cDist{\mathcal{M}} & ::= & \cDist{[A_1, ..., A_n]} \\
\text{Contextos} & \cDist{\tctx} & ::= & \cDist{(.)}\ |\ \cDist{\tctx, \var : \mathcal{M}} & & & & \\
\end{array}
\]
}
\textbf{Tipado}
{\scriptsize
\[
  \indrule{}{
  }{
    \cDist{\var} : \cDist{\lset{A}} \vdash \cDist{\var^A} : \cDist{A}
  }
  \indrule{}{
    \cDist{\tctx \oplus \var} : \cDist{\mtyp} \vdash \cDist{\tm} : \cDist{A}
  }{
    \cDist{\tctx} \vdash \cDist{\lam{\var}{\tm}} : \cDist{\mtyp \to A}
  }
  \indrule{}{
    \cDist{\tctx} \vdash \cDist{\tm} : \cDist{\lset{A_1,\hdots,A_n} \to A}
    \hspace{10pt}
    \left( \cDist{\tctxtwo_i} \vdash \cDist{\tmtwo_i} : \cDist{A_i} \right)_{i=1}^{n}
  }{
    \cDist{\tctx +_{i=1}^{n} \tctxtwo_i} \vdash \cDist{\tm[\tmtwo_1,\hdots,\tmtwo_n]} : \cDist{A}
  }
\]
}
\textbf{Reducción}
\[ \cDist{(\lam{x}{t}) [s_1, ..., s_n]} \todist \cDist{\subs{t}{x}{[s_1, ..., s_n]}} \]
\end{defes}
\end{frame}


\begin{frame}{El cálculo-$\lambda$ distributivo ($\clambdadist$)}
\textbf{Problema! (no es confluente)}
\[
\xymatrix@=0.5cm{
  & \cDist{(\lam{x}{(\lam{y}{f\, y\, x})\, x})\ [a, b]} \ar[dr] \ar[dl]& \\
\cDist{(\lam{y}{f\, y\, a})\, b} \ar@{.>>}[dr] &  & \cDist{(\lam{y}{f\, y\, b})\, a} \ar@{.>>}[dl] \\
  & \cDist{?} & \\
}
\]
\end{frame}


\begin{frame}{El cálculo-$\lambda$ distributivo ($\clambdadist$)}
\begin{defes}[$\clambdadist$]
\textbf{Sintaxis}
{\footnotesize
\[\arraycolsep=4pt
\begin{array}{llllllll}
\text{Términos} & \cGrey{t} & ::= & \cGrey{x^A}\ |\ \cGrey{t \, \ls{t}}\ |\ \cGrey{\lamp{\cDist{\lab}}{x}{t}} & \text{Listas de términos} & \cGrey{\ls{t}} & ::= & \cGrey{[t_1, ..., t_n]} \\
\text{Tipos} & \cGrey{A} & ::= & \cGrey{\alpha^\cDist{\lab}}\ |\ \cGrey{\mathcal{M} \tolab{\cDist{\lab}} A} & \text{Multiconjuntos de tipos} & \cGrey{\mathcal{M}} & ::= & \cGrey{[A_1, ..., A_n]} \\
\text{Contextos} & \cGrey{\tctx} & ::= & \cGrey{(.)}\ |\ \cGrey{\tctx, \var : \mathcal{M}} & & & & \\
\end{array}
\]
}
\textbf{Tipado}
{\scriptsize
\[
  \indrule{}{
  }{
    \cGrey{\var} : \cGrey{\lset{A}} \vdash \cGrey{\var^A} : \cGrey{A}
  }
  \indrule{}{
    \cGrey{\tctx \oplus \var} : \cGrey{\mtyp} \vdash \cGrey{\tm} : \cGrey{A}
  }{
    \cGrey{\tctx} \vdash \cGrey{\lamp{\cDist{\lab}}{\var}{\tm}} : \cGrey{\mtyp \tolab{\cDist{\lab}} A}
  }
  \indrule{}{
    \cGrey{\tctx} \vdash \cGrey{\tm} : \cGrey{\lset{A_1,\hdots,A_n} \tolab{\cDist{\lab}} B}
    \hspace{10pt}
    \left( \cGrey{\tctxtwo_i} \vdash \cGrey{\tmtwo_i} : \cGrey{A_i} \right)_{i=1}^{n}
  }{
    \cGrey{\tctx +_{i=1}^{n} \tctxtwo_i} \vdash \cGrey{\tm[\tmtwo_1,\hdots,\tmtwo_n]} : \cGrey{B}
  }
\]
}
\textbf{Reducción}
\[ \cGrey{(\lam{x}{t}) [s_1, ..., s_n]} \todistl{\cDist{\lab}} \cGrey{\subs{t}{x}{[s_1, ..., s_n]}} \]
\end{defes}
La reducción es orientada por tipos.

\textbf{Ejemplo.} $\cDist{(\lamp{1}{x}{x^{\alpha^2}})\,[y^{\alpha^2}]}$

\end{frame}


\begin{frame}{El cálculo-$\lambda$ distributivo ($\clambdadist$)}
\begin{defes}[Términos correctos]
Decimos que un término tipable $\cDist{t}$ es \defn{correcto} si:
\begin{itemize}
\item Distintos lambdas tienen distintas etiquetas.
\item Para todo multiconjunto de tipos $\cDist{[A_1, ..., A_n]}$ que ocurra como una subfórmula
en cualquier lugar de la derivación de tipos de $\cDist{t}$, si $i\neq j$ entonces
$\cDist{A_i}$ y $\cDist{A_j}$ están decorados con distintas etiquetas en la raíz.
\end{itemize}
\end{defes}

\begin{comes}[Tipado único]
Si $\cDist{\tctx} \vdash \cDist{t} : \cDist{A}$ es derivable y $\cDist{t}$ correcto,
entonces es la única derivación de tipo para $\cDist{t}$.
\end{comes}

\begin{lemes}[\emph{Subject reduction}]
Si $\cDist{\tctx} \vdash \cDist{t} : \cDist{A}$, el término $\cDist{t}$
es correcto y $\cDist{t} \to_{\cDist{\dist}} \cDist{s}$,
entonces $\cDist{\tctx} \vdash \cDist{s} : \cDist{A}$ y $\cDist{s}$ es correcto.
\end{lemes}
\end{frame}


\begin{frame}{El cálculo-$\lambda$ distributivo ($\clambdadist$)}
\begin{proes}[Confluencia]
El cálculo-$\clambdadist$ cumple la propiedad de Church--Rosser.
\end{proes}

\vskip 1cm
\textbf{En el ejemplo que antes no funcionaba:}
\[
\xymatrix{
  \cDist{(\lamp{\cLab1}{x}{(\lamp{\cLab2}{y}{f^{\cLab3}\,[x^{\cLab4}, y^{\cLab5}]}) [x^{\cLab5}]})\ [a^{\cLab5}, b^{\cLab4}]} \ar^-{2}[d] \ar^-{1}[r] & \cDist{(\lamp{\cLab2}{y}{f^{\cLab3}\,[b^{\cLab4}, y^{\cLab5}]}) [a^{\cLab5}]} \ar^-{2}[d] \\
  \cDist{(\lamp{\cLab1}{x}{f^{\cLab3}\,[x^{\cLab4}, x^{\cLab5}]})\ [a^{\cLab5}, b^{\cLab4}]} \ar^-{1}[r] & \cDist{f^{\cLab3}\,[b^{\cLab4}, a^{\cLab5}]}\\
}
\]
\end{frame}

\begin{frame}{El cálculo-$\lambda$ distributivo ($\clambdadist$)}
\begin{proes}[Fuertemente normalizante]
No hay secuencias de reducción infinitas: $\cDist{t_1} \to_\cdist \cDist{t_2} \to_\cdist ...$.
\end{proes}
\begin{defes}[Residuo]
  Los \defn{residuos} pueden definirse en $\clambdadist$ utilizando las etiquetas de los lambdas.
\end{defes}
\begin{proes}[Ortogonalidad \lbrack cf. P.-A. Melli\`es\rbrack]
El cálculo-$\clambdadist$ es un sistema de reescritura abstracto ortogonal.
\end{proes}
\begin{lemes}
No hay duplicación ni borrado en $\clambdadist$.
\end{lemes}
\begin{proes}
En el cálculo-$\clambdadist$, $\ulbDeriv{\cDist{t}}$ es un reticulado distributivo, a saber:
\begin{itemize}
\item existen supremos e ínfimos para cada par de reducciones, y
\item las operaciones de \emph{join} ($\sqcup$) y \emph{meet} ($\sqcap$) distribuyen
una sobre la otra.
\end{itemize}
\end{proes}
\end{frame}

\section{Simulación}
\begin{frame}{Simulación}
\begin{defes}[Refinamiento]
Damos una forma de relacionar términos correctos del cálculo-$\clambdadist$ y el cálculo-$\clambda$.
{\small
\[
  \indrule{}{
  }{
    \cDist{\var^\typ} \refines \cLam{\var}
  }
  \indrule{}{
    \cDist{\tm'} \refines \cLam{\tm}
  }{
    \cDist{\lamp{\lab}{\var}{\tm'}} \refines \cLam{\lam{\var}{\tm}}
  }
  \indrule{}{
    \cDist{\tm'} \refines \tm
    \HS
    \cDist{\tmtwo_i} \refines \cLam{\tmtwo} \text{ para cada } i\in\set{1, ..., n}
  }{
    \cDist{\tm'[\tmtwo_1,\hdots,\tmtwo_n]} \refines \cLam{\tm\,\tmtwo}
  }
\]
}
\end{defes}

Un término $\clambda$ puede tener muchos refinamientos:
\begin{itemize}
  \item[] $\cDist{\lamp{1}{x}{x^2\,[\ ]}} \refines \cLam{\lam{x}{x\,x}}$
  \item[] $\cDist{\lamp{1}{x}{x^2\,[x^3]}} \refines \cLam{\lam{x}{x\,x}}$
  \item[] $\cDist{\lamp{1}{x}{x^2\,[x^3, x^4]}} \refines \cLam{\lam{x}{x\,x}}$
  \item[] $\dots$
\end{itemize}
También puede no tener ninguno, como $\cLam{\Omega} = \cLam{(\lam{x}{x\,x})\ (\lam{x}{x\,x})}$.
\end{frame}

\begin{frame}{Simulación}
\begin{proes}[Simulación]
\textbf{Del $\clambda$ por el $\clambdadist$}.
Si $\cDist{\tm'} \refines \cLam{\tm} \to_{\cLam{\beta}} \cLam{\tmtwo}$,
entonces existe $\cDist{\tmtwo'}$ tal que:
\[
\xymatrix@=1cm{
 \cLam{\tm} \ar@{}[d]|*=0[@]{\rtimes} \ar[r]^{\cLam{\beta}} & \cLam{\tmtwo} \ar@{}[d]|*=0[@]{\rtimes} \\
 \cDist{\tm'} \ar@{.>>}[r]^{\cDist{\dist}} & \cDist{\tmtwo'} \\
}
\]

\textbf{Del $\clambdadist$ por el $\clambda$}.
Si $\cLam{\tm} \refinedby \cDist{\tm'} \to_{\cDist{\dist}} \cDist{\tmtwo'}$,
entonces existen $\cLam{\tmtwo}$ y $\cDist{\tmtwo''}$ tal que:
\[
\xymatrix@=1cm{
 \cLam{\tm} \ar@{}[d]|*=0[@]{\rtimes} \ar@{.>>}[rr]^{\cLam{\beta}} & & \cLam{\tmtwo} \ar@{}[d]|*=0[@]{\rtimes} \\
 \cDist{\tm'} \ar[r]^{\cDist{\dist}} & \cDist{\tmtwo'} \ar@{.>>}[r]^{\cDist{\dist}} & \cDist{\tmtwo''} \\
}
\]
\end{proes}
\end{frame}

\begin{frame}{Simulación -- \emph{Head normal forms}}
\begin{defes}[\emph{Head normal form}]
  Un término del cálculo-$\clambda$ está en \textbf{\emph{head normal form}} si no tiene redexes debajo de un contexto head:
\[ \mathtt{H} ::= \conbase \ | \ \lam{x}{\mathtt{H}} \ | \ \mathtt{H}\, t \]
Se define análogamente para el cálculo-$\clambdadist$.
\end{defes}

\begin{proes}[{\footnotesize La refinabilidad caracteriza la tenencia de \emph{head normal forms}}]
  Dado $\cLam{t}$ un término del cálculo-$\clambda$, son equivalentes:
\begin{enumerate}
\item El término $\cLam{t}$ tiene una \emph{head normal form}.
\item Existe un término $\cDist{t'}$ del cálculo-$\clambdadist$ tal que $\cDist{t'} \refines \cLam{t}$.
\end{enumerate}
\end{proes}
\end{frame}

\begin{frame}{Simulación}
\begin{proes}[Simulación algebraica]
Para cada refinamiento $\cDist{t'} \refines \cLam{t}$, la construcción dada por el resultado
anterior es un morfismo de semirreticulados:
\[
\begin{array}{ccc}
  \ulbDeriv{\cLam{t}} & \to & \ulbDeriv{\cDist{t'}} \\
  \rho & \mapsto & \rho / \cDist{t'} \\
\end{array}
\]
\end{proes}

\textbf{Ejemplo.}
Sean $\cLam{I} = \cLam{\lam{x}{x}}$ y
$\cLam{t} = \cLam{(\lam{\var}{\var\var})\,(I\, \varthree)}$.
Es refinado por $\cDist{t'} = \cDist{(\lamp{1}{x}{x^2 [\,]}) [(\lamp{5}{x}{x^2}) [z^2]]}$.
{\scriptsize
\[
  \xymatrix@=.2cm{
  &
    \cLam{(\lam{\var}{\var\var})\,(I\, \varthree)}
    \ar@/_.25cm/[dl]_-{\redex_1}
    \ar@/^.25cm/[dr]^-{\redextwo}
  &
  &
  \\
    \cLam{(I\, \varthree)\, (I\, \varthree)}
    \ar[d]_-{\redextwo_{11}}
    \ar[r]^-{\redextwo_{21}}
  &
    \cLam{(I\varthree)\varthree}
    \ar[d]^{\redextwo_{12}}
  &
    \cLam{(\lam{\var}{\var\var})\varthree}
    \ar@/^.25cm/[dl]^-{\redex_2}
  \\
    \cLam{\varthree (I\varthree)}
    \ar[r]_-{\redextwo_{22}}
  &
    \cLam{ \varthree \varthree}
  }\hspace{-0.4cm}\xymatrix@=.2cm{
  &
    \cDist{(\lamp{1}{x}{x^2 [\,]}) [(\lamp{5}{x}{x^2}) [z^2]]}
    \ar@/_.25cm/[dl]_-{\redex'_1}
    \ar@/^.25cm/[dr]^-{\redextwo'}
  \\
    \cDist{(\lamp{5}{x}{x^2}) [z^2] [\,]}
    \ar@/_.25cm/[dr]_-{\redextwo'_1}
  &
  &
    \cDist{\hspace{-1cm}(\lamp{1}{x}{x^2 [\,]}) [z^2]}
    \ar@/^.25cm/[dl]^-{\redex'_2}
  \\
  &
    \cDist{z^2 [\,]}
  }
\]
}
\end{frame}

\begin{frame}{Basura}
\begin{defes}[Basura]
Sea $\cDist{t'} \refines \cLam{t}$.
Una derivación $\rho : \cLam{t} \to_{\cLam{\beta}}^* \cLam{s}$ es
\defn{$\cDist{t'}$-basura} si $\rho / \cDist{t'} = \epsilon$.
\end{defes}

\begin{defes}[Libre de basura]
Sea $\cDist{t'} \refines \cLam{t}$.
Una derivación $\rho : \cLam{t} \to_{\cLam{\beta}}^* \cLam{s}$ es
\defn{$\cDist{t'}$-libre de basura} si para cada $\sigma \permle \rho$, 
si $\rho / \sigma$ es $(\cDist{t'}/\sigma)$-basura, entonces $\rho/\sigma = \epsilon$.
\end{defes}

Notemos que las definiciones dependen de la elección de $\cDist{t'}$.
\textbf{Ejemplo.} Los pasos punteados son basura.

{\scriptsize
\[
  \xymatrix@C=.1cm{
  &
    \cLam{(\lam{\var}{\var\var})\,(I\, \varthree)} \refinedby \cDist{(\lamp{1}{x}{x^2 [\,]}) [(\lamp{5}{x}{x^2}) [z^2]]}
    \ar@/_.25cm/[dl]
    \ar@/^.25cm/[dr]
  &
  &
  \\
    \cLam{(I\, \varthree)\, (I\, \varthree)} \refinedby \cDist{(\lamp{5}{x}{x^2}) [z^2] [\,]}
    \ar[d]
    \ar@{.>}[r]
  &
    \cLam{(I\varthree)\varthree} \refinedby \cDist{(\lamp{5}{x}{x^2}) [z^2] [\,]}
    \ar[d]
  &
    \hspace{-1cm}\cLam{(\lam{\var}{\var\var})\varthree} \refinedby \cDist{(\lamp{1}{x}{x^2 [\,]}) [z^2]}
    \ar@/^.25cm/[dl]
  \\
    \cLam{\varthree (I\varthree)} \refinedby \cDist{z^2 [\,]}
    \ar@{.>}[r]
  &
    \cLam{ \varthree \varthree} \refinedby \cDist{z^2 [\,]}
}
\]
}
\end{frame}

\begin{frame}{Basura y factorización}
\begin{teoes}[Factorización]
Si $\cDist{t'} \refines \cLam{t}$, existe un isomorfismo de semirreticulados:
\[\ulbDeriv{\cLam{t}} \simeq \int_{\ulbF}{\ulbG}\]
\noindent donde:
\begin{itemize}
\item $\ulbF$ es el reticulado de derivaciones $\cDist{t'}$-libres de basura.
\item $\ulbG : \ulbF \to \mathsf{Semilattice}$ es un funtor que a cada
  derivación libre de basura $\rho : \cLam{t} \to_{\cLam{\beta}}^* \cLam{s}$ 
  en $\ulbF$ le asigna el semirreticulado de derivaciones basura que salen
  de $\cLam{s}$ (que son las $(\cDist{t'}/\rho)$-basura).
\item $\int_{\ulbF}{\ulbG}$ es la construcción de Grothendieck.
\end{itemize}
\end{teoes}
  \begin{cores}
Toda derivación $\rho$ de $\cLam{t}$
se puede factorizar de manera única en $\rho \permeq \rho_1 \rho_2$ donde
$\rho_1$ es libre de basura y $\rho_2$ es basura.
\end{cores}
\end{frame}

\begin{frame}{Factorización -- Ejemplo}
Sea $\cLam{t} = \cLam{(\lam{\var}{\var\var})\,(I\, \varthree)}$.
Lo refinaba $\cDist{t'} = \cDist{(\lamp{1}{x}{x^2 [\,]}) [(\lamp{5}{x}{x^2}) [z^2]]}$.
{\scriptsize
\[
  \xymatrix@=.2cm{
  &
    \cLam{(\lam{\var}{\var\var})\,(I\, \varthree)}
    \ar@/_.25cm/[dl]_-{\redex_1}
    \ar@/^.25cm/[dr]^-{\redextwo}
  &
  &
  \\
    \cLam{(I\, \varthree)\, (I\, \varthree)}
    \ar[d]_-{\redextwo_{11}}
    \ar@{.>}[r]^-{\redextwo_{21}}
  &
    \cLam{(I\varthree)\varthree}
    \ar[d]^{\redextwo_{12}}
  &
    \cLam{(\lam{\var}{\var\var})\varthree}
    \ar@/^.25cm/[dl]^-{\redex_2}
  \\
    \cLam{\varthree (I\varthree)}
    \ar@{.>}[r]_-{\redextwo_{22}}
  &
    \cLam{ \varthree \varthree}
  }\hspace{-0.4cm}\xymatrix@=.2cm{
  &
    \cDist{(\lamp{1}{x}{x^2 [\,]}) [(\lamp{5}{x}{x^2}) [z^2]]}
    \ar@/_.25cm/[dl]_-{\redex'_1}
    \ar@/^.25cm/[dr]^-{\redextwo'}
  \\
    \cDist{(\lamp{5}{x}{x^2}) [z^2] [\,]}
    \ar@/_.25cm/[dr]_-{\redextwo'_1}
  &
  &
    \cDist{\hspace{-1cm}(\lamp{1}{x}{x^2 [\,]}) [z^2]}
    \ar@/^.25cm/[dl]^-{\redex'_2}
  \\
  &
    \cDist{z^2 [\,]}
  }
\]
}

\[
  \xymatrix@C=1cm{
  &
    ([\epsilon], [\epsilon])
    \ar@/_.25cm/[dl]_-{\redex_1}
    \ar@/^.25cm/[dr]^-{\redextwo}
  &
  &
  \\
    ([\redex_1], [\epsilon])
    \ar[d]_-{\redextwo_{11}}
    \ar@{.>}[r]^-{\redextwo_{21}}
  &
    ([\redex_1], [\redextwo_{21}])
    \ar[d]^{\redextwo_{12}}
  &
    ([\redextwo], [\epsilon])
    \ar@/^.25cm/[dl]^-{\redex_2}
  \\
    ([\redex_1 \redextwo_{11}], [\epsilon])
    \ar@{.>}[r]_-{\redextwo_{22}}
  &
    ([\redex_1 \redextwo_{11}], [\redextwo_{22}])
  }
\]
\end{frame}

\section{Some \LaTeX{} Examples}

\subsection{Final}
\begin{frame}{Trabajo futuro}
\begin{itemize}
  \item Estudiar la relación de los términos que refinan a un $\cLam{t}$ con la noción de
    aproximantes de \emph{head normal forms}.
  \item Intentar hacer lo mismo con otros cálculos de recursos.
  \item Relacionar que la factorización dada con la factorización interna--externa de Melliés.
  \item Obtener resultados cuantitativos con la teoría desarrollada.
\end{itemize}
\end{frame}

\end{document}


